\documentclass{article}

\usepackage{pandekten}

\title{Allgemein}
\author{Ch\=an Taku}

\begin{document}

\maketitle

\section{Quantum Mechanics}

\subsection{Coherence}

\begin{definition}{Coherence Time}{coherence_time}
    The coherence time is given by
    \[ \tau \sim \frac{\hbar}{\Delta E} \]
    where $\Delta E$ is the uncertainty or fluctuation of energy.
\end{definition}

\section{Universal Gates}

\begin{theorem}{Decomposition of Unitary Operations}{decomposition_of_unitary_operations}
    Let $U \in \operatorname{U}\qty(\qty(\mathbb{C}^2)^{\otimes N})$.
    Then $U$ may be decomposed as
    \[ U = U_n \cdots U_1 \]
    where each $U_i$ is either a 1-bit gate or 2-bit gate,
    where a 1-bit gate $V$ is a unitary operator of the of the form
    \[ V = \sum_\mu c_\mu \sigma_0^{\otimes N_1} \otimes \sigma_\mu \otimes \sigma_0^{\otimes N-N_1-1}, \]
    and a 2-bit gate $V$ is a unitary operator of the form
    \[ V = \sum_{\mu,\nu} c_{\mu\nu} \sigma_0^{\otimes N_1} \otimes \sigma_{\mu} \otimes \sigma_0^{\otimes N_2} \otimes \sigma_{\nu} \otimes \sigma_0^{\otimes N - N_1 - N_2 - 2}. \]
\end{theorem}

\begin{theorem}{Hadamard Gate}{hadamard_gate}
    The Hadamard gate acts on $\mathbb{C}^2$ as the matrix
    \[ H = \begin{pmatrix}
        1 & 1 \\ 1 & -1
    \end{pmatrix}. \]
\end{theorem}

\begin{theorem}{T-Gate}{t-gate}
    The T-gate acts on $\mathbb{C}^2$ as the matrix
    \[ H = \begin{pmatrix}
        1 & \\ & e^{i\pi/4}
    \end{pmatrix}. \]
\end{theorem}

\begin{theorem}{CNOT-Gate}{cnot-gate}
    The CNOT-gate acts on $\mathbb{C}^2 \otimes \mathbb{C}^2$ as
    \[ \ket{0}\bra{0} \otimes \sigma_0 + \ket{1}\bra{1} \otimes \sigma_1. \]
\end{theorem}

\section{Atomic}

\subsection{Spin-Orbit Interaction}

When both orbital and spin angular momentum exist, product of irreducible representations of both is not irreducible, and we have to label each irreducible representation by $(\hat{S}^2, \hat{L}^2, (\hat{S} + \hat{L})^2)$.
The dimension of the vector space is $2J+1$ for each irreducible representation.
The split within each $(S,L)$ for different $J$ is proportional to the spin-orbit coupling.

\subsection{Hyperfine Interaction}

After taking the hyperfine interaction into account, the rotation $\operatorname{SO}(3)\times \operatorname{SO}(3)$ on electron and nuclear separately is reduced to a single $\operatorname{SO}(3)$ acting simutaneously on both. Therefore, each $(2J+1) \times (2I+1)$ space decomposed to spaces labelled by
\[ F = \abs{J-I}, \cdots, \abs{J+I}. \]

\subsection{Coupling to External Magnetic Field}

\subsubsection{Small Magnetic Field}

For extremely small (i.e. less than hyperfine interaction) magnetic field, the energy shift for each energy level in a $(2F+1)$ space is given by
\[ \Delta E = g_{F,J,I} \mu_{\mathrm{B}} B \cdot F_z, \]
where $g_{F,J,I}$ is the Land\'e factor.

\subsubsection{Large Magnetic Field}

In such case we have to lift first the degeneration of each $(2J+1)$ space, by
\[ \Delta E = g_J \mu_{\mathrm{B}} B_z J_z, \]
after which we include the first order perturbation
\begin{align*}
    \Delta E' &= A\langle \vb{I}\cdot \vb{J} \rangle \\ 
    &= A I_z J_z.
\end{align*}

\subsubsection{Comparable Case}

Join the lines without crossing.

\subsubsection{Selection Rules}

For hyperfine interaction, the selection rules are given by
\begin{align*}
    \Delta J &= \qty{0,\pm 1}, \\
    \Delta F &= \qty{0,\pm 1}, \\
    \Delta F_z &= \qty{0,\pm 1},
\end{align*}
and $F_z = 0\rightarrow 0$ is forbidden if $\Delta F = 0$.

\section{Manipulating Q-Bit}

\subsection{Electron in Magnetic Field}

The evolution of a spin in magnetic field is given by
\[ \ket{\chi(t)} = e^{-iHt/\hbar} \ket{\chi(0)}. \]
For a static magnetic field, the rotation is around the fixed axis along $\vb{B}$.
To rotate the spin from $\ket{+}$ to $\ket{-}$, an AC magnetic field is required.
\begin{align*}
    H &= \frac{1}{2}g\mu_{\mathrm{B}}\qty[ B_z \sigma_z + B(\sigma_x \cos\omega t + \sigma_y \sin\omega t) ] \\
    &= \begin{pmatrix}
        \omega_{\mathrm{L}}/2 & \Omega e^{-i\omega t} \\
        \Omega e^{i\omega t} & -\omega_{\mathrm{L}}/2
    \end{pmatrix},
\end{align*}
where
\[ \omega_{\mathrm{L}} = g \mu_{\mathrm{B}} B_z,\quad \Omega = \frac{1}{2} g \mu_{\mathrm{B}} B. \]

\section{Spin in Solid}

The Hamiltonian of the quantum dot is given by
\begin{align*}
    U(N) &= \sum_{m=1}^N E_m + \frac{1}{2C}\qty[-e(N-N_0) + C_g V_g]^2.
\end{align*}
The chemical potential is given by
\[ \mu(N) = \frac{e^2}{C} + (E_N - E_{N-1}). \]
Electrons keep coming in until $\mu(N)$ becomes greater than the chemical potential of the reservoir.
\par
Let the chemical potential of the left and right reservoirs be $\mu_{\mathrm{L}}$ and $\mu_{\mathrm{R}}$ respectively.
$V_g$ should be tuned so that $\mu_{\mathrm{L}} = \mu_{\mathrm{R}} = \mu(N)$ so that electrons could be transported from left to right.

% \bibliographystyle{plain}
% \bibliography{main}

\end{document}
