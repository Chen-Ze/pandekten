\documentclass{article}

\usepackage{pandekten}

\title{Field}
\author{Ch\=an Taku}

\begin{document}

\maketitle

\section{Notations, Common Techniques}

Spin operator
\[ \vb{S}(\vb{r}) = \frac{1}{2}\sum_{\sigma,\sigma'} c^\dagger_\sigma \vb*{\sigma}_{\sigma,\sigma'}c_{\sigma'}(\vb{r}). \]
Particle number
\[ n(\vb{r}) = \sum_{\sigma,\sigma'} c^\dagger_\sigma(\vb{r}) \mathbbm{1}_{\sigma,\sigma'} c_{\sigma'}(\vb{r}). \]

\begin{proposition}{Properties of Spin Operator}{properties_of_spin_operator}
    The spin operator $\vb{S}(\vb{r})$ satisfies the following.
    \begin{itemize}
        \item Transformation under $U\in\operatorname{SU}(2)$ is given by
        \[ U S^a(\vb{r}) U^{-1} = \sum_{b} R(U)^a_b S^a(\vb{r}). \]
        \item Relation to particle number:
        \[ \vb{S}(\vb{r})\cdot \vb{S}(\vb{r}) = \frac{3}{4}n(\vb{r}) - \frac{3}{2}n_{\uparrow}(\vb{r})n_{\downarrow}(\vb{r}). \]
        \item In the momentum space
        \[ \vb{S}(\vb{k}) = \frac{1}{2} \int \frac{\dd[d]{\vb{q}}}{(2\pi)^d} c^\dagger_\alpha(\vb{q}) \vb*{\sigma}_{\alpha\beta} c_\beta(\vb{q}+\vb{k}). \]
    \end{itemize}
\end{proposition}

\paragraph*{How do we accomodate magnetic field}
In the presence of a magnetic field, the hopping term is given by
\begin{align*}
    H_0 &= -t \sum_{\substack{\langle \vb{r},\vb{r}'\rangle \\ \sigma}} \Biggl\{
        c^\dagger_\sigma(\vb{r}) \exp\qty[ie \int_{\vb{r}}^{\vb{r}'} \dd{\vb{x}}\cdot \vb{A}(\vb{x})] c_\sigma(\vb{r}') \\
        &\phantom{{}= -t\sum_{\substack{\langle \vb{r},\vb{r}'\rangle \\ \sigma}}\Biggl\{} + c^\dagger_\sigma(\vb{r}') \exp\qty[-ie \int_{\vb{r}}^{\vb{r}'} \dd{\vb{x}}\cdot \vb{A}(\vb{x})] c_\sigma(\vb{r}) \Biggl\}.
\end{align*}

\paragraph*{Correlation Functions}
A few common correlation functions are listed here.
\begin{align*}
    G_{\sigma\sigma'}(\vb{r},t;\vb{r}',t') &= -i \ev**{T c_\sigma(\vb{r},t) c^\dagger_{\sigma'}(\vb{r}',t')}{\mathrm{GS}}, \\
    K_{00}(\vb{r},t;\vb{r}',t') &= \ev**{T \hat{n}(\vb{r},t) \hat{n}(\vb{r}',t')}{\mathrm{GS}}, \\
    K^{aa'}(\vb{r},t;\vb{r}',t') &= \ev**{T S^a(\vb{r},t) S^{a'}(\vb{r}',t')}{\mathrm{GS}},
\end{align*}
where
\[ \hat{n}(\vb{r},t) = n(\vb{r},t) - \ev{n(\vb{r},t)}. \]

\paragraph*{Mean Field Approximation}
Let $H_{\mathrm{i}}(X)$ be the interaction part of $H$.
Applying mean field approximation to $H'$ with respect to field $X$ is equivalent to introducing a field $M$ and replacing
\[ H_{\mathrm{i}}(X) \rightarrow H_{\mathrm{i}}(M) + H'_{\mathrm{i}}(M)(X - M).  \]

\begin{example}{Hubbard Model, Mean Field Approximation}{hubbard_model_mean_field_approximation}
    The Hubbard Hamiltonian is given by
    \[ H = -t\sum_{\substack{\vb{r},\vb{r}'\\ \sigma}} \qty[c^\dagger_\sigma(\vb{r}) c_{\sigma}(\vb{r}') + c^\dagger_\sigma(\vb{r}')c_\sigma(\vb{r})] - \frac{2}{3} U \sum_{\vb{r}} \vb{S}(\vb{r}) \cdot \vb{S}(\vb{r}). \]
    Applying the mean field approximation we find
    \[ H = \int \frac{\dd[d]{\vb{k}}}{(2\pi)^d} \qty(\sum_\sigma \epsilon(\vb{k}) n_\sigma(\vb{k}) + \frac{3}{8U}\vb{M}^*(\vb{k})\cdot \vb{M}(\vb{k}) + \vb{M}^*(\vb{k})\cdot \vb{S}(\vb{k})). \]
\end{example}

\paragraph*{Path Integral}
For boson field we find
\[ Z = \int \mathcal{D}\Psi^* \mathcal{D}\Psi \exp{i\int \dd{t} \qty[\sum_{\vb{r}} \Psi^* i\partial_t \Psi - H(\Psi^*,\Psi)]}. \]
Similarly for fermion field we find
\[ Z = \int \mathcal{D}\overline{\Psi} \mathcal{D}\Psi \exp{i\int \dd{t} \qty[\sum_{\vb{r}} \overline{\Psi} i\partial_t \Psi - H(\overline{\Psi},\Psi)]}. \]

\paragraph*{Wick Rotation, Imaginary Time}
$t = i\tau$.

\paragraph*{Hubbard-Stratonovich Transformation}
We may introduce a bosonic field to reduce a quartic interaction to a Yukawa one.
\[ \int \mathcal{D}\vb{\phi} \exp[-i\qty(\frac{1}{2}\vb{\phi}\cdot \vb{\phi} + \lambda \vb{\phi}\cdot \blacksquare)] = \mathrm{const.} \times \exp[\frac{1}{2}i\lambda^2 (\blacksquare)^2]. \]

\begin{example}{Hubbard Model, Path Integral}{hubbard_model_path_integral}
    The Lagrangian of the Hubbard model is given by
    \begin{align*}
        \mathcal{L}(\vb{r},t) &= \sum_\alpha \overline{\Psi}_\alpha(\vb{r},t) (i\partial_t + \mu) \Psi_\alpha(\vb{r},t) \\
        &\phantom{{}={}} + t \sum_\alpha \sum_{\langle \vb{r},\vb{r}' \rangle} \qty(\overline{\Psi}_\alpha(\vb{r},t) \Psi_\alpha(\vb{r}',t)+\overline{\Psi}_\alpha(\vb{r}',t) \Psi_\alpha(\vb{r},t)) \\
        &\phantom{{}={}} +\frac{U}{6} \sum_{\alpha,\beta} \qty(\overline{\Psi}_\alpha(\vb{r},t) \vb*{\sigma}_{\alpha\beta}\Psi_\beta(\vb{r},t))^2.
    \end{align*}
    Applying the Hubbard-Stratonovich transformation we get
    \begin{align*}
        \mathcal{L}(\vb{r},t) &= \sum_\alpha \overline{\Psi}_\alpha(\vb{r},t) (i\partial_t + \mu) \Psi_\alpha(\vb{r},t) \\
        &\phantom{{}={}} + t \sum_\alpha \sum_{\langle \vb{r},\vb{r}' \rangle} \qty(\overline{\Psi}_\alpha(\vb{r},t) \Psi_\alpha(\vb{r}',t)+\overline{\Psi}_\alpha(\vb{r}',t) \Psi_\alpha(\vb{r},t)) \\
        &\phantom{{}={}} - \sqrt{\frac{U}{3}} \vb*{\phi}(\vb{r},t) \cdot \sum_{\alpha,\beta}\overline{\Psi}_\alpha(\vb{r},t) \vb*{\sigma}_{\alpha\beta} \Psi_\beta(\vb{r},t) - \frac{1}{2}\vb*{\phi}(\vb{r},t) \cdot \vb*{\phi}(\vb{r},t).
    \end{align*}
    Integrating out the fermions, we find
    \[ Z = \int \mathcal{D}\vb*{\phi} e^{iS_{\mathrm{eff}}(\vb*{\phi})}, \]
    where
    \[ S_{\mathrm{eff}}(\vb*{\phi}) = -\int \dd{t} \sum_{\vb{r}} \frac{1}{2}\vb*{\phi}(\vb{r},t)\cdot \vb*{\phi}(\vb{r},t) - i\ln\det\qty(i\partial_t + \mu - \mathcal{M}(\vb*{\phi})), \]
    and
    \begin{align*}
        \bra{\vb{r},t,\alpha} \mathcal{M}(\vb*{\phi})\ket{\vb{r}',t',\beta} &= -\delta_{\alpha\beta}\delta(t-t') t \sum_{1\le j\le d} \qty(\delta_{\vb{r}',\vb{r}+\vb{e}_j} + \delta_{\vb{r}',\vb{r}-\vb{e}_j}) \\
        &\phantom{{}={}} + \sqrt{\frac{U}{3}}\delta(t-t')\delta_{\vb{r},\vb{r}'}\vb*{\phi}(\vb{r},t)\cdot \vb*{\sigma}_{\alpha\beta}.
    \end{align*}
    The stationary condition on $\vb*{\phi}$ gives
    \begin{align*}
        \vb*{\phi}(\vb{r},t) &= i\sqrt{\frac{U}{3}} \sum_{\alpha\beta} \vb*{\sigma}_{\beta\alpha} \mel**{\vb{r},t,\alpha}{\frac{1}{i\partial_t + \mu - \mathcal{M}(\vb*{\phi})}}{\vb{r'},t',\beta} \\
        &= -\sqrt{\frac{U}{3}} \sum_{\alpha\beta} \vb*{\sigma}_{\beta\alpha} G_{\alpha\beta}(\vb{r},t;\vb{r},t;\vb*{\phi}) \\
        &= -\sqrt{\frac{4U}{3}}\ev{\vb{S}(\vb{r},t)}.
    \end{align*}
    In 2-dimensional AFM state,
    \[ \vb*{\phi}(\vb{r},t) = \abs{\vb*{\phi}}\vb{n} (-1)^{x_1+x_2}, \]
    and we find the Green's function
    \[ G(\vb{k},\omega) = \frac{-i}{\omega^2 - \qty(\epsilon^2(\vb{k})+\dfrac{U}{3}\abs{\vb*{\phi}}^2)} \begin{pmatrix}
        \omega + \epsilon(\vb{k}) & \sqrt{U/3}\abs{\vb*{\phi}}\vb{n}\cdot \vb*{\sigma} \\
        \sqrt{U/3}\abs{\vb*{\phi}}\vb{n}\cdot \vb*{\sigma} & \omega - \epsilon(\vb{k})
    \end{pmatrix}. \]
    The spectrum becomes
    \[ E(\vb{k}) = \sqrt{\epsilon^2(\vb{k}) + \frac{U}{3}\abs{\vb*{\phi}}^2}. \]
\end{example}

\paragraph*{Path Integral for Spins}
The partition function is given by
\[ Z = \int \mathcal{D} \vb{n} \exp(i S_{\mathrm{M}}(M)[\vb{n}]) \]
where
\[ S_{\mathrm{M}}(\vb{n}) = S S_{\mathrm{WZ}[\vb{n}]}[\vb{n}] + \frac{S \delta t}{4}\int_0^T \dd{s} (\partial_s \vb{n}(s))^2 - S \int_0^T \dd{s} H(\vb{n}), \]
where $S$ is the spin and $H$ is the Hamiltonian, and the Wess-Zumino term $S_{\mathrm{WZ}}[\vb{n}]$ is the area on the sphere enclosed by the trajectory $\vb{n}$.

\paragraph*{Jordan-Schwinger Construction}
With $[a,a^\dagger] = [b,b^\dagger] = 1$, and that $a$ and $b$ commute, then the angular momentum may be written as
\begin{align*}
    J_z &= \frac{1}{2}(a^\dagger a - b^\dagger b), \\
    J_x + i J_y &= a^\dagger b, \\
    J_x - i J_y &= b^\dagger a,
\end{align*}
and
\[ \ket{j,m} = \frac{1}{\sqrt{(j+m)! (j-m)!}}(a^\dagger)^{j+m} (b^\dagger)^{j-m}\ket{j,0}. \]

\paragraph*{Majorana Operators}
Let $a_k$ be fermionic operators.
Define
\[ c_{2k-1} = a_k + a_k^\dagger,\quad c_{2k} = \frac{a_k - a_k^\dagger}{i}. \]
Then we have
\begin{itemize}
    \item $c_j$ are Hermitian.
    \item $c_j$ obey the follow relations:
    \[ \qty{c_i,c_j} = 2\delta_{ij}. \]
\end{itemize}

Let $b^x$, $b^y$, $b^z$ and $c$ be a set of Majorana operators from the fermionic operators $c_\uparrow$ and $c_\downarrow$.
The 4-dimensional Fock space is denoted by $\tilde{\mathcal{M}}$, while the Hilbert space of a spin is the 2-dimensional subspace $\mathcal{M}\subset \tilde{\mathcal{M}}$ defined by
\[ \mathcal{M} = \Set*{\ket{\xi}}{b^x b^y b^z c\ket{\xi} = \ket{\xi}}. \]
We can replace on each site
\begin{align*}
    \tilde{\sigma}^x &= ib^x c, \\
    \tilde{\sigma}^y &= ib^y c, \\
    \tilde{\sigma}^z &= ib^z c.
\end{align*}
With such replacement, any spin Hamiltonian $H\qty{\sigma^\alpha_j}$ could be replaced by a fermionic one $\tilde{H}\qty{b^\alpha_j, c_j}$.

\section{Note on Gauge Theory}

\subsection{Particle-Vortex System}

The Lagrangian of a particle in a magnetic field is given by
\[ \mathcal{L} = \frac{1}{2}m\vb{v}^2 + \frac{e}{c}\vb{v}\cdot \vb{A}. \]
The Hamiltonian is
\[ H = \frac{1}{2} m \vb{v}^2, \]
with no gauge field present.
What tells this system apart from a free particle is the momentum
\[ \vb{p} = m\vb{v} + \frac{e}{c} \vb{A}. \]
Let $\vb{A}$ be given by that of a monopole
\[ \vb{A} = \frac{\Phi}{2\pi} \qty(\frac{-y}{x^2 + y^2} \dd{x} + \frac{x}{x^2 + y^2}\dd{y}). \]
The canonical momentum is given by
\[ J_{\mathrm{c}} = \star(\vb{r} \wedge \vb{p}) = J + \frac{e\Phi}{2\pi c} \]
where
\[ J = \star(\vb{r}\wedge m\vb{v}). \]
$J_{\mathrm{c}} = -i\hbar \partial_\varphi$ as in the free particle case.
Therefore, the eigenvalues of $J_{\mathrm{c}}$ are quantized to integer.
\[ J = \hbar\qty(m - \frac{e\Phi}{hc}),\quad m\in\mathbb{Z}. \]
The second part is the spin of the cyon, i.e.
\[ s = \frac{-e\Phi}{hc}. \]
Moving the particle around acquires a phase factor
\[ \exp\qty(-i \frac{e}{\hbar c} \int_\Gamma \dd{\vb{r}}\cdot \vb{A}). \]
More generally the parallel transport is done by
\[ \mathcal{P} \exp(-\int_\Gamma \mathcal{A}). \]
Move the cyon around acquires a phase $e^{i\nu\theta}$ where
\[ \nu = -\frac{2e\Phi}{hc}. \]

\subsection{Chern-Simons Action}
The Chern-Simons action in $\mathbb{R}^3$ is given by
\[ S = \int \tr\qty(A \wedge \dd{A} + \frac{2}{3}A\wedge A \wedge A). \]
The Chern-Simons form is invariant under gauge transformation (those connected to the identity).
Such term may be added into the Young-Mills action.
For $\operatorname{U}(1)$ gauge theory,
\[ S = \int \tr\qty(A\wedge \dd{A}) = \int \tr\qty(A\wedge F). \]
With
\begin{align*}
    S &= S_{\mathrm{matter}} + S_{\mathrm{int}} + S_{\mathrm{CS}}, \\
    S_{\mathrm{matter}} &= \int \dd{t} \qty(\sum_{I=1}^N \frac{1}{2} m\vb{v}_I^2), \\
    S_{\mathrm{int}} &= -\frac{1}{c^2} \int \dd[3]{x} j^\alpha(x) A_\alpha(x), \\
    S_{\mathrm{CS}} &= \frac{\kappa}{2c} \int A \wedge \dd{A},
\end{align*}
we find that
\begin{align*}
    E^i &= \frac{1}{\kappa c} \epsilon^{ij} j^j, \\
    B &= -\frac{1}{\kappa} \rho,
\end{align*}
and that every charge curries a magnetic flux.

\paragraph*{Gauge Fixing}%
The Weyl gauge $A_0 = 0$ does not fix the gauge completely.
One could impose further $\partial_i A^i = 0$.
\par
Now we find
\[ H = \sum_{I=1}^N \frac{1}{2m}\qty(\vb{p}_I - \frac{e}{c}\vb{A}_I(\vb{r}_1,\cdots,\vb{r}_N))^2, \]
and
\[ A^i_I(\vb{r}_1,\cdots,\vb{r}_N) = -\frac{e}{2\pi \kappa} \pdv{}{r^i_I} \sum_{J\neq I} \varphi_{IJ}, \]
where $\varphi_{IJ}$ is the winding angle of particle $J$ with respect to particle $I$.
Now the Lagrangian has the anyon form
\[ \mathcal{L} = \sum_{I=1}^N \qty(\frac{1}{2}m\vb{v}_I^2) - \frac{e^2}{2\pi c\kappa}\qty(\sum_{I<J} \dv{}{t}\varphi_{IJ}), \]
and the particles has the anyon statistics
\[ \nu = \frac{\kappa \Phi^2}{2\pi\hbar c}. \]

\paragraph*{Charge}
The following two definitions of charge coincides.
\begin{itemize}
    \item The noether charge of gauge transformation, i.e. $q = j_0$ where
    \[ \delta\mathcal{L} = \int \dd[d-1]{x} j^\alpha \partial_\alpha \Lambda. \]
    \item The charge of parallel transport, as in
    \[ \exp\qty(-iq \int_\Gamma A). \]
\end{itemize}

\paragraph*{Berry Phase}
A frequently used expression for Berry phase (abelian $G$-bundle) is
\[ \gamma = -iq \oint_\Gamma A. \]
If $\Gamma$ may fail to be contained in a single chart, then the gauge transformation
(Note that chart is different from local trivialization.
$\Gamma$ may well be contained in a single trivialization.)
\[ A \rightarrow A + g^{-1} \dd{g} \]
may not have the form
\[ A \rightarrow A + \dd{\Lambda}. \]
Therefore, $\gamma$ may be changed by a nonzero constant.
In the abelian case,
\[ \gamma = -iq \oint_\Gamma \sigma^* \omega = -iq \oint_{\sigma(\Gamma)} \omega \]
and may well depend on the section $\sigma$.

\section{Miscellaneous}

\paragraph*{Global Symmetry and Irreducible Representation}%
If $G$ is the group of global symmetry and $\rho:G\rightarrow \operatorname{Aut}(H)$, then the symmetry breaking groud states may not span irreducible subspaces of $\rho$.
For example, the Ising model has $\mathbb{Z}_2$ global symmetry but the symmetry breaking groud states are $\ket{1\cdots 1}$ and $\ket{0\cdots 0}$, while the irreducible subspaces (one-dimensional) are spanned by
\[ \frac{1}{\sqrt{2}}(\ket{1\cdots 1} \pm \ket{0\cdots 0}). \]

% \bibliographystyle{plain}
% \bibliography{main}

\end{document}
