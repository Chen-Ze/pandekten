\documentclass{article}

\usepackage{pandekten}

\title{Disorder}
\author{Ch\=an Taku}

\begin{document}

\maketitle

\section{Disorder in Hermitian Systems}

\paragraph*{Averaged Action}
After averaging with respect to $V$, the replica action is given by
\begin{align*}
    S[\psi,\overline{\psi}] &= \int \dd{\tau} \dd[d]{r} \overline{\psi}\qty(\partial_\tau - E_F - \frac{\grad^2}{2m}) \psi \\
    &{\phantom{{}={}}} + \frac{1}{4\pi \nu \tau} \int \dd{\tau} \dd{\tau'}\dd[d]{r} (\overline{\psi}\psi)(\tau)(\overline{\psi}\psi)(\tau')
\end{align*}
where
\begin{itemize}
    \item $\tau$ on the denominator is the scattering time,
    \item $\nu$ is the density of states, and
    \item $\psi = \qty{\psi^a}$ carries a replica index $a = 1,\cdots,R$.
\end{itemize}

\paragraph*{Symmetrized Action and Enlarged Field}
The symmetrized action is given by
\begin{align}
    S[\Psi] &= \frac{1}{2} \int \dd{\tau} \dd[d]{r} \overline{\Psi} \qty(\partial_\tau - E_F - \frac{\grad^2}{2m}) \Psi \\
    \label{eq:action_enlarged}
    &{\phantom{{}={}}} + \frac{1}{16\pi\nu\tau} \int \dd{\tau}\dd{\tau'} \dd[d]{r} (\overline{\Psi}\Psi)(\tau) (\overline{\Psi}\Psi)(\tau')
\end{align}
where
\begin{itemize}
    \item the enlarged field is given by
    \[ \Psi(\tau) = \begin{pmatrix}
        \psi(\tau) \\ \overline{\psi}^\intercal(-\tau)
    \end{pmatrix},\quad \overline{\psi}(\tau) = \begin{pmatrix}
        \overline{\psi}(\tau), -\psi^\intercal(-\tau)
    \end{pmatrix}, \]
    \item the components of which are not independent of each other {\color{red}why the following?}
    \begin{equation*}
        \label{eq:psi_bar_relation}
        \overline{\psi}(\tau) = -\psi^\intercal(-\tau)(i\sigma_2^{\mathrm{TR}}),
    \end{equation*}
    where the superscript $\mathrm{TR}$ denotes the time-reversal space.
\end{itemize}

\paragraph*{Hubbard-Stratonovich Transformation}
Keeping only the Cooper and exchange contribution, we find
\[ Z = \int \mathcal{D}{Q} \exp[-\frac{\pi \nu}{8\tau} \int \dd[d]{r} \tr Q^2 + \frac{1}{2} \tr \ln \hat{G}^{-1}[Q]] \]
where
\begin{itemize}
    \item $\hat{G}^{-1}[Q]$ for each $Q$ is an operator acting on $\Psi$ given by
    \[ -\hat{G}^{-1}[Q] = \qty(\partial_\tau - E_F - \frac{\grad^2}{2m} -\frac{i}{2\tau}Q), \]
    \item where $Q = \qty{Q^{\alpha\beta}}(r;\tau,\tau')$ has indices $\alpha = (a,\sigma)$ where $a$ is the replica index and $\sigma = 1,2$ labels the time-reversal space.
\end{itemize}

\paragraph*{Stationary Point}
Solving for the stationary point of $S[Q]$ under the ansatz that $Q=\Lambda$ is diagonal in time-reversal space (i.e. in $\sigma$), replica space (i.e. in $a$), in Matsubara space, and uniform spatially, we find
\[ \pi \nu \Lambda = \frac{i}{L^d} \sum_{\vb{p}} \frac{1}{i\omega_n - \xi_{\vb{p}} + (i/(2\tau))\Lambda}. \]
Therefore, the stationary points of $Q$ satisfy the constraint $Q^2 = \mathbbm{1}$.

\paragraph*{Symmetry of the Original Problem}
By demanding that under the transformation
\[ \Psi \rightarrow T\Psi,\quad \overline{\Psi}\rightarrow \overline{\Psi}\overline{T} \]
both
\begin{itemize}
    \item the relation \eqref{eq:psi_bar_relation}, and
    \item the action \eqref{eq:action_enlarged}
\end{itemize}
should be invariant, we find
\[ T \in \operatorname{Sp}(2\cdot R\cdot (2M)) \]
where $[-2\pi MT, 2\pi MT]$ is the range of $2M$ Matsubara frequencies overwhich the lack of commutativity of the symmetry group and the frequency operator can be safely ignored.

\paragraph*{Symmetry of the Stationary Point}
The subgroup of $\operatorname{Sp}(4RM)$ under which $\Lambda = T\Lambda T^{-1}$ is given by $\operatorname{Sp}(2RM) \times \operatorname{Sp}(2RM)$ where the coset space, i.e. the Goldstone mode, is given by the quotient space
\[ \operatorname{Sp}(4RM)/\qty(\operatorname{Sp}(2RM) \times \operatorname{Sp}(2RM)). \]

\paragraph*{Goldstone Mode Action}
The action of $Q$ with flucuations taken into account is given by
\[ S[Q] = \frac{\pi\nu}{2} \int \dd[d]{r} \tr[\frac{D}{4} (\grad Q)^2 - \hat{\omega}Q]. \]


% \bibliographystyle{plain}
% \bibliography{main}

\end{document}
