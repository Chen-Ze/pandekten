\documentclass{article}

\usepackage{pandekten}

\title{Renormalization Group}
\author{Ch\=an Taku}

\begin{document}

\maketitle

\section{Renormalization Conditions}

\text{We are dealing with massless field.}
Imposing the renormalization condition at scale $M$ we find
\[ M^2 = 0,\quad p_E^2 = M^2 \]
and
\[ \pdv{p^2}M^2 = 0,\quad p_E^2 = M^2. \]
Let $\phi_0$ be the bare field and $\phi = Z^{-1/2}\phi_0$ be the renormalized field.
Since the field is massless, near $p_E^2 = M^2$ we have
\[ \langle \phi_0(p) \phi_0(-p) \rangle = \frac{iZ}{p^2}. \]
Then
\[ G^{(n)}(x_1,\cdots,x_n) = \langle \phi(x_1) \cdots \phi(x_n) \rangle = Z^{-n/2}\langle \phi_0(x_1) \cdots \phi_0(x_n) \rangle. \]
We denote
\[ G_0^{(n)}(x_1,\cdots,x_n) = \langle \phi_0(x_1) \cdots \phi_0(x_n) \rangle. \]
$G^{(n)}$ depends on $\Lambda$ and $\lambda_0$, and should be unaware of the renormalization scale $M$.
Therefore,
\begin{align*}
    M \dv{M} \qty(Z^{n/2} G^{(n)}) &= 0.\\
    M \pdv{M} G^{(n)} + M \pdv{\lambda}{M} \pdv{\lambda} G^{(n)} + \frac{n}{2} M \pdv{Z}{M} Z^{-1} G^{(n)} &= 0,
\end{align*}
where the partial derivatives are all taken at $\lambda_0,\Lambda$.
Denote
\begin{align*}
    \beta(\lambda) &= M \pdv{\lambda}{M},\\
    \gamma(\lambda) &= \frac{1}{2} M \pdv{\log Z}{M}.
\end{align*}
We find the Callan-Symanzik equation
\[ \qty(M \pdv{M} + \beta(\lambda) \pdv{\lambda} + n\gamma(\lambda)) G^{(n)} = 0. \]

Writing in terms of 1PI functions,
\begin{align*}
    \Gamma^{(2)} &= \frac{1}{G^{(2)}(p)} = \frac{Z}{G^{(2)}_0(p)} = Z \Gamma^{(2)}_0(p), \\
    \Gamma^{(3)}(p_1,p_2,p_3) &= -\frac{G^{(3)}(p_1,p_2,p_3)}{G^{(2)}(p_1)G^{(2)}(p_2)G^{(2)}(p_3)} = Z^{3/2} \Gamma^{(3)}_0(p_1,p_2,p_3), \\
    \Gamma^{(n)} &= Z^{n/2} \Gamma_0^{(n)}.
\end{align*}
Therefore,
\[ \qty(M \pdv{M} + \beta(\lambda)\pdv{\lambda} - n\gamma(\lambda)) \Gamma^{(n)}(p_1,\cdots,p_n) = 0. \]
\begin{example}{$\phi^4$ Sunset $\delta$}{phi_4_sunset_delta}
    To the lowest order,
    \[ \Gamma^{(2)}(p) = p^2 + \frac{\lambda^2}{12(4\pi)^2} p^2 \ln \qty(\frac{\Lambda^2}{p^2}) + p^2 \delta_Z = p^2 + \frac{\lambda^2}{12(4\pi)^2}p^2 \ln \frac{M^2}{p^2}. \]
    Therefore,
    \[ \delta_Z = \frac{\lambda^2}{12(4\pi)^2} \ln \frac{M^2}{\Lambda^2}. \]
    To the lowest order,
    \[ \qty(M\pdv{M} - 2\gamma(\lambda))\Gamma^{(2)} = 0. \]
\end{example}

\subsection{\texorpdfstring{$\phi^4$}{phi4} RG}

Imposing $\Gamma^{(4)} = \lambda$ at $p^2 = -M^2$ we find
\[ \Gamma^{(4)} = \lambda - \frac{\lambda^2}{2(4\pi)^2}\qty[\ln \frac{M^2}{s} + \ln \frac{M^2}{t} + \ln \frac{M^2}{u}]. \]
The Callan-Symanzik equation
\[ \qty(M \pdv{M} + \beta \pdv{\lambda} - n\gamma)\Gamma^{(n)} = 0 \]
yields
\[ \beta(\lambda) - 4\gamma(\lambda)\lambda = -M \pdv{M}\delta_\lambda = \frac{3\lambda^2}{(4\pi)^2} + \bigO(\lambda^3), \]
and
\[ \gamma = \frac{1}{2} M \pdv{M} \delta_Z = \frac{\lambda^2}{12(4\pi)^2} \]
from the two-loop sunset diagram.

\subsection{Yukawa RG}

The Callan-Symanzik equation reads
\[ \qty(M\pdv{M} + \beta_g \pdv{g} + \beta_\lambda \pdv{\lambda} - n\gamma_\psi - m\gamma_\phi) \Gamma^{(n,m)} = 0. \]
From $\Gamma^{(2,1)}$ and $\Gamma^{(4,0)}$ we read $\beta_g$ and $\beta_\lambda$.

% \bibliographystyle{plain}
% \bibliography{main}

\end{document}
