\documentclass{article}

\usepackage{pandekten}

\title{Quantization}
\author{Ch\=an Taku}

\externaldocument[Alg-]{../../../../Mathematics/Algebra/Allgemein/main}

\begin{document}

\maketitle

\section{Prerequisites}

\subsection{Quantization of Symmetry}

\begin{definition}{Strong Operator Topology (Hilbert Space)}{strong_operator_topology_hilbert_space}
    Input:
    \begin{itemize}
        \item $H$ is a Hilbert space.
        \item $\mathscr{B}_{\mathbb{R}}(H)$ deote $\mathbb{R}$-linear bounded operators.
    \end{itemize}
    The strong operator topology on $\mathscr{B}_{\mathbb{R}}(H)$ is generated by the subbasis
    \[ \Set*{V_f(U_0,r)}{U_0\in\mathscr{B}_{\mathbb{R}}(H), f\in H,r>0} \]
    where
    \[ V_f(U_0,r) = \Set*{U\in\mathscr{B}_{\mathbb{R}}(H)}{\norm{U_0(f) - U(f)} < r}. \]
\end{definition}

\begin{definition}{Strong Operator Topology (Projective Hilbert Space)}{strong_operator_topology_projective_hilbert_space}
    Input:
    \begin{itemize}
        \item $H$ is a Hilbert space.
        \item $\gamma: H\rightarrow H/\mathbb{C}^\times$ is the projection.
        \item $\operatorname{Aut}(H/\mathbb{C}^\times)$ denotes the group of projective transformations.
    \end{itemize}
    The strong operator topology on $\operatorname{Aut}(H/\mathbb{C}^\times)$ is generated by the subbasis
    \[ \Set*{V_\varphi(U_0,r)}{U_0\in\operatorname{Aut}(H/\mathbb{C}^\times), \varphi\in H/\mathbb{C}^\times,r>0} \]
    where
    \[ V_\varphi(U_0,r) = \Set*{U\in\operatorname{Aut}(H/\mathbb{C}^\times)}{\delta(U_0(\varphi),U(\varphi)) < r}, \]
    and
    \[ \delta(\varphi,\psi) = \frac{\abs{\langle f,g \rangle}^2}{\norm{f}^2 \norm{g}^2}, \]
    where $f,g\in H$ such that $\gamma(\varphi) = f$ and $\gamma(\psi) = g$.
\end{definition}

\begin{theorem}{General Topology of Unitary Groups}{general_topology_of_unitary_groups}
    Input:
    \begin{itemize}
        \item $H$ is a Hilbert space.
        \item $\gamma: H\rightarrow H/\mathbb{C}^\times$ is the projection.
        \item $\operatorname{Aut}(H/\mathbb{C}^\times)$ denotes the group of projective transformations.
        \item $\hat{\gamma}:\operatorname{U}(H) \rightarrow \operatorname{Aut}(H/\mathbb{C}^\times)$ denote the induced projective transformation (\cref{Alg-def:induced_projective_transformation}).
        \item $\operatorname{U}(H/\mathbb{C}^\times) = \hat{\gamma}(\operatorname{U}(H))$.
    \end{itemize}
    Then the following statements hold.
    \begin{itemize}
        \item $\operatorname{U}(H)$, $\operatorname{Aut}(H/\mathbb{C}^\times)$, and $\operatorname{U}(H/\mathbb{C}^\times)$ are all topological groups with respect to the strong operator topology.
        \item $\operatorname{U}(H)$ is connected and is pathwise connected with respect to the strong operator topology.
        Every unitary operator is in the orbit of a suitable one-parameter group $\exp(iAt)$.
        \item $\hat{\gamma}:\operatorname{U}(H) \rightarrow \operatorname{U}(H/\mathbb{C}^\times)$ is a continuous homomorphism.
        \item $\operatorname{U}(H/\mathbb{C}^\times)$ is a connected metrizable group.
        $\operatorname{U}(H/\mathbb{C}^\times)$ is the connected component containing the identity in $\operatorname{Aut}(H/\mathbb{C}^\times)$.
        \item Every continuous homomorphism $T:G\rightarrow \operatorname{Aut}(H/\mathbb{C}^\times)$ on a connected topological group $G$ has its image in $\operatorname{U}(H/\mathbb{C}^\times)$.
    \end{itemize}
\end{theorem}

\begin{theorem}{Quantization of Symmetry Group}{quantization_of_symmetry_group}
    Input:
    \begin{itemize}
        \item $H$ is a Hilbert space.
        \item $\gamma: H\rightarrow H/\mathbb{C}^\times$ is the projection.
        \item $\operatorname{Aut}(H/\mathbb{C}^\times)$ denotes the group of projective transformations.
        \item $\hat{\gamma}:\operatorname{U}(H) \rightarrow \operatorname{Aut}(H/\mathbb{C}^\times)$ denote the induced projective transformation (\cref{Alg-def:induced_projective_transformation}).
        \item $\operatorname{U}(H/\mathbb{C}^\times) = \hat{\gamma}(\operatorname{U}(H))$.
        \item $G$ is a group and $T:G\rightarrow \operatorname{U}(H/\mathbb{C}^\times)$ is a projective representation.
    \end{itemize}
    Then there is
    \begin{itemize}
        \item a central extension $E$ of $G$ by $\operatorname{U}(1)$, and
        \item a unitary representation $S:E\rightarrow\operatorname{U}(H)$,
    \end{itemize}
    such that the following diagram commutes and that the rows are exact.
    \begin{center}
        \begin{tikzcd}
            1 \arrow[r] & \operatorname{U}(1) \arrow[r,"\imath"] \arrow[d,"\operatorname{id}"] & E \arrow[r,"\pi"] \arrow[d,"S"] & G \arrow[r] \arrow[d,"T"] & 1 \\
            %
            1 \arrow[r] & \operatorname{U}(1) \arrow[r] & \operatorname{U}(H) \arrow[r,"\hat{\gamma}"] & \operatorname{U}(H/\mathbb{C}^\times) \arrow[r] & 1
        \end{tikzcd}
    \end{center}
    \paragraph*{With Continuity}
    If
    \begin{itemize}
        \item $G$ is a topological group,
        \item $\operatorname{U}(H)$, $\operatorname{Aut}(H/\mathbb{C}^\times)$, and $\operatorname{U}(H/\mathbb{C}^\times)$ are given the strong operator topology, and
        \item $T$ is continuous,
    \end{itemize}
    then
    \begin{itemize}
        \item $E$ has a structure of a topological group such that $\imath$, $\pi$, and $S$ are continuous.
    \end{itemize}
\end{theorem}

The exact sequence at the bottom has no global section (as a group homomorphism or as a continuous function) \cite{IntroCFT}.
It is nontrivial as fiber bundles (with respect to norm topology or strong topology).

Since $\operatorname{U}(H/\mathbb{C}^\times)$ is connected and is the connected component of $\operatorname{Aut}(H/\mathbb{C}^\times)$ containing the identity, if $G$ is a continuous group, it suffices to study the unitary projective transformations and discard the nonunitary ones.

The Quantization of a classical symmetry group $G$ is a central extension by $\operatorname{U}(1)$ of the universal cover of $G$.

\begin{definition}{Canonical Anticommutation Relation Algebra, CAR Algebra}{car_algebra}
    Let $H$ be a Hilbert space.
    The canonical anticommutation algebra $\mathscr{A}(H)$ is the universal unital $C^*$-algebra generated by the annihilation operators $\Set*{a(f)}{f\in H}$ and creation operators $\Set*{a^*(f)}{f\in H}$, such that
    \begin{itemize}
        \item the following commutations relations are satisfied:
        \begin{align*}
            a(f) a^*(g) + a^*(g) a(f) &= \langle f,g \rangle \mathbf{1}, \\
            a^*(f) a^*(g) + a^*(g) a^*(f) &= 0, \\
            a(f) a(g) + a(g) a(f) &= 0,
        \end{align*}
        \item and that
        \begin{itemize}
            \item $a^*: H \rightarrow \mathscr{A}(H)$ is a complex linear map, and
            \item $a: H \rightarrow \mathscr{A}(H)$ is an complex antilinear-linear map.
        \end{itemize}
    \end{itemize}
\end{definition}

\begin{example}{Quantization of Electrodynamics}{quantization_of_electrodynamics}
    \paragraph*{First-Quantized Hilbert Space}
    The first quantization is given by the one-particle space
    \[ H = H_+ \oplus H_- \]
    where $H_+$ is the positive eigenspace and $H_-$ is the negative eigenspace of the Dirac Hamiltonian on $H = L^2(\mathbb{R}^3,\mathbb{C}^4)$.
    \paragraph*{Second-Quantized Hilbert Space}
    The spinor space $S$ is defined by
    \[ S = \text{completion of } \qty(\wedge H_+) \otimes \qty(\wedge \overline{H_-}) \]
    where
    \[ \wedge W = \bigoplus_{p=0}^\infty \bigwedge^p W \]
    with the induced scalar product on $W$,
    and $\overline{H_-}$ is $H_-$ with scalar multiplication $(\lambda,w) \mapsto \overline{\lambda} w$ and scalar product conjugated.
    \paragraph*{Quantization of CAR Algebra}
    Let $f = f_+ + f_- \in H_+ \oplus H_-$ be a decomposition (polarization) of $f\in H$.
    Then $a^*(f) = a^*(f_+) + a^*(f_-)$ and $a(f) = a(f_+) + a(f_-)$.
    Let $\mathscr{A}(H)$ be the CAR algebra of $H$ and $\mathscr{B}(S)\subset \operatorname{End}(S)$ be the $C^*$-algebra of bounded $\mathbb{C}$-linear endomorphisms of $S$.
    Then we have a representation
    \[ \pi: \mathscr{A}(H) \rightarrow \mathscr{B}(S), \]
    defined by the following.
    \begin{align*}
        \pi(a^*)(f_+) \xi\otimes \eta &= (f_+ \wedge \xi) \otimes \eta, \\
        \pi(a^*)(f_-) (\xi\otimes g_1 \wedge \cdots \wedge g_m) &= \sum_{j=1}^{m} (-1)^{k+j+1} \\
        &\qquad \xi \otimes \langle g_j, f_- \rangle g_1 \wedge \cdots \wedge \cancel{g_j} \wedge \cdots \wedge g_m, \\
        \pi(a)(f_+) (f_1 \wedge \cdots \wedge f_n \otimes \eta) &= \sum_{j=1}^{n} (-1)^{j+1} \\
        &\qquad \langle f_+, f_j \rangle f_1 \wedge \cdots \wedge \cancel{f_j} \wedge \cdots \wedge f_n \otimes \eta, \\
        \pi(a)(f_-)(\xi\otimes \eta) &= (-1)^k \xi \otimes f_- \wedge \eta,
    \end{align*}
    where $f_1,\cdots,f_n\in H_+$, $g_1,\cdots,g_m \in H_-$, $\xi \in \wedge H_+$, and $\eta \in \wedge \overline{H_-}$.
    \paragraph*{Field Operator}
    The field operators are given by $\Phi: H \rightarrow \mathscr{B}(S)$ where
    \begin{align*}
        \Phi(f) &= \pi(a(f)), \\
        \Phi^*(f) &= \pi(a^*(f)).
    \end{align*}
    \paragraph*{Implementation of Unitary Operator}
    An implementation of a unitary operator $U\in \operatorname{U}(H)$ is a unitary operator $U^\sim \in \operatorname{U}(S)$ such that
    \begin{align*}
        U^\sim \circ \Phi(f) &= \Phi(Uf)\circ U^\sim, \\
        U^\sim \circ \Phi^*(f) &= \Phi^*(Uf)\circ U^\sim
    \end{align*}
    for all $f\in H$.
    \paragraph*{Existence of Implementation}
    Each untiary operator $U\in \operatorname{U}(H)$ has an implementation $U^\sim \in \operatorname{U}(H)$ if and only if in the block matrix representation of $U$,
    \[ U = \begin{pmatrix}
        U_{++} & U_{-+} \\
        U_{+-} & U_{--}
    \end{pmatrix}: H_+ \oplus H_- \rightarrow H_+ \oplus H_-, \]
    the off-diagonal components $U_{+-}: H_+ \rightarrow H_-$ and $U_{-+}: H_- \rightarrow H_+$ are Hilbert-Schmidt operators, i.e. $\sum_{i\in I} \norm{U e_i}^2$ is finite, where $\Set*{e_i}{i\in I}$ is a Schauder basis of $H$.
    \paragraph*{Uniqueness of Implementation}
    And two implementations $U^\sim$ and $U^{'\sim}$ of such an operator are the same up to a phase factor $\lambda \in \operatorname{U}(1)$, i.e. $U^\sim = \lambda U^{'\sim}$.
    \paragraph*{Restricted Unitary Group}
    The group of all implementable unitary operators on $H$ is called the restricted unitary group $\operatorname{U}_{\mathrm{res}}(H)$.
    The set of implemented operators
    \[ U^\sim_{\mathrm{res}}(S) = \Set*{V\in \operatorname{U}(S)}{V = U^\sim \text{ for some } U\in \operatorname{U}_{\mathrm{res}}(H_+)} \]
    is a subgroup of $\operatorname{U}(S)$.
    Then we have the following exact sequence
    \[ 1 \longrightarrow \operatorname{U}(1) \xlongrightarrow{\imath} \operatorname{U}^\sim_{\mathrm{res}}(S) \xlongrightarrow{\pi} \operatorname{U}_{\mathrm{res}}(H) \longrightarrow 1, \]
    where $\pi$ is the natural restriction and $\imath(\lambda) = \lambda \operatorname{id}_S$ for $\lambda \in \operatorname{U}(1)$.
    This exact sequence does not split, and therefore no section $s:\operatorname{U}_{\mathrm{res}}(H) \rightarrow \operatorname{U}^\sim_{\mathrm{res}}(S)$ exists.
    As a principal fiber bundle, it does not admit a continuous section.
\end{example}

\section{Scalar Field}

\begin{definition}{Real Klein-Gordon Equation}{real_klein_gordon_equation}
    The real Klein-Gordon equation in $\mathbb{R}^d$ of mass $m$ for unknown $\phi:\mathbb{R}^{d} \rightarrow \mathbb{R}$ is given by
    \[ \qty(\partial^2 + m^2) \phi = 0. \]
\end{definition}

\begin{theorem}{Solution to Real Klein-Gordon Equation}{solution_to_real_klein_gordon_equation}
    For any
    \[ \alpha: \mathbb{R}^{d-1} \rightarrow \mathbb{C}, \]
    the following is a solution to the real Klein gordon solution in $\mathbb{R}^d$ of mass $m$.
    \[ \phi(t,\vb{x}) = \frac{1}{(2\pi)^{d-1}} \int \dd{^{d-1} \vb{p}} \qty(\alpha_{\vb{p}} e^{i(\vb{p}\cdot \vb{x} - \omega_{\vb{p}} t)} + \alpha^*_{\vb{p}} e^{-i(\vb{p}\cdot \vb{x} - \omega_{\vb{p}} t)}), \]
    where
    \[ \omega_{\vb{p}} = \sqrt{\vb{p}^2 + m^2}. \]
\end{theorem}

\section{Spinor Field}

It may seem weird that Dirac spinor is the $\frac{1}{2}\oplus \frac{1}{2}$ reducible representation of $\operatorname{SO}^+(1,3)$ but the Hamiltonian mixes up these two invariant spaces.
However, $\operatorname{SO}^+(1,3)$ is not the symmetry group of the Hamiltonian.
Additional symmetries (e.g. parity) connects these two spaces and therefore the reducible representation of $\operatorname{SU}(2)$ gives rise to a irreducible representation of the whole symmetry group.

\bibliographystyle{plain}
\bibliography{main}

\end{document}
