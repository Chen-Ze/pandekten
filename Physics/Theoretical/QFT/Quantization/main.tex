\documentclass{article}

\usepackage{pandekten}

\title{Quantization}
\author{Ch\=an Taku}

\begin{document}

\maketitle

\section{Scalar Field}

\begin{definition}{Real Klein-Gordon Equation}{real_klein_gordon_equation}
    The real Klein-Gordon equation in $\mathbb{R}^d$ of mass $m$ for unknown $\phi:\mathbb{R}^{d} \rightarrow \mathbb{R}$ is given by
    \[ \qty(\partial^2 + m^2) \phi = 0. \]
\end{definition}

\begin{theorem}{Solution to Real Klein-Gordon Equation}{solution_to_real_klein_gordon_equation}
    For any
    \[ \alpha: \mathbb{R}^{d-1} \rightarrow \mathbb{C}, \]
    the following is a solution to the real Klein gordon solution in $\mathbb{R}^d$ of mass $m$.
    \[ \phi(t,\vb{x}) = \frac{1}{(2\pi)^{d-1}} \int \dd{^{d-1} \vb{p}} \qty(\alpha_{\vb{p}} e^{i(\vb{p}\cdot \vb{x} - \omega_{\vb{p}} t)} + \alpha^*_{\vb{p}} e^{-i(\vb{p}\cdot \vb{x} - \omega_{\vb{p}} t)}), \]
    where
    \[ \omega_{\vb{p}} = \sqrt{\vb{p}^2 + m^2}. \]
\end{theorem}

\section{Spinor Field}

It may seem weird that Dirac spinor is the $\frac{1}{2}\oplus \frac{1}{2}$ reducible representation of $\operatorname{SO}^+(1,3)$ but the Hamiltonian mixes up these two invariant spaces.
However, $\operatorname{SO}^+(1,3)$ is not the symmetry group of the Hamiltonian.
Additional symmetries (e.g. parity) connects these two spaces and therefore the reducible representation of $\operatorname{SU}(2)$ gives rise to a irreducible representation of the whole symmetry group.

% \bibliographystyle{plain}
% \bibliography{main}

\end{document}
