\documentclass{article}

\usepackage{pandekten}

\title{Allgemein}
\author{Ch\=an Taku}

\externaldocument[Int-]{../../../../Mathematics/Calculus/Integrals/main}
\externaldocument[Repr-]{../../../../Mathematics/Algebra/Representation/Allgemein/main}

\begin{document}

\maketitle

\section{Functional Integral}

\subsection{Definitions}

\begin{definition}{Functional Integral}{functional_integral}
    Input:
    \begin{itemize}[nosep]
        \item $M$ is the spacetime.
        \item $F[\varphi,\cdots]$ is a functional, where the nature of $\varphi$ is unknown.
    \end{itemize}
    The functional integral is defined formally by
    \begin{align*}
        \int \mathcal{D}{\varphi}\, F[\varphi,\cdots] = \qty(\prod_{x\in M}\int \dd{\varphi(x)}) F[\varphi,\cdots].
    \end{align*}
\end{definition}

Functional integrals are not well-defined mathematically.
However, in many cases we are entitled to ignore some constant factors and obtain physical results.

We write only one $\int$ when consecutive integrations are carried out, i.e.
\[ \int \mathcal{D}\varphi_1 \cdots \int \mathcal{D}\varphi_n \rightarrow \int \mathcal{D}\varphi_1 \cdots \mathcal{D}\varphi_n. \]

Sometimes we may abuse the notation.
For example, we may write $\mathcal{D}\vb*{m}$ for $\mathcal{D}m_1\mathcal{D}m_2\mathcal{D}m_3$,
and $\mathcal{D}{\overline{\eta}} \mathcal{D}{\eta}$ for $\mathcal{D}\overline{\eta}_1 \mathcal{D}{\eta}_1 \cdots \mathcal{D}\overline{\eta}_r \mathcal{D}{\eta}_r$ if
\[ \eta = \begin{pmatrix} \eta_1 & \cdots & \eta_r \end{pmatrix}^\intercal. \]

\begin{definition}{Generating Functional}{generating_functional}
    Input:
    \begin{itemize}[nosep]
        \item $M = \mathbb{R}^{d}$ is the spacetime.
        \item $S[\varphi_1, \cdots, \varphi_m, \overline{\psi}_1,\psi_1,\cdots,\overline{\psi}_n,\psi_n]$ is the action, where
        \begin{itemize}[nosep]
            \item $\varphi_i: M \rightarrow \mathbb{R}$ for $i = 1,\cdots,m$.
            \item $\psi_i: M \rightarrow$ $\mathbb{C}$ or Grassmann numbers for $i = 1,\cdots, n$.
        \end{itemize}
        \item $J_i: M \rightarrow \mathbb{R}$ for $i = 1,\cdots,m$.
        \item $\eta_i: M \rightarrow$ $\mathbb{C}$ or Grassmann numbers for $i = 1,\cdots, n$, with codomain identical to $\psi_i$ for $S$.
    \end{itemize}
    The generating functional is given by
    \begin{align*}
        &\phantom{{}={}} Z[J_1, \cdots, J_m, \overline{\eta}_1,\eta_1,\cdots,\overline{\eta}_n,\eta_n] \\
        &= \int \mathcal{D}\varphi_1 \cdots \mathcal{D}\varphi_m \mathcal{D} \overline{\psi}_1 \mathcal{D}\psi_1 \cdots \mathcal{D}\overline{\psi}_n \mathcal{D}\psi_n \\
        &\phantom{{}={}\int} \exp\Bigg\{i\bigg[S[\varphi_1, \cdots, \varphi_m, \overline{\psi}_1,\psi_1,\cdots,\overline{\psi}_n,\psi_n] {}+{} \\
        &\phantom{{}={}\int }\int_M \dd{^d x} \qty(\sum_{i=1}^mJ_i(x)\varphi_i(x) + \sum_{i=1}^n \qty(\overline{\eta}_i(x)\psi_i(x) + \overline{\psi}_i(x)\eta(x))) \bigg]\Bigg\}.
    \end{align*}
\end{definition}

The generating functional may be regarded as the Fourier transform of $e^{iS}$.

\begin{definition}{Expectation Value via Generating Functional}{expectation_value_via_generating_functional}
    Input:
    \begin{itemize}[nosep]
        \item $M = \mathbb{R}^{d}$ is the spacetime.
        \item $S[\varphi_1, \cdots, \varphi_m, \overline{\psi}_1,\psi_1,\cdots,\overline{\psi}_n,\psi_n]$ is the action, where
        \begin{itemize}[nosep]
            \item $\varphi_i: M \rightarrow \mathbb{R}$ for $i = 1,\cdots,m$.
            \item $\psi_i: M \rightarrow$ $\mathbb{C}$ or Grassmann numbers for $i = 1,\cdots, n$.
        \end{itemize}
        \item $J_i: M \rightarrow \mathbb{R}$ for $i = 1,\cdots,m$.
        \item $\eta_i: M \rightarrow$ $\mathbb{C}$ or Grassmann numbers for $i = 1,\cdots, n$, with codomain identical to $\psi_i$ for $S$.
        \item $C[\varphi_1, \cdots, \varphi_m, \overline{\psi}_1,\psi_1,\cdots,\overline{\psi}_n,\psi_n]$ is any functional.
    \end{itemize}
    The expectation value of $C$ is given by
    \begin{align*}
        \langle C \rangle &= \int \mathcal{D}\varphi_1 \cdots \mathcal{D}\varphi_m \mathcal{D} \overline{\psi}_1 \mathcal{D}\psi_1 \cdots \mathcal{D}\overline{\psi}_n \mathcal{D}\psi_n \\
        &\phantom{{}={}\int} C[\varphi_1, \cdots, \varphi_m, \overline{\psi}_1,\psi_1,\cdots,\overline{\psi}_n,\psi_n] \\
        &\phantom{{}={}\int} \exp\Bigg\{iS[\varphi_1, \cdots, \varphi_m, \overline{\psi}_1,\psi_1,\cdots,\overline{\psi}_n,\psi_n] \Bigg\}.
    \end{align*}
    The expectation value of $C$ under nonzero source is given by
    \begin{align*}
        &{\phantom{{}={}}} \langle C \rangle_{J_1,\cdots,J_m,\eta_1,\cdots,\eta_n} \\
        &= \int \mathcal{D}\varphi_1 \cdots \mathcal{D}\varphi_m \mathcal{D} \overline{\psi}_1 \mathcal{D}\psi_1 \cdots \mathcal{D}\overline{\psi}_n \mathcal{D}\psi_n \\
        &\phantom{{}={}\int} C[\varphi_1, \cdots, \varphi_m, \overline{\psi}_1,\psi_1,\cdots,\overline{\psi}_n,\psi_n] \\
        &\phantom{{}={}\int} \exp\Bigg\{i\bigg[S[\varphi_1, \cdots, \varphi_m, \overline{\psi}_1,\psi_1,\cdots,\overline{\psi}_n,\psi_n] {}+{} \\
        &\phantom{{}={}\int }\int_M \dd{^d x} \qty(\sum_{i=1}^mJ_i(x)\varphi_i(x) + \sum_{i=1}^n \qty(\overline{\eta}_i(x)\psi_i(x) + \overline{\psi}_i(x)\eta(x))) \bigg]\Bigg\}.
    \end{align*}
\end{definition}

\begin{definition}{Connected Correlation}{connected_correlation}
    Input:
    \begin{itemize}[nosep]
        \item $M = \mathbb{R}^{d}$ is the spacetime.
        \item $S[\varphi_1, \cdots, \varphi_m, \overline{\psi}_1,\psi_1,\cdots,\overline{\psi}_n,\psi_n]$ is the action, where
        \begin{itemize}[nosep]
            \item $\varphi_i: M \rightarrow \mathbb{R}$ for $i = 1,\cdots,m$.
            \item $\psi_i: M \rightarrow$ $\mathbb{C}$ or Grassmann numbers for $i = 1,\cdots, n$.
        \end{itemize}
        \item $C_i[\varphi_1, \cdots, \varphi_m, \overline{\psi}_1,\psi_1,\cdots,\overline{\psi}_n,\psi_n]$ is any functional for $1\le i \le N$.
        \item $\langle \cdots \rangle$ is an expectation value via generating functional for a zero or nonzero source.
    \end{itemize}
    The connected correlation is defined by
    \[ \langle C_1, \cdots, C_N \rangle_{\mathrm{c}} = \kappa(C_1,\cdots,C_N) \]
    where $\kappa$ is the \href{https://en.wikipedia.org/wiki/Cumulant#Joint_cumulants}{joint cumulant}, and the expected value $\operatorname{E}({\cdots})$ thereof is given by the expectation value via generating functional $\langle {\cdots} \rangle$.
\end{definition}

\subsection{Evaluating Functional Integrals}

Some tricks are useful for evaluating functional integrals.

\paragraph*{Gaussian Integrals}
Rigorous result may be found in \cref{Int-thm:real_gaussian_integral} and \cref{Int-thm:complex_gaussian_integral}.
Here we state their unrigorous generalizations.

\begin{theorem}{Real Gaussian Functional Integral}{real_gaussian_functional_integral}
Let $A$ be an invertible symmetric operator acting on $\varphi: M\rightarrow \mathbb{R}$.
Then for any $J: M\rightarrow \mathbb{R}$,
\begin{gather*}
    \int \mathcal{D}\varphi\, \exp[-\frac{1}{2}\int \dd{^d x} \varphi(x) A\varphi(x) + i \int \dd{^d x} J(x)\varphi(x)] \\
    = \frac{(2\pi)^{(\dim A)/2}}{\sqrt{\det A}} \exp{-\frac{1}{2}\int \dd{^d x} J(x) A^{-1}J(x)}.
\end{gather*}
\end{theorem}

\begin{theorem}{Complex Gaussian Functional Integral}{complex_gaussian_functional_integral}
Let $A$ be an invertible symmetric operator acting on $\psi: M\rightarrow \mathbb{C}$ or grassmann numbers.
Then for any $\eta$ from $M$ to the codomain of $\psi$,
\begin{gather*}
    \int \mathcal{D}\overline{\psi}\mathcal{D}\psi\, \exp[-\int \dd{^d x} \overline{\psi}(x) A\psi(x) + i \int \dd{^d x} \qty(\overline{\eta}(x)\varphi(x) + \overline{\varphi}(x)\eta(x))] \\
    = \frac{\pi^{\dim A}}{\det A} \exp{-\int \dd{^d x} \overline{\eta}(x) A^{-1}\eta(x)}.
\end{gather*}
\end{theorem}

In most cases, $A^{-1}$ is just some convolution with the Green's function of $A$, i.e.
\[ A^{-1}\eta(x) = \int \dd{^d x} G(x;y) \eta(y). \]

\paragraph*{Fourier Transform}
Using Fourier transform we write $\eta(x)$ in terms of $\tilde{\eta}(p)$, and therefore obtain a simpler form of $A$ (hence $A^{-1}$ and $\det A$).

\begin{example}{Generating Functional for the Dirac Field}{generating_functional_for_the_dirac_field}
    Let
    \[ S[\overline{\psi},\psi] = \int \dd{^4 x} \overline{\psi}(i\slashed{\partial} - m)\psi. \]
    The generating functional is given by
    \begin{align*}
        Z[\overline{\eta}, \eta] &= \int \mathcal{D}\overline{\psi} \mathcal{D}\psi \\
        &\phantom{{}={}} \exp{i\int \dd{^4 x} \overline{\psi}(i\slashed{\partial} - m)\psi + i \int \dd{^4 x} \qty(\overline{\eta}(x)\varphi(x) + \overline{\varphi}(x)\eta(x))} \\
        &\xlongequal{\text{\cref{thm:complex_gaussian_functional_integral}}} \frac{\pi^\infty}{\det [-i(i\slashed{\partial} - m)]} \exp[-\int \dd{^4 x} \overline{\eta}(x) \frac{i}{i\slashed{\partial} - m} \eta(x)] \\ 
        &\propto \exp\qty[-\int \dd{^4 x} \dd{^4 y} \overline{\eta}(x) S_F(x-y) \eta(y)].
    \end{align*}
\end{example}

\begin{example}{Partition Function of Gaussian Model}{partition_function_of_gaussian_model}
    Let the partition function be given by
    \begin{align*}
        Z &= \int \mathcal{D}\vb*{m} \\
        &\phantom{{}={}}\exp{-\int \dd{^d x} \frac{1}{2}\qty[t{\vb*{m}}^2 + K(\grad \vb*{m})^2 + L(\grad^2 \vb*{m})^2] + \int \dd{^d x} \vb*{h} \cdot \vb*{m}}.
    \end{align*}
    where $\vb*{h}$ is a constant, and $\vb*{h}(x), \vb*{m}(x) \in \mathbb{R}^n$.
    Since the system is microscopically a lattice, we abuse the notation and use summation and integration interchangeably, i.e.
    \begin{align*}
        \tilde{\vb*{m}}(q) &= \int \dd{^d x} e^{iq\cdot x} \vb*{m}(x) \approx \frac{V}{N} \sum_{x} \vb*{m}(x) e^{iq\cdot x}, \\
        \vb*{m}(x) &= \frac{1}{(2\pi)^d}\int \dd{^d q} e^{-iq\cdot x} \tilde{\vb*{m}}(q) \approx \frac{1}{V} \sum_{q} \tilde{\vb*{m}}(q) e^{-iq\cdot x},
    \end{align*}
    where the number of points of $x$ and $q$ are both $N$.
    \par
    Now we apply \cref{thm:real_gaussian_functional_integral} with $A = t - K\grad^2 + L \grad^2 \cdot \grad^2$ and $iJ(x) = \vb*{h}(x)$. We find
    \begin{align*}
        Z &= \frac{(2\pi)^{nN/2}}{\sqrt{\det (t - K\grad^2 + L \grad^2 \cdot \grad^2)}} \\
        &\phantom{{}={}}\exp{\frac{1}{2} \int \dd{^d x} \vb*{h}(x) \frac{1}{t - K\grad^2 + L \grad^2 \cdot \grad^2} \vb*{h}(x)} \\
        &= \prod_{\vb{q}} \qty(\frac{2\pi}{t + Kq^2 + Lq^4})^{n/2} \exp[\frac{Vh^2}{2t}].
    \end{align*}
\end{example}

\section{Wightman Axioms}

\subsection{Minkowski Spacetime}

This section is basically copied from Huzihiro Araki \cite{Araki_Huzihiro2009-07-05}.
Similar formulation may be found from Gerald Folland \cite{Gerald_B_Folland2008-08-26}.
It's better if we could phrase it in terms of superalgebra \cite{Pierre_Deligne1999-06}.

\begin{definition}{Wightman Axioms}{wightman_axioms}
    A Wightman QFT is the following collection of data: \cite{Araki_Huzihiro2009-07-05}
    \begin{itemize}
        \item A Hilbert space $\mathcal{H}$ and a dense subspace $\mathcal{D}$ thereof.
        \item Operator-valued distributions
        \[ \phi_i, \phi_i^\dagger: \mathcal{S} \rightarrow \operatorname{End}(\mathcal{D}), \]
        for $i = 1,\cdots, n$, where $\mathcal{S}$ is the set of $C^\infty$-functions with compact support, and $\phi_i^\dagger(f) = \phi_i(f)^\dagger$.
        \item A \hyperref[Repr-def:unitary_representation]{unitary representation} $\tilde{U}: P^{\uparrow}_+\rightarrow \operatorname{Aut}(\mathcal{H})$, where $P$ is the Poincar\'e group, such that
        \begin{itemize}
            \item $\mathcal{D}$ is a invariant subspace of $U$;
            \item $U$ acts on operators by conjugation, i.e.
            \[ U(a, A) \phi_j(f) U(a, A)^\dagger = \sum_k S(A^{-1})_{jk} \phi_k(f_{(a, A)}), \]
            where $a\in \mathbb{R}^4$ and $\Lambda\in \operatorname{SL}(2,\mathbb{C})$, where
            \[ f_{(a,A)}(x) = f(\Lambda(A)^{-1}(x-a)). \]
        \end{itemize}
        \item If the support of $f$ and $g$ are space-like separated, then for any $\Phi \in \mathcal{D}$,
        \[ [\phi_j(f), \phi_k(g)]_\pm \Phi = 0, \]
        where $\pm$ depends on $j$ and $k$. The same identity holds if we replace $\phi_j$ by $\phi_j^\dagger$ or $\phi_k$ by $\phi_k^\dagger$, or both.
        \item There exists $\Omega \in \mathcal{D}$ such that
        \begin{itemize}
            \item $\Omega$ is $U$-invariant.
            \item The set of all vectors obtained by acting an arbitrary polynomial of $\phi_1,\cdots,\phi_n$ is dense in $\mathcal{H}$.
        \end{itemize}
        \item The spectrum of the translation group $U(a,\mathbbm{1})$ is contained in
        \[ \overline{V}_m = \Set*{p\in \mathbb{R}^4}{p^2 \ge m^2, p^0 > 0}. \]
    \end{itemize}
\end{definition}

\section{Scattering}

\begin{definition}{Mandelstam Variables}{mandelstam_variables}
    For a scatter process $\ket{p,p'}\rightarrow \ket{k,k'}$, the Mandelstam variables are given by
    \begin{align*}
        s &= (p+p')^2 = (k+k')^2; \\
        t &= (k-p)^2 = (k'-p')^2; \\
        u &= (k'-p)^2 = (k-p')^2.
    \end{align*}
\end{definition}

\section{Miscellany}

\subsection{Grassmann Numbers}

Integrals of Grassmann numbers, i.e. Berezin integrals, are related to supermanifolds.
See \href{ref/Losev-Berezin-Integral.pdf}{Losev} for a physicist-friendly introduction.
See also \href{https://math.stackexchange.com/questions/1449312/geometric-meaning-of-berezin-integration}{Geometric meaning of Berezin integration},
as well as this page on \href{https://ncatlab.org/nlab/show/Berezin+integral}{Berezin integral}.

\subsection{Perturbative QFT}

\href{https://en.wikipedia.org/wiki/Haag%27s_theorem}{Haag's theorem} states that no single, universal Hilbert space representation can describe both free and interacting fields.

\bibliographystyle{plain}
\bibliography{main}

\end{document}