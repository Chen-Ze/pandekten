\documentclass{article}

\usepackage{pandekten}

\title{Renormalization}
\author{Ch\=an Taku}

\begin{document}

\maketitle

\begin{center}
    \large \color{gray}
    Violet Divergarden
\end{center}

\section*{Notation}

Let $\operatorname{Span}(\phi_i)$ denote the one dimensional subspace of $\operatorname{End}(\mathcal{D})$ generated by $\phi_i$, i.e. $\Set*{r \phi_i}{r \in\mathbb{R}}$.

\section{Renormalized Perturbation Theory}

It would seem better if I could check out AQFT first before I come back and edit the following definition.
For now, the following phrasing is the best I could obtain from only what I know; the correctness thereof is NOT guaranteed.

\begin{definition}{Renormalization Condition}{renormalization_condition}
    Input:
    \begin{itemize}
        \item $W$ is a Wightman QFT.
        \item $\mathcal{D}$ is the dense subspace in $W$.
        \item $\phi_i,\phi^\dagger_i\in \operatorname{End}(\mathcal{D})$ are the operator-valued distributions in $W$ for $i=1,\cdots,n$.
    \end{itemize}
    A renormalization condition of $W$ contains the following data.
    \begin{itemize}
        \item A mapping
        \[ G: \operatorname{Span}(\phi_1) \times \cdots \times \operatorname{Span}(\phi_n) \rightarrow \operatorname{End}(\mathcal{D}). \]
        \item A real number $\lambda$.
    \end{itemize}
\end{definition}

\begin{definition}{Renormalized Hamiltonian}{renormalized_hamiltonian}
    Input:
    \begin{itemize}
        \item $W$ is a Wightman QFT.
        \item $\mathcal{D}$ is the dense subspace in $W$.
        \item $\Set*{(\lambda_c, G_c)}{c\in C}$ are renormalization conditions of $W$.
    \end{itemize}
    A renormalized Hamiltonian contains the following data.
    \begin{itemize}
        \item $H$ is an Hermitian operator on $\mathcal{D}$ such that the normalized ground state $\Omega\in\mathcal{D}$ of $H$ satisfies
        \[ \bra{\Omega} G_c(Z_1^{-1/2}\phi_1,\cdots,Z_n^{-1/2}\phi_n) \ket{\Omega} = \lambda_c \]
        for all $c\in C$ and some $Z_1,\cdots,Z_n > 0$.
        \item The factors $Z_1,\cdots,Z_n$ are called renormalization factors.
        \item $Z_1^{-1/2}\phi_1,\cdots,Z_n^{-1/2}\phi_n$ are called renormalized fields.
    \end{itemize}
\end{definition}

\begin{definition}{Equivalent Renormalization Conditions}{equivalent_renormalization_conditions}
    Let $\Set*{(G_c,\lambda_c)}{c\in C}$ and $\Set*{(G_d,\lambda_d)}{d\in D}$ be renormalization conditions of the same Wightman QFT $W$.
    They are called equivalent if there exists a single renormalized Hamiltonian $H$ subjected to both of them (possibly with different renormalization factors).
\end{definition}

Equivalence of renormalization conditions is an equivalence relation.

\section{Renormalization Group}

\subsection{Callan-Symanzik Equation}

\begin{definition}{Formal Green's Function}{formal_green_function}
    Input:
    \begin{itemize}
        \item $W$ is a Wightman QFT.
        \item $\mathcal{D}$ is the dense subspace in $W$.
        \item $\phi_i,\phi^\dagger_i\in \operatorname{End}(\mathcal{D})$ are the operator-valued distributions in $W$ for $i=1,\cdots,m$.
        \item $\mathcal{S}$ is the domain of $\phi_i$ for all $i$.
    \end{itemize}
    A formal Green's function of signature $\vb*{d} = (d_1,\cdots,d_m)$ and degree $n = d_1 + \cdots + d_m$ is a homogeneous mapping
    \[ G^{(\vb*{d})}: \operatorname{Span}(\phi_1) \times \cdots \times \operatorname{Span}(\phi_n) \times \mathcal{S}^m \times \mathcal{D} \rightarrow \mathbb{C}. \]
    such that
    \begin{align*}
        &{\phantom{{}={}}} G^{(\vb*{d})}(\lambda_1 \phi_1,\cdots,\lambda_m \phi_m)(x_1,\cdots,x_n)(\ket{S}) \\
        &= \lambda_1^{d_1} \cdots \lambda_m^{d_m} G^{(n)}(\phi_1,\cdots,\phi_m)(x_1,\cdots,x_n)(\ket{S}).
    \end{align*}
\end{definition}

\begin{definition}{Parameterized Renormalization Conditions}{parameterized_renormalization_conditions}
    Let $W$ be a Wightman QFT.
    A parameterized renormalization conditions contains the following data.
    \begin{itemize}
        \item An index set $C$.
        \item For each $c\in C$, a pair of mappings
        \[ (G,\lambda): \mathbb{R}^{N_M}\times \mathbb{R}^{N_\lambda} \rightarrow \text{Renormalization Condition of $W$}. \]
    \end{itemize}
    In other words, for each $\vb*{M} = (M_1,\cdots,M_{N_M})$ and $\vb*{\lambda} = (\lambda_1,\cdots,\lambda_{N_\lambda})$,
    \[ \Set*{(G_c(\vb*{M}, \vb*{\lambda}),\lambda_c(\vb*{M}, \vb*{\lambda}))}{c\in C} \]
    is a set of renormalization conditions of $W$.
\end{definition}

\begin{theorem}{Callan-Symanzik Equation}{callan_symanzik_equation}
    Input:
    \begin{itemize}
        \item $W$ is a Wightman QFT.
        \item $\mathcal{D}$ is the dense subspace in $W$.
        \item $\phi_i,\phi^\dagger_i\in \operatorname{End}(\mathcal{D})$ are the operator-valued distributions in $W$ for $i=1,\cdots,m$.
        \item $G^{(\vb*{d})}$ is a formal Green's function of signature $\vb*{d} = (d_1,\cdots,d_m)$.
        \item $\Set*{(M,\vb*{\lambda}) \mapsto (G_c(M, \vb*{\lambda}),\lambda_c(M, \vb*{\lambda}))}{c\in C}$ is a set of parameterized renormalization conditions of $W$, such that $\vb*{\beta}$ and $\gamma_i$ defined below for each $i$ are independent of $M$.
        \item For each $M_0$ and $\vb*{\lambda}_0$, $\vb*{\lambda}_{M_0,\vb*{\lambda}_0}$ is a mapping such that
        \[ \Set*{(G_c(M, \vb*{\lambda}_{M_0,\vb*{\lambda}_0}(M)),\lambda_c(M, \vb*{\lambda}_{M_0,\vb*{\lambda}_0}(M)))}{c\in C} \]
        is a set of equivalent renormalization conditions, and
        \[ \vb*{\lambda}_{M_0,\vb*{\lambda}_0}(M_0) = \vb*{\lambda}_0. \]
        \begin{itemize}
            \item $\vb*{\lambda}_{M_0,\vb*{\lambda}_0}$ should be compatible for different $M_0$, i.e.
            \[ \lambda_{M'_0,\vb*{\lambda}_{M_0,\vb*{\lambda}_0}(M'_0)}(M) \equiv \vb*{\lambda}_{M_0,\vb*{\lambda}_0}(M). \]
            \item $\ket{\Omega_{M_0,\vb*{\lambda}_0}}$ denotes the ground state of the Hamiltonian shared by the above equivalent renormalization conditions.
        \end{itemize}
        \item For each $M_0$ and $\vb*{\lambda}_0$, and for each $M$, 
        \[ Z_{1,M_0,\vb*{\lambda}_0}(M), \cdots, Z_{m,M_0,\vb*{\lambda}_0}(M) \]
        are the renormalization factors and
        \[ \phi_{1,M_0,\vb*{\lambda}_0}(M), \cdots, \phi_{m,M_0,\vb*{\lambda}_0}(M) \]
        are the renormalized fields in the renormalized Hamiltonian specified by
        \[ \Set*{(G_c(M, \vb*{\lambda}_{M_0,\vb*{\lambda}_0}(M)),\lambda_c(M, \vb*{\lambda}_{M_0,\vb*{\lambda}_0}(M)))}{c\in C}. \]
        \item For each $M_0$ and $\vb*{\lambda}_0$, $\vb*{\beta}$ is defined as follows and independent of $M_0$.
        \[ \vb*{\beta}(M_0,\vb*{\lambda}_0) = \vb*{\beta}(\vb*{\lambda}_0) = M_0 \eval{\pdv{\vb*{\lambda}_{M_0,\vb*{\lambda}_0}(M)}{M}}_{M=M_0}. \]
        \item For each $M_0$ and $\vb*{\lambda}_0$, and for each $1\le i \le m$, $\gamma_i$ is defined as follows and independent of $M_0$.
        \[ \gamma_i(M_0,\vb*{\lambda}_0) = \gamma_i(\vb*{\lambda}_0) = \frac{M}{Z^{1/2}_{i,M_0,\vb*{\lambda}_0}(M_0)} \eval{\pdv{Z^{1/2}_{i,M_0,\vb*{\lambda}_0}(M)}{M}}_{M=M_0}. \]
        \item For each $M$ and $\vb*{\lambda}$,
        \begin{align*}
            &\phantom{{}={}} G^{(n)}(x_1,\cdots,x_n;M,\vb*{\lambda}) \\
            &= G^{(\vb*{d})}(\phi_{1,M,\vb*{\lambda}},\cdots,\phi_{m,M,\vb*{\lambda}})(x_1,\cdots,x_n)(\ket{\Omega_{M,\vb*{\lambda}}}).
        \end{align*}
    \end{itemize}
    Then $G^{(n)}$ satisfies the Callan-Symanzik equation,
    \[ \qty[M \pdv{}{M} + \vb*{\beta}(\vb*{\lambda}) \cdot \pdv{}{\vb*{\lambda}} + \sum_{i=1}^m d_i \gamma_i(\vb*{\lambda})] G^{(n)}(x_1,\cdots,x_n;M,\vb*{\lambda}) = 0. \]
\end{theorem}

Callan-Symanzik equation alone is not sufficient to determine $G$.
However, it sheds light on the form of $G$.
$\vb*{\beta}$ and $\vb*{\gamma}$ may be obtained from $\delta_Z$ and $\delta_\lambda$, and is taken as known in what follows.

\begin{theorem}{Solution to Callan-Symanzik Equation}{solution_to_callan_symanzik_equation}
    The solution to
    \[ \qty[M\pdv{}{M} + \vb*{\beta}(\vb*{\lambda})\cdot \pdv{}{\vb*{\lambda}} + \gamma(\vb*{\lambda})]G(M,\vb*{\lambda}) = 0 \]
    has the form
    \[ G(M,\vb*{\lambda}) = \mathcal{G}(\overline{\vb*{\lambda}}(M,\vb*{\lambda})) \cdot \exp\qty(-\int_{p'=M_0}^{p'=M} \dd{\log(p'/M_0)} \cdot \gamma(\overline{\vb*{\lambda}}(p',\vb*{\lambda}))), \]
    where $\mathcal{G}$ is any smooth function, $M_0$ is some constant and $\overline{\vb*{\lambda}}(M,\vb*{\lambda})$ is determined by
    \begin{align*}
        \dv{}{\log(M/M_0)} \overline{\vb*{\lambda}}(M,\vb*{\lambda}) &= -\vb*{\beta}(\overline{\vb*{\lambda}}(M,\vb*{\lambda})), \\
        \overline{\vb*{\lambda}}(M_0,\vb*{\lambda}) &= \vb*{\lambda}.
    \end{align*}
\end{theorem}

% \bibliographystyle{plain}
% \bibliography{main}

\end{document}
