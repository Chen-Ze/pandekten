\documentclass{article}

\usepackage{pandekten}

\title{Minkowski Conformal}
\author{Ch\=an Taku}

\begin{document}

\maketitle

\section{General Dimensions}

Generators:
\begin{align*}
    P_\mu &= -i\partial_\mu, \\
    D_\mu &= -i x^\mu \partial_\mu, \\
    L_{\mu\nu} &= i(x_\mu \partial_\nu - x_\nu \partial_\mu), \\
    K_\mu &= -i(2x_\mu x^\nu \partial_\nu - x^2 \partial_\mu).
\end{align*}
Commutation relations:
\begin{align*}
    [D,P_\mu] &= iP_\mu, \\
    [D,K_\mu] &= iK_\mu.
\end{align*}
Representation: $\mathcal{O}^a$ be a vector field.
\begin{align*}
    L_{\mu\nu} \mathcal{O}^a(0) &= (S_{\mu\nu}){^a}{_b} \mathcal{O}^b(0), \\
    D \mathcal{O}^a(0) &= -i \Delta \mathcal{O}^a(0), \\
    \tilde{\mathcal{O}}(\lambda x) &= \lambda^{-\Delta} \mathcal{O}(x).
\end{align*}
For actions beyond the origin,
\begin{align*}
    L_{\mu\nu} \mathcal{O}(x) &= \qty[i(x_\nu \partial_\mu - x_\nu \partial_\mu) + S_{\mu\nu}]\mathcal{O}(x), \\
    D \mathcal{O}(x) &= -i(x^\nu \partial_\nu + \Delta) \mathcal{O}(x).
\end{align*}
Correlators:
\begin{align*}
    \langle \mathcal{O}_1(x_1) \mathcal{O}_2(x_2) \rangle &= \frac{C_{12} \delta_{\Delta_1 \Delta_2}}{x^{2\Delta_1}_{12}}, \\
    \langle \mathcal{O}_1(x_1) \mathcal{O}_2(x_2) \mathcal{O}_3(x_3) \rangle &= \frac{f_{123}}{x^{\Delta_1+\Delta_2-\Delta_3}_{12} x^{\Delta_2+\Delta_3-\Delta_1}_{23} x^{\Delta_3+\Delta_1-\Delta_2}_{31}}, \\
    \langle \mathcal{O}_1(x_1) \mathcal{O}_2(x_2) \mathcal{O}_3(x_3) \mathcal{O}_4(x_4) \rangle &= .
\end{align*}
Energy-stress tensor:
\[ T^{\mu\nu} = -\eta^{\mu\nu} \mathcal{L} + \pdv{\mathcal{L}}{(\partial_\mu \Phi)} \partial^\nu \Phi. \]
Or
\[ \delta S = -\frac{1}{2} \int \dd[d]{x} T^{\mu\nu} \delta g_{\mu\nu}. \]

\begin{theorem}{$\tr T = 0$ Implies Conformal Symmetry}{tr_t_0_implies_conformal_symmetry}
    $T{^\mu}{_\mu} = 0$ implies conformal symmetry.
    But not the other way around.
\end{theorem}

Noether's theorem:
\begin{align*}
    &{\phantom{{}={}}}\partial_\mu \langle j{^\mu}{_\alpha} \Phi(x_1) \cdots \Phi(x_n) \rangle \\
    &= -i \sum_{i=1}^n \delta(x-x_i) \langle \Phi(x_1) \cdots G_a \Phi(x_i) \cdots \Phi(x_n) \rangle.
\end{align*}
The charge is defined such that
\[ [Q_a, \Phi] = -i G_a. \]

\section{Two Dimensional}

Conformal Killing field:
\[ \mathcal{L}_X g = \lambda g \]
where
\[ \lambda = \Omega^2 - 1. \]
The infinitesimal conformal transformations are generated by holomorphic functions.
Under the $(z,\overline{z})$ coordinate, the equation for conformal killing field becomes
\begin{align*}
    \partial_z X^z + \partial_{\overline{z}} X^{\overline{z}} &= 2(\partial_z X_{\overline{z}} + \partial_{\overline{z}} X_z), \\
    \partial_{\overline{z}} X^z &= 0, \\
    \partial_{z} X^{\overline{z}} &= 0.
\end{align*}
Therefore we find
\begin{align*}
    \mathrm{CKV} &= \sum_{n\in\mathbb{Z}} c_n \ell_n, \\
    \overline{\mathrm{CKV}} &= \sum_{n\in\mathbb{Z}} \overline{c}_n \overline{\ell}_n, \\
    \ell_n &= -z^{n+1} \partial_z, \\
    \overline{\ell}_n &= -\overline{z}^{n+1} \partial_{\overline{z}}.
\end{align*}
The commutation relations are given by the Witt algebra
\begin{align*}
    [\ell_n,\ell_m] &= (n-m) \ell_{n+m}, \\
    [\overline{\ell}_n, \overline{\ell}_m] &= (n-m) \overline{\ell}_{n+m}, \\
    [\ell_n,\overline{\ell}_m] &= 0.
\end{align*}
The energy-momentum tensor satisfies
\begin{align*}
    T_{z\overline{z}} &= 0, \\
    T_{\overline{z}z} &= 0, \\
    T &= \begin{pmatrix}
        T(z) & \\
        & \overline{T}(\overline{z})
    \end{pmatrix}.
\end{align*}
The Noether current related to the infinitesimal conformal transformation is given by
\[ j_z = \epsilon T,\quad \overline{j}_{\overline{z}} = \overline{\epsilon} \overline{T}. \]
\begin{itemize}
    \item Dilation: $-i x^\mu \partial_\mu$, charge
    \[ x^\sigma T_{\rho\sigma}, \]
    \item Rotation: $2i x_{[\mu}\partial_{\nu]}$,
    \[ x_{\mu} T_{\mu\nu} - x_{\nu} T_{\rho\mu}, \]
    \item Translation: $-i\partial_\mu$,
    \[ T_{\mu\rho}, \]
    \item SCT: $-2x_\mu D + x^2 P_\mu$,
    \[ 2x_\mu x^\sigma T_{\rho\sigma} - x^2 T_{\mu\rho}. \]
\end{itemize}
The charges are given by
\[ Q^a = \int_\Sigma \star J^a. \]
We demand that the dilation operator is diagonalized.
\begin{align*}
    [\tilde{D}, \theta^b(0)] &= i \Delta^b \theta^b, \\
    [\tilde{L}_{\mu\nu}, \theta^b] &= i(S_{\mu\nu}){_c}{^b} \theta^c, \\
    \tilde{D}[\tilde{K}_\mu[\theta^b(0)]] &= i(\Delta^b) - 1.
\end{align*}
A conformal primary is defined by
\[ \tilde{K}_\mu[\theta(0)] = 0. \]
Under coordinate transformation
\[ \theta'(x') = \qty(\det \pdv{x'^\mu}{x^\nu})^{-\Delta^b/d}\theta(x). \]
For primary operators,
\[ \theta(w,\overline{w}) = \theta(z,\overline{z}) \qty(\pdv{w}{z})^{-h}\qty(\pdv{\overline{w}}{\overline{z}})^{-\overline{h}}, \]
where
\[ h = \frac{1}{2} (\Delta + S),\quad \overline{h} = \frac{1}{2}(\Delta - S). \]

\section{Radial Quantization}

The ordered-product is defined by
\[ R\qty{\theta^a(z) \theta^b(w)} = \begin{cases}
    \theta^a(z) \theta^b(w), & \abs{z} > \abs{w}, \\
    \pm \theta^b(w) \theta^a(z), & \abs{w} > \abs{z}.
\end{cases} \]

\section{Unitarity Bounds}

The unitarity bound gives
\[ \Delta^{(\mathcal{O})} \ge 0. \]
Moreover,
\[ \Delta \ge \frac{d - 2}{2}. \]
For spinning operators $\Delta \ge d - 2 + S$.

\section{Central Charge}

The quantized commutation relation we find
\[ [L_m,L_n] = (m-n)L_{m+n} + \frac{c}{12}(m^3 - m)\delta_{m,-n}. \]
The global conformal group $\qty{L_{-1},L_0,L_1}$ is not aware of the central charge.
In terms of the energy-momentum tensor $L_n$ are given by
\begin{align*}
    L_n &= \frac{1}{2\pi i} \oint \frac{\dd{z}}{z} z^{n+2} T(z), \\
    \overline{L}_n &= \frac{1}{2\pi i} \oint \frac{\dd{\overline{z}}}{\overline{z}} \overline{z}^{n+2} T(\overline{z}).
\end{align*}

Conformal Ward identities:
\begin{align*}
    \delta_{\epsilon,\overline{\epsilon}} &= -\frac{1}{2\pi i} \oint \dd{z} \epsilon(z)\langle T(z)X\rangle + \cdots, \\
    T(z) T(w) &= \frac{1/2}{(z-w)^4} + \frac{2T(w)}{(z-w)^2} + \frac{\partial T(w)}{z-w}, \\
    T(z) \mathcal{O}(w,\overline{w}) &= \cdots + h \frac{\mathcal{O}(w,\overline{w})}{(z-w)^2} + \frac{\partial \mathcal{O}(w,\overline{w})}{z-w}.
\end{align*}

If the $\cdots$ part in $T(z)\mathcal{O}(w,\overline{w})$ vanishes, then
\[ \mathcal{O}(w,\overline{w}) = \qty(\pdv{w}{z})^{-h}\qty(\pdv{\overline{w}}{\overline{z}})^{-\overline{h}} \mathcal{O}(z,\overline{z}) \]
and therefore $\mathcal{O}$ is primary.
\par
Under coordinate transformations,
\begin{align*}
    \tilde{T}(w) &= \qty(\pdv{w}{z})^{-2}\qty(T(z) - \frac{c}{12}\qty{w,z}), \\
    \qty{w,z} &= \frac{w'''(z)}{w'(z)} - \frac{3}{2}\qty(\frac{w''(z)}{w'(z)})^2.
\end{align*}

% \bibliographystyle{plain}
% \bibliography{main}

\end{document}
