\documentclass{article}

\usepackage{pandekten}

\geometry{b5paper}

\usepackage{marginnote}

\reversemarginpar

\title{Class AIII}
\author{Ze Chen}

\begin{document}

\newcommand{\mynote}[1]{\marginnote{\color{gray}#1}}

\newcommand{\remove}[1]{}

\maketitle

We will make use of the following two identities, where $X$ and $Y$ are any matrices of c-numbers of size $\mathbb{C}^{N\times N}$:
\begin{align*}
    \int_{\substack{V\in \mathbb{C}^{N\times N} \\ V=V^\dagger}} \dd{V} \exp(-\tr(V^2) + \alpha\tr(VX)) &= \mathrm{const.} \times \exp[\tr(\frac{\alpha^2}{4}X^2)], \\
    \int_{V\in \mathbb{C}^{N\times N}} \dd{V} \exp(-\tr(VV^\dagger) + \alpha\tr(VX) + \beta \tr(V^\dagger Y)) &= \mathrm{const.} \times \exp[\alpha\beta \tr(XY)].
\end{align*}
In the first identity,
\[ \dd{V} = \prod_{i=1}^N \dd{V_{ii}} \prod_{N \ge i>j\ge 1} \dd{\Re V_{ij}} \dd{\Im V_{ij}}, \]
and in the second identity,
\[ \dd{V} = \prod_{i,j=1}^n \dd{\Re V_{ij}} \dd{\Im V_{ij}}. \]

\par
The Non-Hermitian class AIII has the following Hamiltonian:
\begin{equation}
    H = \begin{pmatrix}
        iA & B \\
        B^\dagger & iD
    \end{pmatrix},
\end{equation}
where $A,B,D \in \mathbb{C}^{N\times N}$, and $A$ and $D$ are Hermitian. \mynote{$N$}
\par
In the following we denote by $\Gamma$\mynote{$\Gamma$} the symmetry opreation.
A specific example may be $\Gamma = \sigma_z\otimes \mathbbm{1}_{N\times N}$.

\paragraph*{Class AIII}
The symmetry $\Gamma^{-1}H^\dagger \Gamma = -H$ together with $\Gamma^2 = \mathbbm{1}_{2N\times 2N}$ requires that $i H \Gamma$ be Hermitian.
From the symmetry we also find
\begin{align}
    \det(gH-z) &= \det(gH^\dagger - (-z)), \\
    \det(gH-z) &= \det(gH - (-\overline{z}))^*.
\end{align}
In particular, from the second equation we know $\det(gH - z)$ is real if $z$ is purely imaginary.

\section{First Approach: Brute Force}

\textit{Tip: We should use the symmetry $\Gamma H \Gamma = -H^\dagger$ and merge the second diagonal block into the first one, if $z = iy$ for some $y\in \mathbb{R}$.}
\par
We are going to evaluate
\begin{align}
    Z &= \int \dd{A} \dd{B} \dd{D} \prod_{a=1}^r \qty(\dd\overline{\varphi}_a \dd\overline{\chi}_a \dd\overline{\psi}_a \dd\overline{\xi}_a \dd{\varphi}_a \dd{\chi}_a \dd{\psi}_a \dd{\xi}_a) \mynote{$r$} \\
    &{\phantom{{}={}}} \times \exp\qty[-\tr({A^\dagger A}) - \tr({D^\dagger D}) - 2\tr({B^\dagger B})] \\
    &{\phantom{{}={}}} \mynote{$g$} \hspace{-2em} \times \exp[\sum_{a=1}^r \begin{pmatrix}
        \overline{\varphi}_a & \overline{\chi}_a & \overline{\psi}_a & \overline{\xi}_a
    \end{pmatrix} \begin{pmatrix}
        igA - z & gB & & \\
        gB^\dagger & igD - z & & \\
        & & -igA - \overline{z} & gB \\
        & & gB^\dagger & -igD - \overline{z}
    \end{pmatrix} \begin{pmatrix}
        \varphi_a \\ \chi_a \\ \psi_a \\ \xi_a
    \end{pmatrix}] \\
    &= \int \dd{A} \dd{B} \dd{D} \prod_{a=1}^r \qty(\dd\overline{\varphi}_a \dd\overline{\chi}_a \dd\overline{\psi}_a \dd\overline{\xi}_a \dd{\varphi}_a \dd{\chi}_a \dd{\psi}_a \dd{\xi}_a) \\
    &{\phantom{{}={}}} \times \exp\qty[-\tr({A^\dagger A}) - \tr({D^\dagger D}) - 2\tr({B^\dagger B})] \\
    &{\phantom{{}={}}} \hspace{-3em} \times \exp[\sum_{a=1}^r \begin{pmatrix}
        \overline{\varphi}_a  & \overline{\psi}_a & \overline{\chi}_a & \overline{\xi}_a
    \end{pmatrix} \begin{pmatrix}
        igA - z & & gB & \\
        & -igA - \overline{z} & & gB \\
        gB^\dagger & & -igD - z & \\
        & gB^\dagger & & -igD - \overline{z}
    \end{pmatrix} \begin{pmatrix}
        \varphi_a \\ \psi_a \\ \chi_a \\ \xi_a
    \end{pmatrix}].
\end{align}

The integral in $B$ yields the following (we write $Q$ as $2\times 2$ blocks of $r\times r$):
\begin{align}
    & \phantom{{}={}} \int \dd{B} \exp[-2\tr(B^\dagger B)] \exp[g \cdots] \\
    &= \int_{Q\in \mathbb{C}^{2r\times 2r}} \dd{Q} \\
    &{\phantom{{}={}}} \hspace{-2em} \exp[\frac{g}{\sqrt{2}}\sum_{a,b=1}^r \begin{pmatrix}
        \overline{\varphi}_a & \overline{\psi}_a & \overline{\chi}_a & \overline{\xi}_a
    \end{pmatrix} \begin{pmatrix}
        (Q_{11})_{ab} & (Q_{12})_{ab} & & \\
        (Q_{21})_{ab} & (Q_{22})_{ab} & & \\
        & & -(Q^\dagger_{11})_{ab} & -(Q^\dagger_{12})_{ab} \\
        & & -(Q^\dagger_{21})_{ab} & -(Q^\dagger_{22})_{ab}
    \end{pmatrix} \begin{pmatrix}
        \varphi_b \\ \psi_b \\ \chi_b  \\ \xi_b
    \end{pmatrix}].
\end{align}

The integral in $A$ and $D$ yields the following:
\begin{align}
    & \phantom{{}={}} \int \dd{A} \dd{D} \exp[-\tr(A^\dagger A) - \tr(D^\dagger D)] \exp[g \cdots] \\
    &= \int_{\substack{R,S\in \mathbb{C}^{2r\times 2r} \\ R,S \text{ Hermitian}}} \dd{R} \dd{S} \\
    &{\phantom{{}={}}} \hspace{-2em} \exp[i g\sum_{a,b=1}^r \begin{pmatrix}
        \overline{\varphi}_a & \overline{\psi}_a & \overline{\chi}_a & \overline{\xi}_a
    \end{pmatrix} \begin{pmatrix}
        (R_{11})_{ab} & (R_{12})_{ab} & & \\
        -(R_{21})_{ab} & -(R_{22})_{ab} & & \\
        & & (S_{11})_{ab} & (S_{12})_{ab}  \\
        & & -(S_{21})_{ab} & -(S_{22})_{ab}
    \end{pmatrix} \begin{pmatrix}
        \varphi_b \\ \psi_b \\ \chi_b  \\ \xi_b
    \end{pmatrix}].
\end{align}

Combining these we find
\begin{align}
    Z &= \det(\frac{g}{\sqrt{2}}Q_{2r\times 2r} + ig(\sigma_z \otimes \mathbbm{1}_{r\times r}) R_{2r\times 2r} - \begin{pmatrix}
        z \mathbbm{1}_{r\times r} & \\
        & \overline{z}  \mathbbm{1}_{r\times r}
    \end{pmatrix}) \\
    &{\phantom{{}={}}} \times \det(-\frac{g}{\sqrt{2}}Q^\dagger_{2r\times 2r} + ig(\sigma_z \otimes \mathbbm{1}_{r\times r}) S_{2r\times 2r} - \begin{pmatrix}
        z  \mathbbm{1}_{r\times r} & \\
        & \overline{z}  \mathbbm{1}_{r\times r}
    \end{pmatrix}).
\end{align}

\subsection{Using Symmetry}

For $z = x+iy$,
\begin{align}
    \det \begin{pmatrix}
        & gH-z \\
        gH^\dagger - \overline{z}
    \end{pmatrix} &= \pm \det \begin{pmatrix}
        gH - iy - x & \\ & gH - iy + x
    \end{pmatrix}. \label{eq:symm_appr}
\end{align}

\section{Second Approach: Hermitian Class A}

In the following we set $z = iy$, where $y\in \mathbb{R}$. \mynote{$z,y$}
Then after a unitary transformation,
\begin{align}
    Z &= \int_{H\in \text{AIII}} \dd{H} e^{-\tr({H^\dagger H})} \det \qty[\mathbbm{1}_{r\times r} \otimes \begin{pmatrix}
        & gH - z \\
        gH^\dagger - \overline{z} &
    \end{pmatrix}] \mynote{$r,g$} \\
    &= \int_{H\in \text{AIII}} \dd{H} e^{-\tr({H^\dagger H})} \det \qty[\mathbbm{1}_{r\times r} \otimes \begin{pmatrix}
        & & igA - z & gB \\
        & & gB^\dagger & igD - z \\
        -igA - \overline{z} & gB & & \\
        gB^\dagger & -igD - \overline{z} & &
    \end{pmatrix}] \\
    &= \int_{H\in \text{AIII}} \dd{H} e^{-\tr({H^\dagger H})} \det \qty[\mathbbm{1}_{r\times r} \begin{pmatrix}
        -gA + y & -igB & & \\
        igB^\dagger & gD-y & & \\
        & & gA-y & igB \\
        & & -igB^\dagger & -gD+y
    \end{pmatrix}] \\
    &= \int_{H\in \text{AIII}} \dd{H} e^{-\tr({H^\dagger H})} \det \qty[\mathbbm{1}_{r\times r} \otimes \begin{pmatrix}
        1 & \\ & -1
    \end{pmatrix} \otimes \begin{pmatrix}
        -gA + y & -igB \\
        igB^\dagger & gD-y
    \end{pmatrix}].
\end{align}

The $\begin{pmatrix}
    1 & \\ & -1
\end{pmatrix}$ factor affects the determinant up to a factor of $\pm 1$ (or, equivalent to replacing $\overline{\psi} \rightarrow -\overline{\psi}$ and $\overline{\xi} \rightarrow -\overline{\xi}$).
Denote
\begin{equation}
    \tilde{H} = \begin{pmatrix}
        -A & -iB \\ iB^\dagger & D
    \end{pmatrix} = iH\Gamma. \mynote{$\tilde{H}$}
\end{equation}
Then $\tilde{H}$ is Hermitian, and
\begin{align}
    Z &= \pm \int_{H\in \text{Hermitian}} \dd{H} e^{-\tr({\tilde{H}^\dagger \tilde{H}})} \det[\mathbbm{1}_{2r\times 2r} \otimes \qty(g\tilde{H} + y \sigma_z \otimes \mathbbm{1}_{N\times N})].
\end{align}
Introducing the auxiliary field $Q\in \mathbb{C}^{2r\times 2r}$ where $Q$ is Hermitian, we find \mynote{$Q$}
\begin{align}
    Z &= \pm \int_{\substack{Q\in \mathbb{C}^{2r\times 2r}\\ Q \text{ Hermitian}}} \dd{Q} e^{-\tr({{Q}^\dagger {Q}})} \notag \\
    &{\phantom{{}={}}} \exp\qty[\tr \log\qty(igQ\otimes \mathbbm{1}_{2N\times 2N} + y \mathbbm{1}_{2r\times 2r} \otimes \sigma_z \otimes \mathbbm{1}_{N\times N}) ]. \label{eq:one_pt_z} 
\end{align}
That is, $Q$ acts on the $2r$ replica indices, while the term on $y$ acts on the $2N$ flavor indices.

\section{Two-Point Function}

With the same analysis, for $z_1 = iy_1$ and $z_2 = iy_2$ where $y_1,y_2\in \mathbb{R}$, \mynote{$z_1,z_2$}
the two point function is given by \mynote{$y_1,y_2$}
\begin{align}
    Z &= \pm \int_{\substack{P\in \mathbb{C}^{4r\times 4r}\\ P \text{ Hermitian}}} \dd{P} e^{-\tr({{P}^\dagger {P}})} \mynote{$P$} \\
    &\phantom{{}={}} \quad \exp\qty[\tr \log\qty(igP\otimes \mathbbm{1}_{2N\times 2N} + \begin{pmatrix}
        y_1 & \\ & y_2
    \end{pmatrix} \otimes \mathbbm{1}_{2r\times 2r} \otimes \sigma_z \otimes \mathbbm{1}_{N\times N}) ].
\end{align}

\section{Breaking Symmetry}

After introducing $i\epsilon$ on the diagonal the $2r$ replicas are no longer symmetric.
\par
The first way to introduce $\epsilon$ doesn't break the block chiral symmetry.
\begin{align}
    Z_{0,\epsilon} &= \int_{H\in \text{AIII}} \dd{H} e^{-\tr({H^\dagger H})} \det \qty[\mathbbm{1}_{r\times r} \otimes \begin{pmatrix}
        i\epsilon \Gamma & gH - z \\
        gH^\dagger - \overline{z} & i\epsilon \Gamma
    \end{pmatrix}] \mynote{$Z_{0,\epsilon}$} \\
    \remove{%
    &= \int_{\text{AIII}} \dd{H} e^{-\tr({H^\dagger H})} \det \qty[\mathbbm{1}_{r\times r} \otimes \begin{pmatrix}
        i\epsilon & & igA - z & gB \\
        & -i\epsilon & gB^\dagger & igD - z \\
        -igA - \overline{z} & gB & i\epsilon & \\
        gB^\dagger & -igD - \overline{z} & & -i\epsilon
    \end{pmatrix}] \\
    &\hspace{-5em}= \int_{\text{AIII}} \dd{H} e^{-\tr({H^\dagger H})} \det \begin{pmatrix}
        -gA + (y+i\epsilon) & -igB & & \\
        igB^\dagger & gD-(y+i\epsilon) & & \\
        & & gA-(y-i\epsilon) & igB \\
        & & -igB^\dagger & -gD+(y-i\epsilon)
    \end{pmatrix}^r \\
    }
    &= \pm \int_{\text{AIII}} \dd{H} e^{-\tr({H^\dagger H})} \det \begin{pmatrix}
        g\tilde{H} + (y+i\epsilon)\Gamma & \\ & g\tilde{H} + (y-i\epsilon)\Gamma
    \end{pmatrix}^r.
\end{align}

Another way is to break the block chiral symmetry.
\begin{align}
    Z_{3,\epsilon} &= \int_{H\in \text{AIII}} \dd{H} e^{-\tr({H^\dagger H})} \det \qty[\mathbbm{1}_{r\times r} \otimes \begin{pmatrix}
        \epsilon \Gamma & gH - z \\
        gH^\dagger - \overline{z} & -\epsilon \Gamma
    \end{pmatrix}] \mynote{$Z_{3,\epsilon}$} \\
    &= \pm \int_{\text{AIII}} \dd{H} e^{-\tr({H^\dagger H})} \det \begin{pmatrix}
        gH - i(y + i\epsilon) & \\ & gH - i(y - i\epsilon)
    \end{pmatrix}^r \label{eq:z_3_in_h_y} \\
    &= \pm \int_{\text{AIII}} \dd{H} e^{-\tr({H^\dagger H})} \det \begin{pmatrix}
        g\tilde{H} + (y+i\epsilon)\Gamma & \\ & g\tilde{H} + (y-i\epsilon)\Gamma
    \end{pmatrix}^r.
\end{align}
In \eqref{eq:z_3_in_h_y} the determinant equals $\det(gH - i(y + i\epsilon))\det(gH - i(y + i\epsilon))^*$.
Note that the determinant in \eqref{eq:z_3_in_h_y} is just \eqref{eq:symm_appr} if we set $x=-\epsilon$.
\par
$Z_{0,\epsilon}$ and $Z_{3,\epsilon}$ are the same and we denote $Z_\epsilon = Z_{0,\epsilon} = Z_{3,\epsilon}$.\mynote{$Z_\epsilon$}
Now we introduce the $Q$ field in the same way as before.
\begin{align}
    Z_\epsilon &= \pm \int_{\substack{Q\in \mathbb{C}^{2r\times 2r}\\ Q \text{ Hermitian}}} \dd{Q} e^{-\tr({{Q}^\dagger {Q}})} \\
    &{\phantom{{}={}}} \exp\qty[\tr \log\qty(igQ\otimes \mathbbm{1}_{2N\times 2N} + y \mathbbm{1}_{2r\times 2r} \otimes \Gamma + i\epsilon (\mathbbm{1}_{r\times r}\otimes \sigma_z) \otimes \Gamma) ].  \label{eq:one_pt_z_broken} 
\end{align}
\par
Or shall we expect $i\epsilon(\mathbbm{1}_{r\times r} \otimes \sigma_z)\otimes \mathbbm{1}_{2N\times 2N}$ for the last term?
To achieve this we may replace $\epsilon \Gamma \rightarrow \Gamma$, and we will find $y\Gamma \pm i\epsilon$ in those expressions.
That is,
\begin{align}
    Z'_{0,\epsilon} &= \int_{H\in \text{AIII}} \dd{H} e^{-\tr({H^\dagger H})} \det \qty[\mathbbm{1}_{r\times r} \otimes \begin{pmatrix}
        i\epsilon  & gH - z \\
        gH^\dagger - \overline{z} & i\epsilon 
    \end{pmatrix}] \mynote{$Z'_{0,\epsilon}$} \\
    &= \pm \int_{\text{AIII}} \dd{H} e^{-\tr({H^\dagger H})} \det \begin{pmatrix}
        g\tilde{H} + y\Gamma + i\epsilon & \\ & g\tilde{H} + y\Gamma - i\epsilon
    \end{pmatrix}^r \\
    &= \pm \int_{\substack{Q\in \mathbb{C}^{2r\times 2r}\\ Q \text{ Hermitian}}} \dd{Q} e^{-\tr({{Q}^\dagger {Q}})} \\
    &{\phantom{{}={}}} \exp\qty[\tr \log\qty(igQ\otimes \mathbbm{1}_{2N\times 2N} + y \mathbbm{1}_{2r\times 2r} \otimes \Gamma + i\epsilon (\mathbbm{1}_{r\times r}\otimes \sigma_z) \otimes \mathbbm{1}_{2N\times 2N}) ].
\end{align}

\section{Saddle Point}

\subsection{Saddle Point: One-Point Function}

\paragraph*{Saddle Point for $Z$}
Given that $Q\in \mathbb{C}^{2r\times 2r}$ is Hermitian, we could diagonalize it as
\begin{equation}
    Q = T \Lambda T^{-1},\quad \Lambda = \begin{pmatrix}
        \lambda_1 & & \\
        & \ddots & \\
        & & \lambda_{2r}
    \end{pmatrix},\quad \lambda_1,\cdots,\lambda_{2r}\in \mathbb{R},\quad T\in\operatorname{U}(2r). \mynote{$T,\Lambda,\lambda_i$}
\end{equation}
Then \eqref{eq:one_pt_z} becomes
\begin{align}
    Z &= \pm \int_{\substack{Q\in \mathbb{C}^{2r\times 2r}\\ Q \text{ Hermitian}}} \dd{Q} \exp[-\sum_{i=1}^{2r} \qty(\lambda_i^2 - N \log(g^2 \lambda_i^2 + y^2))] \label{eq:one_pt_z_diag}  \\
    &= \pm \int_{\substack{Q\in \mathbb{C}^{2r\times 2r}\\ Q \text{ Hermitian}}} \dd{Q} \exp[-\sum_{i=1}^{2r} f(\lambda_i; N,g,y)] = \pm \int_{\substack{Q\in \mathbb{C}^{2r\times 2r}\\ Q \text{ Hermitian}}} \dd{Q} \exp[-S[Q]]. \notag
\end{align}
where \mynote{$f,S$} $f(\lambda;N,g,y) = \lambda^2 - N\log(g^2 \lambda^2 + y^2)$ and $S[Q] = \sum_{i=1}^{2r} f(\lambda_i;N,g,y)$, where $\lambda_i$ are the eigenvalues of $Q$.
The saddle point depends on $N,g,y$.
\begin{itemize}
    \item If $N g^2 \le y^2$, then the saddle points are just $\lambda = 0$. \mynote{Case 1}
    \begin{center}
        \begin{tikzpicture}[domain=-0.8:0.8, yscale=2]
            \draw[->] (-2,0) -- (2,0) node[right] {$\lambda$};
            \draw[->] (0,-0.4) -- (0,1) node[above] {$f(\lambda)$};
            \draw   plot (\x,{2*\x*\x - ln(\x*\x+1)});
          \end{tikzpicture}
    \end{center}
    \item If $N g^2 > y^2$, then the saddle points are given by \mynote{Case 2}
    \begin{equation}
        \label{eq:saddle_lambda}
        \lambda = \pm \sqrt{N - \qty(\frac{y}{g})^2}.
    \end{equation}
    \begin{center}
        \begin{tikzpicture}[domain=-2:2, yscale=2]
            \draw[->] (-2,0) -- (2,0) node[right] {$\lambda$};
            \draw[->] (0,-0.4) -- (0,1) node[above] {$f(\lambda)$};
            \draw   plot (\x,{\x*\x - 2*ln(\x*\x+1)});
          \end{tikzpicture}
    \end{center}
\end{itemize}

\remove{
\par
In case 2, we define $M^{(2r)}$  \mynote{$M^{(2r)}$} to be the subset of in $\mathbb{C}^{2r\times 2r} \cap \qty{\text{Hermitian}}$ such that $S[Q]$ attains its minimal value for $Q\in M$, then $M$ admits a disjoint decomposition into connected components, i.e.
\begin{align*}
    M^{(2r)} &= \argmin_{\substack{Q \in \mathbb{C}^{2r\times 2r} \\ 
Q\text{ Hermitian}}} S[Q] = \bigcup_{i=0}^{2r} M^{(2r)}_{i}
\end{align*}
where for each $0\le i\le 2r$, $M^{(2r)}_{i}$ is the subset of $\mathbb{C}^{2r\times 2r} \cap \qty{\text{Hermitian}}$ defined by
\begin{align*}
    M^{(2r)}_i &= \Set*{Q\in \mathbb{C}^{2r\times 2r} \cap \qty{\text{Hermitian}}}{\begin{array}{c}
        \qty{\lambda_j}_{1\le j\le 2r} \text{ of $Q$ all satisfies \eqref{eq:saddle_lambda},} \\
        i \text{ of them positive,} \\
        2r-i \text{ of them negative}
    \end{array}} \mynote{$M^{(2r)}_i$} \\
    &\cong \operatorname{U}(2r)/(\operatorname{U}(i)\times \operatorname{U}(2r-i)).
\end{align*}
% For each $i$, $M_i$ is an orbit under the adjoint action of $\operatorname{U}(r)$ on $\mathbb{C}^{2r\times 2r} \cap \qty{\text{Hermitian}}$.
% Since the stabilizer group of each elements of $M_i$ is $\operatorname{U}(i)\times \operatorname{U}(2r-i)$, 
The dimension of $M^{(2r)}_i$ is given by $\dim M^{(2r)}_i = 2i(2r-i)$,
%\[ \dim M_i = \dim \operatorname{U}(2r) - \dim (\operatorname{U}(i)\times \operatorname{U}(2r-i)) = 2i(2r-i) \]
which attains maximum at $i=r$.
Therefore, we anticipate the main contribution to $Z$ to be given by those $Q$ from $M^{(2r)}_{r}$.
}

\paragraph*{Saddle Point for $Z_\epsilon$}
\remove{%The saddle point for $Z_\epsilon$ are given by (approximately)
\[ \Set*{Q\in \mathbb{C}^{2r\times 2r} \cap \qty{\text{Hermitian}}}{Q = T \qty[\mathbbm{1}_{r\times r} \otimes \begin{pmatrix}
    \lambda_+ & \\ & \lambda_-
\end{pmatrix}] T^{-1}, T\in \operatorname{U}(2r), \lambda_\pm ^2 = N^2 - \qty(\frac{y\pm i\epsilon}{g})^2}. \]
}

To get the saddle point, we may set $y=0$. rescale such that $g=1$, and find (with $\Sigma = \mathbbm{1}_{r\times r} \otimes \sigma_z$)
\begin{align}
    Z_\epsilon &= \int \dd{Q} e^{-\tr(Q^2)} \det(Q\otimes \mathbbm{1}_{2N\times 2N} + \epsilon\Sigma\otimes \Gamma) \\
    &= \int \dd{Q} e^{-\tr(Q^2)} \det(Q\otimes \mathbbm{1}_{2N\times 2N})\det(1 + \epsilon (Q^{-1}\Sigma) \otimes \Gamma) \\
    &= \int \dd{Q} e^{-\tr(Q^2)} \det(Q\otimes \mathbbm{1}_{2N\times 2N}) \exp[\epsilon \tr((Q^{-1}\Sigma) \otimes \Gamma) - \frac{\epsilon^2}{2} \tr((Q^{-1}\Sigma)(Q^{-1}\Sigma))\cdot 2N + \bigO(\epsilon^3)].
\end{align}
If we replace $\Gamma$ by $\mathbbm{1}_{2N\times 2N}$ then the first term is the lowest order of correction.
Otherwise the $\epsilon^2$ is the lowest order.

\subsection{Saddle Point: Two-Point Function}

Define $y_+ = (y_1+y_2)/2$ and $y_- = (y_1 - y_2)/2$, \mynote{$y_+,y_-$} and diagonalize the Hermitian matirx $P\in \mathbb{C}^{4r\times 4r}$ as
\[ P = U \Sigma U^{-1},\quad \Sigma = \begin{pmatrix}
    \sigma_1 & & \\
    & \ddots & \\
    & & \sigma_{4r}
\end{pmatrix},\quad \sigma_1,\cdots,\sigma_{4r}\in \mathbb{R},\quad U\in\operatorname{U}(2r). \mynote{$U,\Sigma,\sigma_i$} \]
then the two-point function is given by
\begin{align*}
    Z &= \pm \int_{\substack{P\in \mathbb{C}^{4r\times 4r}\\ P \text{ Hermitian}}} \dd{P} e^{-\tr({{P}^\dagger {P}})} \\
    &\phantom{{}={}} \quad \qty(\det[i\Sigma\otimes \mathbbm{1}_{2\times 2} + y_+ \mathbbm{1}_{4r\times 4r} \otimes \sigma_z + y_- U^\dagger(\sigma_z \otimes \mathbbm{1}_{2r\times 2r})U \otimes \sigma_z])^N.
\end{align*}
Following the analysis of the previous section, we approximate the saddle point of $P$ by assuming $y_- \approx 0$.
Then $\Sigma = \sigma \cdot \sigma_z \otimes \mathbbm{1}_{2r\times 2r}$ for some $\sigma\in\mathbb{R}$.
%It looks like the manifold for this case is also $M^{(4r)}_{2r}$?

%\paragraph*{Note}
%In the above we assumed small $y$, which corresponds to setting $H$ small in higher dimensional cases.
%However, in higher dimensions we may do
%\begin{align*}
%    \int DQ \det(H+Q) &= \int DT \det(H+T\Lambda T^{-1}) = \int DT \det(T^{-1} H T + \Lambda) \\
%    &= \int DT \det((H+\Lambda) + \vb*{\sigma}\cdot T^{-1}\grad T) \\
%    &= \int DT \det(H+\Lambda) \det(1 + G \vb*{\sigma}\cdot T^{-1}\grad T)
%\end{align*}
%where $G = 1/(H+\Lambda)$ and then expand the second determinant perturbatively.

% \bibliographystyle{plain}
% \bibliography{main}

\end{document}
