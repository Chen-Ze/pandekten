\documentclass{article}

\usepackage{pandekten}

\title{Three Dimensional Effective Action (Question)}
\author{Ze Chen}

\begin{document}

\maketitle

Let $r$ denote the number of replica.
We are now going to evaluate the following determinant
\[ e^{S_{\mathrm{eff}}[U]} = \det\begin{pmatrix}
    (\sigma^i p_i + im - z) \otimes \mathbbm{1}_{r\times r} & g\mathbbm{1}_{2\times 2} \otimes U \\
    -g\mathbbm{1}_{2\times 2} \otimes U^\dagger & (\sigma^i p_i - im - \overline{z}) \otimes \mathbbm{1}_{r\times r}
\end{pmatrix} \]
where $U: \mathbb{R}^3 \rightarrow \operatorname{U}(r)$.
Since $U$ is unitary,
\begin{align*}
    &e^{S_{\mathrm{eff}}[U]} \\ %= \det\Bigg[\begin{pmatrix}
        %\mathbbm{1}_{2\times 2} \otimes U^\dagger & \\
        %& \mathbbm{1}_{2\times 2} \otimes \mathbbm{1}_{r\times r}
    %\end{pmatrix} \\
    %&\quad \begin{pmatrix}
    %    (\sigma^i p_i + im - z) \otimes \mathbbm{1}_{r\times r} & \mathbbm{1}_{2\times 2} \otimes U \\
    %    -\mathbbm{1}_{2\times 2} \otimes U^\dagger & (\sigma^i p_i - im - \overline{z}) \otimes \mathbbm{1}_{r\times r}
    %\end{pmatrix} \begin{pmatrix}
    %    \mathbbm{1}_{2\times 2} \otimes U & \\
    %    & \mathbbm{1}_{2\times 2} \otimes \mathbbm{1}_{r\times r}
    %\end{pmatrix}\Bigg] \\
    &= \det[\begin{pmatrix}
        (\sigma^i p_i + im - z) \otimes \mathbbm{1}_{r\times r} & g\mathbbm{1}_{2\times 2} \otimes \mathbbm{1}_{r\times r} \\
        -g\mathbbm{1}_{2\times 2} \otimes \mathbbm{1}_{r\times r} & (\sigma^i p_i - im - \overline{z}) \otimes \mathbbm{1}_{r\times r}
    \end{pmatrix} + \begin{pmatrix}
        -i \sigma^i \otimes U^\dagger \partial_i U& \\
        & 0
    \end{pmatrix}] \\
    &= \det[G(p)^{-1} \otimes \mathbbm{1}_{r\times r} + \begin{pmatrix}
        -i \sigma^i \otimes A_i& \\
        & 0
    \end{pmatrix}]
\end{align*}
where $A_i = U^\dagger \partial_i U$.
Then we have to evaluate
\begin{align*}
    &\Pi^{ij}(q)\\ &= \int \frac{\dd[3]{p}}{(2\pi)^3} \tr_{\mathbb{C}^4}\qty[G(p) \begin{pmatrix}
        \sigma^i & \\ & 0
    \end{pmatrix} G(p+q) \begin{pmatrix}
        \sigma^j & \\ & 0
    \end{pmatrix}], \\
    &\Pi^{ijk}(q,k)\\ &= \int \frac{\dd[3]{p}}{(2\pi)^3} \tr_{\mathbb{C}^4}\qty[G(p) \begin{pmatrix}
        \sigma^i & \\ & 0
    \end{pmatrix} G(p+q) \begin{pmatrix}
        \sigma^j & \\ & 0
    \end{pmatrix} G(p+q+k) \begin{pmatrix}
        \sigma^k & \\ & 0
    \end{pmatrix}].
\end{align*}
where
\[ G(p) = \begin{pmatrix}
    (\sigma^i p_i + im - z) & g\mathbbm{1}_{2\times 2} \\
    -g\mathbbm{1}_{2\times 2} & (\sigma^i p_i - im - \overline{z})
\end{pmatrix}^{-1}. \]
Although the forms of $\Pi^{ij}$ and $\Pi^{ijk}$ look similar to their QED counterparts like
\[ \int \frac{\dd[d]{p}}{(2\pi)^d} \tr[\frac{1}{\slashed{p} - m}\gamma^\mu \frac{1}{\slashed{p}+\slashed{q}-m}\gamma^\nu], \]
that the vertex is given by $\begin{pmatrix}
    \sigma^i & \\ & 0
\end{pmatrix}$ and that $G(p)^{-1}$ doesn't admit a simple form in our case makes the evaluation of $\Pi^{ij}$ and $\Pi^{ijk}$ very difficult.

% \bibliographystyle{plain}
% \bibliography{main}

\end{document}
