\documentclass{article}

\usepackage{ctex}
\usepackage{pandekten}

\title{Allgemein}
\author{Ch\=an Taku}

\begin{document}

\maketitle

\section{法律部门}

\subsection{公法与私法}

\begin{example}{公法与私法区别判例}{offentliches_und_privat}
    \href{https://judgment.judicial.gov.tw/FJUD/data.aspx?ty=JD&id=TPSV,93%2c%e5%8f%b0%e4%b8%8a%2c1097%2c20040603}{93年度台上字第1097號}.
\end{example}

\section{一般原则}

\subsection{诚实信用}

\begin{proposition}{民法典第199條}{croc_199}
    權利之行使,不得違反公共利益,或以損害他人為主要目的。 行使權利,履行義務,應依誠實及信用方法。
\end{proposition}

\begin{example}{诚实信用原则判例}{treu_und_glauben}
    \begin{itemize}
        \item 权利禁止滥用.
        \begin{itemize}
            \item \href{https://judgment.judicial.gov.tw/FJUD/data.aspx?ty=JD&id=TPSV%2c108%2c%e5%8f%b0%e4%b8%8a%2c1752%2c20191217%2c1&ot=in}{108年度台上字第1752號}.
            \item \href{https://judgment.judicial.gov.tw/FJUD/data.aspx?ty=JD&id=TPSV,108%2c%e5%8f%b0%e4%b8%8a%2c1234%2c20200521%2c1}{108年度台上字第1234號}.
            \item \href{https://judgment.judicial.gov.tw/FJUD/data.aspx?ty=JD&id=TPSV,79%2c%e5%8f%b0%e4%b8%8a%2c2669%2c19901214%2c1}{79 年度台上字第2669號}.
        \end{itemize}
        \item 权利失效.
        \begin{itemize}
            \item \href{https://judgment.judicial.gov.tw/FJUD/data.aspx?ty=JD&id=TPSV,88%2c%e5%8f%b0%e4%b8%8a%2c497%2c19990311}{88年度台上字第497號}.
            \item \href{https://judgment.judicial.gov.tw/FJUD/data.aspx?ty=JD&id=TPSV,85%2c%e5%8f%b0%e4%b8%8a%2c908%2c19960426}{85年度台上字第908號}.
            \item \href{https://judgment.judicial.gov.tw/FJUD/data.aspx?ty=JD&id=TPSV,110%2c%e5%8f%b0%e4%b8%8a%2c551%2c20210311%2c1}{110年度台上字第551號}.
        \end{itemize}
        \item 禁止矛盾行为.
        \begin{itemize}
            \item \href{https://judgment.judicial.gov.tw/FJUD/data.aspx?ty=JD&id=TPSV,109%2c%e5%8f%b0%e4%b8%8a%2c1039%2c20200527%2c1}{109年度台上字第1039號}.
            \item \href{https://judgment.judicial.gov.tw/FJUD/data.aspx?ty=JD&id=TPSV,80%2c%e5%8f%b0%e4%b8%8a%2c342%2c19910222%2c1}{80年度台上字第342號}.
            \item \href{https://judgment.judicial.gov.tw/FJUD/data.aspx?ty=JD&id=TPSV,95%2c%e5%8f%b0%e4%b8%8a%2c1087%2c20060525}{95年度台上字第1087號}.
            \item \href{https://judgment.judicial.gov.tw/FJUD/data.aspx?ty=JD&id=TPSV,106%2c%e5%8f%b0%e4%b8%8a%2c927%2c20171122}{106年度台上字第927號}.
        \end{itemize}
        \item 相互体谅.
        \begin{itemize}
            \item \href{https://judgment.judicial.gov.tw/FJUD/data.aspx?ty=JD&id=TPSV,58%2c%e5%8f%b0%e4%b8%8a%2c2929%2c19691002%2c1}{58年度台上字第2929號}.
            \item \href{https://judgment.judicial.gov.tw/FJUD/data.aspx?ty=JD&id=TPSV,105%2c%e5%8f%b0%e4%b8%8a%2c210%2c20160203}{105年度台上字第210號}.
        \end{itemize}
        \item 创造法律上义务.
        \begin{itemize}
            \item 保护义务.
            \item 忠诚缔约义务.
        \end{itemize}
    \end{itemize}
\end{example}

\section*{杂录}

台湾成年年龄为何为20岁? 见\href{https://www.zhihu.com/question/63447930}{台湾成年年龄为20岁, 这是日据时期的遗存还是国民党带去的?}

% \bibliographystyle{plain}
% \bibliography{main}

\end{document}
