\documentclass{article}

\usepackage{ctex}
\usepackage{pandekten}

\title{Allgemein}
\author{Ch\=an Taku}

\begin{document}

\maketitle

\section{債}

\begin{definition}{債, Schuldverhältnis}{schuldverhaltnis}
    債指特定關係人之間得請求一定給付的法律關係.
\end{definition}

\begin{proposition}{民法典第199條}{croc_199}
    \begin{enumerate}
        \item 債權人基於債之關係,得向債務人請求給付。
        \item 給付,不以有財產價格者為限。
        \item 不作為亦得為給付。
    \end{enumerate}
\end{proposition}

\begin{definition}{狹義債之關係, Schuldverhältnis im engeren Sinne}{schuldverhaltnis_im_engeren_sinne}
    狹義債之關係指個別的給付關係, 自得請求給付的一方而言, 是爲債權, 自負有給付義務的一方而言, 則爲債務.
\end{definition}

\begin{definition}{廣義債之關係, Schuldverhältnis im weiteren Sinne}{schuldverhaltnis_im_weiteren_sinne}
    廣義債之關係指包括多數債權與債務的法律關係.
\end{definition}

\section{Terminologie}

\begin{longtable}{ccc}
    \toprule
    Chinesish & Deutsch & Japanish \\
    \midrule
    意思表示 & Willenserkl\"arung & 意思表示 \\
    撤回 & Widerruf & 撤回 \\
    撤销 & Anfechtung & 取消し \\
    无效 & Unwirksamkeit & 無効 \\
    要约引诱 & invitatio ad offerendum & 申込みの誘引 \\
    要约 & Angebot, Antrag & 申込み \\
    契约 & Vertrag & 契約 \\
    \bottomrule
\end{longtable}

% \bibliographystyle{plain}
% \bibliography{main}

\end{document}
