\documentclass{article}

\usepackage{pandekten}
\usepackage{dashrule}

\makeatletter
\newcommand*{\shifttext}[1]{%
  \settowidth{\@tempdima}{#1}%
  \hspace{-\@tempdima}#1%
}
\newcommand{\plabel}[1]{%
\shifttext{\textbf{#1}\quad}%
}
\newcommand{\prule}{%
\begin{center}%
\hdashrule[0.5ex]{.99\linewidth}{1pt}{1pt 2.5pt}%
\end{center}%
}

\makeatother

\newcommand{\minusbaseline}{\abovedisplayskip=0pt\abovedisplayshortskip=0pt~\vspace*{-\baselineskip}}%

\setlength{\parindent}{0pt}

\title{Assignment 1}
\author{Ze Chen}

\begin{document}

\maketitle

\plabel{1}%
\minusbaseline%
\begin{align*}
    \pdv{\rho}{t} &= -\epsilon_0 \div \pdv{\vb{E}}{t} = \div\qty(\frac{1}{\mu_0}\curl \vb{B} - \vb{J}) = -\div\vb{J}.
\end{align*}

\prule
\plabel{2 (a)}%
With
\[ v = \frac{1}{\sqrt{\epsilon_0 \mu_0}} \]
the equation is rewritten as
\[ \qty(\pdv{}{x} + \frac{1}{v}\pdv{}{t})\qty(\pdv{}{x} - \frac{1}{v}\pdv{}{t}) E = \pdv{}{\epsilon}\pdv{}{\eta}E = 0. \]
Integrating twice we get
\begin{align*}
    E &= f_1(\epsilon) + f_2(\eta) = f_1(x - vt) + f_2(x + vt).
\end{align*}

\plabel{(b)}%
\minusbaseline%
\[ v = \frac{1}{\sqrt{\epsilon_0 \mu_0}}. \]

\plabel{(c)}%
The conditions in terms of $f_1$ and $f_2$ are
\begin{align*}
    F(x) &= f_1(x) + f_2(x), \\
    G(x) &= -v f'_1(x) + v f'_2(x) \\
    \Rightarrow \int_0^x \dd{x'} G(x') &= -v (f_1(x) - f_1(0)) + v (f_2(x) - f_2(0)).
\end{align*}
Therefore,
\begin{align*}
    f_1(x) &= \frac{1}{2}\qty[F(x) + vf_1(0) - vf_2(0) - \int_0^x \dd{x'} G(x')], \\
    f_2(x) &= \frac{1}{2}\qty[F(x) - vf_1(0) + vf_2(0) + \int_0^x \dd{x'} G(x')].
\end{align*}

\plabel{(d)}%
$E = A \cos(x - vt)$ and $E = A \delta(x-vt)$.

\prule

\plabel{3 (a)}%
With $v = \omega / k$,
\[ \grad^2 U - \frac{1}{v^2} \pdv[2]{U}{t} = - k^2 U - \frac{-\omega^2}{v^2} U = 0. \]

\plabel{(b)}%
\minusbaseline%
\[ \frac{\omega}{k} = \dv{\omega}{k} = v = \frac{1}{\sqrt{\epsilon_0 \mu_0}}. \]
In vacuum, $v_{\mathrm{p}} = v_{\mathrm{g}}$.

\plabel{(c.i)}%
The divergences give
\begin{align*}
    \div \vb{E} &= i\vb{k}\cdot\vb{E} = 0 \Rightarrow \vb{k}\cdot \hat{\vb{e}} = 0, \\
    \div \vb{B} &= i\vb{k}\cdot\vb{B} = 0 \Rightarrow \vb{k}\cdot \hat{\vb{b}} = 0.
\end{align*}

\plabel{(c.ii)}%
The curls give
\begin{align}
    \label{eq:curlE} \curl\vb{E} &= i\vb{k}\times\vb{E} = -\pdv{\vb{B}}{t} = i\omega\vb{B} \Rightarrow i\vb{k}\times\hat{\vb{e}} E_0 = ick\hat{\vb{b}} B_0, \\
    \label{eq:curlB} \curl\vb{B} &= i\vb{k}\times \vb{E} = \mu_0\epsilon_0\partial_t E = -i\omega\epsilon_0\mu_0 \vb{E}\Rightarrow ic\vb{k}\times \hat{\vb{b}}B_0 = -ik\hat{\vb{e}}E_0.
\end{align}
$\text{\Cref{eq:curlE}} \cdot \hat{\vb{b}}$ and $\text{\cref{eq:curlB}} \cdot \hat{\vb{e}}$ yield
\begin{align*}
    (\hat{\vb{e}}\times \hat{\vb{b}})\cdot\hat{\vb{k}} E_0 &= cB_0, \\
    (\hat{\vb{e}}\times \hat{\vb{b}})\cdot\hat{\vb{k}} cB_0 &= E_0,
\end{align*}
which requires either
\[ (\hat{\vb{e}}\times \hat{\vb{b}})\cdot\hat{\vb{k}} = 1,\quad E_0 = cB_0, \]
or
\[ (\hat{\vb{e}}\times \hat{\vb{b}})\cdot\hat{\vb{k}} = -1,\quad E_0 = -cB_0. \]
In the first case, we find
\[ \hat{\vb{b}} = \hat{\vb{k}} \times \hat{\vb{e}}. \]

\prule

\plabel{4 (a)}%
For the first equation,
\begin{align*}
    \div \vb{E} &= \frac{\rho}{\epsilon_0}, \\
    &\Downarrow \\
    \grad \div \vb{E} &= \frac{1}{\epsilon_0}\grad\rho, \\
    &\Downarrow \\
    \curl\curl\vb{E} + \grad^2 \vb{E} &= \frac{1}{\epsilon_0}\grad\rho, \\
    &\Downarrow \\
    \grad^2 \vb{E} - \pdv{}{t}\curl \vb{B} &= \frac{1}{\epsilon_0}\grad\rho, \\
    &\Downarrow \\
    \grad^2 \vb{E} - \mu_0\pdv{}{t}\vb{J} - \frac{1}{c^2}\pdv[2]{\vb{E}}{t} &= \frac{1}{\epsilon_0}\grad\rho.
\end{align*}
For the second equation,
\begin{align*}
    \div \vb{B} &= 0, \\
    &\Downarrow \\
    \grad \div \vb{B} &= 0, \\
    &\Downarrow \\
    \curl\curl\vb{B} + \grad^2 \vb{B} &= 0, \\
    &\Downarrow \\
    \grad^2 \vb{B} + \mu_0\curl\vb{J} + \frac{1}{c^2}\pdv{}{t} \curl \vb{E} &= 0, \\
    &\Downarrow \\
    \grad^2 \vb{B} + \mu_0\curl\vb{J} - \frac{1}{c^2}\pdv[2]{\vb{B}}{t} &= 0.
\end{align*}

\plabel{(b)}%
\minusbaseline%
\[ c = \frac{1}{\sqrt{\epsilon_0\mu_0}}. \]

\plabel{(c)}%
Applying Fourier transform we find
\[ \qty(-\vb{k}^2 + \frac{\omega^2}{c^2}) \tilde{\vb{E}}(\omega',\vb{k}) = \mu_0 J_0 (2\pi)^3 \delta^{(2)}(k_y,k_z)\frac{\delta(\omega'-\omega) + \delta(\omega'+\omega)}{2} \hat{\vb{z}}, \]
which may be solved to yield
\[ \vb{E}(t,\vb{x}) = \frac{1}{2} \mu_0 c J_0 \hat{\vb{z}} \frac{1}{\omega} \sin(\frac{\abs{x}\omega}{c}) \cos(\omega t). \]


% \bibliographystyle{plain}
% \bibliography{main}

\end{document}
