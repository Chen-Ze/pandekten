\documentclass{article}

\usepackage{pandekten}
\usepackage{dashrule}

\makeatletter
\newcommand*{\shifttext}[1]{%
  \settowidth{\@tempdima}{#1}%
  \hspace{-\@tempdima}#1%
}
\newcommand{\plabel}[1]{%
\shifttext{\textbf{#1}\quad}%
}
\newcommand{\prule}{%
\begin{center}%
\hdashrule[0.5ex]{.99\linewidth}{1pt}{1pt 2.5pt}%
\end{center}%
}

\makeatother

\setlength{\parindent}{0pt}

\title{Assignment 2}
\author{Ze Chen}

\begin{document}

\maketitle

% \bibliographystyle{plain}
% \bibliography{main}

\plabel{1}%
The equation of motion is given by
\begin{align*}
    &\phantom{{}={}} \pdv{}{t}\pdv{\mathcal{L}}{(\partial_t\psi)} + \div \pdv{\mathcal{L}}{(\grad \psi)} - \pdv{\mathcal{L}}{\psi} = i\pdv{\psi^*}{t} - \frac{1}{2m}\grad^2\psi^* = 0,
\end{align*}
i.e. the Schr\"odinger equation
\[ i\pdv{\psi}{t} = -\frac{1}{2m}\grad^2 \psi. \]
The infinitesimal transformation
\begin{align*}
    \psi &\rightarrow \psi + i\alpha\psi, \\
    \psi^* &\rightarrow \psi^* - i\alpha\psi^*
\end{align*}
gives rise to the charge and current
\begin{align*}
    Q &= -\pdv{\mathcal{L}}{(\partial_t \psi)}\frac{\Delta \psi}{\alpha} - \pdv{\mathcal{L}}{(\partial_t \psi^*)}\frac{\Delta \psi^*}{\alpha} = \abs{\psi}^2, \\
    \vb{j} &= -\pdv{\mathcal{L}}{(\grad \psi)}\frac{\Delta \psi}{\alpha} - \pdv{\mathcal{L}}{(\grad \psi^*)}\frac{\Delta \psi^*}{\alpha} = -\frac{i}{2m}\qty[(\grad\psi)\psi^* - (\grad \psi^*)\psi],
\end{align*}
i.e. the probability current.
The conjugate momentum is given by
\[ \pi = \pdv{\mathcal{L}}{(\partial_t \psi)} = i\psi^*. \]
The Hamiltonian is given by
\[ H = \int \dd{^3 \vb{x}} \qty(\pi \partial_0\psi - \mathcal{L}) = \int\dd{^3\vb{x}} \frac{1}{2m}\abs{\grad\psi}^2. \]
$\psi^\dagger(\vb{x})$ creates a particle at $\vb{x}$.
Therefore, in parallel with $[a,a^\dagger] = 1$, we write
\[ \qty[\psi(\vb{x}),\psi^\dagger(\vb{y})] = \delta^{(3)}(\vb{x} - \vb{y}). \]
Substituting the Fourier transform of $\psi$ into the commutator we find
\begin{align*}
    [a_{\vb{p}},a_{\vb{q}}^\dagger] &= \int \dd{^3\vb{x}} \int \dd{^3\vb{y}} e^{-i\vb{p}\cdot \vb{x} + i\vb{q}\cdot\vb{y}}\qty[\psi(\vb{x}),\psi^\dagger(\vb{y})] \\
    &= \int \dd{^3\vb{x}} \int \dd{^3\vb{y}} e^{-i\vb{p}\cdot \vb{x} + i\vb{q}\cdot\vb{y}}\delta^{(3)}(\vb{x} - \vb{y}) \\
    &= (2\pi)^3\delta^{(3)}(\vb{p} - \vb{q}).
\end{align*}
The Hamiltonian is given by
\begin{align*}
    H &= \int \dd{^3\vb{p}} \frac{\vb{p}^2}{2m} \tilde{\psi}^\dagger(\vb{k})\tilde{\psi}^\dagger(\vb{k}) = \int \dd{^3\vb{p}} \frac{\vb{p}^2}{2m} a_{\vb{p}}^\dagger a_{\vb{p}}.
\end{align*}
It's clear that each particle of momentum $\vb{p}$ contributes $\vb{p}^2/2m$ to the energy.
The other set of creation and annihilation operators is missing because the conjugate momentum of $\psi^*$ is $0$.
In the nonrelativistic case, the scale of energy is too low to create an antiparticle (gap $\sim 2mc^2$).
The $n$ particle states are given by
\[ \operatorname{Sym}\qty(\operatorname{Span}\Set*{\ket{\vb{p}_1}\otimes \cdots \otimes\ket{\vb{p}_n}}{\vb{p}_1,\cdots,\vb{p}_n\in\mathbb{R}^3}), \]
where $\operatorname{Sym}$ denotes symmetrization.
The Energy is given by
\[ H \operatorname{Sym}\qty(\ket{\vb{p}_1}\otimes \cdots \otimes\ket{\vb{p}_n}) = \qty(\frac{\vb{p}_1^2}{2m} + \cdots + \frac{\vb{p}_n^2}{2m})\operatorname{Sym}\qty(\ket{\vb{p}_1}\otimes \cdots \otimes\ket{\vb{p}_n}). \]
The $n$-particle states in quantum mechanics are described by their wavefunction
\[ \Psi(\vb{x}_1,\cdots,\vb{x}_n) \propto \int \frac{\dd{^3 \vb{p}_1}}{2\pi} \cdots \int \frac{\dd{^3 \vb{p}_n}}{2\pi} \Phi(\vb{p}_1,\cdots,\vb{p}_n) e^{i\vb{p}_1\cdot \vb{x}_1 + \cdots + i \vb{p}_n\cdot \vb{x}_n}, \]
which may be written as
\[ \ket{\Psi} \propto \int \frac{\dd{^3 \vb{p}_1}}{2\pi} \cdots \int \frac{\dd{^3 \vb{p}_n}}{2\pi} \Phi(\vb{p}_1,\cdots,\vb{p}_n) \ket{\vb{p}_1}\otimes \cdots \otimes\ket{\vb{p}_n}. \]

\prule

\plabel{2 (a)}%
We have
\[ \phi = \frac{1}{\sqrt{2m}}(a + a^\dagger). \]
The time evolution of $a^\dagger$ is given by $a^\dagger(t) = e^{imt} a^\dagger$.
Therefore,
\[ \phi(t) = \frac{1}{\sqrt{2m}}\qty(e^{-imt} a + e^{imt} a^\dagger). \]
The correlators are given by
\begin{align*}
    \ev{\phi(t)\phi(0)}{0} &= \frac{1}{2m} e^{-imt}, \\
    \ev{\phi(0)\phi(t)}{0} &= \frac{1}{2m} e^{imt}.
\end{align*}
The Green's functions are given by
\begin{align*}
    G_{\mathrm{R}}(t) &= -\frac{i}{m}\sin(mt) \Theta(t), \\
    G_{\mathrm{A}}(t) &= \frac{i}{m}\sin(mt) \Theta(-t), \\
    G_{\mathrm{F}}(t) &= \frac{1}{2m} e^{-imt}\Theta(t) + \frac{1}{2m} e^{imt} \Theta(-t).
\end{align*}

\plabel{(b)}
Differentiate twice we find
\begin{align*}
    \dv[2]{G_{\mathrm{R}}(t)}{t} + m^2 G_{\mathrm{R}}(t) &= -i(\delta(t) - m \sin(mt)\Theta(t)) + m^2 G_{\mathrm{R}}(t) = -i\delta(t), \\
    \dv[2]{G_{\mathrm{A}}(t)}{t} + m^2 G_{\mathrm{A}}(t) &= -i(\delta(t) - m \sin(mt)\Theta(t)) + m^2 G_{\mathrm{A}}(t) = -i\delta(t), \\
    \dv[2]{G_{\mathrm{F}}(t)}{t} + m^2 G_{\mathrm{F}}(t) &= -i\delta(t) - \frac{m}{2}\qty(e^{-imt}\Theta(t) + e^{imt}\Theta(-t)) + m^2 G_{\mathrm{F}}(t) \\ &= -i\delta(t).
\end{align*}
The differences are given by
\begin{align*}
    G_{\mathrm{R}}(t) - G_{\mathrm{A}}(t) &= -\frac{i}{m}\sin(mt), \\
    2G_{\mathrm{F}}(t) - \qty(G_{\mathrm{R}}(t) + G_{\mathrm{A}}(t)) &= \frac{\cos(mt)}{m},
\end{align*}
which are solutions to the homogenous equation.
$G_{\mathrm{R}}(t) = 0$ for all $t<0$.
$G_{\mathrm{A}}(t) = 0$ for all $t>0$.
$G_{\mathrm{F}}(t) \neq 0$ for all $t$.
It's clear that
\[ G_{\mathrm{F}}(t + i\infty) = 0 \]
if $t < 0$ and
\[ G_{\mathrm{F}}(t - i\infty) = 0 \]
if $t > 0$.

\plabel{(c)}%
The equation is transformed into
\[ (-\omega^2 + m^2) \tilde{G}(\omega) = -i. \]
Therefore,
\begin{align*}
    \tilde{G}(\omega) &= \frac{i}{\omega^2 - m^2}, \\
    G(t) &= \int_{-\infty}^\infty \frac{\dd{\omega}}{2\pi} \tilde{G}(\omega) e^{-i\omega t},
\end{align*}
and the path should be completed below the $\Re \omega = 0$ axis for $t>0$ and above for $t<0$.
The residues are given by
\begin{align*}
    \operatorname{Res}_{m} &= -\frac{e^{-imt}}{2m}, \\
    \operatorname{Res}_{-m} &= \frac{e^{imt}}{2m}.
\end{align*}
Now the Green's functions are evaluated using the residue theorem.
\begin{align*}
    G_{\mathrm{R}}(t) &= \begin{cases}
        -\operatorname{Res}_m-\operatorname{Res}_{-m}, & \text{if } t>0, \\
        0, & \text{if } t<0
    \end{cases} = -\frac{i}{m}\sin(mt) \Theta(t), \\
    G_{\mathrm{A}}(t) &= \begin{cases}
        0, & \text{if } t>0, \\
        \operatorname{Res}_m + \operatorname{Res}_{-m}, & \text{if } t<0
    \end{cases} = \frac{i}{m}\sin(mt) \Theta(-t), \\
    G_{\mathrm{F}}(t) &= \begin{cases}
        -\operatorname{Res}_{m}, & \text{if } t>0, \\
        \operatorname{Res}_{-m}, & \text{if } t<0
    \end{cases} = \frac{e^{-imt}}{2m} \Theta(t) + \frac{e^{imt}}{2m} \Theta(-t).
\end{align*}
Therefore,
\begin{align*}
    G_{\mathrm{R}}(t) &= \int_{-\infty}^\infty \frac{\dd{\omega}}{2\pi} \frac{i}{(\omega+i\epsilon)^2 - m^2} e^{-i\omega t}, \\
    G_{\mathrm{A}}(t) &= \int_{-\infty}^\infty \frac{\dd{\omega}}{2\pi} \frac{i}{(\omega-i\epsilon)^2 - m^2} e^{-i\omega t}, \\
    G_{\mathrm{F}}(t) &= \int_{-\infty}^\infty \frac{\dd{\omega}}{2\pi} \frac{i}{\omega^2 - m^2 + i\epsilon} e^{-i\omega t}.
\end{align*}

\plabel{(d)}%
To extend to $t \rightarrow e^{-i\theta}t$ the integration contour should be rotated counterclockwise by $\theta$ such that the integrand remains finite as $\omega \rightarrow \infty$.
To not encounter singularities, the range is given by $0\le \theta < \pi$.
With the substitution $t \rightarrow -i\tau$ and $\omega \rightarrow i \rho$,
\[ \mathcal{G}_{\mathrm{F}}(\tau) = \int_{-\infty}^\infty \frac{\dd{\rho}}{2\pi} \frac{i}{-\rho^2 - m^2} e^{-i\rho \tau}. \]

\prule
\plabel{3 (a)}%
$\phi(y)$ contributes $a^\dagger$ while $\phi(x)$ contributes $a$.
\begin{align}
    \notag \bra{0} \phi(x) \phi(y) \ket{0} &= \int \frac{\dd{^3\vb{p}}}{(2\pi)^3} \int \frac{\dd{^3\vb{q}}}{(2\pi)^3} \frac{1}{\sqrt{2\omega_{\vb{p}}}}\frac{1}{\sqrt{2\omega_{\vb{q}}}} a_{\vb{p}} e^{-ipx} a^\dagger_{\vb{q}} e^{iqy} \\
    \notag &= \int \frac{\dd{^3\vb{p}}}{(2\pi)^3} \int \frac{\dd{^3\vb{q}}}{(2\pi)^3} \frac{1}{\sqrt{2\omega_{\vb{p}}}}\frac{1}{\sqrt{2\omega_{\vb{q}}}} \qty(\cancel{a^\dagger_{\vb{q}}a_{\vb{p}}} + [a_{\vb{p}},a^\dagger_{\vb{q}}]) e^{-ipx} e^{iqy} \\
    \notag &= \int \frac{\dd{^3\vb{p}}}{(2\pi)^3} \int \dd{^3\vb{q}} \frac{1}{\sqrt{2\omega_{\vb{p}}}}\frac{1}{\sqrt{2\omega_{\vb{q}}}} \delta^{(3)}(\vb{p}-\vb{q}) e^{-ipx} e^{iqy} \\
    \label{eq:phiphi} &= \int \frac{\dd{^3\vb{p}}}{(2\pi)^3} \frac{1}{2\omega_{\vb{p}}} e^{-ip\cdot (x-y)}.
\end{align}

\plabel{(b)}%
After doing the angular integrals we find
\[ D(t,r) = (2\pi)^{-3/2} \frac{1}{\sqrt{r}} \int_0^\infty \sqrt{\frac{2}{\pi}} \frac{\sin(pr)}{\sqrt{pr}} \sqrt{p}\cdot p \cdot \frac{e^{-it\sqrt{m^2+p^2}}}{2  \sqrt{m^2+p^2}}\dd{p}. \]
For space-like separations, we take $t=0$ and find
\begin{align}
    \notag D(r) &= \frac{-i}{2(2\pi)^2 r} \int_{-\infty}^\infty e^{ipr} \frac{p}{\sqrt{m^2+p^2}}\dd{p} \\
    \label{eq:space_like_d} &= \frac{m}{(2\pi)^2 r} K_1(mr).
\end{align}
At large $r$,
\[ D(r) \sim e^{-mr}. \]
For time-like separations, we take $r=0$ and find
\begin{align}
    \notag D(t) &= \frac{1}{(2\pi)^2} \int_m^\infty \sqrt{\omega^2 - m^2} e^{-it\omega} \dd{\omega} \\
    \label{eq:time_like_d} &= \frac{im(J_1(mt) - iY_1(m\abs{t}))}{8\pi \abs{t}}.
\end{align}
At large $t$,
\[ D(t) \sim \abs{t}^{-3/2} e^{-it}. \]
Therefore, for space-like $x-y$,
\[ D(x-y) = \frac{m}{(2\pi)^2 \sqrt{-(x-y)^2}} K_1\qty(m\sqrt{-(x-y)^2}). \]
For time-like $x-y$ with $x^0 > y^0$,
\[ D(x-y) = \frac{im}{8\pi \sqrt{(x-y)^2}} H^{(2)}_1\qty(m\sqrt{(x-y)^2}). \]
For time-like $x-y$ with $x^0 < y^0$,
\[ D(x-y) = \frac{-im}{8\pi \sqrt{(x-y)^2}} H^{(1)}_1\qty(m\sqrt{(x-y)^2}). \]

\plabel{(c)}%
From symmetry it's clear that the support is given by
\[ \Set*{x-y}{(x-y)^2 \ge 0}, \]
i.e. all time-like and null-like separations.
Physically, a particle creation could not affect the measurement a space-like distance away.
For time-like separation, with \cref{eq:time_like_d} we find
\[ \bra{0}[\phi(x),\phi(y)]\ket{0} = \frac{imJ_1\qty(m\sqrt{(x-y)^2})}{4\pi\sqrt{(x-y)^2}}. \]

\plabel{(d)}%
For space-like $x-y$, using \cref{eq:space_like_d} we find
\[ D_{\mathrm{F}}(x-y) = \frac{m}{(2\pi)^2 \sqrt{-(x-y)^2}} K_1\qty(m\sqrt{-(x-y)^2}). \]
For time-like $x-y$, using \cref{eq:time_like_d} we find
\[ D_{\mathrm{F}}(x-y) = \frac{im}{8\pi \sqrt{(x-y)^2}} H^{(2)}_1\qty(m\sqrt{(x-y)^2}). \]
The support is the whole spacetime.
For $x^0 > y^0$, $D_{\mathrm{F}}(x-y)$ involves $e^{-i\omega(x^0 - y^0)}$ in \cref{eq:phiphi}, while for $x^0 < y^0$, $D_{\mathrm{F}}(x-y)$ involves $e^{i\omega(x^0 - y^0)}$.
The sign of $\omega$ associated with $x-y$ is different, suggesting the existence of antiparticle of mass $m$.

\plabel{(e)}%
Taking the limit $m\rightarrow 0$ we find $G_{\mathrm{R}}(x - y) = 0$ except for null-like $x-y$.
\[ G_{\mathrm{R}}(x-y) = -\frac{i\Theta(x-y)}{2\pi}\delta((x-y)^2), \]
which is the one for the wave equation (since the Klein-Gordon equation for $m=0$ is the wave equation).
Taking the limit $m\rightarrow 0$ we find that the Feynman propagator is given by
\[ G_{\mathrm{F}}(x-y) = -\frac{1}{4\pi^2 (x-y)^2}. \]
The support is the whole spacetime.

\end{document}
