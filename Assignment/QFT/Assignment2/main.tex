\documentclass{article}

\usepackage{pandekten}
\usepackage{dashrule}

\makeatletter
\newcommand*{\shifttext}[1]{%
  \settowidth{\@tempdima}{#1}%
  \hspace{-\@tempdima}#1%
}
\newcommand{\plabel}[1]{%
\shifttext{\textbf{#1}\quad}%
}
\newcommand{\prule}{%
\begin{center}%
\hdashrule[0.5ex]{.99\linewidth}{1pt}{1pt 2.5pt}%
\end{center}%
}

\makeatother

\setlength{\parindent}{0pt}

\title{Assignment 2}
\author{Ze Chen}

\begin{document}

\maketitle

% \bibliographystyle{plain}
% \bibliography{main}

\plabel{1}%
The equation of motion is given by
\begin{align*}
    &\phantom{{}={}} \pdv{}{t}\pdv{\mathcal{L}}{(\partial_t\psi)} + \div \pdv{\mathcal{L}}{(\grad \psi)} - \pdv{\mathcal{L}}{\psi} = i\pdv{\psi^*}{t} - \frac{1}{2m}\grad^2\psi^* = 0,
\end{align*}
i.e. the Schr\"odinger equation
\[ i\pdv{\psi}{t} = -\frac{1}{2m}\grad^2 \psi. \]
The infinitesimal transformation
\begin{align*}
    \psi &\rightarrow \psi + i\alpha\psi, \\
    \psi^* &\rightarrow \psi^* - i\alpha\psi^*
\end{align*}
gives rise to the charge and current
\begin{align*}
    Q &= -\pdv{\mathcal{L}}{(\partial_t \psi)}\frac{\Delta \psi}{\alpha} - \pdv{\mathcal{L}}{(\partial_t \psi^*)}\frac{\Delta \psi^*}{\alpha} = \abs{\psi}^2, \\
    \vb{j} &= -\pdv{\mathcal{L}}{(\grad \psi)}\frac{\Delta \psi}{\alpha} - \pdv{\mathcal{L}}{(\grad \psi^*)}\frac{\Delta \psi^*}{\alpha} = -\frac{i}{2m}\qty[(\grad\psi)\psi^* - (\grad \psi^*)\psi],
\end{align*}
i.e. the probability current.
The conjugate momentum is given by
\[ \pi = \pdv{\mathcal{L}}{(\partial_t \psi)} = i\psi^*. \]
The Hamiltonian is given by
\[ H = \int \dd{^3 \vb{x}} \qty(\pi \partial_0\psi - \mathcal{L}) = \int\dd{^3\vb{x}} \frac{1}{2m}\abs{\grad\psi}^2. \]
$\psi^\dagger(\vb{x})$ creates a particle at $\vb{x}$.
Therefore, in parallel with $[a,a^\dagger] = 1$, we write
\[ \qty[\psi(\vb{x}),\psi^\dagger(\vb{y})] = \delta^{(3)}(\vb{x} - \vb{y}). \]
Substituting the Fourier transform of $\psi$ into the commutator we find
\begin{align*}
    [a_{\vb{p}},a_{\vb{q}}^\dagger] &= \int \dd{^3\vb{x}} \int \dd{^3\vb{y}} e^{-i\vb{p}\cdot \vb{x} + i\vb{q}\cdot\vb{y}}\qty[\psi(\vb{x}),\psi^\dagger(\vb{y})] \\
    &= \int \dd{^3\vb{x}} \int \dd{^3\vb{y}} e^{-i\vb{p}\cdot \vb{x} + i\vb{q}\cdot\vb{y}}\delta^{(3)}(\vb{x} - \vb{y}) \\
    &= (2\pi)^3\delta^{(3)}(\vb{p} - \vb{q}).
\end{align*}
The Hamiltonian is given by
\begin{align*}
    H &= \int \dd{^3\vb{p}} \frac{\vb{p}^2}{2m} \tilde{\psi}^\dagger(\vb{k})\tilde{\psi}^\dagger(\vb{k}) = \int \dd{^3\vb{p}} \frac{\vb{p}^2}{2m} a_{\vb{p}}^\dagger a_{\vb{p}}.
\end{align*}
It's clear that each particle of momentum $\vb{p}$ contributes $\vb{p}^2/2m$ to the energy.
In the nonrelativistic case, the scale of energy is too low to create an antiparticle, and therefore there is only one set of operators.
{\color{red}Discuss the physical properties of the $n$ particle states,
and compare them with the usual description of $n$ particle states in quantum mechanics.}

\prule

\plabel{2 (a)}%
We have
\[ \phi = \frac{1}{2m}(a + a^\dagger). \]
The time evolution of $a^\dagger$ is given by $a^\dagger(t) = e^{imt} a^\dagger$.
Therefore,
\[ \phi(t) = \frac{1}{2m}\qty(e^{-imt} a + e^{imt} a^\dagger). \]
The correlators are given by
\begin{align*}
    \ev{\phi(t)\phi(0)}{0} &= \frac{1}{2m} e^{-imt}, \\
    \ev{\phi(0)\phi(t)}{0} &= \frac{1}{2m} e^{imt}.
\end{align*}
The Green's functions are given by
\begin{align*}
    G_{\mathrm{R}}(t) &= -\frac{i}{m}\sin(mt) \Theta(t), \\
    G_{\mathrm{A}}(t) &= \frac{i}{m}\sin(mt) \Theta(-t), \\
    G_{\mathrm{F}}(t) &= \frac{1}{2m} e^{-imt}\Theta(t) + \frac{1}{2m} e^{imt} \Theta(-t).
\end{align*}

\plabel{(b)}
Differentiate twice we find
\begin{align*}
    \dv[2]{G_{\mathrm{R}}(t)}{t} + m^2 G_{\mathrm{R}}(t) &= -i(\delta(t) - m \sin(mt)\Theta(t)) + m^2 G_{\mathrm{R}}(t) = -i\delta(t), \\
    \dv[2]{G_{\mathrm{A}}(t)}{t} + m^2 G_{\mathrm{A}}(t) &= -i(\delta(t) - m \sin(mt)\Theta(t)) + m^2 G_{\mathrm{A}}(t) = -i\delta(t), \\
    \dv[2]{G_{\mathrm{F}}(t)}{t} + m^2 G_{\mathrm{F}}(t) &= -i\delta(t) - \frac{m}{2}\qty(e^{-imt}\Theta(t) + e^{imt}\Theta(-t)) + m^2 G_{\mathrm{F}}(t) \\ &= -i\delta(t).
\end{align*}
The differences are given by
\begin{align*}
    G_{\mathrm{R}}(t) - G_{\mathrm{A}}(t) &= -\frac{i}{m}\sin(mt), \\
    2G_{\mathrm{F}}(t) - \qty(G_{\mathrm{R}}(t) + G_{\mathrm{A}}(t)) &= \frac{\cos(mt)}{m},
\end{align*}
which are solutions to the homogenous equation.
{\color{red}They can be distinguished by their behavior as $t\rightarrow\pm\infty$.}

\plabel{(c)}%
The equation is transformed into
\[ (-\omega^2 + m^2) \tilde{G}(\omega) = -i. \]
Therefore,
\begin{align*}
    \tilde{G}(\omega) &= \frac{i}{\omega^2 - m^2}, \\
    G(t) &= \int_{-\infty}^\infty \frac{\dd{\omega}}{2\pi} \tilde{G}(\omega) e^{-i\omega t},
\end{align*}
and the path should be completed below the $\Re \omega = 0$ axis for $t>0$ and above for $t<0$.
The residues are given by
\begin{align*}
    \operatorname{Res}_{m} &= -\frac{e^{-imt}}{2m}, \\
    \operatorname{Res}_{-m} &= \frac{e^{imt}}{2m}.
\end{align*}
Now the Green's functions are evaluated using the residue theorem.
\begin{align*}
    G_{\mathrm{R}}(t) &= \begin{cases}
        -\operatorname{Res}_m-\operatorname{Res}_{-m}, & \text{if } t>0, \\
        0, & \text{if } t<0
    \end{cases} = -\frac{i}{m}\sin(mt) \Theta(t), \\
    G_{\mathrm{R}}(t) &= \begin{cases}
        0, & \text{if } t>0, \\
        \operatorname{Res}_m + \operatorname{Res}_{-m}, & \text{if } t<0
    \end{cases} = \frac{i}{m}\sin(mt) \Theta(-t), \\
    G_{\mathrm{F}}(t) &= \begin{cases}
        -\operatorname{Res}_{m}, & \text{if } t>0, \\
        \operatorname{Res}_{-m}, & \text{if } t<0
    \end{cases} = \frac{e^{-imt}}{2m} \Theta(t) + \frac{e^{imt}}{2m} \Theta(-t).
\end{align*}
{\color{red}Draw the path and write down $\tilde{G}$ with $\epsilon$.}

\plabel{(d)}%
The transform $t \rightarrow e^{-i\theta}t$ is equivalent to a counterclockwise rotation of the countour since

\end{document}
