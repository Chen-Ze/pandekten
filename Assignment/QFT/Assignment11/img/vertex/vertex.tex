\documentclass[preview]{standalone}

\usepackage[compat=1.1.0]{tikz-feynman}
\usepackage{simpler-wick}
\usepackage{physics}

\usepackage{amsmath}
\usepackage{mathtools}

\begin{document}
\abovedisplayskip=0pt
\begin{align*}
    \feynmandiagram [inline=(v.base), small, horizontal=c2 to c3] {
        v[dot] --[cyan!50!black] c1;
        c2 --[cyan!50!black] v;
        c3 --[cyan!50!black] v;
    }; \ &= -6i\lambda v, &
    \feynmandiagram [inline=(v.base), small, horizontal=c2 to c3] {
        v[dot] --[cyan!50!black] c1;
        c2[particle=$i$] -- v;
        c3[particle=$j$] -- v;
    }; \ &= -2i\delta^{ij}\lambda v, \\
    \feynmandiagram [inline=(v.base), small, horizontal=c2 to c3] {
        c3 --[cyan!50!black] v[dot];
        v --[cyan!50!black] c0;
        v --[cyan!50!black] c1;
        c2 --[cyan!50!black] v;
    }; \ &= -6i\lambda, &
    \feynmandiagram [inline=(v.base), small, horizontal=c2 to c3] {
        v[dot] --[cyan!50!black] c0;
        c3[particle=$j$] -- v;
        v --[cyan!50!black] c1;
        c2[particle=$i$] -- v;
    }; \ &= -2i\lambda\delta^{ij}, \\
    \feynmandiagram [inline=(v.base), small, horizontal=c2 to c3] {
        c3[particle=$j$] -- v[dot];
        v -- c0[particle=$k$];
        v -- c1[particle=$l$];
        c2[particle=$i$] -- v;
    }; \ &\mathrlap{= -2i\lambda \qty[\delta^{ij} \delta^{kl} + \delta^{ik}\delta^{jl} + \delta^{il}\delta^{jk}].}
\end{align*}
\end{document}