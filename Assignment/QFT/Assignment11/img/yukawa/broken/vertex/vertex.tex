\documentclass[preview]{standalone}

\usepackage[compat=1.1.0]{tikz-feynman}
\usepackage{simpler-wick}
\usepackage{physics}

\usepackage{amsmath}
\usepackage{mathtools}

\begin{document}
\abovedisplayskip=0pt
\begin{align*}
    \feynmandiagram [inline=(v.base), small, horizontal=i to v] {
        i --[scalar] v[dot];
        v --[fermion] c2;
        c1 --[fermion] v;
    }; \ &= -ig, &
    \feynmandiagram [inline=(v.base), small, horizontal=i to v] {
        i --[boson] v[dot];
        v --[fermion] c2;
        c1 --[fermion] v;
    }; \ &= -ig(i\gamma^5), \\
    \feynmandiagram [inline=(v.base), small, horizontal=c2 to c3] {
        v[dot] --[scalar] c1;
        c2 --[scalar] v;
        c3 --[scalar] v;
    }; \ &= -6i\lambda v, &
    \feynmandiagram [inline=(v.base), small, horizontal=c2 to c3] {
        v[dot] --[scalar] c1;
        c2[] --[boson]v;
        c3[] --[boson]v;
    }; \ &= -2i\lambda v, \\
    \feynmandiagram [inline=(v.base), small, horizontal=c2 to c3] {
        c3 --[scalar] v[dot];
        v --[scalar] c0;
        v --[scalar] c1;
        c2 --[scalar] v;
    }; \ &= -6i\lambda, &
    \feynmandiagram [inline=(v.base), small, horizontal=c2 to c3] {
        v[dot] --[scalar] c0;
        c3[] --[boson]v;
        v --[scalar] c1;
        c2[] --[boson]v;
    }; \ &= -2i\lambda, \\
    \feynmandiagram [inline=(v.base), small, horizontal=c2 to c3] {
        c3[] --[boson]v[dot];
        v --[boson]c0[];
        v --[boson]c1[];
        c2[] --[boson]v;
    }; \ &\mathrlap{= -6i\lambda.}
\end{align*}
\end{document}