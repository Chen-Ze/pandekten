\documentclass[preview]{standalone}

\usepackage[compat=1.1.0]{tikz-feynman}
\usepackage{simpler-wick}
\usepackage{physics}

\usepackage{amsmath}
\usepackage{mathtools}

\usepackage{slashed}

\begin{document}
\abovedisplayskip=0pt
\begin{align*}
    \dv{}{\slashed{p}}\eval{\qty(\feynmandiagram [inline=(v.base), small, layered layout, horizontal=i to v] {
        i --[fermion] v[blob, label=1PI];
        v --[fermion] o;
    };)}_{\slashed{p} = m_\psi} \ &= 0, &
    \eval{\feynmandiagram [inline=(v.base), small, horizontal=i to v] {
        i --[scalar] v[blob];
        v --[fermion] c2;
        c1 --[fermion] v;
    };}_{\text{critical}} \ &= -ig, \\
    \dv{}{\slashed{p}}\eval{\qty(\feynmandiagram [inline=(v.base), small, layered layout, horizontal=i to v] {
        i --[scalar] v[blob, label=1PI];
        v --[scalar] o;
    };)}_{p^2 = m_\sigma^2} \ &= 0, &
    \feynmandiagram [inline=(v.base), small, layered layout, horizontal=v to o] {
        v[blob, label=1PI] --[scalar] o;
    };\ &= 0, \\
    \eval{\Im\qty(\feynmandiagram [inline=(v.base), small, horizontal=c2 to c3] {
        c3 --[scalar] v[blob];
        v --[scalar] c0;
        v --[scalar] c1;
        c2 --[scalar] v;
    };)}_{s=4m^2,s=t=0} \ &= 0.
\end{align*}
\end{document}

\begin{align*}
    \feynmandiagram [inline=(v.base), small, layered layout, horizontal=v to o] {
        v[crossed dot] --[scalar] o;
    }; && 
    \feynmandiagram [inline=(v.base), small, layered layout, horizontal=i to v] {
        i --[fermion] v[crossed dot];
        v --[fermion] o;
    }; && \\
    \feynmandiagram [inline=(v.base), small, layered layout, horizontal=i to v] {
        i --[scalar] v[crossed dot];
        v --[scalar] o;
    }; && 
    \feynmandiagram [inline=(v.base), small, layered layout, horizontal=i to v] {
        i --[boson] v[crossed dot];
        v --[boson] o;
    }; && \\
    \feynmandiagram [inline=(v.base), small, horizontal=i to v] {
        i --[scalar] v[crossed dot];
        v --[fermion] c2;
        c1 --[fermion] v;
    }; &&
    \feynmandiagram [inline=(v.base), small, horizontal=i to v] {
        i --[boson] v[crossed dot];
        v --[fermion] c2;
        c1 --[fermion] v;
    }; \\
    \feynmandiagram [inline=(v.base), small, horizontal=c2 to c3] {
        v[crossed dot] --[scalar] c1;
        c2 --[scalar] v;
        c3 --[scalar] v;
    }; &&
    \feynmandiagram [inline=(v.base), small, horizontal=c2 to c3] {
        v[crossed dot] --[scalar] c1;
        c2[] --[boson]v;
        c3[] --[boson]v;
    }; \\
    \feynmandiagram [inline=(v.base), small, horizontal=c2 to c3] {
        c3 --[scalar] v[crossed dot];
        v --[scalar] c0;
        v --[scalar] c1;
        c2 --[scalar] v;
    }; &&
    \feynmandiagram [inline=(v.base), small, horizontal=c2 to c3] {
        v[crossed dot] --[scalar] c0;
        c3[] --[boson]v;
        v --[scalar] c1;
        c2[] --[boson]v;
    }; \\
    \feynmandiagram [inline=(v.base), small, horizontal=c2 to c3] {
        c3[] --[boson]v[crossed dot];
        v --[boson]c0[];
        v --[boson]c1[];
        c2[] --[boson]v;
    }; &&
\end{align*}
