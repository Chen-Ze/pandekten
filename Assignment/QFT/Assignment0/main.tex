\documentclass{article}

\usepackage{pandekten}

\title{Problem Set 0}
\author{Ze Chen}

\begin{document}

\maketitle

I am taking this course to learn the application of quantum field theory in condensed matter physics. Specifically, I want to learn about the following.

\paragraph*{Axiomatic Treatment of QFT}
How is QFT defined (e.g. Wightman axioms).
How are canonical quantization (second quantization) related to functional integral, and why (if) these two formulations yield the same result.
Moreover, why is Berezin integral defined in such a counterintuitive way.

\paragraph*{Perturbation}
Why divergence occurs.
Does divergence occur in condensed matter systems.
What to do when there is nonzero tadpole diagrams (e.g. how do one evaluate the self-energy).
Moreover, minutiae like symmetry factors of diagrams and sign of fermion diagrams.

\paragraph*{Symmetry}
How exactly are various symmetries defined, e.g. inversion, time-reversal, charge conjugation, chiral.
What do they stand for (and what are the difference) in HEP and CMP.
What is gauge symmetry, and what are its applications in CMP.

\paragraph*{Renormalization}
Why are $\phi^4$ and QED renormalizable.
It's really surprising that the lowest order of $\gamma\gamma\rightarrow\gamma\gamma$ (1-loop) is convergent without any counterterm, after summing all permutations, each of which is divergent.

\paragraph*{Renormalization Group}
How are Wilson's renormalization and Callan-Symanzic equation related.
What are its application in condensed matter physics (beyond critical exponent).
As an example, how is RG used in the Kane \& Mele's QSHE paper to predict a lager gap in graphene.

\paragraph*{Relation to Statistical Physics}
How do one write down the path integral for $Z$ from the second-quantized Hamiltonian.
Why are the Matsubara function given by the second quantization and path integral formulation identical.

% \bibliographystyle{plain}
% \bibliography{main}

\end{document}
