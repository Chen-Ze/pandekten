\documentclass{article}

\usepackage{pandekten}
\usepackage{dashrule}

\makeatletter
\newcommand*{\shifttext}[1]{%
  \settowidth{\@tempdima}{#1}%
  \hspace{-\@tempdima}#1%
}
\newcommand{\plabel}[1]{%
\shifttext{\textbf{#1}\quad}%
}
\newcommand{\prule}{%
\begin{center}%
\hdashrule[0.5ex]{.99\linewidth}{1pt}{1pt 2.5pt}%
\end{center}%
}

\makeatother

\setlength{\parindent}{0pt}

\title{Assignment 1}
\author{Ze Chen}

\begin{document}

\maketitle

% \bibliographystyle{plain}
% \bibliography{main}

\plabel{1 (a)}%
The transformation is a counterclockwise rotation of $(x,y)$ by $\alpha$, i.e.
\[ \begin{pmatrix}
    x' \\ y'
\end{pmatrix} = \begin{pmatrix}
    \cos\alpha & -\sin\alpha \\ \sin\alpha & \cos\alpha
\end{pmatrix} \begin{pmatrix}
    x \\ y
\end{pmatrix}. \]
The Lagrangian is given by
\[ L = \frac{1}{2}\dot{\phi}^*(t)\dot{\phi}(t) - \frac{1}{2}m^2 \phi^*(t) \phi(t). \]
Then
\[ L[e^{i\alpha(t)}\phi(t)] - L[\phi(t)] = \frac{1}{2}i \dot{\alpha}(t)\qty(\dot{\phi}^*(t)\phi(t) - \phi^*(t)\dot{\phi}(t)) + \bigO(\alpha^2). \]
Therefore
\[ Q(t) = \frac{1}{2}i\qty(\dot{\phi}^*(t)\phi(t) - \phi^*(t)\dot{\phi}(t)). \]
Since
\[ \delta S = \int \dd{t} \qty[\dv{}{t}\qty(Q(t)\alpha(t)) - \dot{Q}(t)\alpha(t)] = - \int \dd{t} \dot{Q}(t)\alpha(t) = 0 \]
for arbitrary infinitesimal $\alpha$, $\dot{Q}(t) = 0$.
$Q$ is the angular momentum.

\plabel{(b)}%
Note that $p_\phi = \partial L / \partial \dot{\phi} = \dot{\phi}^*/2$ and $p_{\phi}^* = \dot{\phi}/2$.
Therefore,
\[ Q = i(\phi p_\phi - \phi^* p_\phi^*) = x p_y - y p_x. \]
Since $[\phi,p_\phi] = i$ and $[\phi^*, p^*_\phi] = i$,
\[ [Q,\phi] = \phi,\quad [Q,\phi^*] = -\phi^*. \]
From the BCH formula, $e^{-i\alpha Q} Y e^{i\alpha Q} = e^{-i\alpha s}Y$ if $[Q, Y] = sY$.
Therefore,
\[ e^{-i\alpha Q} \phi e^{i\alpha Q} = e^{-i\alpha} \phi,\quad e^{-i\alpha Q}\phi^* e^{i\alpha Q} = e^{i\alpha} \phi^*. \]

\prule

\plabel{2}%
The Lagrangian is given by
\[ L = -\frac{1}{4}\eta^{\mu\rho}\eta^{\nu\sigma}(\partial_\mu A_\nu - \partial_\nu A_\mu)(\partial_\rho A_\sigma - \partial_\sigma A_\rho). \]
Therefore,
\begin{align*}
    -\partial_\alpha \pdv{L}{(\partial_\alpha A_\beta)} &= \frac{1}{2} \partial_\alpha\qty[ \eta^{\mu\rho}\eta^{\nu\sigma} \qty(\delta^{\alpha\beta}_{[\mu\nu]}(\partial_\rho A_\sigma - \partial_\sigma A_\rho) + \delta^{\alpha\beta}_{[\rho\sigma]}(\partial_\mu A_\nu - \partial_\nu A_\mu)) ] \\
    &= \frac{1}{2} \partial_\alpha\qty[ \eta^{\rho[\alpha}\eta^{\beta]\sigma}(\partial_\rho A_\sigma - \partial_\sigma A_\rho) + \eta^{\mu[\alpha}\eta^{\beta]\nu}(\partial_\mu A_\nu - \partial_\nu A_\mu) ] \\
    &= \partial_\alpha\qty(\partial^\alpha A^\beta - \partial^\beta A^\alpha) \\
    &= \partial_\alpha F^{\alpha\beta} \\
    &= 0.
\end{align*}
$\beta = t$ gives
\[ \div \vb{E} = 0. \]
$\beta = (x,y,z)$ gives
\[ -\pdv{\vb{E}}{t} + \curl B = 0. \]

\prule

\plabel{3 (a)}%
$L = -m \sqrt{-\dot{x}^\mu \dot{x}_\mu}$ and therefore
\begin{align}
    \label{eq:eom_x} p_\mu &= \pdv{L}{\dot{x}^\mu} = \frac{m\dot{x}_\mu}{\sqrt{-\dot{x}^\mu \dot{x}_\mu}}, \\
    \notag p^2 &= \frac{m\dot{x}_\mu\dot{x}^\mu }{{-\dot{x}^\mu \dot{x}_\mu}} = -m^2.
\end{align}

\plabel{(b)}%
$L = \dot{x}^\mu p_\mu - \lambda (p^2+m^2)$ and therefore
\begin{align}
    \notag \dv{}{s}\pdv{L}{\dot{x}^\mu} - \pdv{L}{x^\mu} = \dv{}{s}p_\mu &= 0, \\
    \label{eq:eom_p}\dv{}{s}\pdv{L}{\dot{p}^\mu} - \pdv{L}{p^\mu} = -\dot{x}_\mu + 2\lambda p_\mu &= 0, \\
    \label{eq:pm0} \pdv{L}{\lambda} = p^2 + m^2 &= 0.
\end{align}

Combining \cref{eq:eom_p} and \cref{eq:pm0} we find
\[\dot{x}_\mu \dot{x}^\mu + 4\lambda^2 m^2 = 0 \Rightarrow 2\lambda = \frac{\sqrt{-\dot{x}_\mu \dot{x}^\mu}}{m}, \]
which eliminates $\lambda$ in \cref{eq:eom_p} and yields \cref{eq:eom_x}.

\plabel{(c)}%
The first term $\dot{x}^\mu p_\mu \dd{s}$ is clearly invariant.
The second term is made invariant by noting that
\[ \lambda(s) (p^2 + m^2) \dd{s} = \lambda'(s') (p^2 + m^2) \dd{s'} \]
with
\begin{equation}
    \label{eq:reparam} \lambda'(s') = \frac{\lambda(s) \dd{s}}{\dd{s'}}.
\end{equation}

\plabel{(d)}%
The reparametrization is given by
\[ s' = \int \lambda(s) \dd{s}, \]
which yields $\lambda'(s') = 1$ in \eqref{eq:reparam}.

\plabel{(e)}%
The translational symmetry $P_\alpha$ is generated by
\[ x^\mu \rightarrow x^\mu + t \delta^\mu_\alpha, \]
which yields the conserved quantity
\[ \pdv{L}{\dot{x}^\mu} \delta^\mu_\alpha = p_\alpha. \]
The rotational symmetry $L^{\alpha\beta}$ is generated by
\[ x^\mu \rightarrow x^\mu + \epsilon^{\alpha\beta\mu\nu} x_\nu, \]
which yields the conversed quantity
\[ \pdv{L}{\dot{x}^\mu} \epsilon^{\alpha\beta\mu\nu} x_\nu = \epsilon^{\alpha\beta\mu\nu}p_\mu x_\nu \doteq L^{\alpha\beta}. \]

\plabel{(f)}%
The Hilbert space consists of kets $\Set*{\ket{p}}{p \text{ is any 4-vector}}$.
\[ \bra{x}\ket{p} = \frac{1}{(2\pi)^3} e^{ip \cdot x} = \frac{1}{(2\pi)^3} e^{-iEt + i\vb{p}\cdot \vb{x}} \]
where $E^2 = \vb{p}^2 + m^2$.

\prule

\plabel{4 (a)}%
The momenta are given by
\[ \pi = \pdv{\mathcal{L}}{(\partial_t \phi)} = \partial_t \phi^*,\quad \pi^* = \pdv{\mathcal{L}}{(\partial_t \phi^*)} = \partial_t \phi. \]
The commutators are given by
\begin{align}
    \label{eq:canon1} [\phi(\vb{x}),\pi(\vb{y})] &= i\delta^{(3)}(\vb{x} - \vb{y}), \\
    \label{eq:canon2} [\phi^*(\vb{x}),\pi^*(\vb{y})] &= i\delta^{(3)}(\vb{x} - \vb{y}), \\
    \label{eq:canon3} [\pi(\vb{x}),\pi(\vb{y})] &= [\phi(\vb{x}),\phi(\vb{y})] = [\pi^*(\vb{x}),\pi^*(\vb{y})] = [\phi^*(\vb{x}),\phi^*(\vb{y})] = 0.
\end{align}
The Hamiltonian is given by
\begin{align*}
    H &= \int\dd{^3 \vb{x}} \qty(\pi\partial_t\phi + \pi^*\partial_t\phi^* - \partial_\mu \phi^* \partial^\mu \phi + m^2 \phi^*\phi) \\
    &= \int\dd{^3 \vb{x}} \qty(\pi^* \pi + \grad\phi^* \cdot \grad\phi + m^2 \phi^*\phi).
\end{align*}
Finally the equations of motion are given by
\begin{align*}
    \partial_t \phi(\vb{x}) &= i[H,\phi(\vb{x})] \\
    &= i\int \dd{^3\vb{x}'}\qty[\pi^*(\vb{x}')\pi(\vb{x}'), \phi(\vb{x})] \\
    &= \int \dd{^3\vb{x}'} \pi^*(\vb{x}')\delta^{(3)}(\vb{x} - \vb{x}') \\
    &= \pi^*(\vb{x}). \\
    \partial_t \pi^*(\vb{x}) &= i[H,\pi(\vb{x})] \\
    &= i\int \dd{^3\vb{x}'}\qty[\grad\phi^* \cdot \grad\phi + m^2 \phi^*\phi, \pi^*(\vb{x})] \\
    &= -\int\dd{^3\vb{x}'}\qty[\grad \delta^{(3)}(\vb{x}' - \vb{x}) \cdot \grad\phi(\vb{x}') + m^2 \delta^{(3)}(\vb{x}' - \vb{x}) \phi(\vb{x}')] \\
    &= \grad^2 \phi(\vb{x}) - m^2 \phi(\vb{x}).
\end{align*}
Combining these we get
\[ \partial_t^2 \phi = \grad^2\phi - m^2\phi. \]

\plabel{(b)}%
\newcommand{\removethis}[1]{}%
\removethis{Writing
\[ \phi_1 = \frac{\phi + \phi^*}{2},\quad \phi_2 = \frac{\phi - \phi^*}{2i}, \]
we find their momenta
\[ \pi_1 = \pi + \pi^*,\quad \pi_2 = i(\pi - \pi^*), \]
and the commutators
\begin{align*}
    [\phi_i(\vb{x}), \pi_j(\vb{y})] &= i \delta_{ij}\delta^{(3)}(\vb{x} - \vb{y}), \\
    [\phi_i(\vb{x}), \phi_j(\vb{y})] &= [\pi_i(\vb{x}), \pi_j(\vb{y})] = 0.
\end{align*}
The Hamiltonian may be rewritten as
\[ H = \int \dd{^3 x} \qty(\frac{\pi_1^2}{4} + \frac{\pi_2^2}{4} + \abs{\grad \phi_1}^2 + \abs{\grad \phi_2}^2 + m^2(\phi_1^2 + \phi_2^2)), \]
which may be decoupled as two independent Hamiltonians
\[ H = H_1 + H_2 \]
where
\[ H_i = \int \dd{^3 x} \qty(\frac{\pi_i^2}{4} + \abs{\grad \phi_i}^2 + m^2\phi_i^2). \]
With
\begin{align*}
    \phi_i(\vb{x}) &= \int \frac{\dd{^3\vb{p}}}{(2\pi)^3} \frac{1}{\sqrt{2\omega_{\vb{p}}}} \qty(a_{i,\vb{p}} e^{i\vb{p}\cdot \vb{x}} + a^\dagger_{i,\vb{p}} e^{-i\vb{p}\cdot \vb{x}}), \\
    \pi(\vb{x}) &= \int \frac{\dd{^3\vb{p}}}{(2\pi)^3} (-i) \sqrt{\frac{\omega_{\vb{p}}}{2}} \qty(a_{i,\vb{p}} e^{i\vb{p}\cdot \vb{x}} - a^\dagger_{i,\vb{p}} e^{-i\vb{p}\cdot \vb{x}}),
\end{align*}}%
Writing $\phi$ in the following form that solves the Klein-Gordon equation
\begin{align*}
    \phi(x) &= \int \frac{\dd{^3 \vb{p}}}{(2\pi)^3} \frac{1}{\sqrt{2\omega_{\vb{p}}}}\qty(a_{\vb{p}} e^{-ipx} + b^\dagger_{\vb{p}} e^{ipx}), \\
    \phi^*(x) &= \int \frac{\dd{^3 \vb{p}}}{(2\pi)^3} \frac{1}{\sqrt{2\omega_{\vb{p}}}}\qty(b_{\vb{p}} e^{-ipx} + a^\dagger_{\vb{p}} e^{ipx}),
\end{align*}
we find
\begin{align*}
    \pi(x) &= \int \frac{\dd{^3 \vb{p}}}{(2\pi)^3} (-i) \sqrt{\frac{\omega_{\vb{p}}}{2}} \qty(b_{\vb{p}} e^{-ipx} - a^\dagger_{\vb{p}} e^{ipx}), \\
    \pi^*(x) &= \int \frac{\dd{^3 \vb{p}}}{(2\pi)^3} (-i) \sqrt{\frac{\omega_{\vb{p}}}{2}} \qty(a_{\vb{p}} e^{-ipx} - b^\dagger_{\vb{p}} e^{ipx}).
\end{align*}

Canonical commutators \cref{eq:canon1,eq:canon2,eq:canon3} may be retrieved from
\begin{align*}
    [a_{\vb{p}}, a^\dagger_{\vb{p}'}] &= (2\pi)^3 \delta^{(3)}(\vb{p} - \vb{p}'), \\
    [b_{\vb{p}}, b^\dagger_{\vb{p}'}] &= (2\pi)^3 \delta^{(3)}(\vb{p} - \vb{p}'), \\
    [a_{\vb{p}}, a_{\vb{p}'}] &= [a_{\vb{p}}, b_{\vb{p}'}] = [a_{\vb{p}}, b^\dagger_{\vb{p}'}] = \cdots = 0.
\end{align*}
For example,
\begin{align*}
    [\phi(\vb{x}),\pi(\vb{y})] &= \frac{i}{2} \int \frac{\dd{^3\vb{p}}}{(2\pi)^3} \int \frac{\dd{^3\vb{k}}}{(2\pi)^3} \sqrt{\frac{\omega_{\vb{k}}}{\omega_{\vb{p}}}} \qty([a_{\vb{p}}, a^\dagger_{\vb{k}}] e^{i(\vb{p}\cdot \vb{x} - \vb{k}\cdot \vb{y})} + [b_{\vb{k}},b^\dagger_{\vb{p}}]e^{i(\vb{k}\cdot \vb{y}-\vb{p}\cdot \vb{x})}) \\
    &= \frac{i}{2} \int \frac{\dd{^3\vb{q}}}{(2\pi)^3} 2 e^{i\vb{q}\cdot (\vb{x} - \vb{y})} \\
    &= i \delta^{(3)}(\vb{x} - \vb{y}).
\end{align*}

Note that
\begin{equation}
    \label{eq:ftconvolution} \int \dd{^3 \vb{x}} f(\vb{x}) g(\vb{x}) = \int \frac{\dd{^3\vb{k}}}{(2\pi)^3} \hat{f}(\vb{k}) \hat{g}(-\vb{k})
\end{equation}
where $\hat{\varphi}$ denotes Fourier transform of $\varphi$.
The Hamiltonian is rewritten as
\begin{align*}
    H &= \int \frac{\dd{^3\vb{q}}}{(2\pi)^3} \qty[\pi^*(\vb{q}) \pi(-\vb{q}) + (\vb{q}^2 + m^2)\phi^*(\vb{q})\phi(-\vb{q})] \\
    &= \int \frac{\dd{^3\vb{q}}}{(2\pi)^3} \qty[\frac{\omega_{\vb{q}}}{2}\qty(a_{-\vb{q}}a^\dagger_{-\vb{q}} + b^\dagger_{\vb{q}}b_{\vb{q}}) + \frac{\vb{q}^2+m^2}{2\omega_{\vb{q}}}(a_{\vb{q}}^\dagger a_{\vb{q}} + b_{-\vb{q}}b_{-\vb{q}}^\dagger)] \\
    &= \int \frac{\dd{^3\vb{q}}}{(2\pi)^3} \omega_{\vb{q}}\qty(a_{\vb{q}}^\dagger a_{\vb{q}} + \frac{1}{2}[a_{\vb{q}}, a_{\vb{q}}^\dagger] + b_{\vb{q}}^\dagger b_{\vb{q}} + \frac{1}{2}[b_{\vb{q}}, b_{\vb{q}}^\dagger]) \\
    &= \int \frac{\dd{^3\vb{q}}}{(2\pi)^3} \omega_{\vb{q}}\qty(a_{\vb{q}}^\dagger a_{\vb{q}} + b_{\vb{q}}^\dagger b_{\vb{q}}) + \mathrm{const}.
\end{align*}
Therefore, the theory contains two sets of particles, both of which have dispersion given by $\omega_{\vb{p}} = \sqrt{\vb{p}^2 + m^2}$.

\plabel{(c)}%
Using \eqref{eq:ftconvolution} we find
\begin{align*}
    Q &=  \frac{i}{2} \int \frac{\dd{^3 \vb{q}}}{(2\pi)^3} \qty(\phi^*(\vb{q})\pi^*(-\vb{q}) - \pi(\vb{q})\phi(-\vb{q})) \\
    &= \frac{1}{4} \int \frac{\dd{^3 \vb{q}}}{(2\pi)^3}\qty((a^\dagger_{\vb{q}} + b_{-\vb{q}})(-b^\dagger_{-\vb{q}} + a_{\vb{q}}) - (b_{-\vb{q}} - a^\dagger_{\vb{q}})(a_{\vb{q}} + b^\dagger_{-\vb{q}})) \\
    &= \frac{1}{2} \int \frac{\dd{^3 \vb{q}}}{(2\pi)^3} (a^\dagger_{\vb{q}}a_{\vb{q}} - b_{\vb{q}}b^\dagger_{\vb{q}}) \\
    &= \frac{1}{2} \int \frac{\dd{^3 \vb{q}}}{(2\pi)^3} (a^\dagger_{\vb{q}}a_{\vb{q}} - b^\dagger_{\vb{q}}b_{\vb{q}}) + \mathrm{const}.
\end{align*}

\plabel{(d)}%
Let $X^i$ denotes the generators of $\operatorname{U}(n)$.
Then $\qty(X^i)^\dagger = X^i$.
The infinitesimal transformations are given by
\begin{align*}
    \Delta\phi_a &= i\qty(X^i)_{ab} \phi_b, \\
    \Delta\phi^*_a &= -i\qty(X^i)^*_{ab} \phi^*_b,
\end{align*}
which gives rise to the conserved charge
\begin{align*}
    Q^i &= -\frac{i}{2} \int \dd{^3\vb{x}} \qty(\pi_a \qty(X^i)_{ab} \phi_b - \pi^*_a \qty(X^i)^*_{ab} \phi^*_b) \\
    &= \frac{i}{2} \int \dd{^3\vb{x}} \qty(\phi^*_a \qty(X^i)_{ab} \pi^*_b - \pi_a \qty(X^i)_{ab} \phi_b).
\end{align*}
In particular, for the case of two scalar fields, $X^i = \sigma^i$.
Taking $(X^i)_{ab} = (\sigma^0)_{ab} = \delta_{ab}$ yields
\[ Q = \frac{i}{2} \int \dd{^3\vb{x}} \qty(\phi^*_a \pi^*_a - \pi_a \phi_a). \]
The commutation relations are given by
\begin{align*}
    &{\phantom{{}={}}} [Q^i,Q^j] \\
    &= -\frac{1}{4} \qty[\int \dd{^3\vb{x}} \qty(\phi^*_a \qty(X^i)_{ab} \pi^*_b - \pi_a \qty(X^i)_{ab} \phi_b), \int \dd{^3\vb{x}} \qty(\phi^*_a \qty(X^j)_{ab} \pi^*_b - \pi_a \qty(X^j)_{ab} \phi_b)] \\
    &= -\frac{1}{4} \int \dd{^3\vb{x}} \int \dd{^3\vb{y}} \qty(X^i)_{ab} \qty(X^j)_{cd}\qty([\phi^*_a \pi^*_b, \phi^*_c \pi^*_d] + [\pi_a \phi_b, \pi_c \phi_d]) \\
    &= \frac{i}{4} \int \dd{^3 \vb{x}} \qty(X^i)_{ab} \qty(X^j)_{cd}\qty(\delta_{bc}\phi^*_a \pi^*_d - \delta_{ad}\phi^*_c \pi^*_b - \delta_{bc} \pi_a \phi_d + \delta_{ad}\pi_c\phi_b) \\
    &= \frac{i}{4} \int \dd{^3\vb{x}} \qty(\phi^*_a\qty(\qty(f^{ij})_{k}X^k)_{ab}\pi^*_b - \pi_a\qty(\qty(f^{ij})_{k}X^k)_{ab}\phi_b) \\
    &= \frac{1}{2} \qty(f^{ij})_k Q^k,
\end{align*}
where $(f^{ij})_k$ are the structure constants.
In particular, for $X^i = \sigma^i$, we have $(f^{ij})_k = 2i{\epsilon^{ij}}_k$, and therefore
\[ [Q^i,Q^j] = i{\epsilon^{ij}}_k Q^k. \]

\plabel{(e)}%
It's well known that $\operatorname{U}(1) \cong \operatorname{SO}(2)$ and $\operatorname{O}(2)$ is generated by $\operatorname{SO}(2)$ and a reflection, i.e. $\phi \rightarrow \phi^*$. Since $\phi \rightarrow \phi^*$ and $\pi \rightarrow \pi^*$, it's clear that $Q\rightarrow -Q$ under this symmetry.

\plabel{(f)}%
Let $m_\phi \neq m_\chi$.
\[ \mathcal{L} = \partial_\mu \phi^* \partial^\mu \phi - m_\phi^2 \phi^* \phi + \partial_\mu \chi^* \partial^\mu \chi - m_\chi^2 \chi^* \chi + ig(\phi^*\chi - \phi\chi^*). \]

\prule

\plabel{5}%
It's easy to verify that
\[ \varphi_n(x) = \sqrt{\frac{2}{a}} \sin\qty(\frac{n\pi x}{a}) \]
is an orthonormal basis with respect to the inner product
\[ \langle f,g \rangle = \int_0^a \dd{x} f(x)g(x). \]
With the Parseval's identity we find
\begin{align*}
    L &= \int_0^a \dd{x} \qty[\frac{\sigma}{2} \qty(\sum_n \varphi_n(x) \dot{q}_n(t))^2 - \frac{T}{2} \qty(\sum_n \qty(\frac{n\pi}{a})^2 \varphi_n(x) q_n(t))^2] \\
    &= \sum_n \qty[\frac{\sigma}{2} \dot{q}_n^2(t) - \frac{T}{2}\qty(\frac{n\pi}{a})^2 q^2_n(t)].
\end{align*}
$L$ is clearly decoupled.
\[ \dv{}{t}\pdv{L}{\dot{q}_n} - \pdv{L}{q_n} = \sigma \ddot{q}_n(t) + T \qty(\frac{n\pi}{a})^2 q_n(t) = 0, \]
whence
\[ \omega_n = \frac{n\pi}{a}\sqrt{\frac{T}{\sigma}}. \]

\end{document}
