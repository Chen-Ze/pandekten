\documentclass{article}

\usepackage{pandekten}

\title{Spin Liquid}
\author{Ch\=an Taku}

\begin{document}

\maketitle

\section{Lattice Gauge Theory}

\begin{definition}{Lattice for Gauge Theory}{lattice_for_gauge_theory}
    A lattice for gauge theory is a directed graph $(E,V)$ that satisfies the following conditions:
    \begin{itemize}
        \item Every pair of vertices is either not connected or connected by a single edge.
    \end{itemize}
    The starting vertex and ending vertex of an edge (link) $\ell$ is denoted by $V_{\bullet\rightarrow}(\ell)$ and $V_{\rightarrow\bullet}(\ell)$ respectively.
\end{definition}

\begin{definition}{Lattice Gauge Theory, Configuration, Gauge Transformation}{lattice_gauge_theory}
    A lattice gauge theory contains the following data.
    \begin{itemize}
        \item A lattice for gauge theory $(E,V)$.
        \item A Hilbert space $\mathcal{H}_{\mathrm{site}}$.
        \item A group $G$, called the gauge group.
        \item A faithful unitary representation
        \[ \rho: G \rightarrow \operatorname{Aut}(\mathcal{H}_{\mathrm{site}}). \]
    \end{itemize}
    A configuration $(U,\varphi)\in G^E \times \mathcal{H}_{\mathrm{site}}^V$ is a pair of maps
    \[ U: E \rightarrow G \]
    and
    \[ \varphi: V \rightarrow \mathcal{H}_{\mathrm{site}}. \]
    The group of gauge transformations is $G^V$.
    An element $h\in G^V$, $h: V\rightarrow G$, acts on configurations $(U,\varphi)$ by
    \[ \operatorname{Ad}_h(U)(\ell) = h(V_{\bullet\rightarrow}(\ell)) U(\ell) h(V_{\rightarrow\bullet}(\ell))^{-1} \]
    and
    \[ h(\varphi)(v) = h(v)\varphi(v). \]
\end{definition}

\paragraph*{Quantization of Connections}
The phase space of the gauge configuration space is given by
\[ \mathscr{P}_E = \prod_{\ell \in E} T^* G = \prod_{\ell\in E} (G\times \mathfrak{g}^*). \]
For each $X\in \mathfrak{g}$ and $f\in C^\infty(G)$, the quantization yields a commutation relation
\[ [P_X, T_f] = i T_{X^R(f)}, \]
where $X^R$ denotes the right-invariant vector field.

\begin{example}{Quantization of $\operatorname{U}(1)$}{quantization_of_u_1}
    With $G = \operatorname{U}(1)$ and $X = \partial_\theta$ we find
    \[ [P_X,T_f] = i T_{\partial_\theta f}. \]
    $P_X$ is denoted by $E$ and stands for electric field.
    If we abuse the notation and denote $A = T_\theta$ then we recover
    \[ [E,A] = i. \]
    Loosely speaking, $A = \theta$ takes value in $[0,2\pi)$ and $E = \partial_\theta$ takes value in $i\mathbb{Z}$.
\end{example}

\begin{example}{Quantization of $\mathbb{Z}_2$}{quantization_of_z_2}
    For $G = \mathbb{Z}_2$ the procedure above does not apply.
    Note that We may compare the $\operatorname{U}(1)$ case to quantizing a particle on a ring, and infer that quantizing $\operatorname{Z}_2$ is similar to quantizing an Ising spin.
    \par
    In the $\operatorname{U}(1)$ case, the gauge transformation on the site is generated by
    \[ e^{iq(v)\theta} = \exp{i \theta \sum_{\ell \in \operatorname{star}(v)} E(\ell)}. \]
    Therefore, the corresponding gauge transformation on links is
    \[ \exp{-i \theta \sum_{\ell \in \operatorname{star}(v)} E(\ell)} = \prod_{\ell\in\operatorname{star}(v)} U^{-1}(\theta) \]
    since $E = P_X$ generates left multiplication on $G$.
    This transformation acts on the star.
    For the $\mathbb{Z}_2$ case, going in the reversed direction, we set $U = \sigma_x$, and find that the gauge transformation on links is given by
    \[ \prod_{\ell\in \operatorname{star}(v)} \sigma_x(\ell). \]
    The site transformation is replaced with
    \[ e^{i q(v)\theta} \rightarrow \prod_{\ell\in \operatorname{star}(v)} \sigma_x(\ell) = \pm 1. \]
    Now the charge is only a $\operatorname{Z}_2$ value.
\end{example}

After the quantization we have the following Gau\ss's law:
\[ \div E = \sum_{\ell \in \operatorname{star}(v)} E(\ell) = q(v). \]
The flux is defined by
\[ \prod_{\ell \in \Box} U(\ell) \longrightarrow \prod_{\ell \in \Box} A(\ell). \]

\paragraph*{$\operatorname{U}(1)$-Gauge Theory Phases}
There is a phase transition in $d=3$ (spatial dimension) for the $\operatorname{U}(1)$ gauge theory from the confining phase to the deconfining phase.
However, in $d=2$ there is only confining phase.

\paragraph*{$\mathbb{Z}_2$-Gauge Theory Phases}
There is a phase transition in $d=2$ from the confining phase to the deconfining phase.

\par
It should be noted that gauge transformations on one lattice site act on $2d$ links (star operator).
In the continuum case, $A \rightarrow A + \grad \Lambda$ changes all four components, i.e. values on all four bonds.

\paragraph*{$\mathbb{Z}_2$-Charge}
The physical space is given by those gauge invariant ones, i.e. those satisfying
\[ Q(\vb{r}) \ket{\varphi} = \ket{\varphi}, \]
where $Q(\vb{r})$ is the star operator, i.e. the gauge transformation.
If for some value of $\vb{r}$, $Q(\vb{r})\ket{\varphi} = -\ket{\varphi}$, then there is a $\mathbb{Z}_2$ source at $\vb{r}$.

\paragraph*{$\operatorname{U}(1)$-Charge}
The charge opreator is defined by
\[ Q(\vb{r}) = \sum_j \Delta_j E_j. \]
The gauge transformation is then given by
\[ \exp(i\sum_{\vb{r}}\theta(\vb{r}) Q(\vb{r})). \]
The physical space is given by
\[ Q(\vb{r})\ket{\varphi} = \sum_j \Delta_j E_j \ket{\varphi} = 0. \]
A state that satisfy
\[ Q(\vb{r})\ket{\psi} = n(\vb{r}) \ket{\psi} \]
has sources that carry charge $Q(\vb{r})$.

\begin{example}{Gau\ss's Law of $\mathbb{Z}_2$}
    If matter field is taken into account, then the condition of the physical space is no longer $Q(\vb{r}) \ket{\varphi} = \ket{\varphi}$.
    Either
    \begin{itemize}
        \item the definition of $Q(\vb{r})$ has to be modified to include the gauge transformation of the matter field, or
        \item the Gau\ss's law should be modified to
        \[ Q(\vb{r}) \ket{\text{Phys}} = \ket{\text{Phys}}. \]
    \end{itemize}
    With either solution, the Gau\ss's law should be modified to
    \[ \qty(\text{star}) \qty(\text{site transform}) \ket{\text{Phys}} = \ket{\text{Phys}}. \]
    In particular, if the site has $n$ particle occupation, then
    \[ \qty(\text{star}) \ket{\text{Phys}} = (-1)^{n(\vb{r})} \ket{\text{Phys}}. \]
\end{example}

\paragraph*{Loop Condensation vs RVB}
Two states in the RVB as overlap that forms closed loop.

% \bibliographystyle{plain}
% \bibliography{main}

\end{document}
