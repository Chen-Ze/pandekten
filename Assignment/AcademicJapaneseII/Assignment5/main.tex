\documentclass{ctexart}

\usepackage{pandekten}
\usepackage{dashrule}

\makeatletter
\newcommand*{\shifttext}[1]{%
  \settowidth{\@tempdima}{#1}%
  \hspace{-\@tempdima}#1%
}
\newcommand{\plabel}[1]{%
\shifttext{\textbf{#1}\quad}%
}
\newcommand{\prule}{%
\begin{center}%
\hdashrule[0.5ex]{.99\linewidth}{1pt}{1pt 2.5pt}%
\end{center}%
}

\makeatother

\newcommand{\minusbaseline}{\abovedisplayskip=0pt\abovedisplayshortskip=0pt~\vspace*{-\baselineskip}}%

\setlength{\parindent}{0pt}

\title{Assignment 5}
\author{Ze (Zack) Chen}

\begin{document}

\maketitle

\plabel{$\blacktriangleright$}%
かくの如き變化の結果、\\
As a result of such change,\\
宰相の位置は天子を輔佐するものでなくして、\\
the position of the Chancellor of State is not the one that assists the Son of Heaven, and\\
殆ど祕書官同樣となりたるが、\\
had become almost the same as a secretary, but\\
尚宋代にては唐代の遺風も存在して、\\
still in the Song dynasty the legacy of the Tang dynasty exists too, and\\
宰相は相當の權力を有したるも、\\
the Chancellor of State had had considerable power, but\\
明以後には全く宰相の官を置かぬ事になり、\\
from the Ming dynasty on, it becomes that the official position of the Chancellor of State is not arranged, and\\
事實宰相の仕事をとれるものは殿閣大學士であつて、\\
as a matter of fact, the one that takes the job of the Chancellor of State is the Grand Secretariat, and\\
これは官職の性質としては天子の祕書役、代筆の役で、\\
as for this, as a property of the position it is the secretary of the Son of Heaven, and is the role of ghostwriter, and\\
天子を輔佐し、\\
assists the Son of Heaven, and\\
其責任を分ち、\\
shares his responsibility,\\
若くは責任を全く負擔する古代の宰相の俤はなくなり、\\
or it becomes not that vestige of the Chancellor of State of older ages that completely takes the responsibility, and\\
君權のみが無限に發達した。\\
only the power of the monarch developed infinitely.

\vspace{1em}
\plabel{$\blacktriangleright$}%
唐の時の宰相は、\\
The Chancellor of State of the time of Tang\\
皆貴族階級の中より出で、
all came from the middle of the rank of aristocrats, and\\
一度其位置に到ると、\\
when once it arrives at its position,\\
天子と雖も其權力を自由に動かす事が出來ない習慣であつたが、\\
despite that for the Son of Heaven, there is the custom that it is impossible to freely wield his power, but\\
明以後は如何に強大なる權力を有する宰相でも、\\
from the Ming dynasty on, even if one is a Chancellor of State that has any kind of strong power,\\
天子の機嫌を損ねると、\\
if he hurts the feeling of the Son of the Heaven,\\
忽ち廢黜せられ、\\
he is immediately dismissed, and\\
一個の平民とせられ、\\
is turned into a plebian, or\\
囚人と墜さるゝ。\\
is dropped to a prisoner.

\vspace{1em}
\plabel{$\blacktriangleright$}%
宋代は恰も唐と明清との間に立つので、\\
Since the Song dynasty lies precisely between the Tang and Ming-Qing dynasties,\\
明清の如く宰相に權力がないといふ譯ではないが、\\
it is not the circumstance that to the Chancellor of State there is no power as in the Ming and Qing dynasties, but\\
天子の權力を笠に被て居る間は全盛を極めても、\\
the space that has been covering the power of the son of Heaven by the umbrella carries to the full prosperity, but,\\
天子の背景を失ふと忽ち一匹夫となる。\\
if it lost the background of the Son of Heaven, it becomes immediately a ordinary man.

\vspace{1em}
\plabel{$\blacktriangleright$}%
宋の寇準\textperiodcentered 丁謂、南宋の賈似道などの境遇の變化を見ても分るのである。\\
It is that one can also understand by seeing the change of the situation of Kou Zhun and Ding Wei of the Song dynasty, Jia Sidao of the Southern Song, etc.

\vspace{1em}
\plabel{$\blacktriangleright$}%
地方官なども、唐代には、中央の權力と關係して、\\
Even local officials, etc., in the Tang dynasty, have relation to the power of the central power, and\\
各地方に於て、殆ど君主同樣の權力を有するもの多き習慣なりしが、\\
as for every each region, there was custom that the thing that they have almost the same power as the monarch is much, but\\
宋以後は、如何なるよき位置の地方官も、\\
from the Song dynasty on, it becomes that even for the local officials of any kind of good positions,\\
君主一片の命令で容易に交迭せらるゝ事となつた。\\
they are easily replaced with a single command of the monarch.

\vspace{1em}
\plabel{$\blacktriangleright$}%
宦官は天子の從僕であるが、\\
Eunuchs are servants of the Son of Heaven, but\\
唐代の宦官は天子の眷族の有力なる部分となつて、\\
in the Tang Dynasty, eunuchs became a powerful faction of the family of the Son of Heaven, and\\
定策國老門生天子といふ諺さへ出來たが、\\
even the proverb ``the eunuchs are the ones who make the policies, and the Son of Heaven is their disciple'' was made,\\
後に明の時にも宦官が跋扈せるも、\\
afterwards in the time of Ming, the eunuchs also had been rampant, but,\\
天子の恩寵あるときにのみ權力があつて、\\
the power exists only at the time when there is favor from the Son of Heaven,\\
恩寵が衰へると其勢力は全くなくなる。\\
if such favor faded out, their influence become totally non-existent.

\vspace{1em}
\plabel{$\blacktriangleright$}%
唐と明との宦官にかくの如き相違あるは、\\
That the eunuchs of Tang and Ming has such difference is\\
即ち貴族政治と君主獨裁政治との相違ある結果である。\\
namely the result of that the difference exists between aristocracy and dictatorship of the monarch.

\prule

\plabel{$\blacktriangleright$}%
それと同時に人民の地位も著しく變化して來た。\\
At the same time of this the status of the people also started to change significantly.

\vspace{1em}
\plabel{$\blacktriangleright$}%
元來法治國とは違ひ、\\
Originally it differs from a country of rule of law, and\\
人民の權力を明らかに認める事はないけれども、\\
it is not that it explicitly admits the power of people, but,\\
人民の地位と財産上の私權とは、\\
the status of the people and their rights on properties\\
貴族政治時代と大に趣を異にするやうになつた。\\
become the way that differs significantly in general with the age of aristocracy.

\vspace{1em}
\plabel{$\blacktriangleright$}%
貴族時代には人民は貴族全體の奴隷の如く視られしが、\\
In the age of aristocracy, people were regarded as the slaves of the aristocrat, but\\
隋唐の代となり、\\
in the age of Sui and Tang,\\
人民を貴族の手から解放して國家の直轄とし、\\
(it) freed people from the hand of aristocrat and the country directly reigned over the people, and\\
殊に農民を國家の小作人の如く取扱ふ制度が作られたが、\\
in particular, the system was created that treats farmers as the tenants of the country, but\\
事實は政治の權力は貴族にあつたから、\\
in reality, since the power of politics was in the aristocrat,\\
君主を擁したる貴族團體の小作人といふ状態であつた。\\
the people are in thet state that is called the tenants of the aristocrat that supported the monarch.

\vspace{1em}
\plabel{$\blacktriangleright$}%
土地の分配制度なども、かくの如き意義と密接の關係があり、\\
Also, the close relation exists between the system of land distribution, etc., and such significance, and\\
殊に租税の性質は其意義を尤もよく現して居る。\\
in particular, the nature of taxation is showing this significance the best.

\vspace{1em}
\plabel{$\blacktriangleright$}%
即ち唐代の租\textperiodcentered 庸\textperiodcentered 調の制度は、人民は政府に對して地代を納め、力役に服し、工作品を提供する意味のものであつた。\\
Namely, the system of Zu Yong Diao in the Tang Dynasty was the one of the meaning that people to the government pay land rent, perform labor serives, and provide goods.

\vspace{1em}
\plabel{$\blacktriangleright$}%
唐の中世から此制度自然に壞れて兩税制度となり、\\
From the middle of the Tang Dynasty, this system naturally breaks down and become the two-tax system, and\\
人民の居住が制度上自由に解放さるゝことゝなり、\\
it becomes that residency of the people was liberated freely on the system, and\\
地租などの收納も錢で代納することゝなつたので、\\
since it became that collection of land rent, etc., also pays by money,\\
人民は土地に拘束せらるゝ奴隷小作人たる位置から、自然に解放さるゝ端緒を開きしが、\\
the starting point opened that people were liberated naturally from the position that they are the slave labors bound to the land, and\\
宋代に至り、王安石の新法によりて、\\
entering the Song dynasty, by means of Wang Anshi's New Policies,\\
人民の土地所有の意味が益々確實になつて來た。\\
the meaning of people's land ownership started to become increasingly concrete.

\vspace{1em}
\plabel{$\blacktriangleright$}%
青苗錢の如き低利資金融通法も、人民が土地の收穫を自由に處分する事を認める意味とも解さるゝ。\\
Low-interest loans like Qing Miao Qian can also be understood as the meaning that it admits that people freely dispose the land harvests.


\vspace{1em}
\plabel{$\blacktriangleright$}%
又從來の差役を改めて雇役とし、\\
Moreover it transformed the long-lasting compulsory labor to become hired labor, and\\
隨分反對者の攻撃を受けたが、\\
considerably received attack of the opposition, but\\
此雇役制度は尤も當時の事情に適せるを以て、\\
since this system of hired labor had fitted best the circumstances at that time,\\
後に司馬光が王安石の新法を改めた時に、\\
later at the time when Sima Guang changed Wang Anshi's new Policies,\\
新法反對論者の中にも、\\
even in the middle of the opponents of the new Policies,\\
蘇東坡始め、\\
starting with Su Dongpo,\\
差役を復舊することはこれを否なりとした人が多い。\\
as for reverting to compulsory labor, people are many that took this as negation.

\vspace{1em}
\plabel{$\blacktriangleright$}%
支那は人民の參政權を認むるといふことは全くなかりしも、\\
In China, that it recognizes the people's rights of participation in politics was totally non-existent, but\\
貴族の階級を消滅せしめて、\\
after it makes the class of aristocrat disappear,\\
君主と人民と直接に相對するやうになつたのは、\\
that it became that the monarch and the people directly interact,\\
即ち近世的政治の状態となつたのである。\\
namely, is that it became the status of the politics of early modern age.

\end{document}
