\documentclass{ctexart}

\usepackage{pandekten}
\usepackage{dashrule}

\makeatletter
\newcommand*{\shifttext}[1]{%
  \settowidth{\@tempdima}{#1}%
  \hspace{-\@tempdima}#1%
}
\newcommand{\plabel}[1]{%
\shifttext{\textbf{#1}\quad}%
}
\newcommand{\prule}{%
\begin{center}%
\hdashrule[0.5ex]{.99\linewidth}{1pt}{1pt 2.5pt}%
\end{center}%
}

\makeatother

\newcommand{\minusbaseline}{\abovedisplayskip=0pt\abovedisplayshortskip=0pt~\vspace*{-\baselineskip}}%

\setlength{\parindent}{0pt}

\title{Final}
\author{Ze (Zack) Chen}

\begin{document}

\maketitle

\plabel{$\blacktriangleright$}%
陽明先生日く、\\
Yangming once said that,\\
涵養を專にする者は日に其の足らざるを見、\\
those who focus solely on cultivation see their shortcomings daily, while\\
識見を専らにする者は日に其の餘あるを見る、\\
those who focus solely on knowledge see their surplus daily;\\
日に足らざる者は日に餘りあり矣、\\
those who has shortcomings daily will eventually have surplus, while\\
日に餘ある者は日に足らず矣と。\\
those who has surplus daily will eventually fall short.

\vspace{1em}
\plabel{$\blacktriangleright$}%
惟だ今日の弊、\\
However, today's shortcomings\\
亦然らずや、\\
are not the same, and\\
勝心世に處す、\\
pride is in the world;\\
故に人に尙ぶるを以て自ら慰す、\\
therefore people comfort themselves by looking up to others,\\
而して實功一の擧るなし、\\
but they achieve no real accomplishments;\\
孰れか天下の愚物として、\\
who would, as a fool in the world,\\
自ら天下に先て勞苦する者ぞ、\\
be the one who put in the effort before others?\\
請ふ其の驥尾に附て而して進まん。\\
better follow their tail and then move forward.

\prule
\plabel{$\blacktriangleright$}%
一口に清朝の畫と云っても、\\
In summary, even though we talked about the paintings of the Qing Dynasty,\\
前來述べ來ったやうな關係からして、\\
because of what we mentioned earlier,\\
極、初めの頃は元末四大家の自由なる手法を應用して、\\
in the early stages, the Four Great maters in the late Yuan dynasty applied free techniques, and\\
いろいろな畫風を生じはしたが、\\
various painting styles came about and spread, but\\
矢張りそれは傳統的な支那畫の中に、\\
it is only a period within the traditional Chinese paintings\\
一種の素人趣味の興隆を見たと云ふ丈けの時代であって、\\
where the rise of an amateurish taste was found;\\
其風潮が次の康熙年間になって、\\
the trend was in the following Kangxi era, and\\
王石谷、惲南田などいふ大家の爲めに統一された結果は、\\
as a result of the unification due to the great masters like Wang Shigu and Yun Nantian,\\
何の變化も新味も無い單調時代---\\
into a monotonous period with no changes or new novelty---\\
寧ろ衰微時代に陷り、\\
or rather, a declining period---they fell;\\
そこに初めて支那藝術に一時期を劃する乾隆の破壤時代が現はれたのである。\\
from this on, the breaking period of Qianlong came about which divides the Chinese art. 

\vspace{1em}
\plabel{$\blacktriangleright$}%
そして此の時代には、\\
Then, in this era,\\
前にも云った通り、\\
as what we mentioned before,\\
色々な手法、色々な趣味が雜然として湧起し、\\
various techniques and tastes emerged in a mixture manner;\\
過去の凡ての畫風を離脱して斬新な方途を開拓するために、\\
in order to get rid of all the painting styles of the past and explore brand new paths,\\
盛んな破壞運動を採ったので、\\
they adopted vigorous breaking movements, and therefore\\
此の破壊は從來にない西洋畫風などを採り入れて、\\
this destruction is that they carried out the attempt that they incorporated the unprecedented Western painting style, and\\
支那の傳統的な畫を根柢から破壞し盡さうと云ふやうな試みが、\\
destroyed the foundations of traditional Chinese paintings\\
隨所に行はれたのである。\\
everywhere.

\vspace{1em}
\plabel{$\blacktriangleright$}%
今日より之を見ると、\\
Looking at it from today's perspective,\\
當時の畫が、如何に新しい諸種の試みをしたかといふことは、\\
how many new and various attempt were made in these paintings\\
全く驚くべきものであって、\\
is something what one would certainly be totally astonished;\\
今日の畫家などが、\\
most of the things which contemporary painters\\
新しく試みんとしてゐる大抵のことは、\\
take as something that they attempt anew are,\\
最早や此當時に於て、\\
something that already, at this period,\\
悉く試み盡されて居るのである。\\
had been competely attempted.

\vspace{1em}
\plabel{$\blacktriangleright$}%
されば支那畫の傳統を承けた日本畫が、\\
Therefore when the Japanese paintings, which inherited the tradition of Chinese paintings,\\
從來の畫風に甘んぜずして、\\
without contenting with the traditional painting styles,\\
新しき手法を試み、\\
tried new techniques, and\\
新しき進路を開かんとするには、\\
opened new paths,\\
是非共、この淸初より乾隆の末年に至る支那畫の硏究といふことを度外視してはならないのである。\\
by all means one should not overlook the study of Chinese paintings from the early Qing Dynasty to the end of the Qianlong period.

\prule
\plabel{$\blacktriangleright$}%
以上は大體ながら、\\
While the above is general,\\
歴史的に支那の近代畫を觀察したのであるが、\\
we observed modern Chinese paintings historically, but\\
尙ほ一言附け加へて置きたいのは、\\
there is one more point I would like to add, which is\\
支那畫に對する見方である。\\
the way we view Chinese paintings.

\vspace{1em}
\plabel{$\blacktriangleright$}%
これは日本の畫家などの特に心得置くべきことであって、\\
This is something that Japanese painters should in particular to aware of, and\\
又苟も支那畫を談じ、\\
also for those who discuss Chinese paintings and\\
支那美術を硏究する者に取って、\\
study Chinese arts,\\
頗る根本的な問題である。\\
this is a very fundamental issue.

\vspace{1em}
\plabel{$\blacktriangleright$}%
それは第一に支那の近代畫家の感ずる趣味と、\\
The first point is that between the tastes of modern Chinese painters and\\
日本の現代畫家の感ずる趣味との間にかなり大なる相違のある點である。\\
contemporary Japanese painters there is huge difference.

\vspace{1em}
\plabel{$\blacktriangleright$}%
其原因の一つは、\\
One reason for that is that,\\
普通世に言ふ所の地理的の關係、\\
the reason of geography that common people say,\\
卽ち支那人の趣味は大陸的であり、\\
i.e. the tastes of Chinese are continental, while\\
日本人の趣味は島國的であるといふ事で、\\
the tastes of Japanese are insular;\\
これらも確かに彼我相違の一要素であるが、\\
although these are in fact a factor of the differences between these two,\\
更に又時代に懸隔のあるといふことも、\\
that moreover a gap between these time periods exists\\
大に考へなければならぬ點である。\\
is something that we must carefully consider.

\prule
\plabel{$\blacktriangleright$}%
今日、單に支那を物質的に觀察すると、\\
Today, when we look at China only materialistically,\\
其の文化は、遙かに我國に比して遅れてゐる觀があるが、\\
its culture seems far behind compared with our country, but\\
然し、之を社會組織の上より考へ、\\
(however,) if one considers from the point of social organization,\\
又思想變遷の上より考察すると、\\
and considers from the point of the evolution of thought,\\
支那の文化といふものは、\\
that the thing that we call Chinese culture,\\
日本の文化に比べて、\\
when one compares with Japanese culture,\\
相並んだ時代に於ては、\\
at parallel time periods,\\
遙かに爛熟してゐたといふ事は爭はれない事實である。\\
is far more mature, is a fact that one cannot deny.

\vspace{1em}
\plabel{$\blacktriangleright$}%
で、最近日本の畫家などが支那を遊歴し、\\
In recent times, Japanese painters traveled to China, and\\
歸來色々と其の觀察なり寫生なりを發表するが、\\
upon their return, they published their observations and sketches variously, and\\
それを見ると、\\
when one look at these,\\
其觀察乃至寫生は、\\
that the observations and even their sketches\\
恰も宋の畫院の人々若しくは、\\
are just in the way that the people in the painting office of the Song dynasty, or\\
明初の畫院の人々が見たやうな、\\
the people in the painting office of the Ming dynasty see, and\\
其の同じ目を以て、\\
with the same perspective of theirs,\\
支那の風景を觀てゐるやうに考へられる。\\
they have observed the landscapes of China, is something that we think.

\vspace{1em}
\plabel{$\blacktriangleright$}%
さう云ふ物の見方は支那では、\\
Such way of viewing things is, in China,\\
最早や五百年乃至七百年の昔しに旣に過ぎ去ってゐるので、\\
something that has already passed since as early as 500 or even 700 years ago;\\
其のため支那の畫家中でも、\\
therefore, that even among the painters of China,\\
例へば夫の唐伯虎とか、王石谷とか云ふ人は、\\
e.g. who we call Tang Bohu and Wang Shigu in the past,\\
北宗の畫を採り入れたと言はれてゐるけれども、\\
adopted the painting of Northern school, is something that has been said, but,\\
然も其の採り入れた北宗といふのは、\\
(however,) the Northern school from which they adopted\\
前の所謂畫院などの北宗とは全然別なものであった。\\
is something entirely different from the Northern school of the painting office that we mentioned previously.

\vspace{1em}
\plabel{$\blacktriangleright$}%
斯の如きは卽ち時代による見方の變化であって、\\
These are just the change of way of viewing things due to the time, and\\
今日の日本の畫家の大部分、\\
among the majority of Japanese painters today,\\
特に南畫を學んでゐる人でさへ、\\
in particular even those who have studied Southern painting,\\
多くは支那の明末以後の近代畫の見方には達して居ないのである。\\
most have not reached the perspective of modern paintings of China after the Ming dynasty.

\vspace{1em}
\plabel{$\blacktriangleright$}%
この故に、其の觀察なり寫生なりが、\\
Because of this, their observations and sketches are\\
頗る不徹底なものであって、\\
something very incomplete, and\\
全然、傳統的な日本畫の範圍を脫し切らないのである。\\
they didn't escape from the range of traditional Japanese paintings completely.

\vspace{1em}
\plabel{$\blacktriangleright$}%
されば眞に近代の支那畫を鑑賞せんとするには、\\
Therefore, to truely appreciate modern Chinese painting,\\
先づ第一に、此の見方から改めなければならない。\\
firstly one must change from this way of viewing things.

\vspace{1em}
\plabel{$\blacktriangleright$}%
見方が改まって後、\\
After one changed the way of viewing things,\\
始めて支那畫の時代的變遷の意味も解り、\\
they start to understand the meaning of the evolution of Chinese painting with time,\\
又殊に、近代畫の趣味の奈邊に在るかが諒解されるのである。\\
and especially, they comprehend the whereabouts of the taste of modern painting. 

\vspace{1em}
\plabel{$\blacktriangleright$}%
此の點は支那藝術を硏究するものの、最も注意を要する所である。\\
This point is something that requires the most attention for those who study Chinese art.

\end{document}
