\documentclass{ctexart}

\usepackage{pandekten}
\usepackage{dashrule}

\makeatletter
\newcommand*{\shifttext}[1]{%
  \settowidth{\@tempdima}{#1}%
  \hspace{-\@tempdima}#1%
}
\newcommand{\plabel}[1]{%
\shifttext{\textbf{#1}\quad}%
}
\newcommand{\prule}{%
\begin{center}%
\hdashrule[0.5ex]{.99\linewidth}{1pt}{1pt 2.5pt}%
\end{center}%
}

\makeatother

\newcommand{\minusbaseline}{\abovedisplayskip=0pt\abovedisplayshortskip=0pt~\vspace*{-\baselineskip}}%

\setlength{\parindent}{0pt}

\title{Assignment 6}
\author{Ze (Zack) Chen}

\begin{document}

\maketitle

\plabel{$\blacktriangleright$}%
予が本年の特殊講議題目は兩漢文學考と名け、\\
I name my special lecture topic of this year \textit{A Study of Literature of two Han Dynasties}, and\\
各週二時間を之に費やす豫定なるが、\\
it is the plan to spend two hours per week on this subject, and\\
實は前年度になしたる兩漢學術考の續講に外ならず。\\
in reality, this is nothing but a continuation of the \textit{A Study of Academics of two Han Dynasties} that we had conducted in the previous year.

\vspace{1em}
\plabel{$\blacktriangleright$}%
本年は東漢の經學に就いて其大體を述ぶる筈なるが、\\
This year it should be that we discuss the outline regarding the study of Confucian classics, but,\\
但本年の題目を經學考とせず、文學考としたるは、\\
(however,) as for that we don't take \text{the Study of Confucian Classics} as the topic this year, and we have taken \text{the Study of Literature},\\
元來予の講義に出席し之を聽く義務あるは、\\
because originally the ones who have the obiligation to attend my lectures and listen to them\\
支那文學科の人人なるを以て、\\
are people of the Department of Chinese Literature Study,\\
本年は東漢の經學を敍述すると共に、\\
for this year, together with the discussion of the study of Confucian classics of the Eastern Han dynasty,\\
其文學にも言及したしと思ひ、\\
(I think that) we talk also about its literature, and\\
爲めに文學の字を廣義に用ひ、\\
therefore I use the term literature in a board sense, and\\
文學考となせり。\\
it has become \text{the Study of Literature}.

\vspace{1em}
\plabel{$\blacktriangleright$}%
諸子幸に之を諒せよ。\\
(For you all) please acknowledge this.

\prule

\plabel{$\blacktriangleright$}%
一體西漢と東漢とは、\\
For the Western and Eastern Han dynasties as a single body,\\
其間に王莽の時代が十數年入る丈で、\\
in the gap between them is just that the period of Wang Mang enters for a decade or so, and\\
兩者殆んど相從續したるものと見るを得。\\
it can be seen that the two dynasties are the ones that almost had followed each other.

\vspace{1em}
\plabel{$\blacktriangleright$}%
故に多くの場合には、\\
Therefore in many situations,\\
西漢\textperiodcentered 東漢と分たずして、\\
one doesn't differentiate the Western and Eastern Han dynasties, and\\
唯單に漢といふ。\\
and says just Han.

\vspace{1em}
\plabel{$\blacktriangleright$}%
例せば經學に於いても、\\
For example as for the Study of Confucian classics,\\
宋學に對して漢學といひ、\\
we say the Theory of Han as opposed to the Theory of Song,\\
文に於いても、\\
and as for Literature also,\\
六朝\textperiodcentered 唐\textperiodcentered 宋文に對して漢文といふ。\\
we say the Literature of Han as opposed to the Literature of the Six dynasties, the Tang dynasty and the Song dynasty.

\vspace{1em}
\plabel{$\blacktriangleright$}%
然れども是れ槪括に過ぎたる言葉にして、\\
However this is a word that one had overly simplified, and\\
諸子の知る如く、\\
as you know,\\
經學に於いても、文に於いても、\\
for the study of Confucian classics as well as literature,\\
西漢と東漢とは大に其趣を殊にするものあり。\\
the Eastern and Western Han dynasties has things that we have significantly different view on.

\vspace{1em}
\plabel{$\blacktriangleright$}%
何を以てか之れを言ふ。\\
For what reason do I say this?

\vspace{1em}
\plabel{$\blacktriangleright$}%
經學に就いて之れを見るに、\\
As we see this regarding the study of Confucian classics,\\
前年講義にて述べし如く、\\
as we have explained in the lectures last year,\\
西漢の博士が學校に於いて學生に授けし經書は、\\
the classics that the scholars of the Western Han dynasty teach the student in the school\\
今文を以て書かれたるものなり。\\
are things that had been written in modern script.

\vspace{1em}
\plabel{$\blacktriangleright$}%
東漢に至りても、\\
Reaching the Eastern Han dynasty, however,\\
後に委しく述ぶる如く、\\
as we will discuss in detail later,\\
學官に立てられたるものは、\\
the thins established by the academic officials\\
西漢と何等の差違なく、\\
are not significantly different from the Western Han dynasty, and\\
矢張り西漢に倣うて總べての學制が出來たる事とて、\\
also, even though the entire education system following the Western Han dynasty comes into being,\\
今文の學者が重に博士に採用され、\\
the modern script is highly valued by the scholars and was adopted by the scholars, and\\
それが各西漢より傳來の家法によりて教授したりしが、\\
they had taught according to the family codes passed down from the Western Han dynasty, but,\\
博士流の學問以外に、\\
outside of the academic studies of the \textit{boshi} lineage,\\
東漢には古文派の學問が盛になつた。\\
during the Eastern Han dynasty the study of old script became prominent.

\prule

\plabel{$\blacktriangleright$}%
勿論この古文學は、\\
Of course as for this study of old text\\
西漢の末に劉歆が始めて之れを提唱したけれども、\\
at the end of Western Han dynasty, Liu Xin started to proposed it, however,\\
官學に立籠りし博士等より非常に反對され、\\
(it is) strongly opposed by the scholars who barricaded themself in the official academics, and\\
後彼が王莽に取入りて、\\
afterwards he tries to gain the favor of Wang Mang, and\\
其信用を得たる關係よりして、\\
due to that he gained credits,\\
暫時學官に立てられたる事はありしも、\\
although there existed that for a while he had been put in the position of an academic official,\\
暫らくにして廢せられ、\\
soon it was abolished, and
王莽死して再たび漢の天下となり、\\
after Wang Mang died again it becomes the ruling of Han, and\\
學制は總べて西漢に則る事となつた。\\
it became that the educational system entirely follows the Western Han dynasty.

\vspace{1em}
\plabel{$\blacktriangleright$}%
それで、今文學が學官に立てられ、\\
As a result, the study of modern text is established by the academic officials, but\\
古文學は國家の保護奬勵を受くるといふ事にはならなかつた。\\
it didn't become that the study of classical text receives the protection and encouragement of the country.

\vspace{1em}
\plabel{$\blacktriangleright$}%
然れども縱令劉歆は人物劣等にて取るに足らぬものであつたにせよ、\\
However, even though Liu Xin was a person that is an worthless inferior figure,\\
其倡へし古文學なるものは、\\
the thing he advocated, that is the study of classical text,\\
決して輕視すべきものにあらず。\\
is absolutely not something that one should take lightly.

\vspace{1em}
\plabel{$\blacktriangleright$}%
それで或る特別の學者逹は熱心に之れを修め、\\
Therefore, some exceptional scholars study this heartly, and\\
之れを弟子に傳へたり。\\
had passed this to their disciples.

\prule

\plabel{$\blacktriangleright$}%
彼等は古文學者といはるる通り、\\
As they are called classical text scholars,\\
其經書は古文卽ち古字を以て書かれ、\\
their classic texts are written with old text, i.e. old characters, and\\
經書の文字に古今の差ありしが、\\
the difference existed between the old and the modern ones in the text of classics, and\\
獨り其れのみならず、\\
not only this,\\
所依の經已に同じからず、\\
the original Confucian text they used are already not identical,\\
又經義互ひに異れり。\\
and the meaning of the Confucian texts mutually differ.

\vspace{1em}
\plabel{$\blacktriangleright$}%
易\textperiodcentered 書\textperiodcentered 詩\textperiodcentered 禮\textperiodcentered 春秋、皆然らざるはなし。\\
This holds true for all of the Book of Changes, the Book of Documents, the Book of Odes, the Book of Rituals, and Zuo Zhuan.

\vspace{1em}
\plabel{$\blacktriangleright$}%
施\textperiodcentered 孟\textperiodcentered 梁丘\textperiodcentered 京房の易と\\
As for Shi Cuo, Meng Xi, Liang Qiuhe and Jing Fang's Book of Changes, and\\
費直\textperiodcentered 高相の易、\\
Fei Zhi and Gao Xiang's Book of Changes, plus\\
歐陽\textperiodcentered 大小夏侯の書と、\\
Ouyyang Sheng, Xiahou Sheng, Xiahou Jian's Book of Documents, and\\
孔安國の傳へたる古文尚書、\\
the Book of Documents of old text that Kong Anguo had passed down, plus\\
齊\textperiodcentered 魯\textperiodcentered 韓三家の詩と毛詩、\\
The three schools Qi, Lu, Han's Book of Odes and the Book of Odes with Mao's commentary, plus\\
大小戴\textperiodcentered 慶氏の禮と孔安國が獻ぜし禮古經と周官經、\\
Dai De, Dai Sheng, and Qing's Book of Rituals, and the Book of Rituals of old text that Kong Anguo presented, and the official Book of Rituals of Zhou, plus \\
公羊\textperiodcentered 穀梁春秋と左氏傳と、\\
Gongyang and Guliang Zhuan and Zuo Zhuan,\\
其經義同じからざるのみならず、\\
not only the meaning of the texts are not identical, but\\
周官經の如き、古文家は周公太平を致すの迹として之を尊べども、\\
as in the official Book of Ritual of Zhou, the scholars of old texts revere it as a document that Zhou Gong creates a peaceful society, but\\
今文家は六國陰謀の書として之れを取らず。\\
the modern text scholars take it as a book of consipiracy of the Six states and do not take this.

\end{document}
