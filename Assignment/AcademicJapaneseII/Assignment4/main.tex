\documentclass{ctexart}

\usepackage{pandekten}
\usepackage{dashrule}

\makeatletter
\newcommand*{\shifttext}[1]{%
  \settowidth{\@tempdima}{#1}%
  \hspace{-\@tempdima}#1%
}
\newcommand{\plabel}[1]{%
\shifttext{\textbf{#1}\quad}%
}
\newcommand{\prule}{%
\begin{center}%
\hdashrule[0.5ex]{.99\linewidth}{1pt}{1pt 2.5pt}%
\end{center}%
}

\makeatother

\newcommand{\minusbaseline}{\abovedisplayskip=0pt\abovedisplayshortskip=0pt~\vspace*{-\baselineskip}}%

\setlength{\parindent}{0pt}

\title{Assignment 4}
\author{Ze (Zack) Chen}

\begin{document}

\maketitle

\plabel{$\blacktriangleright$}%
貴族政治の時代には、\\
At the time of aristocracy,\\
貴族が權力を握る習慣であるから、\\
because (there) is the custom that the aristocrat holds the power,\\
隋の文帝\textperiodcentered 唐の太宗の如き英主が出で、\\
wise rulers like Emperor Wen of Sui and Taizong of Tang show up, and\\
制度の上に於ては貴族の權力を認めぬ事にしても、\\
although, with regards to the system, it is the thing that (one) does not admit the power of the aristocrat,\\
際の政治には尚其形式が殘つて、\\
as for the politics of tht time, still its form remains, and\\
政治は貴族との協議體となつた。\\
politics becomes the the entity of consultation of the aristocrats.

\vspace{1em}
\plabel{$\blacktriangleright$}%
勿論この協議體は代議政治ではない。\\
Of course this entity of consultation is not the representation politics.

\prule

\plabel{$\blacktriangleright$}%
唐代に於ける政治上の重要機關は三つあつた。\\
With regards to the Tang dynasty, there are three important agencies on politics.

\vspace{1em}
\plabel{$\blacktriangleright$}%
曰く尚書省、曰く中書省、曰く門下省である。\\
(They) are the so-called Department of State Affairs, the so-called Central Secretariat, and the so-called Chancellery.

\vspace{1em}
\plabel{$\blacktriangleright$}%
その中で中書省は天子の祕書官で、\\
In the middle of these, the Central Secretariat is the private secretary of the Son of Heaven, and\\
詔勅命令の案を立て、\\
(it) makes the bills of imperial edict and command, and\\
臣下の上奏に對して批答を與へることになつて居るが、\\
it has been (decided) that (it) gives the response regarding to reports of officials, but\\
この詔勅が確定するまでには門下省の同意を必要とする。\\
until the time that this imperial edict finalizes, the consent of the Chancellery is regarded as a necessity.

\vspace{1em}
\plabel{$\blacktriangleright$}%
門下省は封駁の權を有して、\\
The Chancellery has the power to refute, and\\
若し中書省の案文が不當と認むるときには、\\
if (at the time that) it recognizes that the bill of the Central Secretariat is not proper,\\
これを駁撃し、これを封還することも出來る。\\
it is possible also that (it) denounces this and return this.

\vspace{1em}
\plabel{$\blacktriangleright$}%
そこで中書と門下とが政事堂で協議して決定する事となる。\\
Then, it becomes that the Central Secretariat and the Chancellery discuss and decide at the Office of Politics.

\vspace{1em}
\plabel{$\blacktriangleright$}%
尚書省はこの決定を受取つて執行する職務である。\\
As for the Department of State Affairs, it is the duty that (it) accepts this decision and execute.

\vspace{1em}
\plabel{$\blacktriangleright$}%
中書省は天子を代表し、\\
The Central Secretariat represents the Son of Heaven, and\\
門下省は官吏の輿論、即ち貴族の輿論を代表する形式になつて居るのではあるが、\\
it exists that the Chancellery has become the form that represents the official's opinion, i.e. the aristocrat's opinion, but\\
勿論、中書\textperiodcentered 門下\textperiodcentered 尚書三省ともに大官は皆貴族の出身であるので、\\
of course, for the three departments, Central Secretariat, the Chancellery, and the Department of State Affairs together, as for import officials, it is all background of aristocrat, and therefore,\\
貴族は天子の命令に絶對に服從したのではない。\\
as for aristocrat, it is not that (they) absolutely obeyed the command of the Son of Heaven.

\vspace{1em}
\plabel{$\blacktriangleright$}%
夫故に天子が臣下の上奏に對する批答なども、極めて友誼的で、決して命令的でない。\\
Because of this, even things such as the response that responds to the reports of officials are exceedingly friendly and absolutely not authoritarian.

\vspace{1em}
\plabel{$\blacktriangleright$}%
然るに明清時代になりては、\\
However, as it reaches the period of Ming and Qing,\\
批答は全く從僕などに對すると同樣、\\
the reponses were completely the same as that responds to things like servants, and\\
ぞんざいな言葉遣ひで命令的となり、\\
they use rude words and become authoritarian, and\\
封駁の權は宋以後益々衰へ、\\
the power to refute gradually fades out starting from the Song dynasty, and\\
明清に在りては殆んどなくなつた。\\
at the time of Ming and Qing almost becomes non-existent.

\end{document}
