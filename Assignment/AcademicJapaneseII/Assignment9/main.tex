\documentclass{ctexart}

\usepackage{pandekten}
\usepackage{dashrule}

\makeatletter
\newcommand*{\shifttext}[1]{%
  \settowidth{\@tempdima}{#1}%
  \hspace{-\@tempdima}#1%
}
\newcommand{\plabel}[1]{%
\shifttext{\textbf{#1}\quad}%
}
\newcommand{\prule}{%
\begin{center}%
\hdashrule[0.5ex]{.99\linewidth}{1pt}{1pt 2.5pt}%
\end{center}%
}

\makeatother

\newcommand{\minusbaseline}{\abovedisplayskip=0pt\abovedisplayshortskip=0pt~\vspace*{-\baselineskip}}%

\setlength{\parindent}{0pt}

\title{Assignment 8}
\author{Ze (Zack) Chen}

\begin{document}

\maketitle

\plabel{$\blacktriangleright$}%
文人畫をその最も發逹した時のものに就て見ると、\\
When we look at literati painting at the time when it is most properous,\\
それには更に若干の特殊なる條件がある。\\
there are even more special conditions.

\vspace{1em}
\plabel{$\blacktriangleright$}%
先づ第一に文人畫はその性質として職業的ならざるを要するもので、\\
Firstly, literati painting is something that, as its nature, requires not being occupational,\\
卽ち作家が人の爲めに畫くと云ふよりも、\\
that is, rather than painting for others,\\
寧ろ自分自身で樂む境涯に於て畫くと云ふ事がなければならん。\\
instead he must paint for the circumstances of his own enjoyment.

\vspace{1em}
\plabel{$\blacktriangleright$}%
外からの拘束を受けないで、\\
Without receiving restrictions from outside,\\
自身の樂む境涯をその儘に吐露すると云ふ事がなければならぬ。\\
he must express the circumstances of his own enjoyment as it is.

\vspace{1em}
\plabel{$\blacktriangleright$}%
古の文人高士の技藝は、\\
It has been thought that the skills and arts of ancient scholars and nobles\\
殊にそうであつたと考へられてゐる。\\
are particularly like this.

\vspace{1em}
\plabel{$\blacktriangleright$}%
事實に於ては、\\
In fact,\\
文人の作る所のもので、\\
the works that literati make\\
時として意外に職業的なるものに近いものもない事はないが、\\
are sometimes something surprisingly close to the professional works, but\\
大體から云へば文人の畫は他人の爲めにするよりも、\\
generally speaking, literati paintings are different from the occupational works in that\\
自分の爲にする點に於て職業的のものとは違はなければならぬ。\\
rather than that they paint for others, they paint for themselves.

\vspace{1em}
\plabel{$\blacktriangleright$}%
文人に相應しい文人畫の約束としては此職業的でない、\\
As the promises of the literati paintings appropriate to literati, the most important is that\\
自ら樂むの境涯を有すると云ふ事が第一に大切である。\\
it is not occupational, and it has the circumstances of his own enjoyment.

\vspace{1em}
\plabel{$\blacktriangleright$}%
倪雲林が友人に答ふる消息の中に、\\
In a letter that Ni Yunlin replies to his friend,\\
自分の畫いた陳子桱剡源の圖に就て云ふ所を見ると、\\
when one see the comment on the picture of Chen Zijing and Yan Yuan that he drew,\\
その文章の中に\\
in that article there is the sentence\\
『僕之所謂畫者。不過逸筆草々。不求形似。聊以自娛耳。』\\
``What I call painting is no more than something drawn by letting my brush run wild.
It is not an attempt to make the painting and real object resemble each other.
Well, it's just something I do for my own enjoyment.''\\
とあるが、丁度それである。\\
which is exactly what we have discussed.

\vspace{1em}
\plabel{$\blacktriangleright$}%
旣に表現が主義であるならば、\\
If expression is already the main principle,\\
自から樂むの境涯に在るのは當然だとも云へるかも知れないが、\\
it is possible to say that it is of course that one exists in the circumstances of his own enjoyment, but\\
併しそれが特に職業的でない爲から來るとなると、\\
if this comes from the purpose not being occupational,\\
それには特殊の性質が、伴ふのである。\\
a special nature accompanies it.

\vspace{1em}
\plabel{$\blacktriangleright$}%
近頃の表現派の藝術の如きは如何に主觀的であつても、\\
No matter how subjective are the things like the art of recent expressionism,\\
强いてそれを非職業的のものとすべく要求してはゐない。\\
there is no strong requirement that it must be something not occupational.

\end{document}
