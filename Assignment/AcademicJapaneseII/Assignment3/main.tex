\documentclass{ctexart}

\usepackage{pandekten}
\usepackage{dashrule}

\makeatletter
\newcommand*{\shifttext}[1]{%
  \settowidth{\@tempdima}{#1}%
  \hspace{-\@tempdima}#1%
}
\newcommand{\plabel}[1]{%
\shifttext{\textbf{#1}\quad}%
}
\newcommand{\prule}{%
\begin{center}%
\hdashrule[0.5ex]{.99\linewidth}{1pt}{1pt 2.5pt}%
\end{center}%
}

\makeatother

\newcommand{\minusbaseline}{\abovedisplayskip=0pt\abovedisplayshortskip=0pt~\vspace*{-\baselineskip}}%

\setlength{\parindent}{0pt}

\title{Assignment 3}
\author{Ze (Zack) Chen}

\begin{document}

\maketitle

\plabel{$\blacktriangleright$}%
この二種の政治状態を比較すると、\\
If we compare these two kinds of political statuses,\\
貴族政治時代に於ける君主の位置は、\\
with regards to the aristocracy age, as for the position of the monarch,\\
時として實力あるものが階級を超越して占むる事ありても、\\
even if sometimes the thing exists that people with power exceed their rank and occupy,\\
既に君主となれば貴族階級中の一の機關たる事を免るゝ事が出來ない。\\
if one already become the monarch, the thing that one escapes from the thing being one agency of the rank of aristocrat is impossible.

\vspace{1em}
\plabel{$\blacktriangleright$}%
即ち君主は貴族階級の共有物で、\\
That is, the monarch is a shared thing of the rank of aristocrat, and\\
その政治は貴族の特權を認めた上に實行し得るのであつて、\\
as for this politics, it is that it is executable on the ground of admitting the privilege of the aristocrat, and \\
一人で絶對の權力を有することは出來ない。\\
the thing that (by) one person possesses the power is impossible.

\vspace{1em}
\plabel{$\blacktriangleright$}%
孟子は嘗て卿に異姓の卿と、貴戚の卿とあつて、\\
Mencius once, as for lord, exists with lord of different family name and lords of the family of the monarch, and\\
後者は君主に不都合あればこれを諫め、\\
the latter admonish the monarch if there is any inconvenience, and\\
聽かざればこれを取り換へるといへる事があるが、\\
although the thing exists that (is called that), if (the monarch) does not listen, (one) replaces it,\\
かゝる事は上代のみならず、\\
it is not that such thing is only of the old age, but\\
中世の貴族政治時代にも屡々實行された。\\
also at the age of aristocracy of the middle age it was executed over and over again.

\vspace{1em}
\plabel{$\blacktriangleright$}%
君主は一族即ち外戚從僕までも含める一家の專有物で、\\
The monarch is the exclusive possession of the clan, i.e. the family that contains even until the far relatives and servants, and\\
從てこれ等一家の意に稱はないと廢立が行はれ、\\
therefore if to the intent of these as a family (it) does not praise, then deposition was carried out,\\
或は弑逆が行はれた。\\
or regicide was carried out.

\vspace{1em}
\plabel{$\blacktriangleright$}%
六朝より唐に至るまで、\\
From the Six Dynasties to the Tang dynasty,\\
弑逆廢立の多いのは、\\
the most thing of regicide and deposition is\\
かゝる事情によるので、\\
by means of such thing, and\\
この一家の事情は多數の庶民とは殆んど無關係であつた。\\
the affairs of this single family was almost without any relation to most of the ordinary people.

\vspace{1em}
\plabel{$\blacktriangleright$}%
庶民は國家の要素として何等の重きをなさず、\\
Ordinary people, regarded as a composition of the country, does not become important (thing) of any degree, and \\
政治とは沒交渉である。\\
with politics there is no negotiation.

\prule

\vspace{1em}
\plabel{$\blacktriangleright$}%
かくの如く君主は單に貴族の代表的位置に立つて居つたのは中世の状態なるが、\\
The thing that such monarch exclusively has standed as the position of representation of aristocrat is the status of the middle age, but\\
近世に入りて其貴族が沒落すると、\\
as it enters the early modern age and its aristocrat declines,\\
君主は直接に臣民全體に對する事となり、\\
the monarch becomes the thing that directly faces the whole citizens, and\\
臣民全體の公の所有物で、\\
is the public possession of the whole citizens, and\\
貴族團體の私有物でなくなつた。\\
becomes not being the private possession of the group of aristocrats.

\vspace{1em}
\plabel{$\blacktriangleright$}%
かくして臣民全體が政治に關係する事となれば、\\
Thus when the whole citizens become the thing that relates to politics,\\
君主は臣民全體の代表となるべき筈のやうであるが、\\
although it becomes as if that it should be that the monarch must become the representative of the whole citizens,\\
支那にはかゝる場合なかりしために、\\
because as for China it was not such circumstance,\\
君主は臣民全體の代表者にあらずして、\\
the monarch does not be the representative of the whole citizens, and \\
夫自身が絶對權力の主體となつた。\\
himself became the main body of the absolute power.

\vspace{1em}
\plabel{$\blacktriangleright$}%
しかし兎も角、君主の位置は貴族時代よりは甚だ安全となり、\\
But anyway, the position of the monarch becomes greatly safer then in the age of aristocracy, and\\
從て廢立も容易に行はれず、\\
therefore deposition is also not easily carried out, and\\
弑逆も殆んどなくなつた事は宋以後の歴史は其然るを證明する。\\
the thing that regicide also became almost nonexistent proves that the history after the Song dyansty is this case.

\vspace{1em}
\plabel{$\blacktriangleright$}%
尤も元代のみは頗る異例がある。\\
Though only for the Yuan dynasty extreme exceptions exists.

\vspace{1em}
\plabel{$\blacktriangleright$}%
これは蒙古文化の程度によるので、\\
This is by means of the level of the Mongolian culture, and\\
蒙古の文化は支那の同時代に比較すると甚しく遲れて、\\
although as for the the Mongolian culture, if one compares it with the Chinese culture of the same age, it is every late, and\\
却て支那の上古時代と同程度であるのに、\\
on the contrary is it is the at same level as of the old age of China,\\
支那を征服せるがために、\\
because it conquers China,\\
突然に近世的の國家組織の上に君臨したのであるから、\\
because of the thing that suddenly it reigns over a country organization of the early modern age,\\
其帝室には依然として貴族政治の形骸が殘つて居り、\\
as for its royal family, as it has been, the skeleton of aristocracy has remained, and\\
民政の方のみが近世的色彩になつたから、\\
because only the aspect of civil politics became the color of early modern age,\\
一種の矛盾した状態をあらはしたのである。\\
the thing exists that it showed one kind of status that contradicted.

\end{document}
