\documentclass{ctexart}

\usepackage{pandekten}
\usepackage{dashrule}

\makeatletter
\newcommand*{\shifttext}[1]{%
  \settowidth{\@tempdima}{#1}%
  \hspace{-\@tempdima}#1%
}
\newcommand{\plabel}[1]{%
\shifttext{\textbf{#1}\quad}%
}
\newcommand{\prule}{%
\begin{center}%
\hdashrule[0.5ex]{.99\linewidth}{1pt}{1pt 2.5pt}%
\end{center}%
}

\makeatother

\newcommand{\minusbaseline}{\abovedisplayskip=0pt\abovedisplayshortskip=0pt~\vspace*{-\baselineskip}}%

\setlength{\parindent}{0pt}

\title{Assignment 8}
\author{Ze (Zack) Chen}

\begin{document}

\maketitle

\plabel{$\blacktriangleright$}%
以上に於て文人畫の起源と並に文人畫の流派化の大要を述べたが、\\
In the above I talked about a summary of the classification of schools of literati painting together with the origin of literati painting, and\\
それで支那に於ける文人畫なるものゝ成立は大畧說明した積である。\\
with that, the intention is that I did a general explanation of the establishment of literati painting in China.

\vspace{1em}
\plabel{$\blacktriangleright$}%
前に云つた如く、\\
As what I just saied previously,\\
文人畫が流派化せられない間は、\\
when literati painting had not been classified into schools,\\
それは文人が畫く畫でさへあれば、\\
so long as it is the painting that a literati paints,\\
どんな畫でも差支ないのである。\\
there is no difficulty with any kinds of paintings.

\vspace{1em}
\plabel{$\blacktriangleright$}%
流派化せられて始めて文人畫に一定の形式が生じて來る譯である。\\
After the classification is done, for the literati painting a certain form started to emerge.

\vspace{1em}
\plabel{$\blacktriangleright$}%
然るにその形式を生ずるに先つて、\\
However, before such a style emerged,\\
文人畫に相應しい畫となる爲には、\\
for a painting to be appropriate to literati painting,\\
その畫に特殊の原理が出來なくてはならぬ。\\
specific principles must be established for that painting.

\vspace{1em}
\plabel{$\blacktriangleright$}%
その原理なるものは、\\
These principles,\\
詮じ詰めれば郭若虛の說くが如く、\\
in short, as what Guo Ruoxu states,\\
人格を本位として所謂ゆる心印の如くに氣韻の現はれのある事がそれであらう。\\
are that the painting should manifest \textit{qiyuan} as the so-called \textit{xinyin} as the principle of personality.

\vspace{1em}
\plabel{$\blacktriangleright$}%
之を今日の言葉に直して云へば、\\
To put it in today's terms,\\
卽ち表現Expressionを眼目とするのである。\\
that is to say that one should take \textit{Expression} as the main point.

\vspace{1em}
\plabel{$\blacktriangleright$}%
要するに文人畫の原理は、\\
In essence, it is that the principles of literati painting\\
近頃西洋に起つた藝術上の表現主義と合致するのである。\\
coincide with the Expressionism of art that emerged in the West in recent times.

\vspace{1em}
\plabel{$\blacktriangleright$}%
表現主義は心印主義と云ふも同じである。\\
Expressionism is the same as what we call the principle of \textit{xinyin}.

\vspace{1em}
\plabel{$\blacktriangleright$}%
西洋の表現主義を說くものにも種々あるが、\\
There are various explanations of Expressionism of the West but,\\
例へばクローチエが說く所に依れば、\\
for example, according to what Benedetto Croce stated,\\
藝術の本義は純眞なる直感に在るので、\\
the essence of art lies in pure intuition, and\\
純眞の直感はそれ自體が卽ち表現である。\\
as for pure intuition, itself is the expression.

\vspace{1em}
\plabel{$\blacktriangleright$}%
表現を基本とするが故に、\\
Because expression is fundamental,\\
藝術は須らく\ruby{舒情詩的}{リリカル}なるべきもので、\\
art must be lyrical, and\\
それには作家の人格から溢れ出でた誠實性がなければならぬと云ふのである。\\
for that, there must be the sincerity that that overflows from the artist's personality.

\vspace{1em}
\plabel{$\blacktriangleright$}%
夫故にその主義は丁度文人畫に當篏まるのである。\\
Therefore, this principle applies exactly to literati painting.

\vspace{1em}
\plabel{$\blacktriangleright$}%
表現派の畫家として有名なるカンディンスキーの唱道する所の内面の音響(Innerer Klang)と云ふものゝ如きも、\\
Just like the concept of Innerer Klang that the famous Expressionist painter Wassily Kandinsky stated, also\\
郭若虛の云ふ氣韻と頗る似寄つた所がある。\\
there are many aspects that are very similar to what Guo Ruoxu called \textit{qiyun}.

\vspace{1em}
\plabel{$\blacktriangleright$}%
そうして見ると文人畫はその主義の上の{\small\relax{}[sic]}於て、\\
When looked at in that way, as for literati paintings, in these principles,\\
頗る近世的なるものであるかの如くに見える。\\
it seems like a very modern concept.

\vspace{1em}
\plabel{$\blacktriangleright$}%
左樣に近世的なるものが支那に於て古くから開けてゐたと云ふ事は、\\
That a modern concept like that is something established in China since the ancient time\\
寧ろ驚くべき事のやうである。\\
is something that must be very surprising.

\vspace{1em}
\plabel{$\blacktriangleright$}%
けれども又文人畫の原理が、\\
But, although also the principles of literati paintings\\
大體に於て最近の西洋表現藝術の主張と一致するものはあるにしても、\\
have something in common with the concepts of the recent Expressionism art of the west in general,\\
文人畫と表現派の藝術とを全く同じものと視る事は出來ない。\\
one can not view the literati painting and the art of Expressionism as something totally the same.

\vspace{1em}
\plabel{$\blacktriangleright$}%
兩者大體に於て主張を同うするものがあるとは雖も、\\
Although there is something identical in the principles of both in general,\\
尚ほその實際に就て詳しく分析して考へて見ると、\\
still when one try to analyze and contemplate regarding the reality in detail,\\
又大いに異ふ所がないとは云へないので、\\
it can't be said that there are no huge differences, and\\
文人畫は文人畫でおのづから特殊の性質を持つてゐる。\\
the literati painting possesses its own special characteristics.

\end{document}
