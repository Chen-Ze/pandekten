\documentclass{ctexart}

\usepackage{pandekten}
\usepackage{dashrule}

\makeatletter
\newcommand*{\shifttext}[1]{%
  \settowidth{\@tempdima}{#1}%
  \hspace{-\@tempdima}#1%
}
\newcommand{\plabel}[1]{%
\shifttext{\textbf{#1}\quad}%
}
\newcommand{\prule}{%
\begin{center}%
\hdashrule[0.5ex]{.99\linewidth}{1pt}{1pt 2.5pt}%
\end{center}%
}

\makeatother

\newcommand{\minusbaseline}{\abovedisplayskip=0pt\abovedisplayshortskip=0pt~\vspace*{-\baselineskip}}%

\setlength{\parindent}{0pt}

\title{Assignment 1}
\author{Ze (Zack) Chen}

\begin{document}

\maketitle

\plabel{$\blacktriangleright$}%
唐宋時代といふことは普通に用ふる語なるが、\\
As for the thing call the period of Tang-Song, although it is a word that (people) commonly use, \\
歴史特に文化史的に考察すると、\\
if (one) considers history in particular cultral history, \\
實は意味をなさぬ語である。\\
in fact it is a word that forms no meaning.

\vspace{1em}
\plabel{$\blacktriangleright$}%
それは唐代は中世の終末に屬し、\\
As for this, with (the thing that) the Tang dynasty belongs to the end of the middle age, \\
而して宋代は近世の發端となりて、\\
but the Song dynasty becomes the beginning of the recent age, and \\
其間に唐末より五代に至る過渡期を含むを以て、\\
in between (there) contains a transition period from the end of the Tang dynasty to the period of Five Dynasties, \\
唐と宋とは文化の性質上著しく異りたる點がある。\\
for Tang and Song there are points of the property of culture that had significantly differentiated.

\vspace{1em}
\plabel{$\blacktriangleright$}%
但し從來の歴史家は多く朝代によりて時代を區劃したから、\\
Although since historians hitherto nevertheless divided ages based mainly on dynasties,\\
唐宋とか元明清とか一の成語になつて居るが、\\
(words like) Tang-Song and Yuan-Ming-Qing have become one set phrase,\\
學術的にはかゝる區劃法を改むる必要がある。\\
academically speaking, the necessity of changing the concerning division method exists.

\vspace{1em}
\plabel{$\blacktriangleright$}%
但し今は便宜上、\\
However, as for now, for convenience,\\
普通の歴史區劃に從ひ唐宋時代の名を用ひて、\\
following the common division of history, (we) use the name of the period of Tang-Song, and\\
支那の中世より近世に移る間の變化の状態を總括して説いて見ようと思ふ。\\
(we) think that let's try to summarize and explain the status of the change in the period of China moving from the middle age to the recent age.

\prule

\plabel{$\blacktriangleright$}%
中世と近世との文化の状態は、\\
As for the status of culture of the middle age and the recent age,\\
如何なる點に於て異るかといふに、\\
for(?) at what kind of points (they) differentiate,\\
政治上よりいへば貴族政治が廢頽して君主獨裁政治が起りたる事で、\\
speaking beyond(?) politically, with the thing that the politics of the lord become(tense?) corrupt and the dictatorship of the monarch had come into being,\\
貴族政治は六朝から唐の中世までを最も盛なる時代とした。\\
as for the politics of the lord, (we) regard (the period) from Six Dynasties to the middle of the Tang dynasty as the most prosperous period.

\vspace{1em}
\plabel{$\blacktriangleright$}%
勿論此貴族政治は、\\
Of course this politics of the lord,\\
上古の氏族政治とは全く別物で、\\
(with that it is) a totally different thing from the politics of the clan of the old age,\\
周代の封建制度とも關係がなく、\\
without relation to the feudal system of the Zhou dynasty either,\\
一種特別のものである。\\
is a kind of special thing.


\vspace{1em}
\plabel{$\blacktriangleright$}%
此時代の支那の貴族は、\\
As for the lord in China at this age,\\
制度として天子から領土人民を與へられたといふのではなく、\\
it is not said that as a system the territory was given to the people by the Son of Heaven,\\
其家柄が自然に地方の名望家として永續したる關係から生じたるもので、\\
with (the thing that) its family (is) naturally the thing that arised because (of the relation that) it perpetuated as the local renowned family,\\
所謂郡望なるものゝ本體がこれである。\\
this is the main body of the thing that is the so-called prefectural gentry.

\vspace{1em}
\plabel{$\blacktriangleright$}%
それ等の家柄は皆系譜を重んじ、\\
These families all treasure the pedigree, and\\
其ために當時系譜學が盛んになつた位である。\\
because of this, at that time genealogy is(tense?) (at) the position that was prosperous.

\vspace{1em}
\plabel{$\blacktriangleright$}%
現に存在する諸書の中でも、\\
Also among the books that exist now,\\
唐書の宰相世系表は即ち其有樣を示したもの、\\
the table of lineage of Chancellors from the book of Tang (is) namely the thing that showed the state of it,\\
又、\\
and,\\
李延壽の南北史の中には、\\
inside \textit{the History of the Northern and Southern Dynasties} by Yanshou Li,\\
朝代に拘らず各家の人を祖先から子孫まで續けて纏まれる傳を書いたから、\\
although, because (he) write biographies that have continuously collected from the ancestors to the descendants of the people of every family regardless of the dynasty,\\
人のために家傳を作つた體裁になつたといふ非難を受けたが、\\
(he) received criticism that says (the book) became (in) the style that (one) makes biographies of families for people,\\
これは南北朝時代の實際状態が無意識の裡に歴史の上に現れたのである。\\
this is (the thing) that the factual status of the Northern and Southern Dynasties shows on history in unconsciousness.

\end{document}
