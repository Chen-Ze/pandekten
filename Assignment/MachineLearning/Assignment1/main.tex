\documentclass{article}

\usepackage{pandekten}
\usepackage{dashrule}

\makeatletter
\newcommand*{\shifttext}[1]{%
  \settowidth{\@tempdima}{#1}%
  \hspace{-\@tempdima}#1%
}
\newcommand{\plabel}[1]{%
\shifttext{\textbf{#1}\quad}%
}
\newcommand{\prule}{%
\begin{center}%
\hdashrule[0.5ex]{.99\linewidth}{1pt}{1pt 2.5pt}%
\end{center}%
}

\makeatother

\setlength{\parindent}{0pt}

\title{Assignment 1}
\author{Ze Chen}

\begin{document}

\maketitle

\plabel{1 (a.i)}%
Neither.
The relation does not contain $\qty(u,u)$ and $(v,v)$.

\plabel{(a.ii)}%
Equivalence relation.
The relation yields $\pi = \qty{\qty{u,v},\qty{x},\qty{y},\qty{z}}$.

\plabel{(a.iii)}%
Similarity relation.
The relation is reflexive and symmetric.
However, $(x,y)$ and $(y,z)$ are in the relation but not $(x,z)$.

\plabel{(b.i)}%
$\qty{\qty{x,y,z},\qty{u},\qty{v}}$.

\plabel{(b.ii)}%
$\qty{\qty{x,y,z},\qty{u,v}}$.

\plabel{(b.iii)}%
$\qty{\qty{x,y,z},\qty{u,v}}$.

\plabel{(c)}%
$\mathcal{E}_1 \land \mathcal{E}_2 = \qty{\qty{x},\qty{y},\qty{z},\qty{u},\qty{v}}$, $\mathcal{E}_1 \lor \mathcal{E}_2 = \qty{\qty{x,y,z,u,v}}$.

\prule

\plabel{2 (a)}%
Let $H=(V,E)$.
Let $\mathcal{E}$ denotes the equivalence relation corresponding to the partition.
Then the iteration of single linkage clustering is equivalent\footnote{Except that $\min E$ may be already contained in $\mathcal{E}$ and the partition doesn't change at such step.} to
\begin{align*}
    \mathcal{E} &\leftarrow \overline{\mathcal{E}\cup\qty{\min E}}, \\
    E &\leftarrow \Set*{e\in E}{e>\min E}.
\end{align*}
The iteration stops when
\[ d(\min E) > r. \]
The final parition $\pi$ is given by the connected components of $H=(V,E')$ where
\[ E' = \Set*{e\in E}{d(e) \le r}, \]
and may therefore be obtained using DFS, or any disjoint-set data structure.

\plabel{(b)}%
A counter should be initialized to $m$ before the iteration starts.
Then, at each iteration of $m$, the counter is
\begin{enumerate}
    \item decreased by one if $\min E\notin \mathcal{E}$, i.e. if the endpoints of $\min E$ belong to different clusters, which may be determined quickly using disjoint-set data structures; or
    \item intact otherwise.
\end{enumerate}
The loop stops when the counter becomes less than $k$.

\prule

\plabel{3 (a)}%
We assume that the orthogonal projection is with respect to the inner product defined by
\[ \langle A,B \rangle = \tr(A^\intercal B). \]
Therefore, the projection is given by
\[ \frac{\langle uv^\intercal, A \rangle}{\norm{uv^\intercal}^2} uv^\intercal = \frac{u^\intercal A v}{\norm{u}^2\norm{v}^2}uv^\intercal. \]

\plabel{(b)}%
Let
\begin{align*}
    U &= \begin{pmatrix}
        u_1 & \cdots & u_n
    \end{pmatrix}, \\
    \Lambda &= \operatorname{diag}(\lambda_1,\cdots,\lambda_n), \\
    V &= \begin{pmatrix}
        v_1 & \cdots & v_n
    \end{pmatrix}.
\end{align*}
Then $A = U\Lambda V^\intercal$ and therefore
\begin{align*}
    \norm{A} &= \sqrt{\tr(A^\intercal A)} \\
    &= \sqrt{\tr(V\Lambda U^\intercal U\Lambda V^\intercal)} \\
    &= \sqrt{\tr{\Lambda {\cancel{U^\intercal U}}\Lambda {\cancel{V^\intercal V}}}} \\
    &= \norm{\lambda}.
\end{align*}

\prule

\plabel{4 (a)}%
Assuming $A\cap B = \varnothing$.
\begin{equation}
    \label{eq:mu_ab}
    \mu_{A\cup B} = \frac{\abs{A} \mu_A + \abs{B}\mu_B}{\abs{A} + \abs{B}} = t \mu_A + (1-t) \mu_B
\end{equation}
and therefore $\mu_{A\cup B}$ lies on the line segment joining $\mu_A$ and $\mu_B$.
\Cref{eq:mu_ab} may not hold if $A\cap B \neq \varnothing$.

\plabel{(b.a)}%
\abovedisplayskip=0pt\abovedisplayshortskip=0pt~\vspace*{-\baselineskip}%
\begin{align*}
    \Delta(A\cup B, C) &= \min\Set*{d(x,c)}{x\in A\cup B, c\in C} \\
    &= \min\qty(\Set*{d(a,c)}{a\in A, c\in C} \cup \Set*{d(b,c)}{x\in B, c\in C}) \\
    &= \min\qty{\min\Set*{d(a,c)}{a\in A, c\in C}, \min\Set*{d(b,c)}{x\in B, c\in C}} \\
    &= \min\qty{\Delta(A,C), \Delta(B,C)}.
\end{align*}

\plabel{(b.b)}%
\abovedisplayskip=0pt\abovedisplayshortskip=0pt~\vspace*{-\baselineskip}%
\begin{align*}
    \Delta(A\cup B, C) &= \max\Set*{d(x,c)}{x\in A\cup B, c\in C} \\
    &= \max\qty(\Set*{d(a,c)}{a\in A, c\in C} \cup \Set*{d(b,c)}{x\in B, c\in C}) \\
    &= \max\qty{\max\Set*{d(a,c)}{a\in A, c\in C}, \max\Set*{d(b,c)}{x\in B, c\in C}} \\
    &= \max\qty{\Delta(A,C), \Delta(B,C)}.
\end{align*}

\plabel{(b.c)}%
We assume that $A\cap B = \varnothing$.
\begin{align*}
    \Delta(A\cup B, C) &= \frac{1}{\abs{A\cup B}\abs{C}} \sum_{(x,c)\in (A\cup B)\times C} d(x,c) \\
    &= \frac{1}{\abs{A\cup B}\abs{C}} \qty(\sum_{(a,c)\in A\times C} d(a,c) + \sum_{(b,c)\in B\times C} d(b,c)) \\
    &= \frac{1}{(\abs{A}+\abs{B})\abs{C}} \qty(\abs{A}\abs{C} \Delta(A,C) + \abs{B}\abs{C} \Delta(B,C)) \\
    &= \frac{\abs{A}}{\abs{A} + \abs{B}} \Delta(A,C) + \frac{\abs{B}}{\abs{A} + \abs{B}} \Delta(B,C).
\end{align*}
Therefore,
\begin{align*}
    \alpha &= \frac{\abs{A}}{\abs{A} + \abs{B}}, \\
    \beta &= \frac{\abs{B}}{\abs{A} + \abs{B}}.
\end{align*}

% \bibliographystyle{plain}
% \bibliography{main}

\end{document}
