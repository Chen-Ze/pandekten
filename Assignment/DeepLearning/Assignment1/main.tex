\documentclass{article}

\usepackage{pandekten}
\usepackage{dashrule}

\makeatletter
\newcommand*{\shifttext}[1]{%
  \settowidth{\@tempdima}{#1}%
  \hspace{-\@tempdima}#1%
}
\newcommand{\plabel}[1]{%
\shifttext{\textbf{#1}\quad}%
}
\newcommand{\prule}{%
\begin{center}%
\hdashrule[0.5ex]{.99\linewidth}{1pt}{1pt 2.5pt}%
\end{center}%
}

\makeatother

\newcommand{\minusbaseline}{\abovedisplayskip=0pt\abovedisplayshortskip=0pt~\vspace*{-\baselineskip}}%

\setlength{\parindent}{0pt}

\title{Assignment 1}
\author{Ze Chen}

\begin{document}

\maketitle

\plabel{1.1}%
Initialization:$\begin{cases}
    \qty{1,2,3,6,7}, & \vb{\mu} = (4,3), \\
    \qty{4}, & \vb{\mu} = (0,4), \\
    \qty{5}, & \vb{\mu} = (8,0).
\end{cases}$.
\par
Step 1: $\begin{cases}
    \qty{1,2,3,6,7}, & \vb{\mu} = (4,4), \\
    \qty{4}, & \vb{\mu} = (0,4), \\
    \qty{5}, & \vb{\mu} = (8,0).
\end{cases}$
\par
It converges after 1 step.

\prule

\plabel{1.2}%
See the document attached at the end.

\prule

\plabel{1.3 (I)}%
Design freedom: $9 \times 3 = 27$.
Number of parameters: $4\times 1 + 4 + 5\times 4 + 5 = 33$.
$f(x)$ is a polynomial of order $27$ since it's a composition of $3$ cubic polynomials.
The formula for intrinsic degree is $\binom{n+d}{n}$ where $n$ is the dimension of the vector space and $d$ is the degree of polynomial.

\plabel{(II)}%
It's a polynomial of order $2^3 = 8$.
The design freedom is $2^3 = 8$.

\prule

\plabel{1.4}%
The iteration is given by
\[ \vb{w}' = \vb{w} - 2\gamma \vb{F}^\intercal (\vb{F}\vb{w} - y). \]
Fixed points exist at the solutions of $\vb{F}^\intercal(\vb{F}\vb{w} - y) = 0$, i.e. stationary points of the target function, which is the minimal since the function is convex.
It remains to show that the iteration is a contraction mapping on the complement of the null space of $\vb{F}^\intercal \vb{F}$, i.e. it converges.
Note that
\[ \norm{\vb{w}' - \vb{v}'} = \norm{(\mathbbm{1} - 2\gamma\vb{F}^\intercal \vb{F}) (\vb{w} - \vb{v})}. \]
For all $0 < \gamma < 1/\sigma_1$ where $\sigma_1$ is the minimal nonzero eigenvalue of $\vb{F}^\intercal \vb{F}$, the iteration is a contraction mapping.
Therefore, the iteration converges for $\gamma$ small enough.

\includepdf[pages=-,pagecommand={}]{2023SpHW1.pdf}

\end{document}
