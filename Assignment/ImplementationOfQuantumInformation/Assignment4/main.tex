\documentclass{article}

\usepackage{pandekten}

\usepackage{quantikz}

\usepackage{dashrule}

\makeatletter
\newcommand*{\shifttext}[1]{%
  \settowidth{\@tempdima}{#1}%
  \hspace{-\@tempdima}#1%
}
\newcommand{\plabel}[1]{%
\shifttext{\textbf{#1}\quad}%
}
\newcommand{\prule}{%
\begin{center}%
\hdashrule[0.5ex]{.99\linewidth}{1pt}{1pt 2.5pt}%
\end{center}%
}

\makeatother

\newcommand{\minusbaseline}{\abovedisplayskip=0pt\abovedisplayshortskip=0pt~\vspace*{-\baselineskip}}%

\setlength{\parindent}{0pt}

\title{Assignment 4}
\author{Ze Chen}

\begin{document}

\maketitle

\plabel{1 (1)}%
\begingroup\minusbaseline
\begin{center}
    \begin{tabular}{c>{\(}c<{_L\)}}
        000 & 0 \\
        001 & 0 \\
        010 & 0 \\
        011 & 1 \\
        100 & 0 \\
        101 & 1 \\
        110 & 1 \\
        111 & 1
    \end{tabular}
\end{center}
\endgroup

\plabel{(2)}%
Circled plus denotes CNOT gate. 
\begin{center}
    \begin{quantikz}
        \lstick{$\ket{\psi}$} & \ctrl{1} & \ctrl{2} & \qw \\
        \lstick{$\ket{0}$} & \targ{} & \qw & \qw \\
        \lstick{$\ket{0}$} & \qw & \targ{} & \qw
    \end{quantikz}
\end{center}

\plabel{(3)}%
\begingroup\minusbaseline
\begin{center}
    \begin{tabular}{>{\(}c<{\)}ccc}
        \toprule
        & $Z_1 Z_2$ & $Z_2 Z_3$ & $Z_1 Z_3$ \\
        \midrule
        \ket{000} & $+1$ & $+1$ & $+1$ \\
        \ket{001} & $+1$ & $-1$ & $-1$ \\
        \ket{010} & $-1$ & $-1$ & $+1$ \\
        \ket{011} & $-1$ & $+1$ & $-1$ \\
        \ket{100} & $-1$ & $+1$ & $-1$ \\
        \ket{101} & $-1$ & $-1$ & $+1$ \\
        \ket{110} & $+1$ & $-1$ & $-1$ \\
        \ket{111} & $+1$ & $+1$ & $+1$ \\
        \bottomrule
    \end{tabular}
\end{center}
\endgroup

\plabel{(4)}%
We need only two parity measurements since the third is always the product of the first two.
\begingroup\minusbaseline
\begin{center}
    \begin{tabular}{>{\(}c<{\)}cccc}
        \toprule
        & $Z_1 Z_2$ & $Z_2 Z_3$ & $Z_1 Z_3$ & Operation \\
        \midrule
        \ket{000} & $+1$ & $+1$ & $+1$ & $1$ \\
        \ket{001} & $+1$ & $-1$ & $-1$ & $X_3$ \\
        \ket{010} & $-1$ & $-1$ & $+1$ & $X_2$ \\
        \ket{011} & $-1$ & $+1$ & $-1$ & $X_1$ \\
        \ket{100} & $-1$ & $+1$ & $-1$ & $X_1$ \\
        \ket{101} & $-1$ & $-1$ & $+1$ & $X_2$ \\
        \ket{110} & $+1$ & $-1$ & $-1$ & $X_3$ \\
        \ket{111} & $+1$ & $+1$ & $+1$ & $1$ \\
        \bottomrule
    \end{tabular}
\end{center}
\endgroup

\prule

\plabel{(1)}%
The state is encoded in a three-qubit entangled state by CPhase gate.
The error is corrected by CCNot gate.

\plabel{(2)}%
\begingroup\minusbaseline%
\begin{enumerate}
    \item $Q_2$ is moved into resonance with the avoided crossing to transfer the population of $\ket{111}$ and $\ket{102}$.
    \item $Q_2$ is moved suddenly further up in frequency to where its two-qubit phase with $Q_3$ is cancelled during the gate by accumulating a multiple of $2\pi$.
    \item $Q_1$ is moved up to initiate the interaction between $\ket{102}$ and $\ket{003}$.
    \item The population in $\ket{102}$ is transferred back to $\ket{111}$ by a reversing swap procedure.
    \item The two-qubit phase between $Q_1$ and $Q_2$ is cancelled with an additional adiabatic interaction.
\end{enumerate}
\endgroup

\plabel{(3)}%
If at most one of $Q_1$ and $Q_3$ is $\ket{1}$, then either no error on $Q_2$ occurred.
If both $Q_1$ and $Q_2$ is $\ket{1}$, then $Q_2$ is flipped and the CCNot gate flips $Q_2$ to get the correct state.

\plabel{(4)}%
If the error only occurs on $Q_2$ with $Q_2 = \ket{0}$ initially, before the CCNot gate the state is $\alpha\ket{000} + \beta\ket{111}$, and then the CCNot gate returns $Q_2$ to the correct state for both components.

\plabel{(5)}%
Final fidelity $\sim p^2$ since correction fails only if at least two errors occurs, which has probability $\sim p^2$.
\par
Even if no qubit occurs, the errors due to the extra gates and in $Q_1$ and $Q_3$ may incur error on $Q_2$.

\plabel{(6)}%
\begingroup\minusbaseline
\begin{itemize}
    \item Energy relaxation of the three qubits during the \SI{85}{\nano\second} measurement.
    \item Qubit transition frequency drift during dataset collection.
    \item Decoherence.
\end{itemize}
\endgroup

\plabel{(7)}
The phase-flip correction circuit is equivalent to the bit-flip one by adding a global $X(\pi/2)$ rotation and $X(-\pi/2)$ rotation before and after the error.

% \bibliographystyle{plain}
% \bibliography{main}

\end{document}
