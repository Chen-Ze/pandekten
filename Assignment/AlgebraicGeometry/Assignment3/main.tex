\documentclass{article}

\usepackage{pandekten}
\usepackage{dashrule}

\makeatletter
\newcommand*{\shifttext}[1]{%
  \settowidth{\@tempdima}{#1}%
  \hspace{-\@tempdima}#1%
}
\newcommand{\plabel}[1]{%
\shifttext{\textbf{#1}\quad}%
}
\newcommand{\prule}{%
\begin{center}%
\hdashrule[0.5ex]{.99\linewidth}{1pt}{1pt 2.5pt}%
\end{center}%
}

\makeatother

\newcommand{\minusbaseline}{\abovedisplayskip=0pt\abovedisplayshortskip=0pt~\vspace*{-\baselineskip}}%

\setlength{\parindent}{0pt}

\title{Assignment 3}
\author{Ze Chen}

\begin{document}

\maketitle

\plabel{6.2}%
Since $X = V(yz, xz, xy)$, it suffices to show that $I = (yz,xz,xy) \subset k[x,y,z]$ cannot be generated by $2$ elements.
\par
Let $M = (x,y,z)$ be a maximal ideal in $k[x,y,z]$.
Then $I/MI$ is a vector space over $k[x,y,z]/M \cong k$.
If $I$ can be generated by $2$ elements, then $I/M$ is at most $2$-dimensional.
However, 
\[ I/MI \cong \Set*{a \cdot yz + b \cdot xz + c \cdot xy}{a,b,c\in k}, \]
which is $3$ dimensional.

\plabel{6.4}%
First we choose coordinate of $\mathbb{A}^2$ such that all points of $S$ have different $x$-coordinates:
Let $L$ denote the set of lines joining (at least) $2$ points in $S$.
Then $L$ is a finite set.
Since $k$ is infinite, we could choose a line $Y = V(ax+by)$ that has slope different from all lines in $L$.
Now we define $\pi(x,y) = ax+by$.
Then $\pi(x,y)$ is injective on $S$, since $\pi(x_1,y_1) = \pi(x_2,y_2)$ implies $Y$ has the same slope with the line joining $(x_1,y_1)$ and $(x_2,y_2)$.
Now we apply a coordinate transformation
\[ \begin{pmatrix}
    x' \\ y'
\end{pmatrix} = \begin{pmatrix}
    a & b \\ c & d
\end{pmatrix} \begin{pmatrix}
    x \\ y
\end{pmatrix} \]
with $c,d$ chosen such that the transformation is invertible (this can be done since $k$ is infinite).
Then points in $S$ have different $x'$-coordinates.

\par
Now we apply the Lagrange interpolation formula to get a polynomial such that $y' = f(x')$ in $S$.
Then $S$ is defined by
\[ S = V(y'-f(x'), \prod_i(x-\alpha_i)), \]
where $\alpha_i$ denotes the $x$-coordinate of the $i$-the point in $S$.
We can pullback and get the polynomial in $(x,y)$.

\plabel{6.10.i}%
First we prove $\dim \Gamma = (\sum_i \nu_{m,n_i}) - 1$.
The fiber of $\psi\colon\Gamma\to\mathbb{P}^n$ is given by $\psi^{-1}(y) = S_0 \times \cdots \times S_n$ where each $S_i \subset \mathbb{P}^{\nu_{n,m_i}}$ is a hypersurface defined by a single linear equation (i.e. the equation for the coefficients such that the homogeneous polynomial is zero at $y$) and is therefore irreducible.
$\psi^{-1}(y)$ is irreducible since it is a product of irreducible projective varieties, and
\[ \dim \psi^{-1}(y) = \sum_{i} (\nu_{m,n_i} - 1) = \qty(\sum_i \nu_{m,n_i}) - 1 - n. \]
With theorem 1.26, we know that $\Gamma$ is irreducible, since $\psi$ is regular, $\mathbb{P}^n$ is irreducible, and each fiber $\psi^{-1}(y)$ is irreducible and has the same dimension.
\par
Now we can use theorem 1.25 (ii) and find that the fiber dimension for each $y$ in some nonempty open set $U\subset \mathbb{P}^n$ is given by
\[ \dim \psi^{-1}(y) = \dim \Gamma - \dim \mathbb{P}^n = \dim \Gamma - n. \]
Therefore,
\[ \dim\Gamma = \qty(\sum_i \nu_{m,n_i}) - 1. \]

\plabel{ii}%
Now we prove $\dim \Gamma = \dim \varphi(\Gamma)$.
We can apply theorem 1.25 since $\varphi(\Gamma)$ is also irreducible.
Therefore, $\dim \Gamma \ge \dim \varphi(\Gamma)$ and for any $z\in \varphi(\Gamma)$,
\[ \dim \varphi(\Gamma) \ge \dim \Gamma - \dim f^{-1}(z). \]
Find any tuple of $n+1$ hypersurfaces such that their intersection is just a single point.
Then we have $\dim\varphi(\Gamma) \ge \dim\Gamma$ and the statement is proven.

\plabel{iii}%
Now we prove $\varphi(\Gamma) \subset \prod_i \mathbb{P}^{\nu_{n,m_i}}$ can be defined by a single polynomial $R$.
This is just theorem 1.21$'$ given the dimension of $\varphi(\Gamma)$.
\par
It is clear that $\varphi(\Gamma)$ consists of all tuples of $n+1$ homogeneous polynomials of given degrees $\qty{m_i}_{0\le i\le n}$ that has a nonzero solution.

\plabel{iv}%
If $F_0,\cdots,F_n$ are linear, then $R$ is just the determinant.

\prule

\plabel{1.10}%
Let the line be given by
\[ L = \Set*{(x_1,\cdots,x_n)}{(x_1,\cdots,x_n) = (a_0,\cdots,a_n) + t (\ell_1,\cdots,\ell_n)}. \]
We may plug this into the definition of the hypersurface $f(x_1,\cdots,x_n) = 0$, and find a cubic polynomial in $t$ which determines the intersection with $t$,
\[ f(t) = f(x_1(t),\cdots,x_n(t)) =  c_3 t^3 + c_2 t^2 + c_1 t^1 + c_0 = 0. \]
However, at the intersection we know
\[ \pdv{t} f(t) = \sum \ell_i \pdv{x_i} f(x_1,\cdots,x_n) = 0 \]
since the intersection is at a singular point.
Let the two intersections be given by $t_1$ and $t_2$ where $t_1\neq t_2$, then $f$ should satisfy
\[ f(t) = h(t) (t-t_1)^2(t-t_2)^2. \]
But $f$ is at most cubic.
Therefore, $h(t)=0$ and $f(t)=0$, i.e. $f$ vanishes on the line, i.e. the line is contained in the hypersurface.

\plabel{1.12}%
We have to assume that $F$ is irreducible.
In such case, a singular point $x$ have $\dim\Theta_{x} > n-1$, i.e. the tangent space is the whole space.
Therefore the defining equation of the tangent space has all coefficients zero, i.e.
\[ \pdv{F}{x_i}(x_0,\cdots,x_n) = 0,\quad i = 0,\cdots,n. \]
That the hypersurface passes through $x$ requires $F(x_0,\cdots,x_n) = 0$.
\par
If $\deg F$ is not divisible by the character of the field, then
\begin{align}
    \label{eq:tangent_through} \sum_i x_i\pdv{F}{x_i}(x_0,\cdots,x_n) = (\deg F) \cdot F(x_0,\cdots,x_n)
\end{align}
and it follows that $F(x_0,\cdots,x_n) = 0$.

\plabel{1.15}%
Using the condition in execrise 1.12 we require
\begin{align*}
    x_1^2 x_2^2 + x_2^2 x_0^2 + x_0^2 x_1^2 - x_0 x_1 x_2 x_3 &= 0, \\
    2 x_1 x_2^2 + 2x_1 x_0^2 - x_0 x_2 x_3 &= 0, \\
    2 x_2 x_1^2 + 2x_2 x_0^2 - x_0 x_1 x_3 &= 0, \\
    2 x_0 x_2^2 + 2x_0 x_1^2 - x_1 x_2 x_3 &= 0, \\
    -x_0 x_1 x_2 &= 0.
\end{align*}
Solving these we find the singular points
\[ S = V(x_1,x_2) \cup V(x_0,x_2) \cup V(x_1,x_2). \]

\plabel{1.18.a}%
It's clear that if $X$ is a line then $\varphi(X)$ is a point.
\par
If $\varphi(X) = (u_0:u_1:u_2)$ is a constant, then let $L = V(u_0 x_0 + u_1 x_1 + u_2 x_2)$.
From equation \eqref{eq:tangent_through} above we know that for each $p\in X$, $L$ passes through $p$.
Therefore $X\subset L$.
Since $X$ is irreducible and has dimension $1$, $X = L$.

\plabel{b}%
From exercise 1.12 we know
\[ x \text{ is nonsingular} \Leftrightarrow u_0\neq 0 \text{ or } u_1\neq 0 \text{ or } u_2 \neq 0. \]
Since $\varphi(x) = (u_0:u_1:u_2)$ are given by polynomials, the condition is equivalent to the requirement that $\varphi$ is regular at $x$.

\end{document}
