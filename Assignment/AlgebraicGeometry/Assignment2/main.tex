\documentclass{article}

\usepackage{pandekten}
\usepackage{dashrule}

\makeatletter
\newcommand*{\shifttext}[1]{%
  \settowidth{\@tempdima}{#1}%
  \hspace{-\@tempdima}#1%
}
\newcommand{\plabel}[1]{%
\shifttext{\textbf{#1}\quad}%
}
\newcommand{\prule}{%
\begin{center}%
\hdashrule[0.5ex]{.99\linewidth}{1pt}{1pt 2.5pt}%
\end{center}%
}

\makeatother

\newcommand{\minusbaseline}{\abovedisplayskip=0pt\abovedisplayshortskip=0pt~\vspace*{-\baselineskip}}%

\setlength{\parindent}{0pt}

\title{Assignment 2}
\author{Ze Chen}

\begin{document}

\maketitle

\plabel{4.1}%
$\Leftarrow$:
Assume $U = X \cup Y$ is a union of proper closed subsets $X$ and $Y$.
Then $\overline{U} = \overline{X} \cup \overline{Y}$ and both are proper subsets since $\overline{X} \cap U = X\cap U \neq U$, $\overline{Y} \cap U = Y \cap U \neq U$.
\par
$\Rightarrow$:
Assume $\overline{U} = A \cup B$ is a union of proper closed subsets $A$ and $B$.
Then $U$ is not contained $A$ nor $B$ since otherwise $A$ or $B$ would be a smaller closed set containing $U$ than $\overline{U}$.
Then
\[ U = (A\cap U) \cup (B\cap U) \]
is a union of proper subsets.

\plabel{4.3}%
Clearly $k[X] \cong k[x,y]$.
But $(x,y) \subsetneq k[x,y]$ defines empty set in $X$.

\plabel{4.9}%
$f^{-1}$ could be defined by
\[ (y_0:y_1:y_2) \mapsto \qty(1:\frac{y_0}{y_1}:\frac{y_0}{y_2}) \text{ or } \qty(\frac{y_1}{y_0}:1:\frac{y_1}{y_2}) \text{ or } \qty(\frac{y_2}{y_0}:\frac{y_2}{y_1}:1). \]
$f$ is not regular on
\[ V = \qty{(1:0:0), (0:1:0), (0:0:1)}. \]
$f^{-1}$ is not regular on
\[ V = \qty{(1:0:0), (0:1:0), (0:0:1)}. \]
The open subsets of $\mathbb{P}^2\setminus V$ are mapped isomorphically by $f$.

\prule

\plabel{5.1}%
If $\varphi(\mathbb{P}^n \times \mathbb{P}^m)$ is contained in a linear space then there are coefficients
\[ \Set*{a_{ij}\in k}{0\le i\le n, 0\le j \le m} \]
such that for all $v\in \mathbb{P}^n$ and $u\in \mathbb{P}^m$,
\[ \sum_{ij} a_{ij} v_i u_j = 0. \]
This can happen only if $a_{ij} \equiv 0$ for all $i$ and $j$.

\plabel{5.8}%
Lines through origin in $\mathbb{A}^r$ corresponds to points in $\mathbb{P}^{r-1}_\infty$.
Projection along the line $L$ is $\pi_L: \overline{X} \rightarrow \mathbb{P}^{r-1}$ in Theorem 1.15.
\par
First we prove $S \subset \overline{X} \cap \mathbb{P}^{r-1}_\infty$.
If $L\in \mathbb{P}^{r-1}_\infty$ is not in $\overline{X}$, then from theorem 1.15 we know $\pi_L$ is finite.
\par
Then we prove $\overline{X} \cap \mathbb{P}^{r-1}_\infty \subset S$.
\textit{%
I am not sure how should I do this part.
Let $X = V(y-x^2) \subset \mathbb{A}^2$.
Then $\overline{X} \cap \mathbb{P}^1_\infty$ is a single point and corresponds to the line $x=0$.
Projection parallel to this line is $\pi(x,y) = x$.
Since $k[x,y]/(y-x^2) \cong k[x]$, the map is finite.
}
\par
For $X = V(xy = 1)$, we find $S = \qty{0,\infty} \in \mathbb{P}^1_\infty$, i.e. the line $x=0$ and the line $y=0$.

\plabel{5.9}%
Let $U,V\subset \mathbb{P}^n$ be isomorphic to affine closed sets.
Then $U\times V \subset \mathbb{P}^n \times \mathbb{P}^n \hookrightarrow \mathbb{P}^N$ is also isomorphic to an affine closed subset under some $\varphi\colon U\times V\to \mathbb{A}^m$, where $\hookrightarrow$ is the Segre embedding.
\par
Let $\Delta \subset \mathbb{P}^n \times \mathbb{P}^n \hookrightarrow \mathbb{P}^N$ be the diagonal.
Then $\varphi(\Delta)$ is a closed set since it's an image of a projective variety, and $\varphi(\Delta) \cap \varphi(U\times V)$ is a closed set in $\mathbb{A}^{m}$.
Therefore the preimage $\Delta \cap (U\times V)$ is an affine variety.
\par
Now $U\cap V$ is isomorphic to $\Delta \cap (U\times V)$ by the map $x\mapsto (x,x)$.
Therefore, $U\cap V$ is also an affine set.

\plabel{5.11}%
Let $D(g)$ be a principal affine open set in $Y$.
Then $f^{-1}(D(g)) = D(f^*(g))$ is also a principal affine open set in $X$.


\end{document}
