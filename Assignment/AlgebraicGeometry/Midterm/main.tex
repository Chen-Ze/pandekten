\documentclass{article}

\usepackage{pandekten}
\usepackage{dashrule}

\makeatletter
\newcommand*{\shifttext}[1]{%
  \settowidth{\@tempdima}{#1}%
  \hspace{-\@tempdima}#1%
}
\newcommand{\plabel}[1]{%
\shifttext{\textbf{#1}\quad}%
}
\newcommand{\prule}{%
\begin{center}%
\hdashrule[0.5ex]{.99\linewidth}{1pt}{1pt 2.5pt}%
\end{center}%
}

\makeatother

\newcommand{\minusbaseline}{\abovedisplayskip=0pt\abovedisplayshortskip=0pt~\vspace*{-\baselineskip}}%

\setlength{\parindent}{0pt}

\title{Midterm MAT 457}
\author{Ze Chen}

\begin{document}

\maketitle

\plabel{1 (1)}%
$C$ is a hypersurface of $\mathbb{P}^3$.
The condition of singularity is given by
\begin{align*}
    F &= y^{n-1}z - x^n = 0, \\
    \partial_x F &= -n x^{n-1} = 0, \\
    \partial_y F &= (n-1) y^{n-2}z  = 0, \\
    \partial_z F &= y^{n-1} = 0.
\end{align*}
Since the characteristic of the field is $0$, these conditions require (since $n\ge 3$)
\[ x = 0, y = 0, z\in k. \]
That is, the only singular point is $(x:y:z) = (0:0:1)$.

\plabel{(2)}%
To prove that it is birational to $\mathbb{P}^1$, it suffices to prove that $C$ and $\mathbb{P}^1$ have a isomorphic open subset.
Let
\[ C' = \Set*{(x,y,z)\in C}{y\neq 0} = C \cap \mathbb{A}^2_1. \]
Then
\[ C' = \Set*{(x,z)\in \mathbb{A}^2}{z - x^n = 0}. \]
This is the image of the regular function on $\mathbb{A}^1$ given by $x \mapsto x^n$, and therefore $C'$ is isomorphic to $\mathbb{A}^1$, i.e. $C$ and $\mathbb{P}^1$ have a isomorphic open subset.

\prule

\plabel{2}%
We may parametrize any line by $(u:v)\in \mathbb{P}^1$ by
\[ \begin{pmatrix}
    w \\ x \\ y \\ z
\end{pmatrix} = \begin{pmatrix}
    c_1 & d_1 \\ c_2 & d_2 \\ c_3 & d_3 \\ c_4 & d_4
\end{pmatrix} \begin{pmatrix}
    u \\ v
\end{pmatrix}, \]
where $(c_1:c_2:c_3:c_4)\in\mathbb{P}^3$ and $(d_1:d_2:d_3:d_4)\in\mathbb{P}^3$, and $(c_1:c_2:c_3:c_4) \neq (d_1:d_2:d_3:d_4)$ (non-degeneracy).
For $w^3+x^3+y^3+z^3=0$ to hold for every $(u:v)$ we require
\begin{align*}
    \sum_{i=1}^4 c_i^3 &= 0, \\
    \sum_{i=1}^4 d_i^3 &= 0, \\
    \sum_{i=1}^4 (c_i+d_i)^3 &= 0, \\
    \sum_{i=1}^4 (c_i-d_i)^3 &= 0.
\end{align*}
To solve the equation we may take $(u:v)$ to be two of the four coordinates $(w,x,y,z)$.
For example, if we take $(u,v) = (w,x)$ then we find
\begin{align*}
    c_3^3 + c_4^3 &= -1, \\
    d_3^3 + d_4^3 &= -1, \\
    c_3^2 d_3 + c_4^2 d_4 &= 0, \\
    c_3 d_3^2 + c_4 d_4^2 &= 0.
\end{align*}
It can be seen that either $c_3=d_4=0$ or $c_4=d_3=0$.
That is
\[ \begin{pmatrix}
    w \\ x \\ y \\ z
\end{pmatrix} = \begin{pmatrix}
    1 & 0 \\ 0 & 1 \\ 0 & -\zeta^i \\ -\zeta^j & 0
\end{pmatrix} \begin{pmatrix}
    u \\ v
\end{pmatrix} \text{ or } \begin{pmatrix}
    w \\ x \\ y \\ z
\end{pmatrix} = \begin{pmatrix}
    1 & 0 \\ 0 & 1 \\ -\zeta^i & 0 \\ 0 & -\zeta^j
\end{pmatrix} \begin{pmatrix}
    u \\ v
\end{pmatrix}, \quad 0\le i \le 2, \quad 0\le j\le 2, \]
where $\zeta$ is a cubic root of $1$ other than $1$ (if it exists).
The other lines could be given by permuting the rows of the coefficient matrix.
Removing duplicating lines we find
\[ \begin{pmatrix}
    w \\ x \\ y \\ z
\end{pmatrix} = \begin{pmatrix}
    1 & 0 \\ 0 & 1 \\ 0 & -\zeta^i \\ -\zeta^j & 0
\end{pmatrix} \begin{pmatrix}
    u \\ v
\end{pmatrix}, \begin{pmatrix}
    1 & 0 \\ 0 & 1 \\ -\zeta^i & 0 \\ 0 & -\zeta^j
\end{pmatrix} \begin{pmatrix}
    u \\ v
\end{pmatrix}, \begin{pmatrix}
    1 & 0 \\ -\zeta^j & 0 \\ 0 & -\zeta^i \\ 0 & 1
\end{pmatrix} \begin{pmatrix}
    u \\ v
\end{pmatrix}, \]
i.e. we have $3\times 3\times 3 = 27$ lines in this way.

\prule

\plabel{3}%
Let $(x_0:\cdots:x_n)$ denote the homogeneous coordinates of $\mathbb{P}^n$.

\plabel{(1)}%
A hypersurface of degree $d$ is defined by a single homogeneous equation
\[ \sum_{\substack{v_0+\cdots+v_n = d \\ v_0,\cdots,v_n\in\mathbb{Z}\\ v_0\ge 0,\cdots,v_n\ge 0}} a_{v_0,\cdots,v_n} x_0^{v_0} \cdots x_n^{v_n} = 0. \]
The number of coefficients is given by the number of ways to write $d$ as the sum of $n+1$ non-negative integers, or writing $d + n + 1$ as the sum of $n+1$ positive integers, i.e. there are $\binom{d+n}{n} = \binom{d+n}{d}$ coefficients.
\par
Simutaneously multiplying all coefficients by a nonzero constant does not change the hypersurface.
Therefore, the set of hypersurfaces of degree $d$ form a projective space $\mathbb{P}^N$, where $N = \binom{d+n}{n} - 1 = \binom{d+n}{d} - 1$.

\plabel{(2)}%
For a given point $x = (x_0:\cdots:x_n)$, the set $S_x$ of hypersurfaces that are singular at $x$ are given by the subset of $\mathbb{P}^N$ that satisfies
\begin{equation}
    \label{eq:singularity_cond}
    \sum_{\substack{v_0+\cdots+v_n = d \\ v_0,\cdots,v_n\in\mathbb{Z}\\ v_0\ge 0,\cdots,v_n\ge 0}} a_{v_0,\cdots,v_n} \pdv{x_i} \qty(x_0^{v_0} \cdots x_n^{v_n}) = 0,\quad 0\le i\le n,
\end{equation}
plus the requirement that the hypersurface should pass through $x$,
\begin{equation}
    \label{eq:passing_through_cond}
    \sum_{\substack{v_0+\cdots+v_n = d \\ v_0,\cdots,v_n\in\mathbb{Z}\\ v_0\ge 0,\cdots,v_n\ge 0}} a_{v_0,\cdots,v_n} x_0^{v_0} \cdots x_n^{v_n} = 0.
\end{equation}
These are all homogeneous linear equations in terms of the coefficients $a_{\cdots}$.
Therefore this is a linear hypersurface.
To know how many independent linear equations are there, we should evaluate the rank of the matrix corresponding to the linear equations given by \eqref{eq:singularity_cond} and \eqref{eq:passing_through_cond}.
The matrix looks like (displaying just one column, there are $\binom{d+n}{n}$ columns)
\[ \begin{pmatrix}
    \vdots & v_0 x_0^{v_0-1}\cdots x_n^{v_n} & \vdots \\
    \vdots & \ddots & \vdots \\
    \vdots & v_n x_0^{v_0}\cdots x_n^{v_n-1} & \vdots \\
    \vdots & x_0^{v_0}\cdots x_n^{v_n} & \vdots
\end{pmatrix}. \]
The first $n+1$ rows add up to $d$ times the last row.
Therefore there can be at most $n+1$ independent equations.
It can be shown that there are at least $n+1$ independent equations.
Therefore $\operatorname{codim} S_x = n+1$.

\prule

\plabel{4 (1)}%
To show it is birational, it suffices to show that they have a isomorphic open subset.
Let $U = \Set*{(z_0:z_1:z_2) \in \mathbb{P}^2}{z_0 \neq 0 \text{ and } z_1 \neq 0 \text{ and } z_2 \neq 0}$.
Then $U$ is mapped isomorphically to itself under $\varphi$, with inverse map also given by $\varphi$, since
\begin{align*}
    \varphi(\varphi(z_0:z_1:z_2)) = \varphi(z_1z_2:z_0z_2:z_0z_1) &= (z_0^2z_1z_2:z_1^2z_0z_2:z_2^2z_0z_1) \\
    &= z_0z_1z_2 (z_0:z_1:z_2).
\end{align*}

\plabel{(2)}%
$\varphi$ is regular on $\mathbb{P}^3$ minus three points, i.e. on
\[ V = \mathbb{P}^2 \setminus \qty{(1:0:0), (0:1:0), (0:0:1)}, \]
since $\varphi(z_0:z_1:z_2)$ has at least one nonzero coordinate for $(z_0:z_1:z_2)\in V$.
%For example, in the plane $\mathbb{A}^2_0$, we have $z_0 \neq 0$ and therefore
%\[ \varphi(z_0:z_2:z_2) = \qty(\frac{z_1z_2}{z_0}: z_2 : z_1) \]
%is regular on $\mathbb{A}^2_0\setminus \qty{(1:0:0)}$.
%It works similarly for $\mathbb{A}^2_1\setminus \qty{(0:1:0)}$ and $\mathbb{A}^2_2\setminus \qty{(0:0:1)}$.

\plabel{(3)}%
The inverse $\varphi^{-1}$ is given by the same expression of $\varphi$, i.e.
\[ \varphi^{-1}(z_0:z_1:z_2) = (z_1z_2:z_0z_2:z_1z_2). \]
The image of $\varphi$ is $U \cup \qty{(1:0:0),(0:1:0),(0:0:1)}$.
This is defined only on $U = \Set*{(z_0:z_1:z_2) \in \mathbb{P}^2}{z_0 \neq 0 \text{ and } z_1 \neq 0 \text{ and } z_2 \neq 0}$.

\plabel{(4)}%
Then $\varphi^2 = \operatorname{id}$ on $U$.

\prule

\plabel{5}%
Recall that a function is called an upper continuous function if for all $x_0$ and for every $y>f(x)$, there exists a neighborhood $U$ of $x_0$ such that $f(x) < y$ for all $x$ in $U$.
Since the set of of singular points of an algebraic variety is closed, the non-singular points form a open set $V$.
We can take $V$ to be the $U$ in the above definition for non-singular points.
\par
To make this work for singular points as well, it suffices to show that the set $\Set*{x\in X}{\dim_x \Theta_x > n}$ for any $n$ is closed.
This can shown by considering the dimension of the fibre of the map $\pi\colon \Theta \to X$, and from the corollary under theorem 1.25 we may show this is closed.

\end{document}
