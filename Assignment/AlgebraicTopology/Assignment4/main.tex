\documentclass{article}

\usepackage{pandekten}
\usepackage{dashrule}

\makeatletter
\newcommand*{\shifttext}[1]{%
  \settowidth{\@tempdima}{#1}%
  \hspace{-\@tempdima}#1%
}
\newcommand{\plabel}[1]{%
\shifttext{\textbf{#1}\quad}%
}
\newcommand{\prule}{%
\begin{center}%
\hdashrule[0.5ex]{.99\linewidth}{1pt}{1pt 2.5pt}%
\end{center}%
}

\makeatother

\newcommand{\minusbaseline}{\abovedisplayskip=0pt\abovedisplayshortskip=0pt~\vspace*{-\baselineskip}}%

\setlength{\parindent}{0pt}

\title{Assignment 4}
\author{Ze Chen}

\begin{document}

\maketitle

\plabel{3.1.3}%
The resolution is given by
\[ \cdots \rightarrow \mathbb{Z}_4 \xrightarrow{2} \mathbb{Z}_4 \xrightarrow{2} \mathbb{Z}_4 \xrightarrow{2} \mathbb{Z}_4 \xrightarrow{1} \mathbb{Z}_2 \rightarrow 0. \]
Applying $\operatorname{Hom}(-,\mathbb{Z}_2)$ we find
\[ \cdots \leftarrow \mathbb{Z}_2 \xleftarrow{2} \mathbb{Z}_2 \xleftarrow{2} \mathbb{Z}_2 \xleftarrow{2} \mathbb{Z}_2 \xleftarrow{1} \mathbb{Z}_2 \leftarrow 0. \]
Therefore, $\operatorname{Ext}^n_{\mathbb{Z}_4}(\mathbb{Z}_2,\mathbb{Z}_2) = \mathbb{Z}_2$ for all $n$.

\plabel{5(a)}%
We assign $f$ and $g$ to two edges of a triangle and $f\cdot g$ to the third edge.
Then
\[ \delta\varphi(\sigma) = \varphi(f\cdot g) - \varphi(f) - \varphi(g) = 0. \]

\plabel{(b)}%
Since $f$ is constant,
\[ \varphi(f) = \varphi(f\cdot f) = \varphi(f) + \varphi(f) \]
and therefore $\varphi(f) = 0$.

\plabel{(c)}%
Triangulating the homotopy square yields a square with diagonal.
$\delta\varphi = 0$ implies $\varphi(f) = \varphi(g) = \varphi(\text{diag})$ since the $\varphi$ of the other two edges are zero by (b).

\plabel{(d)}%
If $\varphi$ is a coboundary, then for some $\psi\in C^0$,
\[ \varphi(f) = \delta \psi(f) = \psi(f(1)) - \psi(f(0)). \]
If $\varphi$ depends on the endpoints only, then we could choose a basepoint $x_0$ (for each path component) and define $\psi(x) = \phi(f)$ where $f$ is any path connecting $x_0$ to $x$.
Then $\varphi = \delta\psi$.

\plabel{8(a)}%
\textbf{Long exact sequence}\quad
We have
\[ \cdots \rightarrow H^i(D^{n+1};G) \rightarrow H^i(S^n;G) \rightarrow H^{i+1}(D^{n+1},S^n;G) \rightarrow H^{i+1}(D^{n+1};G) \rightarrow \cdots, \]
which yields
\[ 0 \rightarrow H^i(S^n;G) \rightarrow H^{i+1}(D^{i+1}/S^n;G) \rightarrow 0, \] 
and therefore
\[ H^i(S^n;G) \cong H^{i+1}(S^{n+1};G). \]

\textbf{Mayer-Vietoris Sequence}\quad
Let $A$ and $B$ be the northern and southern hemisphere, respectively.
Then
\begin{align*}
    \cdots &\rightarrow H^i(A;G) \oplus H^i(B;G) \rightarrow H^i(A\cap B;G) \rightarrow H^{i+1}(S^{n+1};G) \\
    &\rightarrow H^{i+1}(A;G) \oplus H^{i+1}(B;G) \rightarrow \cdots. 
\end{align*}
We find also
\[ 0 \rightarrow H^i(S^n;G) \rightarrow H^{i+1}(S^{n+1};G) \rightarrow 0. \]

\plabel{(b)}%
Since $A$ is a deformation retract of $V$, we have the following commutative diagram.
Denote $q:X\rightarrow X/A$.
\begin{center}
    \begin{tikzcd}
        H^n(X,A;G) & H^n(X,V;G) \arrow[l,"\cong"'] \arrow[r,"\cong"] & H^n(X-A,V-A;G) \\
        H^n(X/A,A/A;G) \arrow[u,"q^*"] & H^n(X/A,V/A;G) \arrow[l,"\cong"'] \arrow[r,"\cong"] \arrow[u,"q^*"] & H^n(X/A-A/A,V/A-A/A;G) \arrow[u,"q^*"]
    \end{tikzcd}
\end{center}
Since $q$ is a homeomorphism of $X-A$,
\[ H^n(X/A-A/A,V/A-A/A;G) \xrightarrow{q^*} H^n(X-A,V-A;G) \]
is a isomorphism, and therefore
\[ H^n(X,A;G) \cong H^n(X/A,A/A;G) \cong \tilde{H}^n(X/A;G). \]

\plabel{(c)}%
The long exact sequence
\[ \cdots \rightarrow H^n(X,A;G) \rightarrow H^n(X;G) \xrightarrow{i^*} H^n(A,G) \rightarrow H^{n+1}(X,A;G) \rightarrow \cdots \]
yield the following short exact sequence
\[ 0 \rightarrow H^n(X,A;G) \rightarrow H^n(X;G) \xrightarrow{i^*} H^n(A;G) \rightarrow 0 \]
since $i$ is injective.
With the splitting lemma we find
\[ H^n(X;G) = H^n(A;G) \oplus H^n(X,A;G). \]

\plabel{9}%
From the universal coefficient theorem, we have the following diagram.
\begin{center}
    \begin{tikzcd}
        0 \arrow[r] & \operatorname{Ext}(H_{n-1}(S^n),G) \arrow[r] & H^n(S^n;G) \arrow[r] & \operatorname{Hom}(H_n(S^n),G) \arrow[r] & 0 \\
        0 \arrow[r] & \operatorname{Ext}(H_{n-1}(S^n),G) \arrow[r] \arrow[u,"(f_*)^*"] & H^n(S^n;G) \arrow[r] \arrow[u,"f^*"] & \operatorname{Hom}(H_n(S^n),G) \arrow[r] \arrow[u,"(f_*)^*"] & 0
    \end{tikzcd}
\end{center}
Therefore, since $H_{n-1}(S^n) = 0$ and $f_* = (d\cdot{})$, we have the following.
\begin{center}
    \begin{tikzcd}
        0 \arrow[r] & H^n(S^n;G) \arrow[r,"\cong"] & \operatorname{Hom}(H_n(S^n),G) \arrow[r] & 0 \\
        0 \arrow[r] & H^n(S^n;G) \arrow[r,"\cong"] \arrow[u,"f^*"] & \operatorname{Hom}(H_n(S^n),G) \arrow[r] \arrow[u,"(d\cdot{})^*"] & 0
    \end{tikzcd}
\end{center}
Therefore, $f^*(\alpha) = d\cdot\alpha$.

\prule

\plabel{3.2.2}%
Let $X = \cup_i A_i$ where each $A_i$ is contractible.
Then we have the following commutative diagram ($k_i>0$ for all $i$).
\begin{center}
    \begin{tikzcd}
        H^{k_1}(X,A_1;R) \times \cdots \times H^{k_n}(X,A_n;R) \arrow[d,"\cong"]\arrow[r,"\smile"] & H^{k_1 + \cdots + k_n}(X,A_1 \cup \cdots \cup A_n;R) = 0 \arrow[d] \\
        H^{k_1}(X;R) \times \cdots \times H^{k_n}(X;R) \arrow[r,"\smile"] & H^{k_1+\cdots+k_n}(X;R)
    \end{tikzcd}
\end{center}
Therefore, the map in the bottom row is zero.

\plabel{3(a)}%
Let $x$ and $y$ be the generators of $H^1(\mathbb{R}P^m;\mathbb{Z}_2)$ and $H^1(\mathbb{R}P^n;\mathbb{Z}_2)$, respectively.
If $f^*x = y$ then
\[ 0 = f^*(0) = f^*(\smile^{m+1} x) = \smile^{m+1} f^*(x) = \smile^{m+1}y, \]
and therefore $m\ge n$.
Similarly, if $m<n$ then there is no nontrivial $g^*:H^2(\mathbb{C}P^m;\mathbb{Z}_2)\rightarrow H^2(\mathbb{C}P^n;\mathbb{Z}_2)$.

\plabel{(b)}%
Let $\tilde{g}:\mathbb{R}P^n\rightarrow \mathbb{R}P^{n-1}$ denote the induced map.
Then since $g(-x) = g(x)$, $\tilde{g}_*$ takes the generator of $\pi_1(\mathbb{R}P^n)$ to the generator of $\pi_1(\mathbb{R}P^{n-1})$, which gives rise to a nontrivial $\tilde{g}^*:H^1(\mathbb{R}P^{n-1};\mathbb{Z}_2)\rightarrow H^1(\mathbb{R}P^n;\mathbb{Z}_2)$, a contradiction.

\plabel{6}%
\newcommand{\RP}{RP}%
\newcommand{\Hom}{Hom}%
Since $H^*(\mathbb{C}P^n;\mathbb{Z}) = \mathbb{Z}[\alpha]/(\alpha^{n+1})$ with $\abs{\alpha}=2$, it suffices to work out $f^*(\alpha)$, i.e. $H^2(\mathbb{C}P^n;\mathbb{Z})$.
Then, since we have the following diagram,
\begin{center}
    \begin{tikzcd}
        H^2(\mathbb{C}P^n;\mathbb{Z}) \arrow[r]\arrow[d,"i^*"] & H^2(\mathbb{C}P^n;\mathbb{Z}) \arrow[d,"i^*"] \\
        H^2(\mathbb{C}P^1;\mathbb{Z}) \arrow[r,"d\cdot"] & H^2(\mathbb{C}P^1;\mathbb{Z})
    \end{tikzcd}
\end{center}
we find $f^*(\alpha) = d\cdot \alpha$.

\plabel{7}%
Since
\[ H^*(\mathbb{R}P^3;\mathbb{Z}_2) = \mathbb{Z}_2[\alpha]/(\alpha^4) \]
and
\[ H^*(\mathbb{R}P^2 \vee S^3;\mathbb{Z}_2) = \mathbb{Z}_2[\alpha]/(\alpha^3) \times \mathbb{Z}_2[\gamma]/(\gamma^2) \]
are not isomorphic (since $x^3 \equiv 0$ for all $x$ in the second ring), they are not homotopy equivalent.

\plabel{11}%
Let $f:S^{k+l} \rightarrow S^k \times S^l$.
It suffices to prove that the induced map on $H^{k+l}(-,\mathbb{Z})$ is zero.
Since $H^{k+l}(S^{k}\times S^{l},\mathbb{Z})\cong \mathbb{Z}$ is generated by $x\smile y$ where $x$ and $y$ are generators of $H^{k}(S^k,\mathbb{Z})$ and $H^{l}(S^l,\mathbb{Z})$, respectively, and
\[ f^*(x\smile y) = f^*(x) \smile f^*(y) = 0 \]
since $H^k(S^{k+l},\mathbb{Z}) = H^l(S^{k+l},\mathbb{Z}) = 0$.
Therefore $f^*=0$ on $H^{k+l}(-,\mathbb{Z})$.

\plabel{15}%
From the K\"unneth formula,
\begin{align*}
    \dim H^k(X\times Y) &= \dim \bigoplus_{i+j=k} H^i(X) \otimes H^j(Y) \\
    &= \sum_{i+j=k} \dim H^i(X) \dim H^j(Y)
\end{align*}
and therefore $p(X\times Y) = p(X)\times p(Y)$.
\begin{itemize}
    \item $p(S^n) = 1+t^n$.
    \item $p(\mathbb{R}P^n) = \begin{cases}
        \sum_{i=0}^n t^i, & \text{if } \operatorname{char} F = 2, \\
        1+t^n, & \text{if }\operatorname{char} F \neq 2 \text{ and } n \text{ odd},\\
        1, & \text{if }\operatorname{char} F \neq 2 \text{ and } n \text{ even}.\\
    \end{cases}$
    \item $p(\mathbb{R}P^n) = \begin{cases}
        \sum_{i=0}^\infty t^i, & \text{if } \operatorname{char} F = 2, \\
        1, & \text{if }\operatorname{char} F \neq 2.
    \end{cases}$
    \item $p(\mathbb{C}P^n) = \sum_{i=0}^\infty t^{2i}$.
    \item $p(S^3\times \mathbb{C}P^\infty) = p(S^3) \times p(\mathbb{C}P^\infty) = (1+t^3)(1+t^2+t^4+\cdots) = 1 + \sum_{i=2}^\infty t^i$.
    \item $p(\mathbb{C}P^\infty/\mathbb{C}P^1) = 1+\sum_{i=2}^\infty t^{2i}$.
    \item $p(S^6\mathbb{H}P^\infty) = p(S^6)\times p(\mathbb{H}P^\infty) = (1+t^6)(1+t^4+t^8+\cdots) = 1+\sum_{i=2}^\infty t^{2i}$.
\end{itemize}

\end{document}
