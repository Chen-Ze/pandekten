\documentclass{article}

\usepackage{pandekten}
\usepackage{dashrule}

\makeatletter
\newcommand*{\shifttext}[1]{%
  \settowidth{\@tempdima}{#1}%
  \hspace{-\@tempdima}#1%
}
\newcommand{\plabel}[1]{%
\shifttext{\textbf{#1}\quad}%
}
\newcommand{\prule}{%
\begin{center}%
\hdashrule[0.5ex]{.99\linewidth}{1pt}{1pt 2.5pt}%
\end{center}%
}

\makeatother

\newcommand{\minusbaseline}{\abovedisplayskip=0pt\abovedisplayshortskip=0pt~\vspace*{-\baselineskip}}%

\setlength{\parindent}{0pt}

\title{Midterm}
\author{Ze Chen}

\begin{document}

\maketitle

\plabel{1 (a)}%
It acts transitively on the fibers.

\plabel{(b)}%
The followings is a normal covering.
\begin{center}
    \begin{tikzpicture}[decoration={
        markings,
        mark=at position 0.5 with {\arrow{latex}}}
        ]
        \draw[postaction={decorate}] (-30:1) to(90:1);
        \draw[postaction={decorate}] (90:1) to(210:1);
        \draw[postaction={decorate}] (210:1) to(330:1);
        \draw[postaction={decorate}] (-30:1) to[bend right] (90:1);
        \draw[postaction={decorate}] (90:1) to[bend right] (210:1);
        \draw[postaction={decorate}] (210:1) to[bend right] (330:1);

        \draw (30:1) node {$a$};
        \draw (150:1) node {$a$};
        \draw (270:1) node {$a$};
        \draw (30:.25) node {$b$};
        \draw (150:.25) node {$b$};
        \draw (270:.25) node {$b$};
    \end{tikzpicture}
\end{center}

\plabel{(c)}%
The following is not normal.
\begin{center}
    \begin{tikzpicture}[decoration={
        markings,
        mark=at position 0.5 with {\arrow{latex}}}
        ]
        \draw[postaction={decorate}] (-30:1.5) to (90:1.5);
        \draw[postaction={decorate}] (90:1.5) to (210:1.5);
        \draw[postaction={decorate}] (210:1.5) to (-30:1.5);
        \draw[postaction={decorate}] (90:1.5) arc (-90:270:0.75);
        \draw (30:1) node {$a$};
        \draw (150:1) node {$a$};
        \draw (270:1) node {$a$};
        \draw (-90:0) node {$b$};
        \draw (90:2.5) node {$b$};
        \draw (-90:1.5) node {$b$};
        \draw[postaction={decorate}] (210:1.5) to[bend left] (-30:1.5);
        \draw[postaction={decorate}] (-30:1.5) to[bend left] (210:1.5);
    \end{tikzpicture}
\end{center}

\plabel{(d)}%
Isomorphic to automorphism groups of $3$-fold covering spaces of $S^1 \vee S^1$.

\plabel{(e)}%
$a^3, b^3, ab^{-1}$.

\plabel{(f)}%
$a^3,b,aba^{-1}b^{-1}$.

\prule

\plabel{2 (a)}%
Since $f$ has no fixed point, it is homotopic to $x \mapsto -x$, which is a composition of $(n+1)$ reflections, and therefore has degree $(-1)^{n+1}$.

\plabel{(b)}%
If for all $x$, $f(x)$ is neither $x$ nor $-x$, then the line joining $x$ and $f(x)$ is not perpendicular to the tangent plane at $x$.
Projecting this line onto the tangent plane yields an tangent vector field nonzero everywhere, which is impossible on $S^{2n}$.

\plabel{(c)}%
$g:\mathbb{R}P^{2m} \rightarrow \mathbb{R}P^{2m}$ induces a map from $g\circ \pi: S^{2m} \mapsto \mathbb{R}P^{2m}$, which can be lifted to a map $f:S^{2m} \rightarrow S^{2m}$.
Since there is some $x\in S^{2n}$ such that $f(x) = \pm x$, we find $g(\pi(x)) = \pi(x)$.

\prule

\plabel{3 (a)}%
\begingroup\minusbaseline
\[ H_n(S^1 \vee S^2) = H_n(S^1) \oplus H_n(S^2) = \begin{cases}
    0, & n>2,\\
    \mathbb{Z},& n=2, \\
    \mathbb{Z},& n=1, \\
    \mathbb{Z}\oplus \mathbb{Z}, & n=0.
\end{cases} \]
\endgroup

\plabel{(b)}%
Since $\mathbb{Z}$ is a free module over $\mathbb{Z}$,
\begin{align*}
    H_0(S^1 \times {S}^1) &= H_0(S^1) \otimes H_0(S^1) = \mathbb{Z}, \\
    H_1(S^1 \times {S}^1) &= H_0(S^1) \otimes H_1(S^1) \oplus  H_1(S^1) \otimes H_0(S^1) = \mathbb{Z} \oplus \mathbb{Z}, \\
    H_2(S^1 \times {S}^1) &= H_1(S^1) \otimes H_1(S^1) = \mathbb{Z}, \\
    H_n(S^1 \times {S}^1) &= 0,\quad \text{if } n>2.
\end{align*}

\plabel{(c)}%
\begingroup\minusbaseline
\[ H_n(S^1 \vee S^1) = H_n(S^1) \oplus H_n(S^1) = \begin{cases}
    0, & n>1,\\
    \mathbb{Z}\oplus \mathbb{Z},& n=1, \\
    \mathbb{Z}\oplus \mathbb{Z}, & n=0.
\end{cases} \]
\endgroup

\plabel{(d)}%
Since $\mathbb{Z}$ is a free module over $\mathbb{Z}$,
\begin{align*}
    H_0(S^1 \times (S^1 \vee S^1)) &= H_0(S^1) \otimes H_0(S^1 \vee S^1) = \mathbb{Z} \oplus \mathbb{Z}, \\
    H_1(S^1 \times (S^1 \vee S^1)) &= H_0(S^1) \otimes H_1(S^1\vee S^1) \oplus  H_1(S^1) \otimes H_0(S^1\vee S^1) \\
    &= \mathbb{Z} \oplus \mathbb{Z} \oplus \mathbb{Z} \oplus \mathbb{Z}, \\
    H_2(S^1 \times (S^1 \vee S^1)) &= H_1(S^1) \otimes H_1(S^1\vee S^1) = \mathbb{Z} \oplus \mathbb{Z}, \\
    H_n(S^1 \times (S^1 \vee S^1)) &= 0,\quad \text{if } n>2.
\end{align*}

\prule

\plabel{4 (a)}%
It has a $0$-cell $e^0$, two $2$-cells $e^2_A$ and $e^2_B$, and a $4$-cell $e^4$.

\plabel{(b)}%
Since it has no cells of adjacent dimensions,
\[ H_n(S^2\times S^2) = \begin{cases}
    \mathbb{Z}, & n=0,4, \\
    \mathbb{Z} \oplus \mathbb{Z}, & n=2, \\
    0, \text{otherwise}.
\end{cases} \]

\plabel{(c)}%
For $n\neq 3,4$, we have
\[ H_n = H_n(S^2\times S^2) \oplus H_n(S^2\times S^2). \]
For $n=3,4$,

% \bibliographystyle{plain}
% \bibliography{main}

\end{document}
