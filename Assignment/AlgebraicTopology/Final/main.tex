\documentclass{article}

\usepackage{pandekten}
\usepackage{dashrule}

\makeatletter
\newcommand*{\shifttext}[1]{%
  \settowidth{\@tempdima}{#1}%
  \hspace{-\@tempdima}#1%
}
\newcommand{\plabel}[1]{%
\shifttext{\textbf{#1}\quad}%
}
\newcommand{\prule}{%
\begin{center}%
\hdashrule[0.5ex]{.99\linewidth}{1pt}{1pt 2.5pt}%
\end{center}%
}

\makeatother

\newcommand{\minusbaseline}{\abovedisplayskip=0pt\abovedisplayshortskip=0pt~\vspace*{-\baselineskip}}%

\setlength{\parindent}{0pt}

\title{Final}
\author{Ze Chen}

\begin{document}

\maketitle

\plabel{1 (a)}%
From $H^p(X,U) \rightarrow H^p(X) \rightarrow H^p(U)$ and that $\alpha$ is mapped to zero in $H^p(U)$ we know $\alpha$ comes from an $\alpha'\in H^p(X,U)$.
Also $\beta$ comes from an $\beta'\in H^p(X,V)$.
Then $\alpha'\smile \beta'\in H^p(X,U\cup V) = H^p(X,X)$ is zero.
Therefore, $\alpha\smile \beta$ is zero.

\plabel{(b)}%
If $U_1,\cdots,U_m$ covers $M$ where each $U_i \cong \mathbb{R}^n$.
Let $\alpha_1,\cdots,\alpha_\ell$ be cohomology classes that satisfy the property.
Then each $\alpha_i$ is zero in $H^{n_i}(U_j)=0$ for each $U_j$.
Then by (a), $\alpha_1 \smile \cdots \smile \alpha_m$ has to be zero.
Therefore $\ell < m$. Taking $k$ to be the minimal $m$ we find $\ell+1\le k$.

\plabel{(c)}%
$k\le 3$ since charts are enough to cover $S^2\times S^2$.
Let $U_N$ denote $S^2$ minus the south pole and $U_S$ denote $S^2$ minus the north pole.
Then $A = U_N \times U_N \cong \mathbb{R}^4$ and $B = U_S\times U_S \cong \mathbb{R}^4$.
$A\cup B$ almost covers $S^2\times S^2$ except for two points: $\qty{\text{north pole}} \times \qty{\text{south pole}}$ and $\qty{\text{south pole}} \times \qty{\text{north pole}}$.
Let $V$ denote $S^2$ minus an arbitrary point that is neither the north pole nor the south pole.
Then $C = V\times V\cong \mathbb{R}^4$ and $A\cup B\cup C$ covers $S^2\times S^2$.
\par
On the other hand, $k\ge 3$ since $H^{4}(S^2\times S^2)$ is generated by $\alpha\smile \beta$ where $\alpha$ is the generator of the first $H^2(S^2)$ and $\beta$ is the generator of the second $H^2(S^2)$.

\plabel{(d)}%
$k\ge n+1$ since $\mathbb{C}P^n$ can be covered by $U_0,\cdots,U_n$ where
\[ U_i = \Set*{[z_0:\cdots:z_n]}{z_i\neq 0}. \]
$k\le n+1$ since $\alpha^n \in H^{2n}(\mathbb{C}P^n)$ is the generator where $\alpha$ is the generator of $H^{2n}(\mathbb{C}P^n)$.

\prule

\plabel{2 (a)}%
Since $S^2\times S^1$ is three dimensional and $\pi_1(S^2\times S^1) = \pi_1(S^2)\times \pi_1(S^1) = \mathbb{Z}$,
\[ \pi_1((S^2\times S^1)\# (S^2\times S^1)) = \pi_1(S^2\times S^1)*\pi_1(S^2\times S^1) = \mathbb{Z}*\mathbb{Z}. \]

\plabel{(b)}%
Since $M$ is connected, $H_0(M) = \mathbb{Z}$.
$H_1(M) = \operatorname{Ab}(\pi_1(M)) = \mathbb{Z}\oplus \mathbb{Z}$.
$H_3(M) = \mathbb{Z}$ since $M$ is compact, connected and orientable.
From the Mayor-Vietoris sequence ($N=S^2\times S^1$)
\[ 0 \rightarrow {H_3}(M) \rightarrow {H_3}(N\vee N) \rightarrow {H}_{2}(S^{2})\rightarrow {H}_{2}(M) \rightarrow {H}_{2}(N\vee N) \to 0 \]
we find (since $H_2(N) = \mathbb{Z}$)
\[ 0 \rightarrow \mathbb{Z} \rightarrow \mathbb{Z}\oplus\mathbb{Z} \rightarrow \mathbb{Z} \rightarrow {H}_{2}(M) \rightarrow \mathbb{Z}\oplus\mathbb{Z} \to 0 \]
and therefore $H_2(M) = \mathbb{Z}\oplus\mathbb{Z}$.

\plabel{(c)}%
From the universal coefficient theorem we find (since $H_*$ are all free)
\[ H^0(M;\mathbb{Z}) = \mathbb{Z},\quad H^1(M;\mathbb{Z}) = \mathbb{Z} \oplus \mathbb{Z},\quad H^2(M;\mathbb{Z}) = \mathbb{Z} \oplus \mathbb{Z},\quad H^2(M;\mathbb{Z}) = \mathbb{Z}. \]
Therefore, $H^*(M,\mathbb{Z}) = \mathbb{Z}[\alpha_1,\alpha_2,\beta_1,\beta_2,\gamma]/I$ where $\alpha_1$ and $\alpha_2$ denote the generators of $H^1(M,\mathbb{Z})$, $\beta_1$ and $\beta_2$ denote the generators of $H^2(M,\mathbb{Z})$, and $\gamma$ denote the generator of $H^3(M,\mathbb{Z})$.
\[ I = (\alpha_1^2,\alpha_2^2,\beta_1^2,\beta_2^2,\gamma^2,\alpha_1\alpha_2,\beta_1\beta_2,\alpha_1\gamma,\alpha_2\gamma,\beta_1\gamma,\beta_2\gamma,\alpha_1\beta_1-\gamma,\alpha_2\beta_2-\gamma). \]

\plabel{(d)}%
Let $\gamma_\alpha$, $\gamma_\beta$, $\gamma_{\alpha\beta}$, and $\gamma_{\beta\alpha}$ be loops at $x_0$ corresponding to $\alpha$, $\beta$, $\alpha\beta$, and $\beta\alpha$, respectively.
They can be lifted to $\tilde{\gamma}_\alpha$, $\tilde{\gamma}_\beta$, $\tilde{\gamma}_{\alpha\beta}$, and $\tilde{\gamma}_{\beta\alpha}$, in the universal cover $(\tilde{X},\tilde{x}_0)$.
$\tilde{\gamma}_\alpha$ ends at $\tilde{x}_{0}'$ and $\tilde{\gamma}_\beta$ ends at $\tilde{x}_{0}''$, and both ending points are projected to $x_0$.
They $\gamma_\alpha$ can be lifted to a path $\tilde{\gamma}_\alpha''$ starting at $\tilde{x}_0''$ and $\gamma_\beta$ can be lifted to a path $\tilde{\gamma}_\beta'$ starting at $\tilde{x}_0'$.
\par
Now we can regard the paths $\tilde{\gamma}_*$ as $1$-simplices.
Since $H_1(\tilde{X}) = 0$, and $\alpha\beta = \beta\alpha$, we have
\[ \partial((\gamma_\alpha + \gamma_\beta') - (\gamma_\beta + \gamma_\alpha'')) = 0 \]
and therefore
\[ (\gamma_\alpha + \gamma_\beta') - (\gamma_\beta + \gamma_\alpha'') = \partial c_2 \]
where $c_2$ is some $2$-complex.
It can be seen that $c_2$ induces a map from $T^2$ to $X$.

\plabel{(e)}%
From (d) we see that there is a map $f:T^2 \rightarrow X$.
Let $\alpha$ and $\beta$ denote the generators of $H_1(X) = \mathbb{Z}\oplus\mathbb{Z}$, while $\alpha_T$ and $\beta_T$ denote the generators of $H_1(T^2) = \mathbb{Z}\oplus\mathbb{Z}$.
Then $f^*(\alpha) = \alpha_T$ and $f^*(\beta) = \beta_T$.
Then we have the following commutative diagram.
\begin{center}
    \begin{tikzcd}
        H_1(T^2) \otimes H_1(T^2) \arrow[r,"\smile"] & H_2(T^2) \\
        H_1(X) \arrow[u,"f^*"] \otimes H_1(X) \arrow[r,"\smile"] & H_2(X) \arrow[u,"f^*"] 
    \end{tikzcd}
\end{center}
Since $\alpha_T \smile \beta_T$ is the generator of $H_2(T^2) = \mathbb{Z}$, we see that $\alpha\smile \beta$ is a nonzero nontorsion element of $H_2(X)$.

\plabel{(f)}%
If $Y$ is a compact connected orientable three-manifold, then by duality we find $H^2(Y) = H_1(Y) = \mathbb{Z}\oplus \mathbb{Z}$.
However, $\alpha \smile \beta$ is a nonzero nontorsion element in $H^2(Y)$ but for any $x = x_\alpha \alpha + x_\beta \beta\in H^1(Y)$, $(\alpha\smile \beta)\smile x = 0$ since $\alpha\smile \alpha = \beta\smile\beta = 0$ because $H^2(Y)$ is free.
But the cup product pairing is nonsingular, leading to a contradiction.

\plabel{(g)}%
If $Y$ is nonorientable, then the orientation bundle $\tilde{Y}$ is a connected orientable two-fold cover of $Y$.
Therefore $\pi_1(\tilde{Y})$ has to be a subgroup of $\pi_1(Y)$ with index $2$.
If $\pi_1(Y) = \mathbb{Z}\oplus \mathbb{Z}$ then $\pi_1(\tilde{Y})$ must also be $\mathbb{Z}\oplus \mathbb{Z}$, leading to a contradiction.

\prule

\plabel{3 (a)}%
If $f$ is not surjective, then there is a ball with preimage the empty set.
By Exercise 3.3.8 of Hatcher (degree equal to the sum of local degrees), $f$ has degree zero.

\plabel{(b)}%
Since $q$ is a homeomorphism on a patch $U$, we find again by local degree that $\operatorname{deg}(f) = 1$.

\plabel{(c)}%
Let $q:S^3\rightarrow T^3$.
Since $S^3$ is simply-connected, we could lift $q$ to the universal cover of $T^3$, i.e. $\tilde{q}:S^3 \rightarrow \mathbb{R}^3$.
Then $\tilde{q}$ is nullhomotopic since $\mathbb{R}^3$ is contractible.
Therefore, $q$ is nullhomotopic and therefore has degree zero.

\prule

\plabel{4 (a)}%
Since $p\circ q: T^3 \rightarrow S^2$, the map on $H_1$ and $H_3$ are zero since $H_1(S^2) = H_3(S^3) = 0.$
The map on $H_2$ is zero since $q:T^3\rightarrow S^3$ and $H_2(S^3)=0$.
The map on $H_0$ is clearly isomorphism since $T^3$ and $S^2$ are connected.

\plabel{(b)}%
Since $\pi_k(T^3)=0$ for $k\ge 2$, and $\pi_0(S^2) = \pi_1(S^2) = 0$, the map $\pi_*$ is trivial.

\plabel{(c)}%
If $p\circ q$ is nullhomotopic, then by the homotopy lifting property it can be lifted to a homotopy $\tilde{g}_t: T^3\rightarrow S^3$ such that $g_0 = q$ and $g_1$ is not surjective.
However, $g_1$ has degree zero since it's not surjective, but $q$ has degree one.
Therefore, $q$ and $g_1$ cannot be homotopic.

\prule

\plabel{5 (a)}%
Since $X_f$ has one $2n$-cell, one $n$-cell, and one $0$-cell, and $n>1$, $X_f$ has no cell in adjacent dimensions, and therefore $H_{2n}(X_f) = H_{n}(X_f) = H_0(X_f) = \mathbb{Z}$.
With the universal coefficient theorem, since $H_{*}(X_f)$ are all free, we find
\[ H^{2n}(X_f;\mathbb{Z}) = H^{n}(X_f;\mathbb{Z}) = H^{0}(X_f;\mathbb{Z}) = \mathbb{Z}. \]

\plabel{(b)}%
Since $f$ is homotopic to $f'$, $X_{f}$ is homotopy equivalent to $X_{f'}$.
Let $g:X_{f}\rightarrow X_{f'}$ be the map of the homotopy equivalence.
Then $g^*$ is an isomorphism, and therefore sends generators to generators.
Let $\xi$ denote the generator of $H^{n}(X_f;\mathbb{Z})$ and $\xi'$ denote the generator of $H^{n}(X_{f'};\mathbb{Z})$, and $\eta$ denote the generator of $H^{2n}(X_f;\mathbb{Z})$ and $\eta'$ denote the generator of $H^{2n}(X_{f'};\mathbb{Z})$.
Then $\xi'\smile \xi' = h\cdot \eta'$ implies
\begin{align*}
    g^*(\xi'\smile \xi') &= g^*(\xi')\smile g^*(\xi') = h \cdot g^*(\eta') \\
    &= \xi\smile \xi = h \eta.
\end{align*}

\plabel{(c)}%
If $n$ is odd then $\xi\in H^{n}(X_f,\mathbb{Z})$ has odd degree and therefore
\[ \xi\smile \xi = -\xi\smile \xi \in H^{2n}(X_f,\mathbb{Z}). \]
Therefore $\xi\smile\xi = 0$ since $H^{2n}(X_f,\mathbb{Z})$ is free.

% \bibliographystyle{plain}
% \bibliography{main}

\end{document}
