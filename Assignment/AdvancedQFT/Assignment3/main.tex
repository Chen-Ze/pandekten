\documentclass{article}

\usepackage{mmacells}

\usepackage{pandekten}
\usepackage{dashrule}

\usepackage[compat=1.1.0]{tikz-feynman}

%workaround from link
\usetikzlibrary{external}
\immediate\write18{mkdir -p pgf-img}
\tikzexternalize[
  prefix=pgf-img/,
  system call={
    lualatex \tikzexternalcheckshellescape -halt-on-error -interaction=batchmode -jobname="\image" "\texsource" || rm "\image.pdf"
  },
]

\makeatletter
\newcommand*{\shifttext}[1]{%
  \settowidth{\@tempdima}{#1}%
  \hspace{-\@tempdima}#1%
}
\newcommand{\plabel}[1]{%
\shifttext{\textbf{#1}\quad}%
}
\newcommand{\prule}{%
\begin{center}%
\hdashrule[0.5ex]{.99\linewidth}{1pt}{1pt 2.5pt}%
\end{center}%
}

\makeatother

\newcommand{\minusbaseline}{\abovedisplayskip=0pt\abovedisplayshortskip=0pt~\vspace*{-\baselineskip}}%

\setlength{\parindent}{0pt}

\title{Assignment 3}
\author{Ze Chen}

\begin{document}

\maketitle

\plabel{1 (a)}%
For $d=4$, the degree of divergence is given by $D = 4 - N_\phi - 3/2 N_\psi$.
The divergent diagrams are as follows (the tadpole diagram of $\phi$ is ignored).
\begin{align*}
    \feynmandiagram [layered layout, inline=(p2.base), horizontal=p1 to p2, small] {
    p1[] --[scalar] c1[blob],
    c1 --[scalar] p2[],
    }; && D = 2 &&
    \feynmandiagram [layered layout, inline=(p2.base), horizontal=p1 to p2, small] {
        p1[] --[fermion] c1[blob],
        c1 --[fermion] p2[],
    }; && D = 1 \\
    \feynmandiagram [inline=(v.base), horizontal=v to p2, small] {
        p1 --[anti fermion] v[blob],
        p2 --[scalar] v,
        p3 --[fermion] v,
    }; && D = 0 &&
    \feynmandiagram [inline=(v.base), horizontal=p1 to p3, small] {
        p1 --[scalar] v[blob],
        p2 --[scalar] v,
        p3 --[scalar] v,
        p4 --[scalar] v,
    }; && D = 0
\end{align*}
The counter terms are as follows.
\begin{gather*}
\feynmandiagram [layered layout, inline=(p2.base), horizontal=p1 to p2, small] {
p1[] --[scalar] c1[crossed dot],
c1 --[scalar] p2[],
}; \quad
\feynmandiagram [layered layout, inline=(p2.base), horizontal=p1 to p2, small] {
    p1[] --[fermion] c1[crossed dot],
    c1 --[fermion] p2[],
}; \\
\feynmandiagram [inline=(v.base), horizontal=v to p2, small] {
    p1 --[anti fermion] v[crossed dot],
    p2 --[scalar] v,
    p3 --[fermion] v,
}; \quad
\feynmandiagram [inline=(v.base), horizontal=p1 to p3, small] {
    p1 --[scalar] v[crossed dot],
    p2 --[scalar] v,
    p3 --[scalar] v,
    p4 --[scalar] v,
};
\end{gather*}

\plabel{(b)}%
\textbf{1PI of $\phi$.}
\begin{align*}
    -i M^2(p^2) &= \feynmandiagram [layered layout, inline=(p2.base), horizontal=p1 to p2, small] {
        p1[particle=$p$\vphantom{$Mg$}] --[scalar] c1[],
        c1 --[scalar] p2[particle=$p$\vphantom{$Mg$}],
        c1 --[scalar,out=135,in=45,loop,min distance=4em] c1,
    }; + \feynmandiagram [layered layout, inline=(p2.base), horizontal=p1 to p2, small] {
        p1[particle=$p$\vphantom{$Mg$}] --[scalar] c1[dot],
        c2[dot] --[scalar] p2[particle=$p$\vphantom{$Mg$}],
        c1 --[half left, fermion] c2;
        c1 --[half right, anti fermion] c2;
    }; + \feynmandiagram [layered layout, inline=(p2.base), horizontal=p1 to p2, small] {
        p1[particle=$p$\vphantom{$Mg$}] --[scalar] c1[crossed dot],
        c1 --[scalar] p2[particle=$p$\vphantom{$Mg$}],
    }; \\
    &= -\frac{i\lambda}{2} \int \frac{\dd[d]{k}}{(2\pi)^d} \frac{i}{k^2 - m^2} - (-ig)^2 \int \frac{\dd[d]{k}}{(2\pi)^d} \tr\qty[\frac{i(\slashed{k} + \slashed{p} + M)\gamma^5 i(\slashed{k}+M) \gamma^5}{\qty((k+p)^2 - M^2)(k^2 - M^2)}] \\
    &\phantom{{}={}} + i(p^2 \delta_{Z,\phi} - \delta_m) \\
    &= \frac{i\lambda m^2}{(4\pi)^2}\frac{1}{\epsilon} + \frac{4ig^2(p^2 - 2M^2)}{(4\pi)^2}\frac{1}{\epsilon} + i(p^2 \delta_{Z,\phi} - \delta_m).
\end{align*}
$\eval{\partial M^2(p^2)/\partial p^2}_{p^2 = m^2} = 0$ yields
\[ \delta_{Z,\phi} = -\frac{4g^2}{(4\pi)^2}\frac{1}{\epsilon}. \]
$M^2(m^2) = 0$ yields
\[ \delta_m = \frac{(\lambda m^2 - 8g^2 M^2)}{(4\pi)^2}\frac{1}{\epsilon}. \]

\textbf{1PI of $\psi$.}
\begin{align*}
    -i\Sigma_2(p) &= \feynmandiagram [layered layout, inline=(p2.base), horizontal=p1 to p2, small] {
        p1[particle=$p$\vphantom{$Mg$}] --[fermion] c1[dot],
        c2[dot] --[fermion] p2[particle=$p$\vphantom{$Mg$}],
        c1 --[scalar,half left] c2;
        c1 --[fermion] c2,
    }; + \feynmandiagram [layered layout, inline=(p2.base), horizontal=p1 to p2, small] {
        p1[particle=$p$\vphantom{$Mg$}] --[fermion] c1[crossed dot],
        c1 --[fermion] p2[particle=$p$\vphantom{$Mg$}],
    }; \\
    &= -(-ig)^2 \int \frac{\dd[d]{k}}{(2\pi)^d} \gamma^5 \frac{i(\slashed{k} + M)}{k^2 - M^2} \gamma^5 \frac{i}{(k-p)^2 - m^2} + i(\slashed{p}\delta_{Z,\psi} - \delta_M) \\
    &= \frac{ig^2(\slashed{p} - 2M)}{(4\pi)^2} \frac{1}{\epsilon} + i(\slashed{p}\delta_{Z,\psi} - \delta_M).
\end{align*}
$\eval{\partial \Sigma(\slashed{p})/\partial \slashed{p}}_{\slashed{p} = m} = 0$ yields
\[ \delta_{Z,\psi} = \frac{-g^2}{(4\pi)^2}\frac{1}{\epsilon}. \]
$\Sigma(\slashed{p} = M) = 0$ yields
\[ \delta_M = -\frac{2g^2 M}{(4\pi)^2} \frac{1}{\epsilon}. \]

\textbf{Vertex Correction.}
\begin{align*}
    &{\phantom{{}={}}} \delta\Gamma^5(p=p'=q=0) \\
    &= \feynmandiagram [inline=(p1.base), horizontal=v1 to p1, small] {
        p1[particle=\vphantom{$Mg$}] --[scalar] v1[dot],
        p2[] --[anti fermion] v2[dot],
        p3[] --[fermion] v3[dot],
        v1 --[fermion] v2,
        v1 --[anti fermion] v3,
        v2 --[scalar, quarter right] v3,
    }; + \feynmandiagram [inline=(v.base), horizontal=v to p2, small] {
        p1[particle=\vphantom{$Mg$}] --[anti fermion] v[crossed dot],
        p2 --[scalar] v,
        p3 --[fermion] v,
    }; \\
    &= g^2 \int \frac{\dd[d]{k}}{(2\pi)^d} \frac{i(\slashed{k}+M)}{k^2 - M^2} \gamma^5 \frac{i(\slashed{k}+M)}{k^2 - M^2} \gamma^5 \frac{i}{k^2 - m^2} + \delta_g \gamma^5 \\
    &= -\frac{g^2\gamma^5 }{(4\pi)^2}\frac{2}{\epsilon} + \delta_g \gamma^5.
\end{align*}
Therefore
\[ \delta_g = \frac{2g^2}{(4\pi)^2}\frac{1}{\epsilon}. \]

\textbf{$\phi^4$ Process.}
\begin{align*}
    &\phantom{{}={}} i\mathcal{M}(\phi^4,p_1=p_2=p_3=p_4=0) \\
    &= 3 \times \feynmandiagram [inline=(c2.base), horizontal=c1 to c2, small] {
        p1[particle=\vphantom{$Mg$}] --[scalar] c1[dot],
        p2[particle=\vphantom{$Mg$}] --[scalar] c1,
        c2[dot] --[scalar] p3,
        c2 -- [scalar]p4,
        c1 --[scalar, half left] c2;
        c1 --[scalar, half right] c2;
    }; + 6\times \feynmandiagram [inline=($0.5*(c2.base)+0.5*(c4.base)$), horizontal=c1 to c2, small] {
        c1 --[scalar] p1,
        c2 --[scalar] p2,
        c3 --[scalar] p3,
        c4 --[scalar] p4,
        p1 --[fermion, quarter right] p2,
        p2 --[fermion, quarter right] p3,
        p3 --[fermion, quarter right] p4,
        p4 --[fermion, quarter right] p1,
    }; + \feynmandiagram [inline=(v.base), horizontal=p1 to p3, small] {
        p1 --[scalar] v[crossed dot],
        p2 --[scalar] v,
        p3 --[scalar] v,
        p4 --[scalar] v,
    }; \\
    &= 3\times \frac{(-i\lambda)^2}{2} \int \frac{\dd[d]{k}}{(2\pi)^d} \qty(\frac{i}{k^2 - m^2})^2 + 6\times (-1) g^4 \int \frac{\dd[d]{k}}{(2\pi)^d} \tr\qty[\qty(\gamma^5 \frac{i(\slashed{k}+M)}{k^2 - M^2})^4] -i\delta_\lambda \\
    &= 3\times \frac{i\lambda^2}{(4\pi)^2}\frac{1}{\epsilon} - 6\times \frac{8ig^4}{(4\pi)^2}\frac{1}{\epsilon} - i\delta_\lambda.
\end{align*}
Therefore,
\[ \delta_\lambda = \frac{3\lambda^2 - 48g^4}{(4\pi)^2}\frac{1}{\epsilon}. \]

\prule

\plabel{2}%
Note that
\begin{align*}
    \delta_2 &= -\frac{e^2}{(4\pi)^{d/2}} \Gamma(2-d/2) \int_0^1 \dd{x} \frac{1}{((1-x)^2 m^2 + x\mu^2)^{2-d/2}} \\
    &{\phantom{{}={}}} \quad \times \qty[(2-\epsilon)x - \frac{\epsilon}{2}\frac{2x(1-x)m^2}{(1-x)^2 m^2 + x \mu^2}(4-2x-\epsilon(1-x))], \\
    \delta_1 &= -\frac{e^2}{(4\pi)^{d/2}} \Gamma(2-d/2) \int_0^1 \dd{x} \frac{1}{((1-x)^2 m^2 + x\mu^2)^{2-d/2}} \\
    &{\phantom{{}={}}} \quad \times (1-x) \times \qty[\frac{(2-\epsilon)^2}{2} + \frac{\epsilon}{2}\frac{1}{(1-x)^2m^2 + x\mu^2} \qty[2(1-4x+x^2) - \epsilon(1-x)^2]m^2].
\end{align*}
The difference is given by
\begin{align*}
    \delta_2 - \delta_1 &= -\frac{e^2}{(4\pi)^{d/2}} \Gamma(\epsilon/2) \int_0^1 \dd{x} \frac{1}{((1-x)^2 m^2 + x\mu^2)^{1+\epsilon/2}} \\
    &{\phantom{{}={}}} \quad \times \qty[2 m^2 (x-1)^2 (x (\epsilon -2)+1)+\mu ^2 x (x (\epsilon -4)-\epsilon +2)] \\
    &= -\frac{e^2}{(4\pi)^{d/2}} \Gamma(\epsilon/2) \times (-2) \int_0^1 \dd{x} \dv{x}\qty[\frac{x(x-1)}{((1-x)^2 m^2 + x\mu^2)^{\epsilon/2}}] \\
    &= 0.
\end{align*}
Expanding around $\epsilon=0$ we find
\begin{align*}
    \delta_1 &= \frac{e^2}{(4\pi)^2} \int_0^1 \dd{x} \\
    &{\phantom{{}={}}} \quad\qty[-\frac{4x}{\epsilon} + 2x\qty(1+\gamma+\frac{2m^2(x-2)(x-1)}{(1-x)^2m^2 + x\mu^2} + \log\qty(\frac{(1-x)^2m^2 + x\mu^2}{4\pi})) + \bigO(\epsilon)].
\end{align*}

\prule

\plabel{3}%
It suffices to show that the following
\[ I^\mu(P,p) = \int \frac{\dd[4]{k}}{(2\pi)^4} \frac{\overline{u}(P)(p-k)_\nu (p-k)_\rho \gamma^\nu (\slashed{k} + \slashed{P} - \slashed{p} + m) \gamma^\mu (\slashed{k}+m) \gamma^\rho u(p)}{\qty((p-k)^2 - \mu^2)^2 \qty((k+P-p)^2 - m^2) (k^2 - m^2)} \]
has the form
\[ I^\mu(P,p) = \int \frac{\dd[4]{k}}{(2\pi)^4} \overline{u}(P) \qty{\text{something doesn't on $P,p,m$}} u(p). \]
This holds since the following input
\begin{mmaCell}[moredefined={ScalarProduct, OneLoopSimplify, SpinorUBarD, Pair, LorentzIndex, Momentum, GAD, GSD, SpinorUD, FAD}]{Input}
ScalarProduct[p,p]=\mmaSup{m}{2};
ScalarProduct[P,P]=\mmaSup{m}{2};
OneLoopSimplify[
    SpinorUBarD[P,m].
    Pair[LorentzIndex[\mmaUnd{\(\pmb{\nu}\)},D],Momentum[p-k,D]].
    Pair[LorentzIndex[\mmaUnd{\(\pmb{\rho}\)},D],Momentum[p-k,D]].
    GAD[\mmaUnd{\(\pmb{\nu}\)}].
    (GSD[k]+GSD[P-p]+m).
    GAD[\mmaUnd{\(\pmb{\mu}\)}].
    (GSD[k]+m).
    GAD[\mmaUnd{\(\pmb{\rho}\)}].
    SpinorUD[p,m].
    FAD[\{p-k,\mmaUnd{\(\pmb{\mu}\)},2\},\{k,m\},\{k+P-p,m\}],
    k
]
\end{mmaCell}
yields the following output.
\begin{mmaCell}{Output}
\mmaFrac{(\(\varphi\)(P,m)).\mmaSup{\(\gamma\)}{\(\mu\)}.(\(\varphi\)(p,m))}{\mmaSup{(\mmaSup{k}{2}-\mmaSup{\(\mu\)}{2})}{2}}
\end{mmaCell}

Let
\[ I^\mu(P,p) =  \overline{u}(P) \gamma^\mu I(\mu) u(p). \]
The divergent bahavior is given by
\begin{align*}
    I(\mu) &= \int \frac{\dd[d]{k}}{(2\pi)^d} \frac{1}{(k^2 - \mu^2)^2} = \frac{i}{(4\pi)^{d/2}}\frac{\Gamma(2-d/2)}{\Gamma(2)} \qty(\frac{1}{\mu^2})^{2-d/2} \\
    &= \frac{i}{(4\pi)^2}\qty(\frac{2}{\epsilon} + \log(\frac{4\pi}{\mu^2}) - \gamma) + \bigO(\epsilon)
\end{align*}
with dimensional regularization ($d=4-\epsilon$) and
\begin{align*}
    I(\mu) &= \int \frac{\dd[d]{k}}{(2\pi)^d} \qty[\frac{1}{(k^2 - \mu^2)^2} - \frac{1}{(k^2 - \Lambda^2)^2}] \approx \frac{i}{(4\pi)^2} \log(\frac{\Lambda^2}{\mu^2})
\end{align*}
with Pauli-Villars regularization.

\prule

\plabel{4 (a)}%
Replacing $e \gamma^\mu$ with the effective vertex $e \Gamma^\mu$, we find
\begin{align*}
    \bra{\vb{p}_2,s_2} H \ket{\vb{p}_1,s_1} &= \int \dd[3]{\vb{x}} e A_\mu(\vb{x}) \overline{u}_2 \Gamma^\mu(p_2,p_1) u_1 e^{-i\vb{q}\cdot \vb{x}} \\
    &= \int \dd[3]{\vb{x}} e \overline{u}_2 \qty(A_0 \gamma^0 F_1 - A_i \gamma^i F_1 - \frac{i \sigma^{\mu\nu} A_\mu q_\nu}{2m} F_2) u_1 e^{-i\vb{q}\cdot \vb{x}} .
\end{align*}
To the first order in $q$, we have
\[ u(p) \approx \sqrt{m} \begin{pmatrix}
    (1-\vb{p}\cdot \vb*{\sigma}/2m)\xi \\
    (1+\vb{p}\cdot \vb*{\sigma}/2m)\xi
\end{pmatrix} \]
and
\begin{align*}
    \bra{\vb{p}_2,s_2} H \ket{\vb{p}_1,s_1} &= 2m \int \dd[3]{\vb{x}} \xi_2^\dagger \\
    & e \qty(A_0 \qty(F_1(0) - \frac{i\sigma^{0\nu} q_\nu}{2m}F_2(0)) - A_i \qty(\frac{-i}{2m} \epsilon^{ijk} q^j \sigma^k) (F_1(0) + F_2(0))) \\
    &\xi_1 e^{-i\vb{q}\cdot \vb{x}} .
\end{align*}
Therefore,
\[ H_I = eA_0 \qty(1 - \frac{i\sigma^{0\nu} q_\nu}{2m}F_2(0)) - eA_i \qty(\frac{-i}{2m} \epsilon^{ijk} q^j \sigma^k) (1 + F_2(0)). \]
For Coulomb interaction, $H_I = e A_0$ to the lowest order of $q$.
For magnetic interaction,
\begin{align*}
    \bra{\vb{p}_2,s_2} H \ket{\vb{p}_1,s_1} &= -2m \int \dd[3]{\vb{x}} \xi^\dagger_2  e A^i\qty(\frac{i}{2m} \epsilon^{ijk} q^j \sigma^k)(1+F_2(0)) \xi_1 e^{-i\vb{q}\cdot \vb{x}} \\
    &= -2m \int \dd[3]{\vb{x}} \xi^\dagger_2  \qty(\frac{e}{2m} B^k \sigma^k)(1+F_2(0)) \xi_1 e^{-i\vb{q}\cdot \vb{x}}
\end{align*}
and therefore
\[ H_I = -\frac{e}{2m} B^k \sigma^k (1+F_2(0)). \]

\plabel{(b)}%
By identifying
\[ H_I = -\vb*{\mu} \cdot \vb{B} \]
we find
\[ \vb{\mu} = \frac{e}{m}\vb{S}(1+F_2(0)). \]
Therefore, $g = 2+2F_2(0)$.
At tree level $F_2(0) = 0$ and therefore $g=2$.
At 1-loop level $F_2(0) = \alpha/2\pi$ and therefore $g = 2+\alpha/\pi$.

\plabel{(c)}%
At one-loop level $g/2 \approx \num{1.001161}$.
The spin frequency is given by $\omega = g(eB)/(2m)$ while the cyclotron frequency is given by $\Omega = eB/(m)$ and therefore $\omega/\Omega = g/2$.

\end{document}
