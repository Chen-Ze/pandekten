\documentclass{article}

\usepackage{pandekten}

\title{Arithmetics}
\author{Ch\=an Taku}

\begin{document}

\maketitle

\section{Hensel's Lemma}

\paragraph*{Statement of the Question}
Let's say we have an equation $f=0$ for $f\in k[x]$.
We have a solution $a$ such that $f(a) = 0$.
\par
Now we add an perturbation to $f$, i.e. consider the equation $f + tg = 0$ where $g\in k[x]$.
Could we solve $x$ as a formal power seires in $t$?
Let $R = \widehat{k[t]} = k\llbracket t \rrbracket$.
Then $h = f+tg\in R[x]$.
The maximal ideal $M$ of $R$ is $(t)$.
\paragraph*{Checking the Condition of Hensel's Lemma}
Now we apply Hensel's lemma.
The condition of Hensel's lemma requires that
\[ h(a) \in h'(a)^2 M. \]
What's $h(a)$? $h(a) = f(a) + t g(a) = t g(a) \in M$.
What's $h'(a)^2 M$? $h'(a)$ is $f'(a) + t g'(a)$.
If $f'(a) \neq 0$ then $h'(a)$ is a unit and therefore $h'(a)^2 M = M$.
That is, the condition is satisfied if $\boxed{f'(a) \neq 0}$.
\paragraph*{Consequence from Hensel's Lemma}
Applying Hensel's lemma we find that there exists $b\in R$ such that
\begin{itemize}
    \item $h(b) = 0$, and
    \item $b - a \in f'(a) M$.
\end{itemize}
The first statement tells us that $b\in k\llbracket t \rrbracket$ solves $f+tg=0$, i.e. $b$ is a formal power series solution.
The second statement tells us, since $f'(a)M = M$ given $f'(a)\neq 0$, that $b-a = t\times (\cdots)$, i.e. the zeroth order approximation is $a$.
\paragraph*{Counterexample}
If $f'(a) = 0$ then there may not be a formal power series solution.
For example, $x^2 + t = 0$ requires $x = \sqrt{t}$ and this does not admit a formal power series expansion.

% \bibliographystyle{plain}
% \bibliography{main}

\end{document}
