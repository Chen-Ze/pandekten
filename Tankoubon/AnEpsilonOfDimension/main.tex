\documentclass{article}

\usepackage{pandekten}

\title{An $\epsilon$ of Dimension\\
\large On Renormalization Group}
\author{Ch\=an Taku}

\begin{document}

\maketitle

We know how to evaluate $G^{(n)}$ for a given renormalization scale $M$ and condition $\lambda$:
\begin{align*}
    G^{(2)}(p;\lambda,M) &= \frac{i}{p^2} + \frac{i}{p^2}\qty(A \log \frac{\Lambda^2}{-p^2} + \text{finite}) + \frac{i}{p^2}(ip^2\delta_Z)\frac{i}{p^2} + \cdots,\\
    G^{(n)}(p_1,\cdots,p_n;\lambda,M) &= \qty(\prod_i \frac{i}{p_i^2}) \times\\
    &\phantom{{}={}}\qty[-ig - iB\log \frac{\Lambda^2}{-p^2} - i\delta_g + (-ig)\sum_i\qty(A_i \log \frac{\Lambda^2}{-p_i^2} - \delta_{Z_i})] \\
    &\phantom{{}={}} + \text{finite}.
\end{align*}
%We also know from the Callan-Symanzik equation how $\lambda$ and $Z$ should flow as $M$ changes.
%Then, could we know how $G^{(n)}$ flow as $M$ changes?
%That is, given the value of $G^{(n)}(\lambda,M)$ at some $\lambda$ and $M$, could we obtain the value of $G^{(n)}(\lambda',M')$, without refering to $M'$, i.e. we know the new value of $\lambda$ but we don't know at which $M'$ does $\lambda$ attain this value?

\par
$\gamma(\lambda_*)$ is called the anomalous dimension of the scalar field (see p.427), or $\gamma(\lambda)$ even if the fixed point doesn't exist.

% \bibliographystyle{plain}
% \bibliography{main}

\end{document}
