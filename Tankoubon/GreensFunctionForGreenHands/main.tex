\documentclass{article}

\usepackage{pandekten}

\usepackage[compat=1.1.0]{tikz-feynman}

\usetikzlibrary{external}
\immediate\write18{mkdir -p pgf-img}
\tikzexternalize[
  prefix=pgf-img/,
  system call={
    lualatex \tikzexternalcheckshellescape -halt-on-error -interaction=batchmode -jobname="\image" "\texsource" || rm "\image.pdf"
  },
]

\title{Green's Function for Green Hands\\
\large On Green's Function in Condensed Matter Field Theory}
\author{Ch\=an Taku}

\newtheorem{myex}{Example}
\newtheorem{myrm}{Remark}

\begin{document}

\maketitle

\section{Perturbation}

The vacuum bubbles satisfy
\begin{align*}
    \sum_v d_v &\ge 3V, \\
    V &= E + (1-L),
\end{align*}
if each vertex has degree greater than or equal to $3$.
Therefore, given the number of loops, there are only finitely many topologies of vacuum bubbles even if we don't post any restriction on the degrees of vertices.
However, \texttt{FaynArts} supports only up to degree $4$.

\subsection{Symmetry Factor}

A feynman diagram with unique indices assigned to
\begin{itemize}
    \item each vertex ($1$, $2$, etc.),
    \item each slot of each vertex ($1^a$, $1^b$, $2^a$, $2^b$, etc.),
    \item each propagator ($A$, $B$, etc.), and
    \item each head of each propagator ($A^{\text{in}}$, $A^{\text{out}}$, $B^{\text{in}}$, $B^{\text{out}}$, etc.)
\end{itemize}
is represented by a one-to-one pairing of (guaranteed by $\sum_v d_v = 2E$)
\begin{equation}
    \label{eq:label_feyn}
    \qty{\text{heads of propagators}} \longleftrightarrow \qty{\text{slots of vertices}}.
\end{equation}
A typical term of perturbation looks like
\begin{align}
    S^{(3)} &= \frac{1}{3!} \qty(\frac{1}{4!}\delta\delta\delta\delta)\qty(\frac{1}{4!}\delta\delta\delta\delta)\qty(\frac{1}{4!}\delta\delta\delta\delta) \label{eq:vertex_factor} \\ & \phantom{{}={}} \times  \frac{1}{6!}\qty(\frac{1}{2}jGj)\qty(\frac{1}{2}jGj)\qty(\frac{1}{2}jGj)\qty(\frac{1}{2}jGj)\qty(\frac{1}{2}jGj)\qty(\frac{1}{2}jGj). \label{eq:propagator_factor}
\end{align}
Now we should sum over $12!$ diagrams.
Each diagram is a way to contract
\[ (\delta\delta\delta\delta)(\delta\delta\delta\delta)(\delta\delta\delta\delta)(jGj)(jGj)(jGj)(jGj)(jGj)(jGj). \]
These $12!$ diagrams are divided into orbits of the group $G$ generated by
\begin{itemize}
    \item $G_V$, interchanging vertices (of the same kind),
    \item $G^v_S$, for each vertex $v$, interchanging the slots of $v$ (of the same kind),
    \item $G_P$, interchanging propagators (of the same kind),
    \item $G^p_H$, for each propagator $p$, interchanging the heads of $p$ (the trivial group for fermions)
\end{itemize}
that acts on the set of pairing \eqref{eq:label_feyn}.
$G = G_V \times G^v_S \times G_P \times G^p_H$ since the factor groups commute with each other.
The factorial factors of the perturbation term are just
\[ \frac{1}{\abs{G}} = \frac{1}{\abs{G_V}} \frac{1}{\abs{{G^v_S}}^V} \frac{1}{\abs{G_P}} \frac{1}{\abs{G^p_G}^P} = \frac{1}{V!} \frac{1}{(d!)^V} \frac{1}{P!} \frac{1}{2^P}. \]
Each element in the orbit contributes
\[ \frac{1}{\abs{G}}\times \langle {}\cdots \rangle. \]
The whole contribution of the orbit $O$ is therefore
\[ \frac{\abs{O}}{\abs{G}}\times \langle {}\cdots \rangle = \frac{1}{S}\langle {}\cdots \rangle, \]
where $S$ is the order of stabilizer of the orbit $O$, i.e. the number of elements in $G$ that keeps the set of pairing \eqref{eq:label_feyn} invariant.

\paragraph*{Why we have stabilizer}
The reason why $\abs{G} \neq \abs{O}$ (or $S\neq 1$) is that some permutations may results in exactly the same diagram, e.g. for the 1-loop of $\phi+\phi\rightarrow\phi+\phi$ in $\phi^4$ theory we could interchange two slots of each vertices and interchange two propagators and find the same diagram.

\par
The expression above for $V^{(3)}$ has only one kind of vertex and one kind of propagator.
In the following subsubsections we will find that the same factor $1/S$ applies to more complex cases.
We write
\[ S^{(V)} = S^{(V)}_{V} S^{(V)}_{P} \]
to separate the contribution $S^{(V)}_V$ from the vertices, i.e. \eqref{eq:vertex_factor}, and $S^{(V)}_P$ from the propagators, i.e. \eqref{eq:propagator_factor}.
\par
Moreover, we should also consider the cases where we have external vertices.

\subsubsection{Symmetry Factor: Different Types of Vertices}

The \textit{vertex} part of the expansion is given by
\[ S^{(\abs{V})}_V = \frac{1}{\abs{V}!} (V^{\text{type }\alpha} + V^{\text{type }\beta} + V^{\text{type }\gamma} + \cdots)^{\abs{V}}, \]
where $\abs{V}$ is the number of vertices, which equals to the order of the taylor expansion of $e^{\sum_{i\in I} V^{\text{type} i} }$.
\par
We may assume that vertices of different types commute with each other (they should, since they are just $\int \delta\delta\delta\delta\cdots$).
Therefore
\begin{align*}
    & S^{(\abs{V})}_V = \frac{1}{\abs{V}!} \frac{\abs{V}!}{\abs{\alpha}! \abs{\beta}! \abs{\gamma}! \cdots} \qty(V^{\text{type }\alpha})^{\abs{\alpha}}\qty(V^{\text{type }\beta})^{\abs{\beta}}\qty(V^{\text{type }\gamma})^{\abs{\gamma}}\cdots.
\end{align*}
That is, the prefactor is still
\[ \frac{1}{\abs{\alpha}! \abs{\beta}! \abs{\gamma}! \cdots} = \frac{1}{\abs{G_V}}, \]
where $\abs{\alpha}$ is the number of vertices of type $\alpha$, etc.
Of course they are subject to
\[ \abs{\alpha} + \abs{\beta} + \abs{\gamma} + \cdots = \abs{V}. \]

\subsubsection{Symmetry Factor: Different Types of Propagators}

The \text{propagator} part of the expansion is given by
\[ S^{(\abs{V})}_P = \frac{1}{\abs{P}!} (P^{\text{type }\alpha} + P^{\text{type }\beta} + P^{\text{type }\gamma} + \cdots)^{\abs{P}}, \]
where $\abs{P}$ is the number of propagators, which equals to the order of the taylor expansion of $e^{\sum_{i\in I} P^{\text{type} i} }$.
\par
We may assume that propagators of different types commute with each other (they should, since they are just $\int \overline{\eta} G \eta$---not even operators).
\begin{align*}
    & S^{(\abs{V})}_P = \frac{1}{\abs{P}!} \frac{\abs{P}!}{\abs{\alpha}! \abs{\beta}! \abs{\gamma}! \cdots} \qty(V^{\text{type }\alpha})^{\abs{\alpha}}\qty(V^{\text{type }\beta})^{\abs{\beta}}\qty(V^{\text{type }\gamma})^{\abs{\gamma}}\cdots.
\end{align*}
That is, the prefactor is still
\[ \frac{1}{\abs{\alpha}! \abs{\beta}! \abs{\gamma}! \cdots} = \frac{1}{\abs{G_P}}, \]
where $\abs{\alpha}$ is the number of propagators of type $\alpha$, etc.
Of course they are subject to
\[ \abs{\alpha} + \abs{\beta} + \abs{\gamma} + \cdots = \abs{P}. \]

\subsubsection{Symmetry Factor: External Vertices}
\label{sssec:ext_vertex}

With external vertices, each term looks like
\[ \delta_1\delta_2\delta_3\delta_4(\delta\delta\delta\delta)(\delta\delta\delta\delta)(\delta\delta\delta\delta)(jGj)(jGj)(jGj)(jGj)(jGj)(jGj)(jGj)(jGj). \]
Now we can regard each external vertices as ordinary (internal) vertices with a single slot, except that different external vertices are not interchangable.
The difference is from that each internal vertex is
\[ \int \dd{x} \delta_{j(x)} \delta_{j(x)} \delta_{j(x)} \delta_{j(x)} \]
and it is therefore legitimate to interchange two internal vertices.
However, an external vertex has the form
\[ \delta_1 = \frac{\delta}{\delta j(x_1)}, \]
where $x_1$ is a fixed value (not integrated).

\paragraph{What if we have duplicate coordinates}
If $x_1 = x_2$, e.g. if we are evaluating
\[ \langle \phi(x_1) \phi(x_1) \phi(x_3) \phi(x_4) \rangle, \]
then we may treat $\phi(x_1) \phi(x_1)$ as a vertex with two slots.
\par
Now we have two approaches.
One is to write
\[ \langle \phi(x_1) \phi(x_1) \phi(x_3) \phi(x_4) \rangle = 2 \times \langle \frac{1}{2}\phi(x_1) \phi(x_1) \phi(x_3) \phi(x_4) \rangle, \]
Here $1/2 = 1/\abs{G^{\text{external}}}$.
In this case, the existence of stabilizer depends on whether the diagram is \textit{symmetric} enough.
For example, if the two propagators attached to $\phi(x_1)\phi(x_1)$ are connected to the same internal vertex, then there is a stabilizer and $\abs{O} = 1$.
So the factor $1/2$ is retained and cancels the factor of $2$.
Otherwise, $\abs{O} = 2$ and the factor of $2$ will be retained.
\par
We could also evaluate first $\langle \phi(x_1) \phi(x_2) \phi(x_3) \phi(x_4) \rangle$ and then take $x_1 = x_2$.
With this approach, the symmetry factor also depends on if the diagram is \textit{symmetric} enough.
If it is, then we have to multiply by a factor of $2$.
\par
It can be seen that these two approaches coincides with each other, i.e. stabilizers in the first approach are in one-to-one correspondence to stabilizer in the second approach.

\subsubsection{Symmetry Factor: Momentum Space}

Note that (up to a constant factor)
\[ \int \dd{x} \phi(x) j(x) = \int \dd{p} \phi(p) j(-p) \]
so $\phi(p) = \delta_{j(-p)}$, and that (up to a constant factor)
\begin{align*}
    \int \dd{x} \dd{x'} j(x) G(x-x') j(x') &= \int \dd{p} j(-p) (G*j)(p) = \int \dd{p} j(-p) G(p) j(p) \\
    &= \int \dd{p} \dd{q} \delta(p+q) j(q) G(p) j(p).
\end{align*}
Each vertices becomes
\begin{align*}
    \int \dd{x} \phi(x) \phi(x) \phi(x) \phi(x) &= \int \dd{p_1} \cdots \dd{p_4} \delta(p_1 + \cdots + p_4) \phi(p_1) \cdots \phi(p_4) \\
    &= (-1)^{\cdots} \int \dd{p_1} \cdots \dd{p_4} \delta(p_1 + \cdots + p_4) \delta_{j(p_1)} \cdots \delta_{j(p_4)}.
\end{align*}
The arguments in previous sections still apply---we still write a diagram loosely as
\[ \delta_1\delta_2\delta_3\delta_4(\delta\delta\delta\delta)(\delta\delta\delta\delta)(\delta\delta\delta\delta)(jGj)(jGj)(jGj)(jGj)(jGj)(jGj)(jGj)(jGj), \]
except that everything carries a momemtum index now.
\par
We consider the tree level $\phi(p_1)+\phi(p_2) \rightarrow \phi(p_3)+\phi(p_4)$, i.e.
\begin{align*}
    \langle \phi(p_1) \cdots \phi(p_4) \rangle &= \delta_1\delta_2\delta_3\delta_4(\delta\delta\delta\delta)(jGj)(jGj)(jGj)(jGj) \\
    &= \delta(p_1 + \cdots + p_4) G(p_1) \cdots G(p_4).
\end{align*}
For more complex fields and interactions, we are still guaranteed to see a factor of
\begin{equation}
    \delta\qty(\sum_{\text{external } p} p) \prod_{\text{external } p} G(p). \label{eq:amp_factor}
\end{equation}
The LSZ formula tells us to discard $\prod G(p)$ to get $\bra{f}\ket{i}$, and further discard $\delta(\sum p)$ to get the $\mathcal{M}$ matrix.

\par
The conclusion is that, when there are incident particles of given momenta, the discussion in \cref{sssec:ext_vertex} still works.
We just regard the incident particle $i$ as an external vertex $\delta_{p_i}$, and the symmetry factor is still determined as described in \cref{sssec:ext_vertex}.
\par
The discussion is much simpler if we use canonical quantization, i.e. we have external operators $a^\dagger(p_i)_{t=-\infty}$ and $a(p_j)_{t=+\infty}$.
With this formulation, everything in \cref{sssec:ext_vertex} carries on.

\subsection{Finding Effective Action}

\label{ssec:finding_gamma}

\paragraph*{The problem---correspondence between field theory and statistics}
We find the following correspondence.
\begin{longtable}{cc}
    Field Theory & Statistics \\
    \midrule
    Removing vacuum bubble & Normalizing by $1/Z$ \\
    Diagrams (without bubble) & Moment $\langle \phi_1\cdots \phi_n\rangle $ \\ Connected diagrams & Joint cumulant \\
    1PI diagrams & ? \\
    Tree level contribution & ? \\
    Classical solutions of motion & Stationary points of PDF \\
    Amputated diagrams & Devided by $\langle \phi\phi \rangle \cdots \langle \phi\phi \rangle$ \\
\end{longtable}

\paragraph*{The attempt}
Let $X$ be a random variable with PDF $f_X$.
$\kappa_n(X)$ denotes the $n$-th cumulant of $X$.
Then
\[ \kappa^{\text{tree}}_n(X) = \lim_{t\rightarrow\infty} \frac{\kappa_n(Y_t)}{t \kappa_2(Y_t)^n} \]
is called the tree-level cumulant of $X$, where $Y_t$ is the random variable with PDF $f_X^t$ normalized.
Let $Z$ be a random variable with PDF $f_Z$.
$X$ is called the effective random variable of $Z$ if
\[ \kappa_n^{\text{tree}}(X) = \kappa_n(Z), \]
and $f_X$ is called the effective PDF.

\begin{itemize}
    \item Note that as $t\rightarrow\infty$, $Y_t$ is concentrated on the values where $f_X$ reaches its maximal, i.e. the classical limit.
    Therefore, the tree-leve cumulant may be regarded as the cumulant when quantum fluctuations are ignored.
    The effective action is the one which gives the correct cumulant (and expectation values) when regarded as a classical theory (i.e. ignoring its quantum fluctuations and evaluate only the tree-level).
    \item In particular, the effective action (i.e. $X$) gives the correct expectation value of $Z$ with the mode of $X$, i.e. the classical limit.
    \item The contribution from a single vertex of $\Gamma$ is just
    \[ \fdv{\phi}\cdots \fdv{\phi}\Gamma[\phi], \]
    i.e. getting the coefficient of $\phi \cdots \phi$.
    \item Put it in the other way,
    \[ \langle \phi \rangle_{\text{tree}} = 
    \hbar \lim_{\hbar\rightarrow 0} \langle \phi \rangle = \argmax_{\varphi} L = \text{solution of Euler-Lagrange}. \]
\end{itemize}

\section{Spontaneous Symmetry Breaking}

\subsection{Broken Lagrangian}

We consider the following theory that has the following (of course this is from the $\phi^4$ theory but let's forget this for the moment).
\begin{itemize}
    \item A scalar field $\sigma$, and a collection of $N-1$ scalar fields $\pi$ (i.e. $\pi^1,\cdots,\pi^{N-1}$).
    \item Two fixed parameters $v$ and $\lambda$.
    \item Three sliding (counterterm) parameters $\delta_Z$, $\delta_\mu$, and $\delta_\lambda$.
    \item Lagrangian
    \begin{align*}
        & \mathcal{L} = \\
        & \quad \frac{1}{2}(\partial \pi)^2 + \frac{1}{2} (\partial \sigma)^2 - \frac{1}{2} (2\lambda v^2) \sigma^2 \\
        & \quad - \lambda v \sigma^3 - \frac{\lambda}{4}\sigma^4 - \cdots \\
        & \quad + \delta_Z\qty[\frac{1}{2}(\partial\pi)^2 + \frac{1}{2}(\partial \sigma)^2] + \delta_\mu\qty[\frac{1}{2} \pi^2 + \frac{1}{2}\sigma^2 - v \sigma + \cdots] + \delta_\lambda\qty[v^3 \sigma - v \sigma^3 + \cdots].
    \end{align*}
    \item The dimension is $d = 4-\epsilon$.
    \item The propgator of $\pi$ has an infrared regulator $\zeta$, i.e.
    \[ \pi^i\pi^j \sim \frac{i \delta^{ij}}{k^2 - \zeta^2}. \]
\end{itemize}

The three sliding parameters are not determined yet.
Their values could be fixed by imposing renormalization conditions.
The names for the sliding (counterterm) parameters are not important here.
We could even rename them to $x$, $y$, and $z$.
The fixed parameters are just numbers.
We can safely replace them with any numerical values like \num{3.14} or \num{2.72} (of course we don't replace the subscripts---Mathematica does that), or even better, with $2$ or $3$.
If we take $\lambda = 3$ and $v=2$, and denote $\delta_Z = x$, $\delta_\mu = y$, and $\delta_\lambda = z$, then the theory looks innocent:
\begin{align*}
    & \mathcal{L} = \\
    & \quad \frac{1}{2}(\partial \pi)^2 + \frac{1}{2} (\partial \sigma)^2 - \frac{1}{2} (24) \sigma^2 \\
    & \quad - 6 \sigma^3 - \frac{3}{4}\sigma^4 - \cdots \\
    & \quad + x \qty[\frac{1}{2}(\partial\pi)^2 + \frac{1}{2}(\partial \sigma)^2] + y \qty[\frac{1}{2} \pi^2 + \frac{1}{2}\sigma^2 - 2 \sigma + \cdots] + z \qty[8 \sigma - 2 \sigma^3 + \cdots].
\end{align*}

\paragraph*{Renormalization}

Now we impose the renormalization conditions, at some scale $M$ (in Peskin it is $m^2$, and I don't understand what $m$ refers to here).
\begin{align*}
    \feynmandiagram [layered layout, inline=(p.base), horizontal=v to p, small] {
        v[blob] -- p[particle=\vphantom{$Mg$}],
    }; &= 0, \\
    \eval{\dv{p^2} \qty(\feynmandiagram [layered layout, inline=(p.base), horizontal=p1 to p, small] {
        p1[particle=\vphantom{$Mg$}] -- v[blob] -- p[particle=\vphantom{$Mg$}],
    };)}_{p^2=M^2} &= 0, \\
    \eval{\Im \qty(\feynmandiagram [inline=($0.5*(p1.base)+0.5*(p4.base)$), horizontal=p1 to p3, small] {
        p1[particle=\vphantom{$a$}] -- v[blob],
        p2[particle=\vphantom{$a$}] -- v,
        p3[particle=\vphantom{$a$}] -- v,
        p4[particle=\vphantom{$a$}] -- v,
    };)}_{s=4M^2,t=u=0} &= -6i\lambda.
\end{align*}
We hope that the renormalization conditions give rise to counterterm coefficients of lower order, i.e. we want to see $\delta_\lambda \sim \lambda^2$ rather than $\delta_\lambda \sim \lambda$.
If the latter case happens, we know the renormalization conditions are not well chosen (in principle we could do blind calculation and obtain $\delta_\lambda$, but the counterterm may fail to cancel the divergences?).

\par
Then indeed we could find the counterterms (the divergent part, i.e. determined by requiring a finite amplitude)
\begin{align*}
    \delta_\lambda &\sim \lambda^2 (N+8) \frac{\Gamma(2-d/2)}{(4\pi)^2}, \\
    \delta_\mu + v^2 \delta_\lambda &= -\lambda \frac{\Gamma(1-d/2)}{(4\pi)^{d/2}} \qty(\frac{3}{(2\mu^2)^{1-d/2}} + \frac{N-1}{(\zeta^2)^{1-d/2}}), \\
    \delta_Z &\sim 0.
\end{align*}

\subsection{Returning to Symmetric Lagrangian}

Now we have obtained a set of values for $\delta_Z$, $\delta_\mu$, and $\delta_\lambda$.
No matter what their values are, we could make a change of variable
\[ (\phi^1,\cdots,\phi^{N-1},\phi^N) = (\pi^1,\cdots,\pi^{N-1}, v+\sigma) \]
and find
\begin{align*}
    \mathcal{L} &= \frac{1}{2}(\partial \phi)^2 + \frac{1}{2}(\lambda v^2) \phi^2 - \frac{\lambda}{4}(\phi^2)^2 \\
    &\phantom{{}={}} + \frac{1}{2}\delta_Z (\partial\phi)^2 - \frac{1}{2}\delta_\mu \phi^2 - \frac{\delta_\lambda}{4}(\phi^2)^2.
\end{align*}

\subsection{Effective Action}

\subsubsection{Definitions and Properties}

The effective action $\Gamma\colon \qty{\phi} \to \mathbb{R}$ has the following properties (not sure if they are independent).
\begin{itemize}
    \item The minimal of the action (i.e. its classical value) gives $\langle \phi \rangle$ (with quantum corrections).
    \item The tree level prediction of $\Gamma$, i.e.
    \[ \langle \phi_1 \cdots \phi_n \rangle = \sum \text{tree level of }\Gamma. \]
    This may be equivalent to say (really?), the prediction under the limit $\hbar \rightarrow 0$ is the LHS (see \cref{ssec:finding_gamma}).
    \item The coefficients of $\Gamma$ are the 1PI diagrams (this is related to the tree level property).
    \item $\Gamma$ equals to the one given by the background field method.
\end{itemize}

\subsubsection{Calculation of Effective Action}

The Lagrangian $\mathcal{L} = \mathcal{L}_1 + \delta\mathcal{L}$, i.e. renormalized terms plus counterterms.
If the Lagrangian contains bosonic fields $(\phi_i)_{i\in I}$ and fermionic fields $(\psi_j)_{j\in J}$, then around classical solution $\phi_{\textnormal{cl}}$, we write $\phi = \phi_{\textnormal{cl}} + \tilde{\phi}$ and there are no linear terms of $\tilde{\phi}$ (since it is a stationary point of $\mathcal{L} + J\phi$) and of $\psi$ (fermionic fields should occur in pair) in $\mathcal{L}$.
Given this, the effective action have the familiar form of Gaussian plus perturbation, i.e. ({\color{red}Question: is the first factor a number? or a c-number?})
\begin{align*}
    e^{i\Gamma[\phi_{\textnormal{cl}}]} &= e^{-i(E[J] + \int \dd[d]{x} J\phi_{\textnormal{cl}})} \\
    &= \exp[i\int \dd[d]{x} \mathcal{L}[\phi_{\textnormal{cl}}]] \\
    &\phantom{{}={}} \int D\tilde{\phi} D\psi \exp[i\int \dd[d]{x}\qty(\text{interactions} + \text{Gaussian terms in }(\tilde{\phi},\psi))],
\end{align*}
where $J$ is the classical source such that $\phi_{\textnormal{cl}}$ is the solution that corresponds to that source.
But the interactions here are not just the ones given by expanding $\mathcal{L}$ around $\phi_{\textnormal{cl}}$.
It can be shown that we could indeed put interactions to be just the expansion of $\mathcal{L}$ around $\phi_{\textnormal{cl}}$, except that the path integral is now summing the 1PI diagrams only.
We note that this has the form
\[ \Gamma[\phi_{\textnormal{cl}}] = \int \dd[d]{x} \mathcal{L}[\phi_{\textnormal{cl}}] -i\log \tilde{Z}[0], \]
where $\tilde{Z}$ is the generating functional for $(\tilde{\phi},\psi)$, given by (see section 16.1 of Weiberg)
\[ \tilde{Z} = \int_{\text{1PI bubbles}} D\tilde{\phi} D\psi \exp[i\int \dd[d]{x}\qty(\text{interactions} + \text{Gaussian in }(\tilde{\phi},\psi))], \]
where interactions are in given by the expansion.
Remember that
\begin{align*}
    \log \tilde{Z} &= \log \int e^{\text{interactions} + \text{Gaussian}} \\
    &= \log \int e^{\text{Gaussian}} + \log \frac{\int e^{\text{interactions} + \text{Gaussian}}}{\int e^{\text{Gaussian}}} \\
    &= \log \int e^{\text{Gaussian}} + \log \langle e^{\text{interactions}} \rangle \\
    &= \log \int e^{\text{Gaussian}} + \text{cumulants i.e. connected bubbles}.
\end{align*}
Then we could expand $\Gamma$ like the following.
\begin{itemize}
    \item $0$-loop order:
    \[ \Gamma[\phi_{\textnormal{cl}}] = \int \dd[d]{x} \mathcal{L}[\phi_{\textnormal{cl}}]. \]
    \item $1$-loop order:
    \[ \Gamma[\phi_{\textnormal{cl}}] = \int \dd[d]{x} \mathcal{L}[\phi_{\textnormal{cl}}] - i \log \tilde{Z}_{\text{Gaussian}}[0]. \]
    \item $n$-loop order:
    \[ \Gamma[\phi_{\textnormal{cl}}] = \int \dd[d]{x} \mathcal{L}[\phi_{\textnormal{cl}}] - i \log \tilde{Z}_{\text{Gaussian}}[0] - i\cdot \qty(\text{connected 1PI bubbles}). \]
\end{itemize}
We have to note the following when actually doing calculation.
\begin{itemize}
    \item The expansion of $\mathcal{L}$ itself may have a linear term.
    We intentionally drop this term.
    \item The Gaussian part is evaluated at renormalized parameters, while the $0$-loop part is evaluated at bare parameters, i.e. this is equation (11.63) of Peskin
    \[ \Gamma[\phi_{\textnormal{cl}}] = \int \dd[d]{x} \mathcal{L}_1[\phi_{\textnormal{cl}}] + \int \dd[4]{x} \delta\mathcal{L}[\phi_{\textnormal{cl}}] + \frac{i}{2} \log \det[-\frac{\delta^2\mathcal{L}_1}{\delta\phi\delta\phi}] - i\cdot (\text{bubbles}). \]
    \item The counterterms should be added such that $\Gamma$ (or the effective potential) is finite, although the bare Lagrangian is divergent.
\end{itemize}

\subsubsection{Examples}

\begin{myex}[Effective Action of Linear Sigma Model]
    Let $\phi = (\phi^i)_{1\le i \le N}$ and
    \[ \mathcal{L}[\phi] = \frac{1}{2}(\partial \phi)^2 + \frac{1}{2}\mu^2 \phi^2 - \frac{\lambda}{4}(\phi^2)^2. \]
    Expanding around $\phi_{\textnormal{cl}}$ (constant independent of position) we find
    \begin{align*}
        \mathcal{L} &= \frac{1}{2}\mu^2 \phi_{\textnormal{cl}}^2 - \frac{\lambda}{4}(\phi_{\textnormal{cl}}^2)^2 \\
        &{\phantom{{}={}}} + \frac{1}{2} (\partial \eta)^2 + \frac{1}{2}\mu^2 \eta^2 - \frac{\lambda}{2}\qty(\phi^2_{\textnormal{cl}} \eta^2 + 2\qty(\phi\cdot \eta)^2) \\
        &{\phantom{{}={}}} + \qty(\text{linear terms in }\eta) + \qty(\text{interactions i.e. terms cubic or greater}).
    \end{align*}
    We may diagonalize the mass term and find (linear and interactions dropped)
    \begin{align*}
        \mathcal{L} &= \frac{1}{2}\mu^2 \phi_{\textnormal{cl}}^2 - \frac{\lambda}{4}(\phi_{\textnormal{cl}}^2)^2 \\
        &{\phantom{{}={}}} + \frac{1}{2}(\partial \eta)^2  - \frac{1}{2}\eta^\intercal \begin{pmatrix}
            (\lambda \phi^2_{\textnormal{cl}} - \mu^2) & & & \\
            & (3\lambda \phi^2_{\textnormal{cl}} - \mu^2) & & \\
            & & \ddots \\
            & & & (3\lambda \phi^2_{\textnormal{cl}} - \mu^2)
        \end{pmatrix}\eta.
    \end{align*}
    Then we use the standard way to evaluate the determinant.
    With this we obtain the effective potential
    \begin{align*}
        V_{\mathrm{eff}}(\phi_N) &= -\frac{1}{2} \mu^2 \phi^2_{\textnormal{cl}} + \frac{\lambda}{4} \phi^2_{\textnormal{cl}} \\
        &{\phantom{{}={}}} -\frac{1}{2} \frac{\Gamma(-d/2)}{(4\pi)^{d/2}}\qty[(N-1)(\lambda \phi^2_{\textnormal{cl}} - \mu^2)^{d/2} + (3\lambda \phi^2_{\textnormal{cl}} - \mu^2)^{d/2}] \\
        &{\phantom{{}={}}} + \frac{1}{2}\delta_\mu \phi^2_{\textnormal{cl}} + \frac{1}{4} \delta_\lambda \phi^4_{\textnormal{cl}}.
    \end{align*}
    The divergent parts of $\delta_\mu$ and $\delta_\lambda$ may be determined by requiring that $V_{\textnormal{eff}}$ while the finite part depends on the renormalization conditions.
\end{myex}

\subsection{Questions}

\begin{itemize}
    \item Why does the following definitions of critical dimension matches?
    (Notice that some of the definitions may result in fractional dimension but others are integral.)
    \begin{itemize}
        \item The dimension s.t. the fluctuation of heat capacity is comparable to the singlar contribution.
        \item The dimension s.t. the $\beta$ functions shows irrelevance.
        \item The dimension such that that are only finitely many divergent irreducible diagrams, i.e. renormalizability.
    \end{itemize}
    \item What is an appropriate description of the effective action?
    A classical action that takes into account quantum fluctuation?
    And what does it mean by \textit{Goldstone bosons remains massless under renormalization}?
    Does it mean that the goldstone modes are still massless in the effective action?
\end{itemize}

\section{Renormalization Group}

\subsection{Calculations}

The following is a brief summary of pp. 5 ff. of Peskin.
\par
In the following, $\log(\Lambda^2/M^2)$ should be interpreted as (see equation (12.34) of Peskin)
\[ \log(\Lambda^2/M^2) = \frac{1}{2-d/2} - \log(M^2). \]

\paragraph{Calculating $\gamma$ Function}
To the lowest order,
\begin{equation}
    \label{eq:gamma_lowest_delta}
    \gamma = \frac{1}{2} M\pdv{M} \delta_Z.
\end{equation}
In particular, given
\[ \delta_Z = A \log \frac{\Lambda^2}{M^2} + \text{finite}, \]
we have
\begin{equation}
    \label{eq:gamma_lowest}
    \boxed{\gamma = \frac{1}{2} M\pdv{M} \delta_Z = -A.}
\end{equation}
This applies to both fermions and bosons, no matter if they have a nonzero $\delta_Z$ starting from one-loop or higher---just plug in the lowest order of nonzero $\delta_Z$.

\paragraph{Calculating $\beta$ Function}
To the lowest order,
\begin{equation}
    \label{eq:beta_lowest_delta}
    \beta(g) = M\pdv{M}\qty(-\delta_g + \frac{1}{2}g\sum_i \delta_{Z_i}),
\end{equation}
where the summation is done for propagators connected to this vertex.
In particular, given
\begin{align*}
    \delta_g &= -B \log \frac{\Lambda^2}{M^2} + \text{finite}, \\
    \delta_{Z_i} &= A_i \log \frac{\Lambda^2}{M^2} + \text{finite}, 
\end{align*}
we have
\begin{equation}
    \label{eq:beta_lowest}
    \boxed{\beta(g) = M\pdv{M}\qty(-\delta_g + \frac{1}{2}g\sum_i \delta_{Z_i}) = -2B - g\sum_i A_i.}
\end{equation}
This applies scalars as well as vertices involving fermions.

\begin{myrm}
    Since the finite parts of $\delta$'s are independent of $M$, they don't contribute to $\beta$ and $\gamma$.
    Therefore, to the leading order the exact form of renormalization condition is not important.
    That is to say, we need only the divergent part of $\delta$, and therefore only the divergent part of the loop diagrams.
\end{myrm}

\begin{myrm}
    \Cref{eq:gamma_lowest} applies to QED, i.e. we have
    \[ \gamma_2 = \frac{1}{2}M\pdv{M}\delta_2,\quad \gamma_3 = \frac{1}{2}M\pdv{M}\delta_3. \]
    \Cref{eq:beta_lowest} applies to QED as well.
    However, since $\delta_1$ in QED is defined by $\mathcal{L} = -(e+e\delta_1) \overline{\psi}A_\mu \gamma^\mu \psi$, we should put $\delta_g = e\delta_1$ and find
    \[ \beta(e) = M\pdv{M}\qty(-e\delta_1 + e\delta_2 + \frac{e}{2}\delta_3). \]
    Note that there are two fermion lines and therefore we have $e\cdot 2\times (1/2)\delta_2$.
\end{myrm}

% \bibliographystyle{plain}
% \bibliography{main}

\end{document}
