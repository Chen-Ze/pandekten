\documentclass{article}

\usepackage{pandekten}

\title{Chain Saw Manifold\\
\large On Algebraic Topology}
\author{Ch\=an Taku}

\begin{document}

\maketitle

\section{Homology Theory}

We write $(X,A) \simeq (Y,B)$ if there is $f\colon (X,A) \to (Y,B)$ where $f(A)\subset B$ and $g\colon (Y,B) \to (X,A)$ where $g(B)\subset A$ such that $f\circ g \simeq \operatorname{id}$ and $g\circ f \simeq \operatorname{id}$ where in each homotopy, $A$ is mapped inside $A$ and $B$ is mapped inside $B$.

\subsection{Basic Homology Groups}

$[f]$ stands for homotopy class and $\llbracket x \rrbracket$ stands for homology class.
$\Delta_p$ denotes the $p$-simplex.

\begin{corollary}{Zeroth Homology Group}{zeroth_homology_group}
    \begin{itemize}
        \item If $X$ is arcwise connected then $H_0 = \mathbb{Z}$ and is generated by $\llbracket x \rrbracket$ for any $x\in X$.
        \item For general $X$, $H_0(X)$ is canonically isomorphic to the free abelian group based on the arc components of $X$.
    \end{itemize}
\end{corollary}

\begin{corollary}{First Homology Group}{first_homology_group}
    If $X$ is arcwise connected then $H_1(X) = \operatorname{Ab}(\pi(X))$.
\end{corollary}

\begin{corollary}{Homology of Contractible Spaces}{homology_of_contractible_spaces}
    If $X$ is contractible then $\tilde{H}_*(X) = 0$.
\end{corollary}

\begin{definition}{Acyclic Space}{acyclic_space}
    A space $X$ is said to be acyclic if $\tilde{H}_*(X) = 0$.
\end{definition}

\subsection{Homological Algebra}

\begin{theorem}{SES of Chain to LES}{ses_of_chain_to_les}
    Let
    \[ 0 \xlongrightarrow{} A_* \xlongrightarrow{i} B_* \xlongrightarrow{j} C_* \xlongrightarrow{} 0 \]
    be a short exact sequence of chain complexes.
    Then we have the following long exact sequence
    \[ \cdots \xlongrightarrow{\partial_*} H_p(A_*) \xlongrightarrow{i_*} H_p(B_*) \xlongrightarrow{j_*} H_p(C_*) \xlongrightarrow{\partial_*} H_{p-1}(A_*) \xlongrightarrow{i_*} \cdots \]
    where
    \[ \partial_*\llbracket c \rrbracket = \llbracket i^{-1} \circ \partial \circ j^{-1}(c) \rrbracket \]
    and is called the connecting homomorphism.
\end{theorem}

\begin{definition}{Homology Theory, Coefficient Group}{homology_theory}
    A homology theory (on the category of all pairs of topological spaces and continuous maps) is
    \begin{itemize}
        \item a functor $H$ assigning to each pair $A\subset X$ of spaces, a graded abelian group $\Set*{H_p(X,A)}{p\in I}$, and
        \item to each map $f\colon (X,A) \to (Y,B)$, homomorphisms $f_*\colon H_p(X,A) \to H_p(Y,B)$ for each $p\in I$, together with
        \item a natural transformation of functors $\partial_*\colon H_p(X,A) \to H_{p-1}(A)$, called the connecting homomorphism, where $H_*(A) = H_*(A,\varnothing)$,
    \end{itemize}
    such that all of the following axioms are satisfied.
    \begin{itemize}
        \item The homotopy axiom: If $f\simeq g: (X,A) \rightarrow (Y,B)$, then
        \[ f_*=g_*: H_*(X,A) \rightarrow H_*(Y,B). \]
        \item Exactness axiom: For inclusions $\imath: A \hookrightarrow X$ and $\jmath: X \hookrightarrow (X,A)$ the following sequence
        \[ \cdots \xlongrightarrow{\partial_*} H_p(A) \xlongrightarrow{i_*} H_p(X) \xlongrightarrow{j_*} H_p(X,A) \xlongrightarrow{\partial_*} H_{p-1}(A) \xlongrightarrow{i_*} \cdots \]
        is exact.
        \item Excision axiom: Given a pair $(X,A)$ and an open set $U\subset X$ such that $\overline{U}\subset \operatorname{int}(A)$, the inclusion
        \[ k: (X-U,A-U) \rightarrow (X,A) \]
        induces an isomorphism
        \[ k_*: H_*(X-U,A-U) \xlongrightarrow{\cong} H_*(X,A). \]
        \item Dimension axiom: For a one-point space $P$, $H_i(P) = 0$ for all $i\neq 0$.
        \item Additivity axiom: For a topological sum $X = \coprod_\alpha X_\alpha$, the homomorphism
        \[ \bigoplus_\alpha(\imath_\alpha)_*: \bigoplus_\alpha H_n(X_\alpha) \rightarrow H_n(X) \]
        is an isomorphism, where $\imath_\alpha: X_\alpha \hookrightarrow X$ is the inclusion.
    \end{itemize}
    For a homology theory, $H_0(P) = G$ is called the coefficient group of the theory, where $P$ is the one-point space.
\end{definition}

Each map $f:(X,A) \rightarrow (Y,B)$ induced a natural transformation of the LES in the exactness axiom.

\subsection{Relative Homology}

\begin{definition}{Relative Homology}{relative_homology}
    Let $A\subset X$ be a pair of spaces.
    The relative homology is defined by
    \[ H_p(X,A) = H_p(\Delta_*(X,A)) \]
    where $\Delta_i(X,A) = \Delta_i(X)/\Delta_i(A)$.
\end{definition}
$\Delta_i(X,A)$ is the free abelian group generated by the singular $p$-simplices whose image are not completely in $A$.

\begin{theorem}{LES of Relative Homology}{les_of_relative_homology}
    Let $A\subset X$ be a pair of spaces.
    Then we have the following LES.
    \[ \cdots \xlongrightarrow{\partial_*} H_p(A) \xlongrightarrow{i_*} H_p(X) \xlongrightarrow{j_*} H_p(X,A) \xlongrightarrow{\partial_*} H_{p-1}(A) \xlongrightarrow{i_*} \cdots. \]
\end{theorem}

\subsection{Homology With Coefficients}

\begin{definition}{Homology With Coefficients}{homology_with_coefficients}
    Let $G$ be an abelian group.
    Let $A\subset X$ be a pair of spaces.
    Then
    \[ H_p(X;G) = H_p(\Delta_*(X) \otimes G) \]
    and
    \[ H_p(X,A;G) = H_p(\Delta_*(X,A) \otimes G). \]
    where $\Delta_*(X)\otimes G$ has differential map $\partial \otimes \mathbbm{1}$.
\end{definition}

\begin{theorem}{LES of Homology with Coefficients}{les_of_homology_with_coefficients}
    Let $A\subset X$ be a pair of spaces.
    Then we have the following LES.
    \begin{align*}
        \cdots &\xlongrightarrow{\partial_*} H_p(A;G) \xlongrightarrow{i_*} H_p(X;G) \xlongrightarrow{j_*} H_p(X,A;G) \\ &\xlongrightarrow{\partial_*} H_{p-1}(A;G) \xlongrightarrow{i_*} \cdots
    \end{align*}
\end{theorem}

\subsection{Homology of Sphere}

\begin{theorem}{Homotopy Preserves Homology}{homotopy_preserves_homology}
    If $(X,A) \simeq (Y,B)$ then
    \[ H_*(X,A) \cong H_*(Y,B). \]
\end{theorem}
\begin{proof}
    This is a consequence of the homotopy axiom.
\end{proof}

\begin{theorem}{Homology of Sphere}{homology_of_sphere}
    For $n\ge 0$, we have
    \begin{align*}
        \tilde{H}_i(S^n) &= \begin{cases}
            G, & \text{if } i = n,\\
            0, & \text{if } i \neq n,
        \end{cases} \\
        {H}_i(D^n, S^{n-1}) &= \begin{cases}
            G, & \text{if } i = n,\\
            0, & \text{if } i \neq n,
        \end{cases} \\
        \tilde{H}_i(S^n,D^n_+) &= \begin{cases}
            G, & \text{if } i = n,\\
            0, & \text{if } i \neq n,
        \end{cases}
    \end{align*}
    where $D^n_+$ stands for the closed upper hemisphere.
\end{theorem}
\begin{proof}
    We prove the statements using the axioms.
    \begin{enumerate}
        \item $H_i(S^0,D^0_+)$:
        In the excision axiom we choose $U=D^0_+$ we find $H_i(P) \cong H_i(S^0,D^0_+)$.
        Then we use the dimension axiom.
        \item $H_i(S^n,D^n_+) = \tilde{H}_i(S^n)$:
        Using the exactness axiom we find
        \[ 0 = \tilde{H}_i(D^n_+) \longrightarrow \tilde{H}_i(S^n) \longrightarrow H_i(S^n,D^n_+) \longrightarrow \tilde{H}_{i-1}(D^n_+) = 0. \]
        Note that $0 = \tilde{H}_i(D^n_+)$ is a consequence of \cref{thm:homotopy_preserves_homology}.
        \item $\tilde{H}_i(D^n, S^{n-1}) = H_i(S^n,D^n_+)$:
        In the excision axiom we choose $U$ to be a small disk neighborhood of the north pole of $S^n$, then
        \[ H_i(S^n,D^n_+) \cong H_i(S^n - U, D^n_+ - U). \]
        Now we apply \cref{thm:homotopy_preserves_homology} and find
        \[ H_i(S^n - U, D^n_+ - U) \cong H_i(D^n_-,S^{n-1}). \]
        \item $\tilde{H}_i(D^n,S^{n-1}) \cong \tilde{H}_{i-1}(S^{n-1})$:
        Using the exactness axiom and \cref{thm:homotopy_preserves_homology} we find
        \[ 0 = \tilde{H}_i(D^n) \longrightarrow H_i(D^n,S^{n-1}) \longrightarrow \tilde{H}_{i-1}(S^{n-1}) \longrightarrow \tilde{H}_{i-1}(D^n) = 0. \qedhere \]
    \end{enumerate}
\end{proof}

\begin{theorem}{Stronger Exactness}{stronger_exactness}
    If $B\subset A \subset X$ and we let
    \[ \partial_*: H_i(X,A) \rightarrow H_{i-1}(A,B) \]
    be the composition of $\partial_*:H_i(X,A) \rightarrow H_{i-1}(A)$ with the map $H_{i-1}(A) \rightarrow H_{i-1}(A,B)$ induced by inclusion, then we have the following exact sequence, where $\imath_*$ and $\jmath_*$ are induced by inclusions:
    \begin{align*}
        \cdots &\xlongrightarrow{\partial_*} H_p(A,B) \xlongrightarrow{\imath_*} H_p(X,B) \xlongrightarrow{\jmath_*} H_p(X,A) \\
        &\xlongrightarrow{\partial_*} H_{p-1}(A,B) \xlongrightarrow{\imath_*}\cdots.
    \end{align*}
\end{theorem}

\subsection{CW-Complexes}

$f_{\partial \sigma}: S^{n-1} \rightarrow K^{(n-1)}$ is called the attaching map while $f_\sigma: D^n \rightarrow K$ is called the canonical map.
A closed cell is the image $K_\sigma = f_\sigma(D)$ while an open cell is the image $U_\sigma = f_\sigma(\operatorname{int}(D))$.

\begin{example}{CW-Complexes}{cw_complexes}
    \begin{itemize}
        \item $S^n = e^0 \cup e^n$.
        \item $S^n = (e^0 \oplus e^0) \cup \cdots \cup (e^n \oplus e^n)$.
        \item $T^2 = e^0 \cup (e^1 \oplus e^1) \cup e^2$.
        \item $\mathbb{C}P^n = e^0 \cup e^2 \cup \cdots \cup e^{2n}$.
    \end{itemize}
\end{example}

\begin{theorem}{Covering of CW-Complex}{covering_of_cw_complex}
    The coverint of a CW-complex is a CW-complex.
\end{theorem}

\subsection{Cellular Homology}

This subsection is on the computation of homology of CW-complexes.
\begin{itemize}
    \item Let $K$ be a CW-complex and $A$ be a subcomplex.
    \item Let $K^{(n)}$ denote the union of the $n$-skeleton of $K$ with the entire space $A$.
    \item $\coprod_\sigma I^n$ be the disjoint union of $n$-disks over the $n$-cells of $K$ not in $A$.
    \item $p_\sigma$ denotes the collapse $K^{(n )} \rightarrow S^n$.
\end{itemize}
From the axiom of additivity we find
\[ H_n\qty(\coprod_\sigma (I^n,\partial I^n)) = \bigoplus_\sigma H_n(I^n,\partial I^n), \]
which is a free abelian group on the $n$-cells of $K$ not in $A$.

\begin{lemma}{Relative Cellular Homology as Free Abelian}{relative_cellular_homology_as_free_abelian}
    The map
    \[ \bigoplus_\sigma f_{\sigma *}: \bigoplus_\sigma H_n(I^n,\partial I^n) \rightarrow H_n(K^{(n)}, K^{(n-1)}) \]
    is an isomorphism.
    Moreover, $H_i(K^{(n)}, K^{(n-1)}) = 0$ for $i\neq n$.
\end{lemma}

Note also that the following diagram commutes.
\begin{center}
    \begin{tikzcd}
        H_n(K^{(n)}, K^{(n-1)}) \arrow[r,"\partial_*"] & \tilde{H}_{n-1}(K^{(n-1)},A) \\
        \bigoplus_\sigma H_n(I^n,\partial I^n ) \arrow[u,"\bigoplus_\sigma f_{\sigma *}"'] \arrow[r,"\bigoplus \partial_*"] & \bigoplus_\sigma \tilde{H}_{n-1}(\partial I^n) \arrow[u,"\bigoplus_\sigma f_{\partial \sigma *}"']
    \end{tikzcd}
\end{center}

$H_n(K,A)$ can be computed from the chain complex whose
\begin{itemize}
    \item $n$th term is $H_n(K^{(n )}, K^{(n-1)})$ and
    \item whose boundary operator is $\beta = j \circ \partial$.
\end{itemize}

We have the following isomorphisms between
\begin{itemize}
    \item $C_n(K,A)$, the free abelian group based on the $n$-cells of $K$ not in $A$, and
    \item $H_n(K^{(n )}, K^{(n-1)})$.
\end{itemize}
The map
\[ \Psi: C_n(K,A) \rightarrow H_n(K^{(n )}, K^{(n-1)}) \]
is defined by
\[ \Psi\qty(\sum_\sigma n_\sigma \sigma) = \sum_\sigma n_\sigma f_{\sigma *}[I^n] \]
and the inverse
\[ \Phi: H_n(K^{(n )}, K^{(n-1)}) \rightarrow C_n(K,A) \]
is defined by
\[ \Phi(\alpha) = \sum_\sigma \phi_n(p_{\sigma *}(\alpha)) \sigma, \]
where $\phi_n: H_n(S^n,*) \rightarrow \mathbb{Z}$ is the unique homomorphism such that $\phi_n[S^n] = 1$.

\begin{theorem}{Computation of Cellular Homology}{computation_of_cellular_homology}
    $H_*(K,A)$ is isomorphic to $H_*(C_*(K,A))$, i.e. the homology of the chain complex $C_*(K,A)$, where the boundary operator
    \[ \partial: C_{n+1}(K,A) \rightarrow C_n(K,A) \]
    is given by
    \[ \partial \sigma = \sum_\tau [\tau:\sigma] \tau, \]
    where
    \[ [\tau:\sigma] = \operatorname{deg}(p_\tau f_{\partial \sigma}). \]
\end{theorem}

\begin{example}{Cellular Homology of Projective Space}{cellular_homology_of_projective_space}
    The complex $C_*$ is given by
    \[ 0 \xlongrightarrow{} \mathbb{Z} \xlongrightarrow{\times 2} \mathbb{Z} \xlongrightarrow{0} \mathbb{Z} \longrightarrow 0. \]
    Therefore, the homology is given by
    \[ H_2(\mathbb{R}P^2) = 0,\quad H_1(\mathbb{R}P^2) = \mathbb{Z}_2,\quad H_0(\mathbb{R}P^2) = \mathbb{Z}. \]
\end{example}

\begin{example}{Cellular Homology of Torus}{cellular_homology_of_torus}
    The complex $C_*$ is given by
    \[ 0 \xlongrightarrow{} \mathbb{Z} \xlongrightarrow{0} \mathbb{Z} \oplus \mathbb{Z} \xlongrightarrow{0} \mathbb{Z} \longrightarrow 0. \]
    Therefore,
    \[ H_2(T^2) = \mathbb{Z},\quad H_1(T^2) = \mathbb{Z} \oplus \mathbb{Z},\quad H_0(T^2) = \mathbb{Z}. \]
\end{example}

\begin{example}{Cellular Homology of Klein Bottle}{cellular_homology_of_klein_bottle}
    The complex $C_*$ is given by
    \[ 0 \xlongrightarrow{} \mathbb{Z} \xlongrightarrow{0 \oplus \times 2} \mathbb{Z} \oplus \mathbb{Z} \xlongrightarrow{0} \mathbb{Z} \longrightarrow 0. \]
    Therefore,
    \[ H_2(T^2) = 0,\quad H_1(T^2) = \mathbb{Z} \oplus \mathbb{Z}/2,\quad H_0(T^2) = \mathbb{Z}. \]
\end{example}

\subsection{Cellular Maps}

\begin{definition}{Cellular Map}{cellular_map}
    A map $f:K\rightarrow L$ between CW-complexes is said to be cellular if $f(K^{(n)}) \subset L^{(n)}$ for all $n$.
\end{definition}

\begin{theorem}{Cellular Approximation Theorem}{cellular_approximation_theorem}
    Input:
    \begin{itemize}
        \item $K$ and $Y$ are CW-complexes.
        \item $L\subset K$ is a subcomplex.
        \item $\phi:K\rightarrow Y$ is a map such that $\phi\vert_L$ is cellular.
    \end{itemize}
    Then $\phi$ is homotopic to a cellular map $\psi: K\rightarrow Y$ rel $L$.
\end{theorem}

\begin{theorem}{Cellular Chain Map}{cellular_chain_map}
    Input:
    \begin{itemize}
        \item $g:K\rightarrow L$ is cellular.
        \item $A\subset K$ is a subcomplex of $K$.
        \item $B\subset L$ is a subcomplex of $L$.
        \item $g(A)\subset B$.
    \end{itemize}
    Then the induced chain map
    \[ g_{\Delta}: C_*(K,A) \rightarrow C_*(L,B) \]
    is given by
    \[ g_\Delta(\sigma) = \sum_\tau \operatorname{deg}(g_{\tau,\sigma})\tau. \]
\end{theorem}

\subsection{Product of CW-Complexes}

\begin{theorem}{Product of CW-Complexes}{product_of_cw_complexes}
    The boundary operator satisfies
    \[ \partial(\sigma \times \mu) = \partial \sigma \times \mu + (-1)^{\operatorname{deg}\sigma}\sigma \times \partial \mu. \]
\end{theorem}

\begin{example}{Homology of $\mathbb{R}P^n$}{homology_of_r_p_n}
    Let the complex of $S^n$ be given by two cells in each dimension.
    Let the cells in each dimension be denoted $\sigma_k$ and $T\sigma_{k}$, where $T$ denotes the antipodal map.
    Then
    \[ \partial \sigma_k = (1+(-1)^k T)\sigma_{k-1}. \]
    The map $\pi:S^n \rightarrow \mathbb{R}P^n$ induces the chain map
    \[ \pi_\Delta:C_k(S^n) \rightarrow \mathbb{R}P^n \]
    taking both $\sigma_k$ and $T\sigma_k$ to $\tau_k$.
    Therefore,
    \[ \partial \tau_k = \pi_\Delta (1+(-1)^k T) \sigma_{k-1} = (1+(-1)^k) \tau_{k-1}. \] 
\end{example}

\subsection{Cross Product}

\begin{theorem}{Cross Product}{cross_product}
    There exist bilinear maps
    \[ \times: \Delta_p(X) \times \Delta_q(Y) \rightarrow \Delta_{p+q}(X\times Y) \]
    such that
    \begin{itemize}
        \item for $x\in X$, $y\in Y$, $\sigma:\Delta_q\rightarrow Y$ and $\tau:\Delta_p\rightarrow X$, $x\times \sigma$ and $\tau\times y$ are given in the obvious way;
        \item if $f:X\rightarrow X'$ and $g:Y\rightarrow Y'$ and if $f\times g:X\times Y \rightarrow X'\times Y'$ denotes the product map then
        \[ (f\times g)_\Delta(a\times b) = f_\Delta(a)\times g_\Delta(b); \]
        and
        \item $\partial(a\times b) = (\partial a)\times b + (-1)^{\operatorname{deg}a} a\times (\partial b)$.
    \end{itemize}
\end{theorem}

\begin{definition}{Product of Pair}{product_of_pair}
    $(X,A) \times (Y,B)$ denotes
    \[ (X\times Y, X\times B \cup A\times Y). \]
\end{definition}
 
\subsection{Mayer-Vietoris Sequence}

\begin{theorem}{Mayer-Vietoris Sequence}{mayer_vietoris_sequence}
    Input:
    \begin{itemize}
        \item $A,B\in X$ and $X = \operatorname{int}(A) \cup \operatorname{int}(B)$.
        \item $\imath^A: A\cap B \hookrightarrow A$ and $\imath^B: A\cap B \hookrightarrow B$.
        \item $\jmath^A: A \hookrightarrow A\cup B$ and $\jmath^B: B \hookrightarrow A\cup B$.
    \end{itemize}
    Then we have the long exact sequence
    \begin{align*}
        \cdots &\rightarrow H_p(A\cap B) \xlongrightarrow{\imath^A_* \oplus \imath^B_*} H_p(A) \oplus H_p(B) \xlongrightarrow{\jmath^A_* - \jmath^B_*} H_p(A\cup B) \\
        & \rightarrow H_{p-1}(A\cap B) \rightarrow \cdots.
    \end{align*}
\end{theorem}

\begin{example}{Klein Bottle with Mayer-Vietoris}{klein_bottle_with_mayer_vietoris}
    A Klein bottle $K = A \cup B$ where $A$ and $B$ are M\"obius strips and $A\cap B$ is also a M\"obius strip.
    Therefore
    \begin{align*}
        \cdots &\rightarrow H_2(A\cap B) \xlongrightarrow{\imath^A_* \oplus \imath^B_*} H_2(A) \oplus H_2(B) (=0) \xlongrightarrow{\jmath^A_* - \jmath^B_*} H_2(K) \\
        & \rightarrow H_1(A\cap B) (=\mathbb{Z}) \xlongrightarrow{2\oplus 2} H_1(A) \oplus H_1(B) (=\mathbb{Z}\oplus \mathbb{Z}) \xlongrightarrow{1 - 1} H_1(K) \\
        & \xlongrightarrow{0} H_0(A\cap B) \xlongrightarrow{\imath^A_* \oplus \imath^B_*} H_0(A) \oplus H_0(B) \xlongrightarrow{\jmath^A_* - \jmath^B_*} H_0(K) \rightarrow 0.
    \end{align*}
    Therefore, $H_2(K) = 0$ and $H_1(K)= (\mathbb{Z}\oplus \mathbb{Z})/((2,2)) = \mathbb{Z}\oplus \mathbb{Z}_2$.
\end{example}

\begin{example}{Projective Plane with Mayer-Vietoris}{projective_plane_with_mayor_vietoris}
    Since $\mathbb{R}P^n = \mathbb{R}P^{n-1} \cup_f D^n$ we have (let $A = D^n = \mathbb{R}P^n - \mathbb{R}P^{n-1}$ while $B$ is the mapping cylinder over $\mathbb{R}P^{n-1}$, i.e. the complement of $[1:0:\cdots:0]$)
    \begin{align*}
        0 &\rightarrow H_p(S^{n-1}) \xlongrightarrow{\imath^A_* \oplus \imath^B_*} H_p(\mathbb{R}P^{n-1}) \oplus H_p(D^n) \xlongrightarrow{\jmath^A_* - \jmath^B_*} H_p(\mathbb{R}P^n) \\
        & \rightarrow H_{p-1}(S^{n-1}) \rightarrow \cdots.
    \end{align*}
    For $p < n-1$, $H_p(\mathbb{R}P^n) \cong H_p(\mathbb{R}P^{n-1})$.
    For $p=n-1$, we have
    \[ 0 \rightarrow H_{n-1}(S^{n-1}) (=\mathbb{Z}) \xrightarrow{\imath} H_{n-1}(\mathbb{R}P^{n-1}) \rightarrow H_{n-1}(\mathbb{R}P^n) \rightarrow 0. \]
    Therefore, if $n$ is even, we have
    \[ H_{n-1}(\mathbb{R}P^n) = H_{n-1}(\mathbb{R}P^{n-1})/2\mathbb{Z} \]
    while if $n$ is odd, we have
    \[ H_{n-1}(\mathbb{R}P^n) \cong H_{n-1}(\mathbb{R}P^{n-1}) \cong 0. \]
    For $p=n$ we have
    \begin{align*}
        0 &\rightarrow H_n(\mathbb{R}P^n) \rightarrow H_{n-1}(S^{n-1})(=\mathbb{Z}) \rightarrow H_{n-1}(\mathbb{R}P^{n-1}) \\ &\rightarrow H_{n-1}(\mathbb{R}P^{n}) \rightarrow 0.
    \end{align*}
    If $n$ is even then
    \[ 0 \rightarrow H_n(\mathbb{R}P^n) \rightarrow \mathbb{Z} \xlongrightarrow{2} \mathbb{Z} \rightarrow \mathbb{Z}_2 \rightarrow 0 \]
    and therefore $H_n(\mathbb{R}P^n) \cong 0$ while if $n$ is odd then
    \[ 0 \rightarrow H_n(\mathbb{R}P^n) \rightarrow \mathbb{Z} \xlongrightarrow{0} \cdots \]
    and therefore $H_n(\mathbb{R}P^n) \cong \mathbb{Z}$.
\end{example}

\section{Cohomology}

\subsection{Some Homological Algebra}

\begin{proposition}{Right Exactness of $\otimes$}{right_exactness_of_otimes}
    A short exact sequence
    \[ 0 \longrightarrow A' \longrightarrow A \longrightarrow A'' \longrightarrow 0 \]
    induces the exact sequence
    \[ A'\otimes B \longrightarrow A\otimes B \longrightarrow A''\otimes B \longrightarrow 0. \]
\end{proposition}

\begin{proposition}{Left Exactness of $\operatorname{Hom}$}{left_exactness_of_hom}
    A short exact sequence
    \[ 0 \longrightarrow A' \longrightarrow A \longrightarrow A'' \longrightarrow 0 \]
    induces the exact sequence
    \[ 0 \longrightarrow \operatorname{Hom}(A'',B) \longrightarrow \operatorname{Hom}(A,B) \longrightarrow \operatorname{Hom}(A',B). \]
\end{proposition}

Let $(C_*,\partial)$ be a chain complex.
Then $C_*\otimes G$ is a chain complex with the differential $\partial\otimes 1$ while $\operatorname{Hom}(C_*,G)$ is a chain complex with the differential $\delta = \operatorname{Hom}(\partial,1)$.
\par
Then
\[ \phi\otimes 1: A_* \otimes G \rightarrow B_*\otimes G \]
is a chain map and
\[ \phi'=\operatorname{Hom}(\phi,1):\operatorname{Hom}(B_*,G) \rightarrow \operatorname{Hom}(A_*,G) \]
is also a chain map.
\par
Thus there is a induced homomorphism
\[ \phi^*: H^*(\operatorname{Hom}(B_*,G)) \rightarrow H^*(\operatorname{Hom}(A_*,G)) \]
and
\[ \phi_*: H_*(A_*\otimes G) \rightarrow H_*(B_*\otimes G). \]
\par
Let $0_* \longrightarrow A_* \longrightarrow B_* \longrightarrow C_* \longrightarrow 0$ be a split short exact sequence where the arrows are chain maps (the retraction or section map is not necessarily a chain map).
Then we have the induced short exact sequences
\[ 0 \longrightarrow \operatorname{Hom}(C_*,G) \longrightarrow \operatorname{Hom}(B_*,G) \longrightarrow \operatorname{Hom}(A_*,G) \longrightarrow 0 \]
and
\[ 0 \longrightarrow A_*\otimes G \longrightarrow B_*\otimes G \longrightarrow C_*\otimes G \longrightarrow 0. \]
\par
These short exact sequences turn out to induce
\begin{align*}
    \cdots &\longrightarrow H^i(\operatorname{Hom}(C_*,G)) \longrightarrow H^i(\operatorname{Hom}(B_*,G)) \longrightarrow H^i(\operatorname{Hom}(A_*,G)) \\
    &\longrightarrow H^{i+1}(\operatorname{Hom}(C_*,G)) \longrightarrow \cdots
\end{align*}
and
\begin{align*}
    \cdots &\longrightarrow H^i(A_*\otimes G) \longrightarrow H^i(B_*\otimes G) \longrightarrow H^i(C_*\otimes G) \\
    &\longrightarrow H_{i-1}(A_*\otimes G) \longrightarrow \cdots.
\end{align*}

We can regard $\Delta^p(X,A;G)$ as the set of functions on a singular $p$-simplices of $X$ to $G$ which vanish on simplices totally in $A$.

\begin{theorem}{Collection of Long Exact Sequences}{collection_of_long_exact_sequences}
    Given the exact sequence $0 \rightarrow G' \rightarrow G \rightarrow G'' \rightarrow 0$ we have
    \begin{align*}
        \cdots &\longrightarrow H_i(X,A;G) \longrightarrow H_i(X;G) \longrightarrow H_i(A;G) \\
        &\longrightarrow H_{i-1}(X,A;G) \longrightarrow \cdots,\\
        \cdots &\longrightarrow H_i(X,A;G') \longrightarrow H_i(X,A;G) \longrightarrow H_i(X,A;G'') \\
        &\longrightarrow H_{i-1}(X,A;G') \longrightarrow \cdots, \\
        \cdots &\longrightarrow H^i(X,A;G) \longrightarrow H^i(X;G) \longrightarrow H^i(A;G) \\
        &\longrightarrow H^{i+1}(X,A;G) \longrightarrow \cdots,\\
        \cdots &\longrightarrow H^i(X,A;G') \longrightarrow H^i(X,A;G) \longrightarrow H^i(X,A;G'') \\
        &\longrightarrow H^{i+1}(X,A;G') \longrightarrow \cdots.
    \end{align*}
\end{theorem}

\begin{proposition}{Injective iff Divisible}{injective_iff_divisible}
    An abelian group is injective if and only if it is divisible.
\end{proposition}

\begin{proposition}{Embedding into Injective Group}{embedding_into_injective_group}
    Any abelian group $G$ is a subgroup of an injective group.
\end{proposition}

Let
\[ 0\rightarrow G \rightarrow I \rightarrow J \rightarrow 0 \]
be a injective resolution of $G$.
Then we have the exact sequence
\[ 0 \longrightarrow \operatorname{Hom}(A,G) \longrightarrow \operatorname{Hom}(A,I) \longrightarrow \operatorname{Hom}(A,J). \]
The $\operatorname{Ext}$ functor is defined such that the following sequence is exact.
\[ 0 \longrightarrow \operatorname{Hom}(A,G) \longrightarrow \operatorname{Hom}(A,I) \longrightarrow \operatorname{Hom}(A,J) \longrightarrow \operatorname{Ext}(A,G) \longrightarrow 0. \]
There is a canonical isomorphism between $\operatorname{Ext}(A,G)$ obtained from two different resolutions of of $G$.
Moreover, $\operatorname{Ext}(A,G)$ is covariant in $G$ and contravariant in $A$.
\par
Let
\[ 0\rightarrow R\rightarrow F\rightarrow A \rightarrow 0 \]
be a projective resolution of $A$.
Then the $\operatorname{Tor}$ functor is defined such that the following sequence is exact.
\[ 0\longrightarrow \operatorname{Tor}(A,B) \longrightarrow R\otimes B \longrightarrow F\otimes B \longrightarrow A\otimes B \longrightarrow 0. \]

\begin{theorem}{Evaluation of $\operatorname{Ext}$ and $\operatorname{Tor}$}{evaluation_of_ext_and_tor}
    Let $G$ and $A$ be abelian groups, and $d=\operatorname{gcd}(n,m)$.
    \begin{itemize}
        \item $\operatorname{Tor}(A,\mathbb{Z}) = 0$, $\operatorname{Ext}(\mathbb{Z},A) = 0$.
        \item $\operatorname{Ext}(\mathbb{Z}_n,G) = G/nG$.
        \item $\operatorname{Ext}(\mathbb{Z}_n,\mathbb{Z}_m) = \mathbb{Z}_d = \operatorname{Hom}(\mathbb{Z}_n,\mathbb{Z}_m)$.
        \item $\operatorname{Tor}(\mathbb{Z}_n,\mathbb{Z}_m) = \mathbb{Z}_d = \operatorname{Tor}(\mathbb{Z}_n,\mathbb{Z}_m)$.
        \item $A$ is torsion free $\Leftrightarrow$ $\operatorname{Tor}(A,B) = 0$ for all $B$.
        \item If $A$ is projective then $\operatorname{Tor}(A,B) = 0$ for all $B$.
        \item $A$ is projective $\Leftrightarrow$ $\operatorname{Ext}(A,G) = 0$ for all $G$.
        \item $G$ is injective $\Leftrightarrow$ $\operatorname{Ext}(A,G) = 0$ for all $A$.
    \end{itemize}
\end{theorem}

\begin{theorem}{Induced Exact Sequence}{induced_exact_sequence}
    Let
    \[ 0 \rightarrow A' \rightarrow A \rightarrow A'' \rightarrow 0 \]
    be exact.
    Then we have the long exact sequences
    \begin{align*}
        0 &\rightarrow \operatorname{Hom}(A',G) \rightarrow \operatorname{Hom}(A,G) \rightarrow \operatorname{Hom}(A'',G) \\
        &\rightarrow \operatorname{Ext}(A',G) \rightarrow \operatorname{Ext}(A,G) \rightarrow \operatorname{Ext}(A'',G) \rightarrow 0.
    \end{align*}
    and
    \begin{align*}
        0 &\rightarrow \operatorname{Tor}(A',G) \rightarrow \operatorname{Tor}(A,G) \rightarrow \operatorname{Tor}(A'',G) \\
        &\rightarrow A'\otimes G \rightarrow A\otimes G \rightarrow A''\otimes G\rightarrow 0.
    \end{align*}
    Moreover, if
    \[ 0\rightarrow G' \rightarrow G \rightarrow G'' \rightarrow 0 \]
    is a short exact sequence then we have the long exact sequence
    \begin{align*}
        0 &\rightarrow \operatorname{Hom}(A,G') \rightarrow \operatorname{Hom}(A,G) \rightarrow \operatorname{Hom}(A,G'') \\
        &\rightarrow \operatorname{Ext}(A,G') \rightarrow \operatorname{Ext}(A,G) \rightarrow \operatorname{Ext}(A,G'')\rightarrow 0.
    \end{align*}
\end{theorem}

\begin{example}{Measure the Torsion}{measure_the_torsion}
    $\operatorname{Tor}(G,\mathbb{Q}/\mathbb{Z})$ is the torsion subgroup of $G$.
    Therefore, from the universal coefficient theorem for homology we have
    \[ 0\rightarrow H_n(X,A) \otimes \mathbb{Q}/\mathbb{Z} \rightarrow H_n(X,A;\mathbb{Q}/\mathbb{Z}) \rightarrow TH_{n-1}(X,A) \rightarrow 0, \]
    where $TH_{n-1}$ denotes the torsion subgroup of $H_{n-1}$.
\end{example}

\begin{theorem}{Universal Coefficient Theorem}{universal_coefficient_theorem}
    \paragraph*{For Cohomology}
    We have the exact sequence
    \[ 0 \rightarrow \operatorname{Ext}(H_{n-1}(C_*),G) \rightarrow H^n(\operatorname{Hom}(C_*,G)) \xrightarrow{\beta} \operatorname{Hom}(H_n(C_*),G) \rightarrow 0 \]
    that is natural in $C_*$ and $G$, and splits.
    \paragraph*{For Homology}
    We have the exact sequence
    \[ 0 \rightarrow H_n(C_*) \otimes G \xrightarrow{\alpha} H_n(C_*\otimes G) \rightarrow \operatorname{Tor}(H_{n-1}(C_*), G) \rightarrow 0 \]
    that is natural in $C_*$ and $G$, and splits.
\end{theorem}

\begin{corollary}{Cohomology Decomposition}{cohomology_decomposition}
    If $H_{n-1}(X,A)$ and $H_n(X,A)$ are fintiely generated, then $H^n(X,A;\mathbb{Z})$ is also finitely generated, and
    \[ H^n(X,A;\mathbb{Z}) = F_n \oplus T_{n-1}, \]
    where $F_i$ and $T_i$ are the free part and torsion part, respectively, of $H_i(X,A)$.
\end{corollary}

\subsection{Excision and Homotopy for Cohomology}

\begin{theorem}{Excision}{excision}
    Input:
    \begin{itemize}
        \item $B\subset A\subset X$.
        \item $\overline{B}\subset \operatorname{int}(A)$.
    \end{itemize}
    Then the inclusion map $(X-B,A-B) \hookrightarrow (X,A)$ induces isomorphisms
    \begin{align*}
        H^p(X,A;G) &= H^p(X-B,A-B; G),\\
        H_p(X,A;G) &= H_p(X-B,A-B,G)
    \end{align*}
    for all $p$ and $G$.
\end{theorem}

\begin{theorem}{Mayer-Vietoris}{mayer_vietoris_sequence}
    If $X=\operatorname{int}(A) \cup \operatorname{int}(B)$ then there is the long exact sequence for any coefficient group $G$
    \begin{align*}
        \cdots &\longrightarrow H^p(A\cup B) \xlongrightarrow{\jmath^*_A - \jmath^*_B} H^p(A) \oplus H^p(B) \xlongrightarrow{\imath^*_A + \imath^*_B} H^p(A\cap B) \\
        &\longrightarrow H^{p+1}(A\cup B) \longrightarrow \cdots.
    \end{align*}
\end{theorem}

\begin{theorem}{Cohomology of Coproduct}{cohomology_of_coproduct}
    If $X=\coprod_\alpha X_\alpha$ then the inclusions and projections induce an isomorphism
    \[ H^p(X;G) = \prod_\alpha H^p(X_\alpha;G). \]
\end{theorem}

\begin{theorem}{Cohomology is Homotopy Invariant}{cohomology_is_homotopy_invariant}
    If $f_0\cong f_1:(X,A)\rightarrow (Y,B)$ then, for any coefficient group $G$,
    \begin{align*}
        f^*_0 \cong f^*_1: H^p(Y,B;G) \rightarrow H^p(X,A;G);\\
        f_{0*}\cong f_{1*}: H_p(X,A;G) \rightarrow H_p(Y,B;G).
    \end{align*}
\end{theorem}

% \bibliographystyle{plain}
% \bibliography{main}

\end{document}
