\documentclass{article}

\usepackage{pandekten}

\title{Allgemein}
\author{Ch\=an Taku}

\begin{document}

\maketitle

\section{Style}

This section deals with questions related to style.

\paragraph{Citation}
Citations should not hold grammatical positions in sentences.
See \href{https://academia.stackexchange.com/questions/49487/can-cited-works-hold-grammatical-positions-in-sentences}{Can cited works hold grammatical positions in sentences?}

\section{Tools}

\subsection{Find Citation}

To look for a BibTeX file of a given book, visit \href{https://lead.to/amazon/com/?key=978-0199566402&si=al&bn=&la=en&cu=usd&op=bt&so=re#first}{Lead2Amazon}.
More tools may be found \href{https://tex.stackexchange.com/questions/58200/what-is-the-best-way-to-get-bibtex-entries-from-isbn-number}{here}.

\section{Admonitions}

\subsection{Covariant Properties}

Although Pandekten pratices the DRY principle,
sometimes a definition has to be stated twice over different categories.
For example, the pullback bundle of a fiber bundle is a fiber bundle.
Applying the same definition we have no guarantee that the pullback bundle of a smooth fiber bundle is again a smooth fiber bundle.
We have to restate the same definition for smooth fiber bundles, and replace every occurrence homeomorphism with diffeomorphism.
\par
This problem is mitigated by using the language of category theory.
However, to make the text more accessible, we have to sometimes surrender to the plain language.

\end{document}