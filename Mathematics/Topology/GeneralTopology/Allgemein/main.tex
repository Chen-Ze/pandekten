\documentclass{article}

\usepackage{pandekten}

\title{Allgemein}
\author{Ch\=an Taku}

\begin{document}

\maketitle

\section{Topological Space}

\subsection{Definitions and Properties}

\begin{definition}{Topological Space, Topology}{topological_space}
    Data:
    \begin{enumerate}
        \item $X$ is a set.
        \item $\mathcal{T}$ is a subset of the power set of $X$.
    \end{enumerate}
    Then $(X,\mathcal{T})$ is called a topological space if $\mathcal{T}$ is a topology for $X$, i.e. if all of the following hold.
    \begin{enumerate}
        \item $\varnothing \in \mathcal{T}$ and $X\in\mathcal{T}$.
        \item If $\mathcal{S}\subset \mathcal{T}$ (finite or infinite), then
        \[ \bigcup_{S\in\mathcal{S}} S \in \mathcal{T}. \]
        \item If $\mathcal{S}\subset \mathcal{T}$ is finite, then
        \[ \bigcap_{S\in\mathcal{S}} S \in \mathcal{T}. \]
    \end{enumerate}
\end{definition}

\begin{definition}{Continuous Map}{continous_map}
    Let $f:X\rightarrow Y$ be a map between two topological spaces.
    $f$ is called a continuous map if $f^{-1}(V)$ is open in $X$ whenever $V$ is open in $Y$.
\end{definition}

\begin{definition}{Category of Topological Spaces \badge{\Cat}, Homeomorphism, Homeomorphic Topological Spaces}{category_of_topological_spaces}
    The category $\mathbf{Top}$ of topological spaces is the category with
    \begin{itemize}
        \item all topological spaces as object, and
        \item continuous maps between them as morphisms.
    \end{itemize}
    Isomorphisms in $\mathbf{Top}$ are called homeomorphisms.
    Isomorphic objects in $\mathbf{Top}$ are called homeomorphic topological spaces.
\end{definition}

\begin{definition}{Closure}{closure}
    Input:
    \begin{enumerate}
        \item $(X,\mathcal{T})$ is a topological space.
        \item $A\subset X$.
    \end{enumerate}
    Then the closure of $A$ is defined by
    \[ \overline{A} = \bigcap_{S\in\mathcal{A}} S, \]
    where
    \[ \mathcal{A} = \Set*{S}{A\subset S\in \mathcal{T}}. \]
\end{definition}

\begin{theorem}{Equivalent Definitions of Continuous Map}{equivalent_definitions_of_continuous_map}
    Let $X$ and $Y$ be a topological spaces, and $f:X\rightarrow Y$ be a map between them.
    The following conditions are all equivalent to $f$ being continuous.
    \begin{enumerate}
        \item $f^{-1}(V)$ is open in $X$ whenever $V$ is open in $Y$.
        \item $f(\overline{A}) \subset \overline{f(A)}$ for any $A\subset X$.
        \item $f^{-1}(C)$ is closed in $X$ whenever $C$ is closed in $Y$.
        \item For each $x\in X$ and each neighborhood $V\supset f(x)$, there is a neighborhood $U\ni x$ such that $f(U) \in V$.
    \end{enumerate}
\end{theorem}

\begin{definition}{Relative Topology}{relative_topology}
    Input:
    \begin{enumerate}
        \item $(X,\mathcal{T})$ is a topological space.
        \item $X'\subset X$.
    \end{enumerate}
    The relative topology $\mathcal{T}'$ for $X'$ is given by
    \[ \mathcal{T}' = \Set*{X'\cap S}{S\in\mathcal{T}}. \]
\end{definition}

\begin{lemma}{The Gluing Lemma}{the_gluing_lemma}
    Input:
    \begin{enumerate}
        \item $X$ and $Y$ are topological spaces.
        \item $X = A_1 \cup A_2$, where $A_1$ and $A_2$ are either both open or both closed.
        \item $f_1:A_1 \rightarrow Y$ and $f_2:A_2\rightarrow Y$ are continuous.
        \item $\eval{f_1}_{A_1\cap A_2} = \eval{f_2}_{A_1\cap A_2}$.
    \end{enumerate}
    Then the map defined by
    \[ f(x) = \begin{cases}
        f_1(x), & \text{if } x\in A_1, \\
        f_2(x), & \text{if } x\in A_2
    \end{cases} \]
    is continous.
\end{lemma}

\begin{definition}{Basis, Countable Basis}{basis}
    Let $(X,\mathcal{T})$ be a topological space.
    $\mathcal{B} \subset \mathcal{T}$ is called a basis if every $T\in \mathcal{T}$ could be written as $\cup_{i\in \alpha} B_i$ where $\Set*{B_i}{i\in\alpha}\subset \mathcal{B}$.
    $\mathcal{B}$ is called a countable basis if it is a basis and it is countable.
\end{definition}

\begin{definition}{Open Map, Closed Map}{open_map_closed_map}
    Let $X$ and $Y$ be topological spaces.
    A map $f:X\rightarrow Y$ is called a open (closed) map if $U\subset X$ is open (closed) implies that $f(U)\subset Y$ is open (closed).
\end{definition}

\subsubsection{Separation Axiom}

\begin{definition}{Fr\'echet Space, $\mathrm{T}_1$ Space}{frechet_space}
    A topological space $(X,\mathcal{T})$ is called a Fr\'echet space if for all $x,y\in X$ such that $x\neq y$, there exists $U,V$ such that $x\in U \in \mathcal{T}$, $y\in V\in \mathcal{T}$, $y\notin U$, and $x\notin V$.
\end{definition}

\begin{definition}{Hausdorff Space, $\mathrm{T}_2$ Space}{hausdorff_space}
    A topological space $(X,\mathcal{T})$ is called a Hausdorff space if for all $x,y\in X$ such that $x\neq y$, there exists $U,V$ such that $x\in U \in \mathcal{T}$, $y\in V\in \mathcal{T}$, and $U\cap V = \varnothing$.
\end{definition}

\begin{definition}{Regular Space}{regular_space}
    A topological space $(X,\mathcal{T})$ is called a regular space if for each $x\in X$ and closed set $B\subset X$ such that $x\notin B$, there exists $U,V$ such that $x\in U \in \mathcal{T}$, $B\subset V\in \mathcal{T}$, and $U\cap V = \varnothing$.
\end{definition}

\begin{definition}{Normal Space}{normal_space}
    A topological space $(X,\mathcal{T})$ is called a normal space if for all closed sets $A,B\subset X$ such that $A\cap B = \varnothing$, there exists $U,V$ such that $A\subset U \in \mathcal{T}$, $B\subset V\in \mathcal{T}$, and $U\cap V = \varnothing$.
\end{definition}

\subsubsection{Countability Axiom}

\begin{definition}{Second-Countable Space}{second_countable_space}
    Let $(X,\mathcal{T})$ be a topological space.
    If there exists a countable basis $\mathcal{B}$ of $(X,\mathcal{T})$ then $(X,\mathcal{T})$ is called a second-countable space.
\end{definition}

\begin{definition}{Dense Set}{dense_set}
    Let $X$ be a topological space.
    A subset $A\subset X$ is called a densed subset if $\overline{A} = X$.
\end{definition}

\begin{definition}{Separable Space}{separable_space}
    A topological space $X$ is separable if there is a countable dense subset $A\subset X$.
\end{definition}

\begin{theorem}{Second-Countable Space is Separable}{second_countable_space_is_separable}
    If $X$ is a second-countable space, then $X$ is separable.
\end{theorem}

\subsection{Product Topology}

\begin{definition}{Product Topology}{product_topology}
    Let $\Set*{(X_i,\mathcal{T}_i)}{i\in\alpha}$ be $n$ topological spaces.
    Then the product topology $\mathcal{T}$ for $X = \prod_{i\in\alpha} X_i$ is generated by the basis
    \[ \mathcal{B} = \Set*{\prod_{i\in\alpha} U_i}{U_i\in\mathcal{T}_i \text{ and } U_i = X_i \text{ except for finitely many } i}. \]
\end{definition}

\begin{theorem}{Relative Product Topology}{relative_product_topology}
    Let $X_i$ be subspace of $Y_i$ for each $i=1,\cdots,n$ with relative topology.
    Then the following two topologies coincides:
    \begin{itemize}
        \item the product topology $X_1 \times \cdots \times X_n$, and
        \item the relative topology of the set $X_1 \times \cdots \times X_n$ with respect to $Y_1 \times \cdots \times Y_n$.
    \end{itemize}
\end{theorem}

\begin{lemma}{Projection is Open Map}{projection_is_open_map}
    Let $X_1,\cdots,X_n$ be topological spaces, $X = X_1 \times \cdots \times X_n$, and $\operatorname{proj}_i: X \rightarrow X_i$ be the projection maps.
    Then for each $i$, $\operatorname{proj}_i$ is a open map.
\end{lemma}

\begin{counterexample}{Projection is not Closed Map}{projection_is_not_closed_map}
    The projection of the closed set $\Set*{(x,1/x)}{x>0}\subset \mathbb{R}^2$ to the first component is not closed.
\end{counterexample}

\begin{lemma}{Continuity of Components}{continuity_of_components}
    Let $X,Y_1,\cdots,Y_n$ be topological spaces, and $f:X\rightarrow Y_1 \times \cdots \times Y_n$.
    Then $f$ is continuous if and only if for each $i$, $\operatorname{proj}_i \circ f$ is continuous.
\end{lemma}

\subsection{Topological Manifold}

\begin{definition}{Chart}{chart}
    Input:
    \begin{enumerate}
        \item $(X,\mathcal{T})$ is a topological space.
    \end{enumerate}
    An $n$-dimensional chart on $X$ is a pair $(U,\varphi)$ where $U\in\mathcal{T}$ and
    \[ \varphi: U \rightarrow \mathbb{R}^n \]
    is a homeomorphism.
    An $n$-dimensional chart at $x\in X$ is an $n$-dimensional chart $(U,\varphi)$ where $U\supset x$.
\end{definition}

\begin{definition}{Locally Euclidean Space, Overlap Function}{locally_euclidean_space}
    Input:
    \begin{enumerate}
        \item $(X,\mathcal{T})$ is a topological space.
    \end{enumerate}
    $X$ is an $n$-dimensional locally Euclidean space if there is an $n$ such that for each $x\in X$ there is an $n$-dimensional chart at $x$.
    Let $(U_1,\varphi_1)$ and $(U_2,\varphi_2)$ be two charts of $X$.
    The overlap function for them is given by $\varphi_2^{-1}\circ \varphi_1$.
\end{definition}

\begin{definition}{Topological Manifold}{topological_manifold}
    Let $(X,\mathcal{T})$ be a topological space.
    $X$ is called a topological manifold if it satisfies all of the following.
    \begin{itemize}
        \item $X$ is Hausdorff.
        \item $X$ is locally Euclidean.
        \item $X$ is second countable.
    \end{itemize}
\end{definition}

\begin{example}{$S^n$}{s_n}
    The $n$-dimensional sphere $S^n$ is defined by
    \[ S^n = \Set*{(x^1,\cdots,x^{n+1})}{(x^1)^2 + \cdots + (x^{n+1})^2 = 1}. \]
    Let $S = (0,\cdots,0,-1) \in S^n$ and $N = (0,\cdots,0,1) \in S^n$.
    Define $U_{\mathrm{N}} = S^n \setminus \qty{S}$ and $U_{\mathrm{S}} = S^n \setminus \qty{\mathrm{N}}$.
    The charts $\varphi_{\mathrm{S}}: U_{\mathrm{S}} \rightarrow \mathbb{R}^n$ and $\varphi_{\mathrm{N}}: U_{\mathrm{N}}\rightarrow \mathbb{R}^n$ are given by
    \begin{align*}
        \varphi_{\mathrm{S}}(x^1,\cdots,x^{n+1}) &= \qty(\frac{x^1}{1-x^{n+1}},\cdots, \frac{x^n}{1-x^{n+1}}), \\
        \varphi_{\mathrm{N}}(x^1,\cdots,x^{n+1}) &= \qty(\frac{x^1}{1+x^{n+1}},\cdots, \frac{x^n}{1+x^{n+1}}).
    \end{align*}
    The inverses are given by
    \begin{align*}
        \varphi^{-1}_{\mathrm{S}}(y) &= \frac{1}{1+\norm{y}^2}\qty(2y^1,\cdots,2y^n,\norm{y}^2 - 1), \\
        \varphi^{-1}_{\mathrm{N}}(y) &= \frac{1}{1+\norm{y}^2}\qty(2y^1,\cdots,2y^n,1-\norm{y}^2).
    \end{align*}
    The overlap is given by
    \[ \varphi_{\mathrm{S}}\circ \varphi^{-1}_{\mathrm{N}}(y) = \frac{y}{\norm{y}^2}. \]
    In particular, for the case of $S^1$, $S^2$, and $S^3$, the codomain of $\varphi$ may be regarded as $\mathbb{R}$, $\mathbb{C}$, and $\mathbb{H}$, respectively, while the overlap is given by
    \[ y\rightarrow \overline{y}^{-1}. \]
\end{example}

\subsection{Quotient Space}

\begin{definition}{Quotient Space, Quotient Topology, Quotient Map, Fiber \badge{UMP}}{quotient_space}
    Input:
    \begin{enumerate}
        \item $(X,\mathcal{T})$ is a topological space.
        \item $Y$ is a set.
        \item $\mathcal{Q}:X\rightarrow Y$ is a surjective map.
    \end{enumerate}
    The quotient topology on $Y$ is defined by
    \[ \mathcal{S} = \Set*{S\subset Y}{\mathcal{Q}^{-1}(S) \in \mathcal{T}}. \]
    The topological space $(Y,\mathcal{S})$ is called the quotient space.
    $\mathcal{Q}$ is called the quotient map. The fiber of $\mathcal{Q}$ over $y\in Y$ is given by $\mathcal{Q}^{-1}(y)$.
\end{definition}

\begin{lemma}{Continuous Function from Quotient Pulls Back}{continuous_function_from_quotient_pulls_back}
    Input:
    \begin{enumerate}
        \item $X\xrightarrow{\mathcal{Q}} Y$ is a quotient space.
        \item $Z$ is any topological space.
    \end{enumerate}
    Then $g:Y\rightarrow Z$ is continuous if and only if $g\circ \mathcal{Q}: X\rightarrow Z$ is continous.
\end{lemma}

\begin{lemma}{UMP of Quotient Space}{ump_of_quotient_space}
    Input:
    \begin{enumerate}
        \item $X\xrightarrow{\mathcal{Q}} Y$ is a quotient space.
    \end{enumerate}
    Let $f:X\rightarrow Z$ be any continous function that is constant over each fiber of $\mathcal{Q}$.
    Then there exists a unique continuous function $\overline{f}:Y\rightarrow Z$ such that $\overline{f}\circ\mathcal{Q} = f$.
    \begin{center}
        \begin{tikzcd}[execute at end picture={\draw[dashed,-latex,lightgray] (q1) -- (q2) -- (q3);}]
            X \arrow[r, "\mathcal{Q}"] \arrow[rd, "\forall f"'{name=q2}] & Y \arrow[d, "\exists! \overline{f}"{name=q3}] \\
            & |[alias=q1]| \forall Z
        \end{tikzcd}
    \end{center}
\end{lemma}

\begin{definition}{Suspension}{suspension}
    Let $X$ be a topological space.
    The suspension of $X$ is given by
    \[ S = X \times \qty[-1, 1] / \sim, \]
    where $\sim$ is the equivalence relation defined by
    \begin{align*}
        {\sim} &= \Set*{((x_1, t), (x_2, t))}{x_1,x_2\in X, t\in\qty{-1,1}} \\
        &\phantom{{}={}} \cup \Set*{((x, t), (x, t))}{x\in X, t\in\qty[-1,1]}.
    \end{align*}
\end{definition}

\subsection{Connectivity}

\subsubsection{Connectivity}

\begin{definition}{Connected Space, Disconnected Space\\ \defextends{Topological Space}}{connect_space}
    Let $(X,\mathcal{T})$ be a topological space.
    $(X,\mathcal{T})$ is disconnected if $X = U\cup V$ for some $U,V\in \mathcal{T}\setminus\qty{\varnothing}$.
    If $(X,\mathcal{T})$ is not disconnected, then $X$ is said to be connected.
\end{definition}

\begin{theorem}{Image of Connected Space is Connected}{image_of_connected_space_is_connected}
    Input:
    \begin{itemize}
        \item $X$ and $Y$ are topological spaces.
        \item $f:X\rightarrow Y$ is a surjective continuous map.
    \end{itemize}
    Then if $X$ is connected, $Y$ is connected.
\end{theorem}

\begin{definition}{Path}{path}
    Let $X$ be a topological space.
    A path in $X$ is a continuous map $\gamma:[0,1]\rightarrow X$.
\end{definition}

\begin{definition}{Pathwise Connected Space\\ \defextends{Connected Space}}{pathwise_connected_space}
    Let $X$ be a topological space.
    $X$ is said to be a pathwise connected space if for each $x_0,x_1\in X$, there exists a path $\gamma:[0,1]\rightarrow X$ with $\gamma(0) = x_0$ and $\gamma(1) = x_1$.
\end{definition}

\begin{theorem}{Pathwise Connected Space is Connected}{pathwise_connected_space_is_connected}
    Let $X$ be a pathwise connected space.
    Then $X$ is connected.
\end{theorem}

\begin{counterexample}{Topologist's Sine Curve}{topologist_s_sine_curve}
    The topologist's sine curve is connected but not pathwise connected.
\end{counterexample}

\begin{theorem}{Image of Pathwise Connected Space is Pathwise Connected}{image_of_pathwise_connected_space_is_pathwise_connected}
    Input:
    \begin{itemize}
        \item $X$ and $Y$ are topological spaces.
        \item $f:X\rightarrow Y$ is a surjective continuous map.
    \end{itemize}
    Then if $X$ is pathwise connected, $Y$ is pathwise connected.
\end{theorem}

\begin{lemma}{Pathwise Connected to a Basepoint}{pathwise_connect_to_a_basepoint}
    Let $X$ be a topological space and $x_0\in X$.
    Then the following are equivalent.
    \begin{itemize}
        \item $X$ is a pathwise connected space.
        \item For each $x_1\in X$ there exists a path $\gamma:[0,1] \rightarrow X$ such that $\gamma(0) = x_0$ and $\gamma(1) = x_1$.
    \end{itemize}
\end{lemma}

\begin{definition}{Connected Component}{connected_component}
    Let $X$ be a topological space.
    The connected components are given by the partition by the equivalence relation defined by $x\sim y$ if $x$ and $y$ are contained in some connected open subspace of $X$.
\end{definition}

\begin{counterexample}{Connected Component is Not Open}{connected_component_is_not_open}
    Connected components of $\mathbb{Q}$ are single points and are not open.
\end{counterexample}

\begin{theorem}{Connected Subspace Intersects Only One Component}{connected_subspace_intersects_only_one_component}
    Let $X$ be a topological space.
    The connected components give rise to a partition of $X$.
    If $Y$ is a nonempty connected subspace of $X$, then $Y$ intersects with exactly one connected component of $X$.
\end{theorem}

\begin{definition}{Pathwise Connected Component}{pathwise_connected_component}
    Let $X$ be a topological space.
    The pathwise connected components are given by the partition by the equivalence relation defined by $x\sim y$ if there exists a path $\alpha:[0,1]\rightarrow X$ such that $\alpha(0) = x$ and $\alpha(1) = y$.
\end{definition}

\begin{theorem}{Pathwise Connected Subspace Intersects Only One Component}{pathwise_connected_subspace_intersects_only_one_component}
    Let $X$ be a topological space.
    The pathwise connected components give rise to a partition of $X$.
    If $Y$ is a nonempty pathwise connected subspace of $X$, then $Y$ intersects with exactly one pathwise connected component of $X$.
\end{theorem}

\subsubsection{Local Connectedness}

\begin{definition}{Locally Connected Space}{locally_connected_space}
    Let $X$ be a topological space.
    $X$ is said to be locally connected at $x\in X$ if for each neighborhood $U$ of $x$, there is a connected neighborhood $V$ of $x$ contained in $U$.
    If $X$ is locally connected at all $x\in X$, then $X$ is said to be locally connected.
\end{definition}

\begin{theorem}{Equivalence of Local Connectivity}{equivalence_of_local_connectivity}
    Let $X$ be a topological space.
    The following are equivalent.
    \begin{itemize}
        \item $X$ is locally connected.
        \item For each open set $U$ of $X$, the connected components of $U$ are all open in $X$.
    \end{itemize}
\end{theorem}

\begin{definition}{Locally Pathwise Connected Space}{locally_pathwise_connected_space}
    Let $X$ be a topological space.
    $X$ is said to be locally pathwise connected at $x\in X$ if for each neighborhood $U$ of $x$, there is a pathwise connected neighborhood $V$ of $x$ contained in $U$.
    If $X$ is locally pathwise connected at all $x\in X$, then $X$ is said to be locally pathwise connected.
\end{definition}

\begin{theorem}{Equivalence of Local Pathwise Connectivity}{equivalence_of_local_pathwise_connectivity}
    Let $X$ be a topological space.
    The following are equivalent.
    \begin{itemize}
        \item $X$ is locally pathwise connected.
        \item For each open set $U$ of $X$, the pathwise connected components of $U$ are all open in $X$.
    \end{itemize}
\end{theorem}

\begin{theorem}{Connectivity is Pathwise with Local Pathwise Connectivity}{connectivity_is_pathwise_with_local_pathwise_connectivity}
    If $X$ is a topological space, then each pathwise connected component lies in a connected component of $X$.
    If $X$ is locally pathwise connected, then each connected component is pathwise connected.
\end{theorem}

\begin{definition}{Weak Local Connectivity}{weak_local_connectivity}
    Let $X$ be a topological space.
    $X$ is said to be weakly locally connected at $x$ if for every neighborhood $U$ of $x$, there is a connected subspace of $X$ contained in $U$ that contains a neighborhood of $x$.
\end{definition}

The difference between local connectivity and weak local connectivity is that for weak local connectivity, the subspace containing $x$ may not necessarily be open.

\begin{proposition}{Weak Local Connectivity Everywhere is Local Connectivity}{weak_local_connectivity_everywhere_is_local_connectivity}
    If $X$ is weakly locally connected at all $x\in X$, then $X$ is locally connected.
\end{proposition}
\begin{proof}
    If $X$ is not locally connected, there is a connected component $C$ that is not open, and therefore a $x\in C$ such that no neighborhood of $x$ is contained in $C$.
    However, $X$ is weakly locally connected at $x$, and therefore there is a connected subspace $Y$ containing a neighborhood $U$ of $X$.
    $Y$ intersects with exactly one connected component which is $C$.
    Therefore, $U$ is a neighborhood of $X$ contained in $C$.
\end{proof}

\begin{counterexample}{Infinite Broom, Weak Local Connectivity is not Local Connectivity}{infinite_broom}
    The infinite broom is not locally connected at $a_\infty$.
    But it is weakly locally connected at $a_\infty$.
\end{counterexample}

\begin{theorem}{Connected and Locally Pathwise Connected is Pathwise Connected}{connected_and_locally_pathwise_connected_is_pathwise_connected}
    If $X$ is a connected and locally pathwise connected space, then $X$ is pathwise connected.
\end{theorem}

\begin{theorem}{Pathwise Connectivity of Topological Manifold}{pathwise_connectivity_of_topological_manifold}
    Let $X$ be a topological manifold.
    Then the following are equivalent.
    \begin{itemize}
        \item $X$ is connected.
        \item $X$ is pathwise connected.
    \end{itemize}
\end{theorem}

\begin{theorem}{Product of Connected is Connected}{product_of_connected_is_connected}
    Let $X_1,\cdots,X_n$ be topological spaces, and $X = X_1\times \cdots \times X_n$.
    Then the following are equivalent.
    \begin{itemize}
        \item $X$ is connected.
        \item $X_1,\cdots,X_n$ are all connected.
    \end{itemize}
\end{theorem}

\begin{theorem}{Product of Pathwise Connected is Pathwise Connected}{product_of_pathwise_connected_is_pathwise_connected}
    Let $X_1,\cdots,X_n$ be topological spaces, and $X = X_1\times \cdots \times X_n$.
    Then the following are equivalent.
    \begin{itemize}
        \item $X$ is pathwise connected.
        \item $X_1,\cdots,X_n$ are all pathwise connected.
    \end{itemize}
\end{theorem}

\subsection{Compactness}

\subsubsection{Compactness}

\begin{definition}{Cover, Open Cover, Subcover}{cover}
    Let $(X,\mathcal{T})$ be a topological space.
    A collection $\Set*{U_\alpha}{\alpha\in I}$ is called a cover of $X$ if
    \begin{itemize}
        \item for each $\alpha\in I$, $U_\alpha$ is a subset of $X$, and
        \item $\cup_\alpha U_\alpha = X$.
    \end{itemize}
    $\Set*{U_\alpha}{\alpha\in I}$ is an open cover of $X$ if it is a cover of $X$ and it is a subset of $\mathcal{T}$.
    A subcover of a cover of $X$ is a subset therefore that is also a cover of $X$.
\end{definition}

\begin{theorem}{Second-Countable Space Has Countable Subcover}{second_countable_space_has_countable_subcover}
    If $X$ is a second-countable topological space, then every open cover of $X$ has a countable subcover.
\end{theorem}

\begin{definition}{Compact Space \defextends{Topological Space}}{compact_space}
    A topological space $X$ is compact if any open cover of $X$ has a finite subcover.
\end{definition}

\begin{theorem}{Heine-Borel Theorem}{heine_borel_theorem}
    A subspace $X$ of $\mathbb{R}^n$ is compact if and only if it is closed and bounded.
\end{theorem}

\begin{theorem}{Image of Compact Space is Compact}{image_of_compact_space_is_compact}
    Input:
    \begin{itemize}
        \item $X$ and $Y$ are topological spaces.
        \item $f:X\rightarrow Y$ is a surjective continuous map.
    \end{itemize}
    Then if $X$ is compact, $Y$ is compact.
\end{theorem}

\begin{theorem}{Closed Set of Compact Space is Compact}{closed_set_of_compact_space_is_compact}
    Input:
    \begin{itemize}
        \item $X$ is a topological space.
        \item $X'$ is a subspace of $X$ with relative topology.
    \end{itemize}
    If $X'$ is closed in $X$, then $X'$ is compact.
\end{theorem}

\begin{theorem}{Compact Set of Hausdorff Space is Closed}{compact_set_of_hausdorff_space_is_closed}
    Input:
    \begin{itemize}
        \item $X$ is a Hausdorff space.
        \item $X'$ is a compact subspace of $X$ with relative topology.
    \end{itemize}
    Then $X'$ is closed.
\end{theorem}

\begin{theorem}{Compact to Hausdorff Bimorphism is Homeomorphism}{compact_to_hausdorff_bimorphism_is_homeomorphism}
    Input:
    \begin{itemize}
        \item $X$ is a compact space.
        \item $Y$ is a Hausdorff space.
        \item $f:X\rightarrow Y$ is a continuous bijection.
    \end{itemize}
    Then $f$ is a homeomorphism.
\end{theorem}

\begin{theorem}{Product of Compact is Compact}{product_of_compact_is_compact}
    If $X_1,\cdots,X_n$ are compact spaces, then $X = X_1 \times \cdots \times X_n$ with product topology is compact.
\end{theorem}

\begin{theorem}{Compact Hausdorff Space is Normal}{compact_hausdorff_space_is_normal}
    If $X$ is a compact Hausdorff space then $X$ is normal.
\end{theorem}

\subsubsection{Limit Point Compactness}

\begin{definition}{Limit Point Compact Space}{limit_point_compact_space}
    Let $X$ be a topological space.
    $X$ is called a limit point compact space if every infinite subset of $X$ has a limit point.
\end{definition}

\begin{definition}{Sequentially Compact Space}{sequentially_compact_space}
    Let $X$ be a topological space.
    $X$ is called a sequentially compact space if every sequence in $X$ has a convergent subsequence.
\end{definition}

\subsubsection{Local Compactness}

\begin{definition}{Locally Compact Space}{locally_compact_space}
    Let $X$ be a topological space.
    $X$ is said to be locally compact at $x\in X$ if there is
    \begin{itemize}
        \item a neighborhood $U$ of $x$, and
        \item a compact subspace of $X$ that contains $U$.
    \end{itemize}
    $X$ is said to be locally compact if $X$ is locally compact at each $x\in X$.
\end{definition}

\begin{theorem}{Locally Compact Hausdorff Space}{locally_compact_hausdorff_space}
    Let $X$ be a Hausdorff space.
    The following conditions are equivalent.
    \begin{itemize}
        \item $X$ is locally compact.
        \item For each $x\in X$ and each neighborhood $U$ of $x$, there is a neighborhood $V$ of $x$ such that
        \begin{itemize}
            \item $\overline{V}$ is compact, and
            \item $\overline{V}\subset U$.
        \end{itemize}
    \end{itemize}
\end{theorem}

\begin{theorem}{One-Point Compactification}{one_point_compactification}
    Let $(X,\mathcal{T})$ be a topological space.
    Then the following conditions are equivalent.
    \begin{itemize}
        \item $X$ is locally compact Hausdorff.
        \item There exists $(Y,\mathcal{S})$ such that
        \begin{itemize}
            \item $X$ is a subspace of $Y$,
            \item $Y-X$ consists of a single point, and
            \item $Y$ is a compact Hausdorff space.
        \end{itemize}
    \end{itemize}
\end{theorem}

The topology of $Y$ in the above is given by
\[ \mathcal{S} = \mathcal{T} \cup \Set*{Y-C}{C \text{ is a compact subspace of } X}. \]

\subsubsection{Paracompactness}

\begin{definition}{Paracompact Space \defextends{Compact Space}}{paracompact_space}
    Let $X$ be a topological space.
    $X$ is paracompact if every open cover of $X$ has a locally finite open refinement that covers $X$.
\end{definition}

\begin{theorem}{Paracompact Space is Compact}{paracompact_space_is_compact}
    Let $X$ be a paracompact space.
    Then $X$ is a compact space.
\end{theorem}

\begin{theorem}{Paracompact Hausdorff is Normal}{paracompact_hausdorff_is_normal}
    Let $X$ be a paracompact Hausdorff space.
    Then $X$ is a normal space.
\end{theorem}

\begin{theorem}{Equivalence of Paracompactness in Hausdorff Space}{equivalence_of_paracompactness_in_hausdorff_space}
    Let $X$ be a Hausdorff space.
    Then the following are equivalent.
    \begin{itemize}
        \item $X$ is paracompact.
        \item $X$ admits partition of unity subordinate to any open cover.
    \end{itemize}
\end{theorem}

\subsection{Metric Space}

\begin{lemma}{Lebesgue's Number Lemma}{lebesgue_s_number_lemma}
    Input:
    \begin{itemize}
        \item $X$ is a compact metric space.
        \item $\mathcal{U}$ is an open cover of $X$.
    \end{itemize}
    There exists $\delta > 0$ such that every subset of $X$ having diameter less than $\delta$ is contained in some member of $\mathcal{U}$.
\end{lemma}

\begin{theorem}{Equivalence of Compactness in Metric Space}{equivalence_of_compactness_in_metric_space}
    Let $X$ be a metrizable space.
    Then the following are equivalent.
    \begin{itemize}
        \item $X$ is compact.
        \item $X$ is limit point compact.
        \item $X$ is sequentially compact.
    \end{itemize}
\end{theorem}

\begin{theorem}{Urysohn's Metrization Theorem}{Urysohn_s_metrization_theorem}
    A topological space $X$ is metrizable if it satisfies all of the following.
    \begin{itemize}
        \item $X$ is Hausdorff.
        \item $X$ is regular.
        \item $X$ is second-countable.
    \end{itemize}
\end{theorem}

\section{Locally Trivial Bundle}

\begin{definition}{Locally Trivial Bundle, Total Space, Base Space, Fiber, Local Trivializing Neighborhood, Local Trivialization}{locally_trivial_bundle}
    A locally trivial bundle $(P, X, \pi, Y)$ consists of the following data.
    \begin{itemize}
        \item A topological space $P$, the total space.
        \item A Hausdorff space $X$, the base space.
        \item A continuous map $\pi: P\rightarrow X$, the projection.
        \item A Hausdorff space $Y$, the fiber.
    \end{itemize}
    $(P, X, \pi, Y)$ should satisfy the following properties.
    \begin{itemize}
        \item For each $x_0\in X$ there exists an open set $V\subset X$ containing $x_0$ and a homeomorphism $\Phi: V \times Y \rightarrow \pi^{-1}(V)$ such that $\pi\circ \Phi(x, y) = x$ for all $(x, y) \in V \times Y$.
    \end{itemize}
    In the above, $V$ is called a local trivializing neighborhood in $X$ and the pair $(V, \Phi)$ is a local trivialization of the bundle.
\end{definition}

\begin{definition}{Trivial Bundle}{trivial_bundle}
    $(P,X,\pi,Y)$ is called a trivial bundle if $P = X\times Y$ and $\pi = \operatorname{proj}_1$.
\end{definition}

\begin{example}{Hopf Bundle}{hopf_bundle}
    The Hopf bundles are $S^1 \rightarrow S^{2n-1} \xrightarrow{\pi} \mathbb{C}\mathrm{P}^{n-1}$ and $S^3 \rightarrow S^{4n-1} \xrightarrow{\pi} \mathbb{H}\mathrm{P}^{n-1}$.
\end{example}

\begin{theorem}{Compact Base and Fiber Gives Compact Total}{compact_base_and_fiber_gives_compact_total}
    Let $(P,X,\pi,Y)$ be a locally trivial bundle.
    If $X$ and $Y$ are compact, then $P$ is compact.
\end{theorem}

\begin{theorem}{Connected Base and Fiber Gives Connected Total}{connected_base_and_fiber_gives_connected_total}
    Let $(P,X,\pi,Y)$ be a locally trivial bundle.
    If $X$ and $Y$ are connected, then $P$ is connected.
\end{theorem}

\begin{definition}{Lift}{lift}
    \begin{itemize}
        \item Let $(P,X,\pi,Y)$ be a locally trivial bundle.
        \item $Z$ is a topological space.
        \item $f:Z\rightarrow X$ is a continuous map.
    \end{itemize}
    A lift of $f$ to $P$ is a continuous map $\tilde{f}$ such that the following diagram commutes.
    \begin{center}
        \begin{tikzcd}
            & P \arrow[d, "\pi"] \\
            Z \arrow[r,"f"] \arrow[ur,"\tilde{f}"] & X
        \end{tikzcd}
    \end{center}
\end{definition}

\begin{definition}{Cross Section}{cross_section}
    Let $(P,X,\pi,Y)$ be a locally trivial bundle.
    A cross section $\sigma:X\rightarrow P$ is a lift of $\operatorname{id}:X\rightarrow X$ to $P$.
\end{definition}

\begin{theorem}{Path Lifting Theorem}{path_lifting_theorem}
    Input:
    \begin{itemize}
        \item $(P,X,\pi,Y)$ is a locally trivial bundle.
        \item $\alpha:[0,1]\rightarrow X$ is a path in $X$.
    \end{itemize}
    For any $p$ in the fiber $\pi^{-1}(\alpha(0))$, there exists a lift $\tilde{\alpha}:[0,1]\rightarrow P$ of $\alpha$ to P with $\tilde{\alpha}(0) = p$.
\end{theorem}

\begin{theorem}{Pathwise Connected Base and Fiber Gives Pathwise Connected Total}{pathwise_connected_base_and_fiber_gives_pathwise_connected_total}
    Let $(P,X,\pi,Y)$ be a locally trivial bundle.
    If $X$ and $Y$ are pathwise connected, then $P$ is pathwise connected.
\end{theorem}

\section{Covering Space}

\begin{definition}{Evenly Covered Open Set, Sheet}{evenly_covered_open_set}
    Input:
    \begin{itemize}
        \item $E$ and $B$ are topological spaces.
        \item $p:E\twoheadrightarrow B$ is a continuous surjective map.
        \item $U$ is an open set of $B$.
    \end{itemize}
    $U$ is said to be evenly covered by $p$ if
    \[ p^{-1}(U) = \bigcup_{\alpha\in I} V_\alpha \]
    where $V_\alpha$ are disjoint open set of $E$ for each $\alpha$ and that $\eval{p}_{V_\alpha}: V_\alpha\rightarrow U$ is a homeomorphism.
    The collection $\Set*{V_\alpha}{\alpha\in I}$ is called a partition of $p^{-1}(U)$ into sheets.
\end{definition}

\begin{definition}{Covering Space, Covering Map}{covering_space}
    \begin{itemize}
        \item $E$ and $B$ are topological spaces.
        \item $p:E\twoheadrightarrow B$ is a continuous surjective map.
    \end{itemize}
    If every point $b$ of $B$ has a neighborhood $U$ that is evenly covered by $p$, then $p$ is called a covering map and $E$ is said to be a covering space of $B$.
\end{definition}

\begin{example}{Hopf Bundle is not Covering Space}{hopf_bundle_is_not_covering_space}
    The Hopf bundles $S^1 \rightarrow S^{2n-1} \xrightarrow{\pi} \mathbb{C}\mathrm{P}^{n-1}$ and $S^3 \rightarrow S^{4n-1} \xrightarrow{\pi} \mathbb{H}\mathrm{P}^{n-1}$ are not covering map.
    The preimage of single points are $S^1$ and $S^3$, which are not discrete.
\end{example}

\begin{proposition}{Fibers of Connected Covering are Identical}{fibers_of_connected_covering_are_identical}
    Let $p:E\twoheadrightarrow B$ be a convering and $E$ be connected.
    Then for each $x\in B$, $p^{-1}(x)$ has the same cardinality.
\end{proposition}

\begin{theorem}{Connected Covering is a Locally Trivial Bundle}{connect_covering_is_a_locally_trivial_bundle}
    Input:
    \begin{itemize}
        \item $p:E\twoheadrightarrow B$ is a connected covering.
        \item $D$ is the discrete space isomorphic to any fiber $p^{-1}(x)$ for $x\in B$.
    \end{itemize}
    Then $(E,B,p,D)$ is a locally trivial bundle.
\end{theorem}

\begin{theorem}{Unique Lifting Theorem}{unique_lifting_theorem}
    Input:
    \begin{itemize}
        \item $p:E\twoheadrightarrow B$ is a connected covering, and $(E,B,p,D)$ is the locally trivial bundle.
        \item $x_0 \in B$.
        \item $\tilde{x}_0 \in p^{-1}(x_0)$.
        \item $Z$ is a connected space.
        \item $f:Z\rightarrow X$ is a continuous map with $f(z_0) = x_0$ for some $z_0\in Z$.
    \end{itemize}
    If there is a lift $\tilde{f}$ of $f$ to $E$ with $\tilde{f}(z_0) = \tilde{x}_0$, then the lift is unique.
\end{theorem}

\begin{corollary}{Connected Covering Admits Path Lift}{connected_covering_admits_path_lift}
    Input:
    \begin{itemize}
        \item $p:E\twoheadrightarrow B$ is a connected covering, and $(E,B,p,D)$ is the locally trivial bundle.
        \item $x_0 \in B$.
        \item $\tilde{x}_0 \in p^{-1}(x_0)$.
        \item $\alpha:[0,1]\rightarrow X$ is a path in $X$ with $\alpha(0) = x_0$
    \end{itemize}
    Then there exists a unique lift $\tilde{\alpha}:[0,1]\rightarrow E$ of $\alpha$ such that $\tilde{\alpha}(0) = \tilde{x}_0.$
\end{corollary}

\section{Topological Group}

\begin{definition}{Topological Group \\ \defextends{Group} \defextends{Fr\'echet Space}}{topological_group}
    A topological group $G$ is both
    \begin{itemize}
        \item a group, and
        \item a Fr\'echet space,
    \end{itemize}
    such that both of the following are continuous.
    \begin{itemize}
        \item $(x,y) \mapsto x\times y$, and
        \item $x \mapsto x^{-1}$.
    \end{itemize}
\end{definition}

\begin{theorem}{Topological Group is Hausdorff and Regular}{topological_group_is_hausdorff_and_regular}
    Let $G$ be a Topological group.
    Then $G$ is both Hausdorff and regular.
\end{theorem}

\begin{proposition}{Subgroup is Topological Group}{subgroup_is_topological_group}
    Let $G$ be a topological group, and $H$ be a subgroup thereof.
    Then both $H$ and $\overline{H}$ are topological subgroups.
\end{proposition}

\begin{proposition}{Quotient is Topological Group}{quotient_is_topological_group}
    Let $G$ be a topological group, and $H$ be a normal subgroup thereof.
    Then $G/H$ is a topological group.
\end{proposition}

\begin{proposition}{Connected Topological Group is Exponential}{connected_topological_group_is_exponential}
    Input:
    \begin{itemize}
        \item $G$ is a connected topological group.
        \item $U$ is a subgroup of $G$ containing $e\in G$.
    \end{itemize}
    Then
    \[ G = \bigcup_{n=1}^\infty U^n. \]
\end{proposition}

\begin{theorem}{Component of Identity is Normal Subgroup}{component_of_identity_is_normal_subgroup}
    Input:
    \begin{itemize}
        \item $G$ is a topological group.
        \item $H$ is the connected component of $G$ containing $e$.
    \end{itemize}
    Then $H$ is a closed, connected, and normal subgroup of $G$. If $G$ is locally connected, then $H$ is open.
\end{theorem}

\begin{proposition}{Subgroup and Quotient Connectivity Implies Connectivity}{subgroup_and_quotient_connectivity_implies_connectivity}
    Let $G$ be a topological group and $H$ be a closed subgroup thereof.
    If $H$ and $G/H$ (quotient as a topological space) are connected, then $G$ is also connected.
\end{proposition}

\begin{theorem}{Orbit-Stabilizer Theorem for Topological Group}{orbit_stabilizer_theorem_for_topological_group}
    Input:
    \begin{itemize}
        \item $G$ is a topological group.
        \item $X$ is a topological space.
        \item $(g,x)\mapsto g\cdot x$ is a transitive left action of $G$ on $X$.
    \end{itemize}
    For a fixed $x_0\in X$, let
    \begin{itemize}
        \item $H = \Set*{g\in G}{g\cdot x_0 = x_0}$,
        \item $Q':G\rightarrow X$ be defined by $Q'(g) = g\cdot x_0$, and
        \item $Q:G\rightarrow G/H$ be the canonical projection.
    \end{itemize}
    Then there exists a unique continuous bijection $\varphi:G/H\rightarrow X$ such that the following diagram commutes.
    \begin{center}
        \begin{tikzcd}
            G \arrow[d,"Q"'] \arrow[rd,"Q'"] & \\
            G/H \arrow[r,"\varphi"'] & X
        \end{tikzcd}
    \end{center}
    Moreover, if either
    \begin{itemize}
        \item $G$ is compact, or
        \item $Q'$ is an open map,
    \end{itemize}
    then $\varphi$ is a homeomorphism.
\end{theorem}

% \bibliographystyle{plain}
% \bibliography{main}

\end{document}
