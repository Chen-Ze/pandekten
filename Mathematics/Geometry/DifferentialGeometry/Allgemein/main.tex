\documentclass{article}

\usepackage{pandekten}

\title{Allgemein}
\author{Ch\=an Taku}

\begin{document}

\maketitle

\section{}

\section{Miscellany}

\subsection{Homeomorphism and Diffeomorphism}

\begin{theorem}{Homeomorphic Manifold Admits Diffeomorphism}{homeomorphic_manifold_admits_diffeomorphism}
    Let $M$ be a smooth manifold, $N$ be a topological space and $f:M\rightarrow N$ be a homeomorphism.
    Then there is a smooth structure on $N$ such that $f$ becomes a diffeomorphism.
\end{theorem}

A proof may be found in \href{https://math.stackexchange.com/questions/762585/are-homeomorphic-differentiable-manifolds-actually-diffeomorphic}{Are homeomorphic differentiable manifolds actually diffeomorphic?}
Diffeomorphism depends on the smooth structure on both manifolds.
Therefore, homeomorphic manifolds are not necessarily diffeomorphic.

\subsection{Visualization of Manifolds}

\subsubsection{Visualization of Spheres}

\begin{example}{Visualization of $S^3$}{visualization_of_s_3}
    $S^3$ may be identified with
    \[ \Set{(z^1,z^2)}{\abs{z^1}^2 + \abs{z^2}^2 = 1} \]
    and therefore be parameterized by
    \begin{align*}
        z^1 &= \cos \frac{\phi}{2} e^{i\xi_1}, \\
        z^2 &= \sin \frac{\phi}{2} e^{i\xi_2},
    \end{align*}
    where
    \[ 0 \le \frac{\phi}{2} \le \frac{\pi}{2}. \]
    The parameterization is visualized as the following.
    \begin{enumerate}
        \item $\phi/2 = 0$ is a circle in the plane $z=0$ in $\mathbb{R}^3 \cup \qty{\infty}$.
        \item $\phi/2 = \pi/4$ is a torus that encircles the above circle.
        The torus grows larger as $\phi$ increases.
        \item $\phi/2 = \pi/2$ is the largest torus which is identified with the line through infinity $\Set*{(0,0,z)}{z\in\mathbb{R}}\cup\qty{\infty}$.
    \end{enumerate}
    $\phi$ stands for the radius of the torus, and $(\xi_1,\xi_2)$ for the latitude and longitude of a torus.
\end{example}

\begin{example}{$S^3$ as a Principal $\operatorname{U}(1)$-bundle over $S^2$}{s_3_as_a_principal_u_1_bundle_over_s_2}
    Let $(z^1,z^2)$ be the parameterization given as in \cref{ex:visualization_of_s_3}.
    The right action of $g\in\operatorname{U}(1)$ on $S^3$ is given by
    \[ (z^1,z^2) g = (z^1 g, z^2 g). \]
    The quotient $S^3/\sim$ with respect to this action is $S^2$.
    Let $\mathcal{P}: S^3\rightarrow S^2$ be this projection, then $S^3 \xrightarrow{\mathcal{P}} S^2$ is a principal $\operatorname{U}(1)$-bundle.
\end{example}

\subsubsection{Visualization of Groups}

\begin{example}{Visualization of $\operatorname{SU}(2)$}{visualization_of_su_2}
    The following manifolds are homeomorphic.
    \begin{enumerate}
        \item $\operatorname{SU}(2)$.
        \item $\Set*{x\in \mathbb{H}}{\norm{x} = 1}$.
        \item $S^3$.
    \end{enumerate}
\end{example}

% \bibliographystyle{plain}
% \bibliography{main}

\end{document}
