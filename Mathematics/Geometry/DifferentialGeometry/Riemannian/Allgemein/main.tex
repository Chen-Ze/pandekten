\documentclass{article}

\usepackage{pandekten}

\title{Allgemein}
\author{Ch\=an Taku}

\begin{document}

\maketitle

\section{Curvature}

\begin{definition}{Algebraic Curvature Tensor, Curvature-like Tensor}{curvature_like_tensor}
    Let $V$ be a linear space. A tensor $R\in \mathscr{T}^0_4(V)$ is curvature-like (or an algebraic curvature tensor) if for any $x,y,z,w\in V$,
    the following identities hold:
    \begin{gather}
        \label{eq:curvature_like_antisymmetry} R(x, y, z, w) = -R(y, x, z, w) = -R(x, y, w, z); \\
        \label{eq:curvature_like_symmetry} R(x, y, z, w) = R(z, w, x, y); \\
        \label{eq:curvature_like_jacobi} R(x, y, z, w) + R(y, z, x, w) + R(z, x, y, w) = 0;
    \end{gather}
    Curvature-like tensors over $V$ are denoted $\operatorname{Curv}(V)$.
\end{definition}

\begin{proposition}{Pair Symmetry of Curvature-like Tensors}{pair_symmetry_of_curvature_like_tensors}
    In \cref{def:curvature_like_tensor},
    \cref{eq:curvature_like_antisymmetry} and \cref{eq:curvature_like_jacobi} implies \cref{eq:curvature_like_symmetry}.
\end{proposition}

\begin{definition}{Kulkarni-Nomizu Product}{kulkarni_nomizu_product}
    $\owedge: \operatorname{Sym}^2(V) \times \operatorname{Sym}^2(V) \rightarrow \operatorname{Curv}(V)$ is defined by 
    \begin{align*}
        T\owedge S(x_1,x_2,x_3,x_4) &= T(x_1,x_3)S(x_2,x_4) + S(x_2,x_4)T(x_1,x_3) \\
        &{\phantom{{}={}}} - T(x_1,x_4)S(x_2,x_3) - S(x_2,x_3)T(x_1,x_4).
    \end{align*}
\end{definition}

\begin{proposition}{Properties of Kulkarni-Nomizu Product}{properties_of_kulkarni_nomizu_product}
    \begin{itemize}
        \item Commutativity: $T\owedge S = S\owedge T$.
    \end{itemize}
\end{proposition}

\subsection{Affine Connections}

\subsection{Curvature Tensors}

\begin{definition}{Riemann Curvature Tensor}{riemann_curvature_tensor}
    Let $(M,g)$ be a pseudo-Riemannian manifold.
\end{definition}

\begin{definition}{Ricci Tensor}{ricci_tensor}
    Let $(M,g)$ be a pseudo-Riemannian manifold.
\end{definition}

\begin{definition}{Ricci Scalar}{ricci_scalar}
    Let $(M,g)$ be a pseudo-Riemannian manifold.
\end{definition}

% \bibliographystyle{plain}
% \bibliography{main}

\end{document}
