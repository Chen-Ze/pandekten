\documentclass{article}

\usepackage{pandekten}

\externaldocument[Top-]{../../../../Topology/GeneralTopology/Allgemein/main}

\title{Projective Spaces}
\author{Ch\=an Taku}

\begin{document}

\maketitle

\section{Real Projective Space}

\begin{definition}{Real Projective Space}{real_projective_space}
    The real projective space $\mathbb{R}\mathrm{P}^n$ is the quotient space $\mathbb{R}^{n+1}\setminus\qty{0}\rightarrow \mathbb{R}\mathrm{P}^n$ with fibers given by the partition induced by the equivalence relation
    \[ \sim = \Set*{(x,y)}{\exists\alpha\in\mathbb{R},x=y\alpha}. \]
\end{definition}

\begin{theorem}{Real Projective Space from Sphere}{real_projective_space_from_sphere}
    $\mathbb{R}\mathrm{P}^n \cong S^n/\sim$ with fibers given by the partition induced by the equivalence relation
    \[ \sim = \Set*{(x,y)}{\exists\alpha\in\qty{1,-1},x=y\alpha}. \]
\end{theorem}

Therefore, $\mathbb{R}\mathrm{P}^n$ may be identified with $S^n/\mathbb{Z}_2$.

\begin{example}{$\mathbb{R}\mathrm{P}^1\cong S^1$}{rp_1_is_s_1}
    Note that the inhomogeneous coordinates of $\mathbb{R}\mathrm{P}^1$ coincides with the stereographic projection of $S^1$, i.e.
    \[ \varphi_1 \circ \varphi_2^{-1} = \varphi_{\mathrm{N}}\circ\varphi_{\mathrm{S}}^{-1}. \]
    A homeomorphism may be defined between $\mathbb{R}\mathrm{P}^1\cong S^1$ using the gluing lemma (\cref{Top-lem:the_gluing_lemma}).
\end{example}

\section{Complex Projective Space}

\begin{definition}{Complex Projective Space}{complex_projective_space}
    The complex projective space $\mathbb{C}\mathrm{P}^n$ is the quotient space $\mathbb{C}^{n+1}\setminus\qty{0}\rightarrow \mathbb{C}\mathrm{P}^n$ with fibers given by the partition induced by the equivalence relation
    \[ \sim = \Set*{(x,y)}{\exists\alpha\in\mathbb{C},x=y\alpha}. \]
\end{definition}

\begin{theorem}{Complex Projective Space from Sphere}{complex_projective_space_from_sphere}
    $\mathbb{C}\mathrm{P}^n \cong S_{\mathbb{C}}^n/\sim$ with fibers given by the partition induced by the equivalence relation
    \[ \sim = \Set*{(x,y)}{\exists\alpha\in U,x=y\alpha}, \]
    where
    \[ S_{\mathbb{C}}^n = \Set*{x\in\mathbb{C}^{n+1}}{\norm{x}^2 = 1} \]
    and
    \[ U = \Set*{z\in\mathbb{C}}{\abs{z} = 1} = \operatorname{U}(1). \]
\end{theorem}

Therefore, $\mathbb{C}\mathrm{P}^{n-1}$ may be identified with $S^{2n-1}/\operatorname{U}(1)$ where $\operatorname{U}(1)\cong S^1$.

\begin{example}{$\mathbb{C}\mathrm{P}^1\cong S^2$}{cp_1_is_s_2}
    Note that the inhomogeneous coordinates of $\mathbb{C}\mathrm{P}^1$ coincides with the stereographic projection of $S^2$, i.e.
    \[ \overline{\varphi}_1 \circ \varphi_2^{-1} = \varphi_{\mathrm{N}}\circ\varphi_{\mathrm{S}}^{-1}. \]
    A homeomorphism may be defined between $\mathbb{C}\mathrm{P}^1\cong S^2$ using the gluing lemma (\cref{Top-lem:the_gluing_lemma}).
\end{example}

\section{Quaternion Projective Space}

\begin{definition}{Quaternion Projective Space}{quaternion_projective_space}
    The quaternion projective space $\mathbb{H}\mathrm{P}^n$ is the quotient space $\mathbb{H}^{n+1}\setminus\qty{0}\rightarrow \mathbb{H}\mathrm{P}^n$ with fibers given by the partition induced by the equivalence relation
    \[ \sim = \Set*{(x,y)}{\exists\alpha\in\mathbb{H},x=y\alpha}. \]
\end{definition}

\begin{theorem}{Quaternion Projective Space from Sphere}{quaternion_projective_space_from_sphere}
    $\mathbb{H}\mathrm{P}^n \cong S_{\mathbb{H}}^n/\sim$ with fibers given by the partition induced by the equivalence relation
    \[ \sim = \Set*{(x,y)}{\exists\alpha\in U,x=y\alpha}, \]
    where
    \[ S_{\mathbb{H}}^n = \Set*{x\in\mathbb{H}^{n+1}}{\norm{x}^2 = 1} \]
    and
    \[ U = \Set*{z\in\mathbb{H}}{\abs{z} = 1} = \operatorname{Sp}(1). \]
\end{theorem}

Therefore, $\mathbb{H}\mathrm{P}^{n-1}$ may be identified with $S^{4n-1}/\operatorname{SU}(2)$ where $\operatorname{SU}(2)\cong S^3$.

\begin{example}{$\mathbb{H}\mathrm{P}^1\cong S^4$}{hp_1_is_s_4}
    Note that the inhomogeneous coordinates of $\mathbb{H}\mathrm{P}^1$ coincides with the stereographic projection of $S^4$, i.e.
    \[ \overline{\varphi}_1 \circ \varphi_2^{-1} = \varphi_{\mathrm{N}}\circ\varphi_{\mathrm{S}}^{-1}. \]
    A homeomorphism may be defined between $\mathbb{H}\mathrm{P}^1\cong S^4$ using the gluing lemma (\cref{Top-lem:the_gluing_lemma}).
\end{example}

% \bibliographystyle{plain}
% \bibliography{main}

\end{document}
