\documentclass{article}

\usepackage{pandekten}

\title{Allgemein}
\author{Ch\=an Taku}

\begin{document}

\maketitle

\section{Lie Algebra}

\begin{definition}{Adjoint Representation}{adjoint_representation}
    Input:
    \begin{itemize}
        \item $G$ is a Lie group.
        $\Phi_g \in \operatorname{Aut}(G)$ be the conjutation $\Phi_g(h) = ghg^{-1}$.
        \item $\mathfrak{g}$ is the Lie algebra of $G$.
        $\operatorname{Ad}_g$ denotes the derivative of $\Phi_g$ at origin, i.e.
        \[ \operatorname{Ad}_g = \dd_e \Phi_g: \mathfrak{g} \rightarrow \mathfrak{g}. \]
    \end{itemize}
    The adjoint representation of $G$
    \[ \operatorname{Ad}: G \rightarrow \operatorname{Aut}(\mathfrak{g}) \]
    is given by
    \[ \operatorname{Ad}(g) = \operatorname{Ad}_g. \]
\end{definition}

\begin{definition}{Fundamental Vector Field}{fundamental_vector_field}
    Input:
    \begin{itemize}
        \item $G$ is a Lie group.
        \item $\mathfrak{g}$ is the Lie algebra of $G$.
        \item $M$ is a smooth manifold.
        \item $A: G\times M \rightarrow M$ is a smooth action.
        $A_p: G\rightarrow M$ is defined by $A_p(g) = A(g, p)$.
    \end{itemize}
    The fundamental vector field for $X\in \mathfrak{g}$ is
    \[ X^\sharp_p = \dd_e A_p(X). \]
\end{definition}

\section{Haar Measure}

\begin{definition}{Left (Right) Invariant Measure}{left_right_invariant_measure}
    Let $G$ be an $n$-dimensional Lie group, and $\Omega$ be a volume form on $G$.
    $\Omega$ is called a left (right) invariant measure if
    \[ L_g^* \Omega = \Omega \quad (R_g^* \Omega = \Omega) \]
    for all $g\in G$.
\end{definition}

\begin{theorem}{Construction of Left (Right) Invariant Measure}{constrcution_of_left_right_invariant_measure}
    Let $G$ be an $n$-dimensional Lie group.
    Define $\omega^\alpha(g_0)\in \Omega^1(G)$ for all $g_0\in G$ by
    \[ \eval{\dd_g(g_0^{-1}g)}_{g = g_0} = \sum_{\alpha} X_\alpha \omega^\alpha(g_0) \quad \qty(\eval{\dd_g(g g_0^{-1})}_{g = g_0} = \sum_{\alpha} X_\alpha \omega^\alpha(g_0)), \]
    where $\Set*{X^\alpha}{\alpha}$ is a basis of the Lie algebra of $G$.
    Then
    \[ \Omega = \bigwedge_{\alpha} \omega^\alpha \]
    is an left (right) invariant measure on $G$.
\end{theorem}

In particular, we may let $\xi: U\subset G \rightarrow \mathbb{R}^n$ be a chart.
Then
\[ \eval{\Omega}_U = \xi^*(\rho), \]
where
\[ \rho = (\det A) \dd \xi^1 \wedge \cdots \wedge \dd \xi^n \]
and the $n\times n$ matrix $A$ is given by
\[ \eval{\omega^\alpha}_U(g) = \xi^*\qty(\sum_{i=1}^n A^\alpha_i(\xi(g)) \dd{\xi^i}). \]

\begin{example}{Haar Measure on $\operatorname{SO}(3)$}{haar_measure_on_so_3}
    $\operatorname{SU}(2)$ is the universal cover of $\operatorname{SO}(3)$.
    Therefore, $\operatorname{SU}(2) \cong S^3$ is homeomorphic to a $\mathbb{Z}_2$-bundle over $\operatorname{SO}(3)$ since $\pi_1(\operatorname{SO}(3)) \cong \mathbb{Z}_2$.
    A Haar measure $\rho$ on $\operatorname{SU}(2)$ may be given by the Euclidean one since $S^3\subset \mathbb{R}^4$.
    For an arbitrary section $\sigma$, $\sigma^* \rho$ induces a Haar measure on $\operatorname{SO}(3)$.
\end{example}

% \bibliographystyle{plain}
% \bibliography{main}

\end{document}
