\documentclass{article}

\usepackage{pandekten}

\title{Integrals}
\author{Ch\=an Taku}

\begin{document}

\maketitle

\section{Fourier Transform}

\begin{definition}{Fourier Transform, Inverse Fourier Transform}{fourer_transform}
    The Fourier transform is given by
    \[ \mathcal{F}[f](k) = \int \dd{^d x} f(x) e^{-ik\cdot x}. \]
    The inverse Fourier transform is given by
    \[ \mathcal{F}^{-1}[\tilde{f}](x) = \frac{1}{(2\pi)^d} \int \dd{^d k} \tilde{f}(k) e^{ik\cdot x}. \]
\end{definition}

\begin{theorem}{Fourier Transform of Radial Function}{fourier_transform_of_radial_function}
    If $f: \mathbb{R}^d \rightarrow \mathbb{R}$ has the form $f(x) = F(\abs{x})$, then its Fourier transform $\tilde{f}(k) = \hat{F}(\abs{k})$, \href{https://www.math.arizona.edu/~faris/methodsweb/hankel.pdf}{where}
    \[ s^{\frac{d-2}{2}} \hat{F}(s) = (2\pi)^{\frac{d}{2}} \int_0^\infty J_{\frac{d-2}{2}}(sr) r^{\frac{d-2}{2}} F(r) r \dd{r}. \]
\end{theorem}

\begin{theorem}{Fourier Transform of Power}{fourier_transform_of_power}
    Let $f:\mathbb{R}^d \rightarrow \mathbb{R}$ be given by
    \[ f(x) = \frac{1}{\abs{x}^\alpha}, \]
    where $1 < \alpha < d$.
    Then its Fourier transform \href{https://www.math.arizona.edu/~faris/methodsweb/hankel.pdf}{is given by}
    \[ \mathcal{F}[f](k) = \frac{C_\alpha}{\abs{k}^{d-\alpha}}, \]
    where
    \[ C_\alpha = (2\pi)^{\frac{d}{2}} \frac{2^{\frac{d-\alpha}{2}}\Gamma\qty(\frac{d-\alpha}{2})}{2^{\frac{\alpha}{2}}\Gamma\qty(\frac{\alpha}{2})}. \]
\end{theorem}

\section{Collection}

\begin{theorem}{Real Gaussian Integral}{real_gaussian_integral}
    Let $A$ be an invertible complex $n\times n$ matrix such that $A=A^\intercal$,
    and $\Re A$ is positive semidefinite.
    Then for any $y\in \mathbb{R}^n$,
    \[ \int e^{-x^\intercal A x / 2} e^{iy\cdot x} \dd{^n x} = \frac{(2\pi)^{n/2}}{\sqrt{\det A}} e^{-y^\intercal A^{-1} y/2}. \]
\end{theorem}

\begin{theorem}{Complex Gaussian Integral}{complex_gaussian_integral}
    Let $A$ be an invertible complex $n\times n$ matrix such that $A=A^\intercal$,
    and $\Re A$ is positive semidefinite.
    Then for any $w\in \mathbb{C}^n$,
    \[ \int e^{-z^\dagger A z} e^{i(w^\dagger z + z^\dagger w)} \dd{^{2n} z} = \frac{\pi^n}{\det A} e^{-w^\dagger A^{-1} w}. \]
\end{theorem}

\end{document}