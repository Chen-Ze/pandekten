\documentclass{article}

\usepackage{pandekten}

\title{Allgemein}
\author{Ch\=an Taku}

\begin{document}

\maketitle

\section{Foundations}

To define the notion of class, we have to go beyond ZFC and adopt, for example, \href{https://en.wikipedia.org/wiki/Von_Neumann%E2%80%93Bernays%E2%80%93G%C3%B6del_set_theory}{von Neumann-Bernays-G\"odel set theory}.

\begin{definition}{Diagram, Small Diagram, Finite Diagram}{diagram}
    Input:
    \begin{itemize}
        \item $\mathbf{C}$ and $\mathbf{J}$ are categories.
    \end{itemize}
    A diagram of shape $\mathbf{J}$ in $\mathbf{C}$ is a functor $F:\mathbf{J}\rightarrow\mathbf{C}$.
    A diagram is said to be small or finite whenever $\mathbf{J}$ is.
\end{definition}

\begin{definition}{Cone, Cocone}{cone}
    Input:
    \begin{itemize}
        \item $\mathbf{C}$ and $\mathbf{J}$ are categories.
        \item $F:\mathbf{J}\rightarrow\mathbf{C}$ is a diagram.
    \end{itemize}
    \paragraph*{Cone}
    A cone $(N,\psi)$ from $N$ to $F$ is the following collection of data.
    \begin{itemize}
        \item An object $N$ of $\mathbf{C}$.
        \item For each $X$ an object of $\mathbf{J}$, a morphism $\psi_X: N\rightarrow F(X)$, subjected to the following condition.
        \begin{itemize}
            \item For every morphism $f:X\rightarrow Y$ in $\mathbf{J}$, the following diagram commutes.
            \begin{center}
                \begin{tikzcd}
                    & N \arrow[dl, "\psi_X"'] \arrow[dr, "\psi_Y"] & \\
                    F(X) \arrow[rr, "F(f)"]  & & F(Y) \\
                    X \arrow[rr, "f"] & & Y
                \end{tikzcd}
            \end{center}
        \end{itemize}
    \end{itemize}
    \tcblower
    \paragraph*{Cocone}
    A cocone $(N,\psi)$ from $F$ to $N$ is the following collection of data.
    \begin{itemize}
        \item An object $N$ of $\mathbf{C}$.
        \item For each $X$ an object of $\mathbf{J}$, a morphism $\psi_X: F(X) \rightarrow N$, subjected to the following condition.
        \begin{itemize}
            \item For every morphism $f:X\rightarrow Y$ in $\mathbf{J}$, the following diagram commutes.
            \begin{center}
                \begin{tikzcd}
                    X \arrow[rr, "f"] & & Y \\
                    F(X) \arrow[dr, "\psi_X"'] \arrow[rr, "F(f)"]  & & F(Y) \arrow[dl, "\psi_Y"] \\
                    & N &
                \end{tikzcd}
            \end{center}
        \end{itemize}
    \end{itemize}
\end{definition}

\section{Universals}

\begin{definition}{Universal from an Object to a Functor}{universal_from_an_object_to_a_functor}
    Input:
    \begin{itemize}
        \item $\mathbf{C}$ and $\mathbf{D}$ are categories.
        \item $F$ is a functor from $\mathbf{C}$ to $\mathbf{D}$.
        \item $B$ is an object of $\mathbf{D}$.
    \end{itemize}
    A universal from $B$ to $F$ is a pair $(U,u)$ where
    \begin{itemize}
        \item $U$ is an object of $\mathbf{C}$, and
        \item $u$ is a morphism from $B$ to $FU$
    \end{itemize}
    such that if $A$ is an object of $\mathbb{C}$,
    and $g$ is any morphism from $B$ to $FA$,
    then there exists a unique morphism $\tilde{g}$ of $U$ into $A$
    such that the following diagram is commutative.
    \begin{center}
        \begin{tikzcd}[execute at end picture={\draw[dashed,-latex,lightgray] (q1) -- (q2) -- (q3);}]
            B \arrow[r, "u"] \arrow[rd, "\forall g"'{name=q2}] & FU \arrow[d, "F(\exists! \tilde{g})"{name=q3}] \\
            & |[alias=q1]| F(\forall A)
        \end{tikzcd}
    \end{center}
\end{definition}

\begin{definition}{Universal from a Functor to an Object}{universal_from_a_functor_to_an_object}
    Input:
    \begin{itemize}
        \item $\mathbf{C}$ and $\mathbf{D}$ are categories.
        \item $G$ is a functor from $\mathbf{D}$ to $\mathbf{C}$.
        \item $A$ is an object of $\mathbf{C}$.
    \end{itemize}
    A universal from $G$ to $A$ is a pair $(V,v)$ where
    \begin{itemize}
        \item $V$ is an object of $\mathbf{D}$, and
        \item $u$ is a morphism from $GV$ to $A$
    \end{itemize}
    such that if $B$ is an object of $\mathbf{D}$,
    and $g$ is any morphism from $GB$ to $A$,
    then there exists a unique morphism $\tilde{g}$ of $B$ into $V$
    such that the following diagram is commutative.
    \begin{center}
        \begin{tikzcd}[execute at end picture={\draw[dashed,-latex,lightgray] (q1) -- (q2) -- (q3);}]
            A & GV \arrow[l, "u"'] \\
            & |[alias=q1]|G(\forall B) \arrow[u, "G(\exists!\tilde{g})"'{name=q3}] \arrow[ul, "\forall g"{name=q2}]
        \end{tikzcd}
    \end{center}
\end{definition}

\begin{example}{Field of Fractions as a Universal}{fields_of_fractions_as_a_universal}
    Let $\mathbf{Dom}$ denote the category of commutative domains.
    Let $D$ be an object of $\mathbf{D}$.
    The field of fractions $\operatorname{Frac}(D)$ is the universal object from $D$ to the injection functor of $\mathbf{Field}$ into $\mathbf{Dom}$.
\end{example}

\begin{example}{Polynomial over a Ring as a Universal}{polynomial_over_a_ring_as_a_universal}
    Let $K$ be a commutative ring, and $K$-$\mathbf{comalg}$ be the category of commutative algebras over $K$.
    A polynomial ring $K[X_1,\cdots,X_n]$ over $K$ is the universal object from $\qty{x_1,\cdots,x_n}$ to the forgetful functor from $K$-$\mathbf{comalg}$ to $\mathbf{Set}$.
    The universal map $u$ is given by $x_i \mapsto X_i$.
\end{example}

\begin{example}{Coproduct as a Universal}{coproduct_as_a_universal}
    Let $I$ be a set of indices.
    $A_\alpha$ is an object of a category $\mathbf{C}$ for all $\alpha \in I$.
    The coproduct $\displaystyle \coprod_{\alpha\in I} A_\alpha$ is the universal object from $\alpha \mapsto A_\alpha$ to the diagonal functor from $\mathbf{C}$ to $\mathbf{C}^I$.
\end{example}

\begin{example}{Product as a Universal}{product_as_a_universal}
    Let $I$ be a set of indices.
    $A_\alpha$ is an object of a category $\mathbf{C}$ for all $\alpha \in I$.
    The product $\displaystyle \prod_{\alpha\in I} A_\alpha$ is the universal object from the diagonal functor from $\mathbf{C}$ to $\mathbf{C}^I$ to $\alpha \mapsto A_\alpha$.
\end{example}

\begin{example}{Quotient Ring as a Universal}{quotient_ring_as_a_universal}
    Let $R$ be a commutative ring and $I$ be an ideal thereof.
    Let $\mathbf{C}$ be a subcategory of $R$-$\mathbf{comalg}$ where $IA = 0$ for all $A$ an object of $\mathbf{C}$.
    The quotient ring $R/I$ is the universal object from $R$ to the injection functor from $\mathbf{C}$ to $\mathbf{Ring}$ (commutative ring).
    For a generalization of quotient, see \href{https://uwseminars.com/archive/ly-UPQ/}{this seminar}.
\end{example}

\subsection{Limit}

\begin{definition}{Limit \defextends{Cone}, Colimit \defextends{Cocone}}{limit_colimit}
    Input:
    \begin{itemize}
        \item $\mathbf{C}$ and $\mathbf{J}$ are categories.
        \item $F:\mathbf{J}\rightarrow\mathbf{C}$ is a diagram.
    \end{itemize}
    \paragraph*{Limit}
    A limit of $F$ is a cone $(L,\phi)$ from $L$ to $F$, such that for each cone $(N,\psi)$ to $F$, there exists a unique morphism $u: N\rightarrow L$ such that the following diagram commutes.
    \begin{center}
        \begin{tikzcd}
            & \forall N \arrow[ddl, bend right, "\psi_X"'] \arrow[ddr, bend left, "\psi_Y"] \arrow[d, dashed, "\exists!u"] & \\
            & L \arrow[dl, "\phi_X"'] \arrow[dr, "\phi_Y"] & \\
            F(X) \arrow[rr, "F(f)"]  & & F(Y)
        \end{tikzcd}
    \end{center}
    The limit $L$ is denoted by $\lim F$.
    \tcblower
    \paragraph*{Colimit}
    A colimit of $F$ is a cocone $(L,\phi)$ from $F$ to $L$, such that for each cocone $(N,\psi)$ to $F$, there exists a unique morphism $u: L\rightarrow N$ such that the following diagram commutes.
    \begin{center}
        \begin{tikzcd}
            F(X) \arrow[dr, "\phi_X"'] \arrow[ddr, bend right, "\psi_X"'] \arrow[rr, "F(f)"]  & & F(Y) \arrow[dl, "\phi_Y"] \arrow[ddl, bend left, "\psi_Y"] \\
            & L \arrow[d, dashed, "\exists!u"] & \\
            & \forall N &
        \end{tikzcd}
    \end{center}
    The colimit $L$ is denoted by $\operatorname{colim} F$.
\end{definition}

\begin{definition}{Existence of Limit, Existence of Colimit, Complete Category, Cocomplete Category}{existence_of_limit}
    Input:
    \begin{itemize}
        \item $\mathbf{C}$ is a category.
    \end{itemize}
    \paragraph*{Existence of Limit}
    Let $\mathbf{J}$ be a category.
    $\mathbf{C}$ is said to have limits of shape $\mathbf{J}$ if every diagram $F:\mathbf{J}\rightarrow \mathbf{C}$ of shape $\mathbf{J}$ has a limit in $\mathbf{C}$.
    \par
    $\mathbf{C}$ is said to be a complete category if $\mathbf{C}$ has limit of shape $\mathbf{J}$ for all small category $\mathbf{J}$.
    \tcblower
    \paragraph*{Existence of Limit}
    Let $\mathbf{J}$ be a category.
    $\mathbf{C}$ is said to have colimits of shape $\mathbf{J}$ if every diagram $F:\mathbf{J}\rightarrow \mathbf{C}$ of shape $\mathbf{J}$ has a colimit in $\mathbf{C}$.
    \par
    $\mathbf{C}$ is said to be a cocomplete category if $\mathbf{C}$ has colimit of shape $\mathbf{J}$ for all small category $\mathbf{J}$.
\end{definition}

\begin{definition}{Preservation of Limit, Preservation of Colimit, \\ Continuous Functor, Cocontinuous Functor}{preservation_of_limit}
    Input:
    \begin{itemize}
        \item $\mathbf{C}$ and $\mathbf{D}$ are categories.
        \item $G:\mathbf{C}\rightarrow \mathbf{D}$ is a functor.
    \end{itemize}
    \paragraph*{Preservation of Limit}
    Let $\mathbf{J}$ be a category.
    $G$ is said to preserve all limit of shape $\mathbf{J}$ if for all diagrams $F:\mathbf{J}\rightarrow \mathbf{C}$ of shape $\mathbf{J}$, $(G\lim F, G\phi)$ is a limit of $GF$ whenever $(\lim F, \phi)$ is a limit of $F$.
    \par
    $G$ is said to be a continuous functor if $G$ preserves all limit of shape $\mathbf{J}$ for all small category $\mathbf{J}$.
    \tcblower
    \paragraph*{Preservation of Colimit}
    Let $\mathbf{J}$ be a category.
    $G$ is said to preserve all colimit of shape $\mathbf{J}$ if for all diagrams $F:\mathbf{J}\rightarrow \mathbf{C}$ of shape $\mathbf{J}$, $(G\operatorname{colim} F, G\phi)$ is a colimit of $GF$ whenever $(\operatorname{colim} F, \phi)$ is a colimit of $F$.
    \par
    $G$ is said to be a cocontinuous functor if $G$ preserves all colimit of shape $\mathbf{J}$ for all small category $\mathbf{J}$.
\end{definition}

\begin{example}{Coproduct as Colimit}{coproduct_as_colimit}
    Let $\Lambda$ be a discrete category and $\mathbf{C}$ be the category $R$-$\mathbf{Mod}$. 
    Then the colimit of $F:\Lambda\rightarrow \mathbf{C}$ is just the direct sum
    \[ \operatorname*{colim} F = \bigoplus_{\lambda\in\Lambda} F(\lambda). \]
\end{example}

\begin{example}{Coequalizer as Colimit}{coequalizer_as_colimit}
    Let $\Lambda$ be a category with
    \begin{itemize}
        \item two objects $\qty{X, Y}$,
        \item two morphisms $\alpha,\beta: X\rightrightarrows Y$ between $X$ and $Y$,
    \end{itemize}
    $\mathbf{C}$ be the category $R$-$\mathbf{Mod}$.
    Then the colimit of $F:\Lambda \rightarrow \mathbf{C}$ is just the coequalizer
    \[ \operatorname*{colim} F = \operatorname{coeq}(F(\alpha), F(\beta)). \]
\end{example}
A cocone from $F$ to $N$ should satisfy the following commutative diagram.
\begin{center}
    \begin{tikzcd}
        F(X) \arrow[rr, yshift=0.7ex, "f = F(\alpha)"] \arrow[rr, yshift=-0.7ex, "g = F(\beta)"'] \arrow[rrd, "\varphi_X=\varphi_{Y}f=\varphi_{Y}g"'] & & F(Y) \arrow[d, "\varphi_Y"] \\
        & & N
    \end{tikzcd}
\end{center}
Then the colimit $L$ satisfies the following commutative diagram.
\begin{center}
    \begin{tikzcd}[execute at end picture={\draw[dashed,-latex,lightgray] (q1) -- (q2) -- (q3);}]
        & & L \arrow[dd, bend left, dashed, "\exists!u"{name=q3}] \\
        F(X) \arrow[rru] \arrow[rr, yshift=0.7ex, "f"] \arrow[rr, yshift=-0.7ex, "g"'] \arrow[rrd] & & F(Y) \arrow[d, "\forall \varphi"'{name=q2}] \arrow[u, "q"] \\
        & & |[alias=q1]|\forall N
    \end{tikzcd}
\end{center}
To put it another way, $L$ satisfies the following.
\begin{center}
    \begin{tikzcd}[execute at end picture={\draw[dashed,-latex,lightgray] (q1) -- (q2) -- (q3);}]
        F(X) \arrow[rr, yshift=0.7ex, "f"] \arrow[rr, yshift=-0.7ex, "g"'] \arrow[rrd] & & F(Y) \arrow[rr, "q", "qf=qg"'] \arrow[d,"\forall\varphi"{name=q2}, "\varphi f=\varphi g"'] & & L \arrow[lld, dashed, "\exists!u"{name=q3}] \\
        & & |[alias=q1]|\forall N
    \end{tikzcd}
\end{center}

\begin{theorem}{Existence Theorem for Limits, Existence Theorem for Colimits}{existence_theorem_for_limits}
    Input:
    \begin{itemize}
        \item $\mathbf{C}$ is a category.
        \item $\mathbf{J}$ is a category.
    \end{itemize}
    \paragraph*{Existence of Limits}
    The following are equivalent.
    \begin{itemize}
        \item $\mathbf{C}$ has limits of shape $\mathbf{J}$.
        \item $\mathbf{C}$ has both
        \begin{itemize}
            \item products indexed by $\operatorname{ob}(\mathbf{J})$ and $\operatorname{hom}(\mathbf{J})$, and
            \item equalizers.
        \end{itemize}
    \end{itemize}
    \tcblower
    \paragraph*{Existence of Colimits}
    The following are equivalent.
    \begin{itemize}
        \item $\mathbf{C}$ has colimits of shape $\mathbf{J}$.
        \item $\mathbf{C}$ has both
        \begin{itemize}
            \item coproducts indexed by $\operatorname{ob}(\mathbf{J})$ and $\operatorname{hom}(\mathbf{J})$, and
            \item equalizers.
        \end{itemize}
    \end{itemize}
\end{theorem}
\begin{proof}
    The limit $\lim F$ of diagram $F:\mathbf{J}\rightarrow \mathbf{C}$ is given by $\operatorname{eq}(s,t)$ where
    \[ s,t: \prod_{Z\in\operatorname{ob}(\mathbf{J})} F(Z) \rightrightarrows \prod_{(X\rightarrow Y)\in \operatorname{hom}(\mathbf{J})} F(Y) \]
    are defined by
    \begin{align*}
        s &= \prod_{(X\rightarrow Y)\in\operatorname{hom}(\mathbf{J})} F(f) \circ \operatorname{proj}_X, \\
        t &= \prod_{(X\rightarrow Y)\in\operatorname{hom}(\mathbf{J})} \operatorname{proj}_Y. \qedhere
    \end{align*}
\end{proof}

\begin{theorem}{Continuity Theorem, Cocontinuity Theorem}{existence_theorem_for_limits}
    Input:
    \begin{itemize}
        \item $\mathbf{C}$ is a complete category.
        \item $\mathbf{D}$ is a category.
        \item $G:\mathbf{C}\rightarrow\mathbf{D}$ is a functor.
    \end{itemize}
    \paragraph*{Preservation of Limits}
    The following are equivalent.
    \begin{itemize}
        \item $G$ is continuous.
        \item $G$ preserves both
        \begin{itemize}
            \item products indexed by all small categories, and
            \item equalizers.
        \end{itemize}
    \end{itemize}
    \tcblower
    \paragraph*{Preservation of Colimits}
    The following are equivalent.
    \begin{itemize}
        \item $G$ is cocontinuous
        \item $G$ preserves both
        \begin{itemize}
            \item coproducts indexed by all small categories, and
            \item equalizers.
        \end{itemize}
    \end{itemize}
\end{theorem}

\begin{theorem}{Small Colimits Commute}{small_colimit_commute}
    Input:
    \begin{itemize}
        \item $\mathbf{I}$ and $\mathbf{J}$ are small categories.
        \item $\mathbf{C}$ is a category that has colimits of shape $\mathbf{I}$ and $\mathbf{J}$.
        \item $F: \mathbf{I} \rightarrow \mathbf{C}^\mathbf{J}$ is a diagram of shape $\mathbf{I}$ in the category of diagram of shape $\mathbf{J}$ in $\mathbf{C}$.
        \item $G: \mathbf{J} \rightarrow \mathbf{C}^\mathbf{I}$ is a diagram of shape $\mathbf{J}$ in the category of diagram of shape $\mathbf{I}$ in $\mathbf{C}$.
        \item $F$ and $G$ satisfy $F(i)(j) = G(j)(i)$ for all $i\in \mathbf{I}$ and $j\in\mathbf{J}$.
    \end{itemize}
    Then
    \[ \operatorname*{colim} \operatorname*{colim} F = \operatorname*{colim} \operatorname*{colim} G. \]
\end{theorem}
This could also be written as
\[ \operatorname*{colim}_{i\in I} \operatorname*{colim}_{j\in J} F(i)(j) = \operatorname*{colim}_{j\in J} \operatorname*{colim}_{i\in I} F(i)(j). \]

\section{Adjoints}

\begin{definition}{Left Adjoint, Right Adjoint, Adjugant, and Adjunction}{adjoint}
    Data:
    \begin{itemize}
        \item $F$ a functor from $\mathbf{C}$ to $\mathbf{D}$.
        \item $G$ a functor from $\mathbf{D}$ to $\mathbf{C}$.
    \end{itemize}
    $F$ is called a right adjoint of $G$ and $G$ a left adjoint of $F$ if
    \begin{itemize}
        \item for every $(B,A)$ where $B$ is an object of $\mathbf{D}$ and $A$ is an object of $\mathbf{C}$,
        we have a bijective map $\eta_{B,A}: \operatorname{Hom}_{\mathbf{C}}(GB,A) \rightarrow \operatorname{Hom}_{\mathbf{D}}(B,FA)$ such that
        \item for every $B$, $A \mapsto \eta_{B,A}$ is a natural isomorphism of $\operatorname{Hom}_{\mathbf{C}}(GB, -)$ to $\operatorname{Hom}_{\mathbf{D}}(B, F-)$, and
        \item for every $A$, $B \mapsto \eta_{B,A}$ is a natural isomorphism of $\operatorname{Hom}_{\mathbf{C}}(G-, A)$ to $\operatorname{Hom}_{\mathbf{D}}(-, FA)$.
    \end{itemize}
    The map $\eta: (B,A) \mapsto \eta_{B,A}$ is called an adjugant from $G$ to $F$.
    The triple $(F,G,\eta)$ is an adjunction.
\end{definition}

The conditions may be represented by the following commutative diagram.
\begin{center}
    \begin{tikzcd}
        \operatorname{Hom}_{\mathbf{C}}(GgB,A) \arrow[d, "\eta_{gB,A}", "\simeq"'] \arrow[rr,"{\operatorname{Hom}_{\mathbf{C}}(Gg,A)}"] & &\operatorname{Hom}_{\mathbf{C}}(GB,A) \arrow[d, "\eta_{B,A}", "\simeq"'] \arrow[rr,"{\operatorname{Hom}_{\mathbf{C}}(GB,f)}"] & & \operatorname{Hom}_{\mathbf{C}}(GB,fA) \arrow[d, "\eta_{B,fA}", "\simeq"'] \\
        \operatorname{Hom}_{\mathbf{D}}(gB,FA) \arrow[rr,"{\operatorname{Hom}_{\mathbf{D}}(g,FA)}"] & & \operatorname{Hom}_{\mathbf{D}}(B,FA) \arrow[rr,"{\operatorname{Hom}_{\mathbf{D}}(B,Ff)}"] & & \operatorname{Hom}_{\mathbf{D}}(B,FfA)
    \end{tikzcd}
\end{center}
This is equivalent to that morphisms $C\rightarrow C'$ and $D\rightarrow D'$ induces the following commutative diagram.
\begin{center}
    \begin{tikzcd}
        \operatorname{Hom}_{\mathbf{C}}(GD',C) \arrow[r, leftrightarrow, "\simeq"] \arrow[d] & \operatorname{Hom}_{\mathbf{D}}(D',FC) \arrow[d] \\
        \operatorname{Hom}_{\mathbf{C}}(GD,C') \arrow[r, leftrightarrow, "\simeq"] & \operatorname{Hom}_{\mathbf{D}}(D,FC')
    \end{tikzcd}
\end{center}

\begin{example}{Semidirect Product as an Adjoint}{semidirect_product_as_an_adjoint}
    This example is copied from \href{https://math.stackexchange.com/questions/284315/universal-property-of-n-rtimes-k}{Universal property of $N\rtimes K$}.
    See \href{https://ncatlab.org/nlab/show/semidirect+product+group#as_a_left_adjoint}{semidirect product group} for more fancy words.
    \par
    The category $\mathbf{GrpAction}$ is defined by the following.
    \begin{itemize}
        \item Objects: $(G, \Gamma, \rho)$ where $G$ and $\Gamma$ are groups and $\rho: G\rightarrow \operatorname{Aut}(\Gamma)$.
        \item Morphism: $(G,\Gamma,\rho) \rightarrow (G',\Gamma',\rho')$ are $G$-equivariant pairs of morphisms $f: G\rightarrow G'$ and $h: \Gamma\rightarrow \Gamma'$,
        i.e. such that $h(\rho(g)(\gamma)) = \rho'(f(g))(h(\gamma))$ for all $g\in G$ and $\gamma\in \Gamma$.
    \end{itemize}
    The forgetful functor $U: \mathbf{Arr}(\mathbf{Grp}) \rightarrow \mathbf{GrpAction}$ is defined by the following.
    \begin{itemize}
        \item An object $f: G\rightarrow \Gamma$ is sent to $(G,\Gamma,\rho)$ where $\rho$ is given by $\rho(g)(\gamma) = f(g)\gamma f(g^{-1})$.
    \end{itemize}
    Now the left adjoint of $U$ maps $(G,\Gamma,\rho)$ to the inclusion $(G,\Gamma,\rho)\xhookrightarrow{} \Gamma \rtimes_\rho G$.
\end{example}

\begin{example}{Colimit as Left Adjoint}{colimit_as_left_adjoint}
    Let $\mathbf{C}$ be a diagram and $\Delta$ be the diagonal functor (where $\Lambda$ is a small category)
    \[ \Delta: \mathbf{C} \rightarrow \mathbf{C}^\Lambda. \]
    The colimit
    \[ \operatorname{colim}: \mathbf{C}^\Lambda \rightarrow \mathbf{C} \]
    is the left adjoint of $\Delta$.
\end{example}
Let $F:\Lambda\rightarrow\mathbf{C}$ be a diagram (an object of $\mathbf{C}^\Lambda$), $N$ be an object of $\mathbf{C}$ and $\Delta(N)$ be an diagonal functor (an element of $\mathbf{C}^\Lambda$).
Then $\operatorname{Hom}_{\mathbf{C}^\Lambda}(F, \Delta(N))$ consists of all natural transformations from $F$ to $\Delta(N)$.
Let $\eta$ be one of such natural transformations.
Then for each $X\xrightarrow{f}Y$ of $\Lambda$, the following diagram commutes.
\begin{center}
    \begin{tikzcd}
        F(X) \arrow[rr, "F(f)"] \arrow[rd, "\eta_X"'] & & F(Y) \arrow[ld, "\eta_Y"]  \\
        & N &
    \end{tikzcd}
\end{center}
Therefore, $\operatorname{Hom}_{\mathbf{C}^\Lambda}(F, \Delta(N))$ consists of the cones to $F$.
\par
To verify that  $\lim$ is the left adjoint of $\Delta$, firstly we need a bijection
\[ \operatorname{Hom}_{\mathbf{C}}(\lim F, N) \xlongleftrightarrow{\simeq} \operatorname{Hom}_{\mathbf{C}^\Lambda}(F, \Delta(N)) = \qty{\text{Cocones from } F \text{ to } N}. \]
This is just the definition of colimit.
Secondly, we need to verify that each
\begin{itemize}
    \item morphism $N\rightarrow N'$ in $\mathbf{C}$, and
    \item natural transformation $F\rightarrow F'$
\end{itemize}
induces the following commutative diagram.
\begin{center}
    \begin{tikzcd}
        \operatorname{Hom}_{\mathbf{C}}(\lim F', N) \arrow[r, leftrightarrow, "\simeq"] \arrow[d] & \operatorname{Hom}_{\mathbf{C}^\Lambda} (F', \Delta(N)) = \qty{\text{Cocones from $F'$ to $N$}} \arrow[d] \\
        \operatorname{Hom}_{\mathbf{C}}(\lim F, N') \arrow[r, leftrightarrow, "\simeq"] & \operatorname{Hom}_{\mathbf{C}^\Lambda} (F, \Delta(N')) = \qty{\text{Cocones from $F$ to $N$}} \\
    \end{tikzcd}
\end{center}

\end{document}