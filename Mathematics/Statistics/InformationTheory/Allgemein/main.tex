\documentclass{article}

\usepackage{pandekten}

\title{Allgemein}
\author{Ch\=an Taku}

\begin{document}

\maketitle

\section{Miscellany}

\begin{theorem}{Shannon-Hartley Theorem}{shannon_hartley_theorem}
    Input:
    \begin{itemize}
        \item $B$ is the bandwidth of the channel in hertz (passband bandwidth in case of a bandpass signal).
        \item $S$ is the average received signal power over the bandwidth.
        \item $N$ is the average power of the noise and interference over the bandwidth, measured in watts.
    \end{itemize}
    The channel capacity in bits per second, a theoretical upper bound on the net bit rate, is given by
    \[ C = B \log_2\qty(1+\frac{S}{N}). \]
\end{theorem}

See \href{https://en.wikipedia.org/wiki/Shannon%E2%80%93Hartley_theorem}{Shannon-Hartley theorem} for more information.
That's why \href{https://electronics.stackexchange.com/questions/272658/why-does-more-bandwidth-guarantee-high-bit-rate}{bit rate and bandwidth are sometimes used interchangeably}.

% \bibliographystyle{plain}
% \bibliography{main}

\end{document}
