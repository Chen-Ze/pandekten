\documentclass{article}

\usepackage{pandekten}

\title{Central Extension}
\author{Ch\=an Taku}

\externaldocument[Group-]{../Allgemein/main}
\externaldocument[Alg-]{../../Allgemein/main}

\begin{document}

\maketitle

\section{Central Extension of Group}

\begin{definition}{Central Extension of Group}{central_extension_of_group}
    An group extension (\cref{Group-def:group_extension})
    \[ 1 \rightarrow A \xrightarrow{\imath} E \xrightarrow{\pi} G \rightarrow 1 \]
    is central if $\imath(A)$ is in the center of $E$.
\end{definition}

\begin{example}{Universal Covering of $\operatorname{SO}(1,3)$}{universal_covering_of_so_1_3}
    The following is a central extension:
    \[ 1 \rightarrow \mathbb{Z}/2\mathbb{Z} \rightarrow \operatorname{SL}(2,\mathbb{C}) \xrightarrow{\pi} \operatorname{SO}(1,3) \rightarrow 1 \]
    where $\pi$ is the 2-to-1 covering.
\end{example}

\begin{example}{Central Extension of Projective Linear Group}{central_extension_of_projective_linear_group}
    Let $K^\times$ denote the multiplicative group of $K$ and $V$ a vector space over $K$.
    Then the following is a central extension:
    \[ 1\rightarrow K^\times \rightarrow \operatorname{GL}(V) \rightarrow \operatorname{PGL}(V) \rightarrow 1. \]
\end{example}

\begin{lemma}{Unitary as Central Extension of Projective}{unitary_as_central_extension_of_projective}
    \begin{itemize}
        \item Let $H$ be a Hilbert space over $\mathbb{C}$.
        \item $\gamma: H\rightarrow H/\mathbb{C}^\times$ is the projection.
        \item $\hat{\gamma}$ denotes the induced projective transformation (\cref{Alg-def:induced_projective_transformation}).
    \end{itemize}
    Then the following is a central extension:
    \[ 1 \rightarrow \operatorname{U}(1) \xrightarrow{\imath} \operatorname{U}(H) \xrightarrow{\hat{\gamma}} \operatorname{U}(H/\mathbb{C}^\times) \rightarrow 1, \]
    where $\imath$ is the inclusion, and
    \[ \operatorname{U}(H/\mathbb{C}^\times) = \hat{\gamma}(\operatorname{U}(H)). \]
\end{lemma}

\begin{definition}{Equivalence of Central Extension}{equivalence_of_central_extension}
    Two central extensions of $G$ by $A$
    \[ 1 \rightarrow A \xrightarrow{\imath} E \xrightarrow{\pi} G \rightarrow 1 \]
    and
    \[ 1 \rightarrow A \xrightarrow{\imath'} E' \xrightarrow{\pi'} G \rightarrow 1 \]
    are equivalent if there exists an isomorphism $\psi:E\rightarrow E'$ such that the following diagram commutes.
    \begin{center}
        \begin{tikzcd}
            1 \arrow[r] & A \arrow[r,"\imath"] \arrow[d,"\operatorname{id}"] & E \arrow[r,"\pi"] \arrow[d,"\psi"] & G \arrow[r] \arrow[d,"\operatorname{id}"] & 1 \\
            %
            1 \arrow[r] & A \arrow[r,"\imath'"] & E' \arrow[r,"\pi'"] & G \arrow[r] & 1
        \end{tikzcd}
    \end{center}
\end{definition}

\begin{lemma}{Central Extension Splits iff Trivial}{central_extension_splits_iff_trivial}
    A central extension splits if and only if it is equivalent to a trivial central extension.
\end{lemma}

\begin{example}{Nontrivial Central Extensions}{nontrivial_central_extensions}
    If $n>1$ and $H$ is an infinite dimensional Hilbert space, then
    \[ 1 \rightarrow \operatorname{U}(1) \rightarrow \operatorname{U}(n) \xrightarrow{\hat{\gamma}} \operatorname{U}(\mathbb{C}^n/\mathbb{C}^\times)\rightarrow 1 \]
    and
    \[ 1 \rightarrow \operatorname{U}(1) \rightarrow \operatorname{U}(H) \xrightarrow{\hat{\gamma}} \operatorname{U}(H/\mathbb{C}^\times)\rightarrow 1 \]
    are not equivalent to the trivial extension.
\end{example}

\begin{example}{Central Extension and Cohomology}{central_extension_and_cohomology}
    \paragraph*{Cocycle and Cohomology}
    Let $G$ be a group and $A$ be an abelian group.
    A map $\omega: G\times G \rightarrow A$ having the properties
    \begin{align*}
        \omega(1,1) &= 1, \\
        \omega(x,y)\omega(xy,z) &= \omega(x,yz)\omega(y,z)
    \end{align*}
    is called a 2-cocycle.
    The central extension associated with a 2-cocycle $\omega$ is given by
    \[ 1 \longrightarrow A \xlongrightarrow{\imath} A \times_\omega G \xlongrightarrow{\operatorname{proj}_2} G \longrightarrow 1 \]
    where $\imath(a) = (a,1)$ and $A \times_\omega G$ is $A\times G$ endowed with the multiplication
    \[ (a,x)(b,y) = (\omega(x,y)ab, xy). \]
    The second cohomology group is defined by
    \[ H^2(G,A) = \Set*{\omega:G\times G \rightarrow A}{\omega \text{ is a 2-cocycle}}/\sim, \]
    where $\omega \sim \omega'$ if and only if there is a $\lambda:G\rightarrow A$ with
    \[ \lambda(xy) = \omega(x,y)\omega'(x,y)^{-1}\lambda(x)\lambda(y). \]
    $H^2(G,A)$ is an abelian group with the multiplication induced by the pointwise multiplication of the maps $\omega:G\times G \rightarrow A$.
    \paragraph*{For a Central Extension to Split}
    Let $\omega:G\times G\rightarrow A$ be a 2-cocycle.
    Then $A \times_\omega G$ splits if and only if there is a map $\lambda:G\rightarrow A$ with
    \[ \lambda(xy) = \omega(x,y)\lambda(x)\lambda(y). \]
    \paragraph*{Finding the Cocycle of a Central Extension}
    The first row in
    \begin{center}
        \begin{tikzcd}
            1 \arrow[r] & \operatorname{U}(1) \arrow[r,"\imath"] \arrow[d,"\operatorname{id}"] & E \arrow[r,"\pi"] \arrow[d,"S"] & G \arrow[r] \arrow[d,"T"] & 1 \\
            %
            1 \arrow[r] & \operatorname{U}(1) \arrow[r] & \operatorname{U}(H) \arrow[r,"\hat{\gamma}"] & \operatorname{U}(H/\mathbb{C}^\times) \arrow[r] & 1
        \end{tikzcd}
    \end{center}
    has the form $E = A \times_\omega G$ (since $E$ is the pullback).
    To find $\omega$, we select for each $g\in G$ a $U_g\in \operatorname{U}(H)$ such that $\hat{\gamma}(U_g) = Tg$.
    Then we define $\tau:G\rightarrow G$ by $\tau_g = (U_g,g)$.
    Now $\tau$ satisfies $\tau(1) = 1$ and $\pi\circ \tau = \operatorname{id}_G$.
    $\omega:G\times G \rightarrow \operatorname{U}(1)$ is defined by
    \[ \omega(g,h) = \tau_g \tau_h \tau^{-1}_{gh} = (U_g U_h U^{-1}_{gh}, 1_G). \]
    $g\mapsto U_g$ is not continuous in general, but for some important cases it can be chosen to yield a continous homomorphism.
    \par
    $H^2(G,A)$ is in one-to-one correspondence with the equivalence classes of central extensions of $G$ by $A$.
\end{example}

% \bibliographystyle{plain}
% \bibliography{main}

\end{document}
