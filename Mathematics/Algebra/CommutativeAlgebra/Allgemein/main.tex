\documentclass{article}

\usepackage{pandekten}

\title{Allgemein}
\author{Ch\=an Taku}

\begin{document}

\maketitle

\section{Commutative Rings}

The following definitions may also be rewritten in terms of monoid objects.

\begin{definition}{Commutative Ring}{commutative_ring}
    A commutative ring $(R,+,\cdot)$ is a ring $(R,+,\cdot)$ where the monoid $(R,\cdot)$ is abelian.
\end{definition}

In the following context, we denote $(R,+,\cdot)$ by simply $R$.

\begin{definition}{Category of Commutative Ring}{category_of_commutative_ring}
    The category of commutative rings, denoted $\mathbf{ComRing}$,
    is the full subcategory of $\mathbf{Ring}$ whose objects are all commutative rings.
\end{definition}

\begin{definition}{Subring of a Commutative Ring}{subring}
    A subring $S$ of a ring $R$ is a subobject of $R$ in $\mathbf{ComRing}$.
\end{definition}

\begin{definition}{Commutative Algebra over a Commutative Ring}{commutative_algebra_over_a_commutative_ring}
    Let $R$ be a commutative ring.
    A commutative algebra over $R$ is a tuple $(S,\varphi)$ where $S$ is a commutative ring and $\varphi: R\rightarrow S$ is a homomorphism.
\end{definition}

In the following context, we denote $(S,\varphi)$ by simply $S$.

\begin{definition}{Polynomial Ring \badge{UMP}}{polynomial_ring}
    Let $R$ be a commutative ring.
    $R[X_1,\cdots,X_n]$ denotes the polynomials with $n$ indeterminates over $R$, i.e. formal sums
    \[ \Set*{\sum_{i\in \qty{0,1,\cdots,m}^n} a_{i_1,\cdots,i_n}X_1^{i_1}\cdots X_n^{i_n}}{m\in \mathbb{N}, a: \qty{0,1,\cdots,m}^n \rightarrow R}. \]
\end{definition}

\begin{proposition}{UMP of Polynomial Ring}{ump_of_polynomial_ring}
    Let $R$ be a commutative ring.
    The polynomial ring with $n$ indeterminates over $R$ is a tuple $(P,\qty{p_1,\cdots,p_n})$ where $\qty{p_1,\cdots,p_n}\subset P$ and $R\xrightarrow{\iota} P$ is an $R$-algebra,
    such that for any tuple $(R',\qty{r'_1,\cdots,r'_n})$ where $\qty{r'_1,\cdots,r'_n}\subset R'$ and $R\xrightarrow{\varphi} R'$ is an $R$-algebra, there exists a unique $P\xrightarrow{f} R'$ such that $f(p_i) = r'_i$ for all $i$ and that the following diagram commutes.
    \begin{center}
        \begin{tikzcd}[execute at end picture={\draw[dashed,-latex,lightgray] (q1) -- (q2) -- (q3);}]
            R \arrow[r, "\iota"] \arrow[rd, "\forall \varphi"'{name=q2}] & P \arrow[d, "\exists! \tilde{f}"{name=q3}] & p_1 \arrow[d,"f"] & \cdots & p_n \arrow[d,"f"] \\
            & |[alias=q1]| \forall R' & r'_1 & \cdots & r'_n
        \end{tikzcd}
    \end{center}
    The tuple $(P,\qty{X_1,\cdots,X_n})$ is denoted by $R[X_1,\cdots,X_n]$.
\end{proposition}

\begin{example}{Polynomial Vanishing Everywhere}{polynomial_vanishing_everywhere}
    $X^7-X\in \mathbb{Z}_7[X]$ evaluates to $0$ for all $X\in \mathbb{Z}_7$.
\end{example}

\begin{definition}{Formal Power Series}{formal_power_series}
    Let $R$ be a commutative ring.
    $R\llbracket X \rrbracket$ denotes the formal power series over $R$.
\end{definition}

\begin{definition}{Formal Laurent Series}{formal_laurent_series}
    Let $R$ be a commutative ring.
    $R\llparenthesis X \rrparenthesis$ denotes the formal Laurent series over $R$, i.e. formal sums
    \[ \Set*{\sum_{i=-m}^\infty a_i X^i}{a:\qty{-m,-m+1,\cdots}\rightarrow R}. \]
\end{definition}

\begin{definition}{Zero Divisor}{zero_divisor}
    A zero divisor $z\in R$ of a commutative ring $R$ is an element $r\in R$ such that there exists $x\in R$, $x\neq 0$, and $zx=0$.
\end{definition}

\begin{definition}{Unit of Commutative Ring}{unit_of_commutative_ring}
    A unit $u$ of a commutative ring $R$ is an element $u\in R$ such that there exists $x\in R$ and $ux=1$.
\end{definition}

\begin{definition}{Idempotent}{idempotent}
    Input:
    \begin{itemize}
        \item $R$ and $R'$ are commutative rings.
        \item $\varphi: R\rightarrow R'$ is a homomorphism.
    \end{itemize}
    Then
    \begin{itemize}
        \item $e\in R$ is an idempotent if $e^2 = 1$;
        \item $e'=1-e$ is also an idempotent and is called the complementary idempotent of $e$;
        \item the set of all idempotents of $R$ is denoted by $\operatorname{Idem}(R)$;
        \item $\operatorname{Idem}(\varphi): \operatorname{Idem}(R) \rightarrow \operatorname{Idem}(R')$ is the restriction of $\varphi$.
    \end{itemize}
\end{definition}

\subsection{Integral Domains}

\begin{definition}{Integral Domain}{integral_domain}
    An integral domain is a nonzero commutative ring $R$ where $0$ is the only zero divisor.
\end{definition}

\begin{definition}{Field of Fraction \badge{UMP}}{field_of_fraction}
    Let $R$ be an integral domain.
    The field of fraction $\operatorname{Frac}(R)$ of $R$ consists of equivalence classes of the form $a/b$ where $a,b\in R$ where $a/b = \Set*{(c,d)\in R}{ad=bc}$.
    The addition is given by
    \[ \frac{n}{d} + \frac{m}{b} = \frac{nb + md}{db}, \]
    the multiplication is given by
    \[ \frac{n}{d}\cdot \frac{m}{b} = \frac{nm}{db} \]
    and the inverse is given by
    \[ \frac{d}{n}\cdot \frac{n}{d} = 1. \]
\end{definition}

\begin{proposition}{UMP of Field of Fraction}{ump_of_field_of_fraction}
    Let $R$ be an integral domain.
    The field of fraction $\operatorname{Frac}(R)$ of $R$ is a field which is also an $R$-algebra $R\xrightarrow{\iota} \operatorname{Frac}(R)$ such that for any $R$-algebra $R\xrightarrow{\varphi} R'$, there exists a unique $\operatorname{Frac}(R) \xrightarrow{f} R'$ such that the following diagram commutes.
    \begin{center}
        \begin{tikzcd}[execute at end picture={\draw[dashed,-latex,lightgray] (q1) -- (q2) -- (q3);}]
            R \arrow[r, "\iota"] \arrow[rd, "\forall \varphi"'{name=q2}] & \operatorname{Frac}(R) \arrow[d, "\exists! \tilde{f}"{name=q3}] \\
            & |[alias=q1]| \forall R'
        \end{tikzcd}
    \end{center}
\end{proposition}

\begin{definition}{Prime Element}{prime_element}
    Let $R$ be a commutative ring.
    A prime element $p$ of $R$ is an element such that
    \begin{itemize}
        \item $p$ is not a unit of $R$, and
        \item $p \mid xy$ implies $p \mid x$ or $p \mid y$.
    \end{itemize}
\end{definition}

\begin{definition}{Irreducible Element}{irreducible_element}
    An irreducible element $p\in R$ of an integral domain $R$ is an element such that
    \begin{itemize}
        \item $p\neq 0$ and $p$ is not a unit of $R$, and
        \item if $a,b\in R$ and $p=ab$, then either $a$ or $b$ is a unit of $R$.
    \end{itemize}
\end{definition}

\begin{definition}{Unique Factorization Domain}{unique_factorization_domain}
    A unique factorization domain is an integral domain $R$ where
    \begin{itemize}
        \item for each $x\in R$, there exists irreducible elements $p_1,\cdots,p_s\in R$ such that
        \[ x = p_1\cdots p_s, \]
        and
        \item if $p_1,\cdots,p_s,q_1,\cdots,q_t\in R$ are irreducible elements and
        \[ p_1\cdots p_s = q_1 \cdots q_t, \]
        then $s = t$ and for each $1\le i \le s$, there exists a unit $u\in R$ such that
        \[ p_i = u_i q_i. \]
    \end{itemize}
\end{definition}

\begin{definition}{Euclidean Domain}{euclidean_domain}
    An Euclidean domain $R$ is an integral domain with a degree function $d: R\backslash \qty{0} \rightarrow \mathbb{N}_0$ such that
    \begin{itemize}
        \item if $a,b,c\in R$ and $b = ac$ then $d(a) \le d(b)$, and
        \item if $a,b\in R$ and $b\neq 0$, then there exists $q,r\in R$ such that
        \[ a = bq + r \]
        such that either $r=0$ or $d(r) < d(b)$.
    \end{itemize}
\end{definition}

\begin{definition}{Primitive Polynomial}{primitive_polynomial}
    A polynomial $p$ over an integral domain $R$ is primitive if that $a\in R$ divides all coefficient of $p$ implies $a$ is a unit of $R$.
\end{definition}

\begin{theorem}{Every Euclidean Domain is a UFD}{every_euclidean_domain_is_a_ufd}
    Every Euclidean domain is a UFD.
\end{theorem}

\begin{proposition}{Zero Divisor Polynomial}{zero_divisor_polynomial}
    Let $R$ be a commutative ring.
    If $f$ is a zero divisor in $R[X]$, then there exists $c\in R$ such that $cf = 0$.
\end{proposition}
The following proof is from \href{https://math.stackexchange.com/questions/83121/zero-divisor-in-rx/83171#83171}{Zero divisor in $R[X]$}.
\begin{proof}
    Let $g = b_0 + b_1 X + \cdots + b_m X^m$ be the polynomial of least degree such that $fg = 0$.
    Let $a_k$ be the coefficient of highest degree such that $a_k g \neq 0$.
    Then $a_k b_m = 0$ and therefore $\deg a_k g < \deg g$ but $f a_k g = 0$.
\end{proof}

\begin{lemma}{Gauss's Lemma}{gauss_lemma}
    \begin{itemize}
        \item Primitive polynomials are closed under multiplication.
        \item A nonconstant polynomial $p\in R[X]$ is irreducible if and only if it is irreducible in $\operatorname{Frac}(R)[X]$ and primitive in $R[X]$.
    \end{itemize}
\end{lemma}

\begin{theorem}{Polynomial Ring over UFD is UFD}{polynomial_ring_over_ufd_is_ufd}
    A polynomial ring over a UFD is a UFD.
\end{theorem}

\section{Ideals}

\begin{definition}{Maximal Ideal}{maximal_ideal}
    Let $R$ be a commutative ring and $I\subset R$ be an ideal.
    $I$ is a maximal ideal if
    \begin{itemize}
        \item $I$ is proper, and
        \item there is no proper ideal $J$ such that $I\subsetneq J$.
    \end{itemize}
\end{definition}

\begin{proposition}{Ideal of Field}{ideal_of_field}
    A commutative ring $R$ is a field if and only if $\langle 0 \rangle$ is a maximal ideal.
\end{proposition}

\begin{corollary}{Quotient by Maximal Ideal}{quotient_by_maximal_ideal}
    Let $R$ be a commutative ring, and $I\subset R$ be an ideal.
    $I$ is a maximal ideal if and only if $R/I$ is a field.
\end{corollary}

\begin{theorem}{Proper Ideal is Contained in a Maximal Ideal}{proper_ideal_is_contained_in_a_maximal_ideal}
    Every proper ideal $I$ of a commutative ring $R$ is contained in a maximal ideal thereof.
\end{theorem}

\begin{corollary}{Unit Belongs to No Maximal Ideal}{unit_belongs_to_no_maximal_ideal}
    Let $R$ be a commutative ring.
    $x\in R$ is a unit if and only if $x$ belongs to no maximal ideal.
\end{corollary}

\begin{counterexample}{Non-Finitely Generated Subideal}{non_finitely_generated_subideal}
    See \href{https://math.stackexchange.com/questions/1206194/finitely-generated-ideal-containing-non-finitely-generated-ideal}{Finitely generated ideal containing non finitely generated ideal}.
    An example of a commutative ring $R$ containing proper ideals $I\subsetneq J \subsetneq R$ with $J$ finitely generated but with $I$ not finitely generated was given by
    \begin{itemize}
        \item $R = F[X_1,X_2,\cdots]$ where $F$ is a field,
        \item $J = \langle X_1 \rangle$, and
        \item $I = \langle X_1 X_2, X_1 X_3, X_1 X_4,\cdots \rangle$.
    \end{itemize}
\end{counterexample}

\subsection{Prime Ideals}

\begin{definition}{Prime Ideal}{prime_ideal}
    A prime ideal $I$ of a commutative ring $R$ is an ideal such that 
    \begin{itemize}
        \item $1\notin I$ and
        \item $xy \in I$ where $x,y\in R$ implies $x\in I$ or $y\in I$.
    \end{itemize}
\end{definition}

\begin{proposition}{Prime Ideal Under Homomorphism}{prime_ideal_under_homomorphism}
    Let $R$ and $R'$ be commutative rings, and $\varphi: R\rightarrow R'$ be a homomorphism.
    \begin{itemize}
        \item If $I'\subset R'$ is a prime ideal, then $\varphi^{-1}(I')$ is a prime ideal of $R$.
        \item If $\varphi$ is surjective and $I\subset R$ is a prime ideal, then $\varphi(I)$ is a prime ideal of $R'$.
    \end{itemize}
\end{proposition}

\begin{corollary}{Quotient by Prime Ideal}{quotient_by_prime_ideal}
    Let $R$ be a commutative ring, and $P\subset R$ be an ideal.
    $P$ is prime if and only if $R/P$ is a integral domain.
\end{corollary}

\begin{corollary}{Maximal Ideal is Prime}{maximal_ideal_is_prime}
    Every maximal ideal is a prime ideal.
\end{corollary}

\subsection{Radicals}

\begin{definition}{Jacobson Radical}{jacobson_radical}
    Let $R$ be a commutative ring.
    Its Jacobson radical $\operatorname{rad}(R)$ is defined to be the intersection of all its maximal ideals.
\end{definition}

\begin{proposition}{Jacobson Radical Addition Preserves Units}{jacobson_radical_addition_preserves_units}
    Let $R$ be a commutative ring.
    $x\in R$ and $u$ is a unit of $R$.
    \begin{itemize}
        \item $x\in\operatorname{rad}(R)$ if and only if $u - xy$ is a unit for all $y\in R$.
        \item In particular, the sum of an element of $\operatorname{rad}(R)$ and a unit is a unit.
    \end{itemize}
\end{proposition}

\begin{proposition}{Quotient by Jacobson Radical is Injective on Idempotents}{quotient_by_jacobson_radical_is_injective_on_idempotents}
    Input:
    \begin{itemize}
        \item $R$ is a commutative ring.
        \item $I\subset \operatorname{rad}(R)$ is an ideal.
        \item $\kappa: R\rightarrow R/I$ is the quotient map.
    \end{itemize}
    Then $\operatorname{Idem}(\kappa)$ is injective.
\end{proposition}

\begin{definition}{Local Ring, Semilocal Ring}{local_ring}
    A commutative ring $A$ is called local if it has exactly one maximal ideal, and semilocal if it has at least one and at most finitely many.
\end{definition}

\begin{lemma}{Nonunit Criterion}{nonunit_criterion}
    Let $A$ be a commutative ring.
    $N$ be the set of nonunits.
    Then
    \begin{itemize}
        \item $A$ is local if and only if $N$ is an ideal; and
        \item if so, then $N$ is the maximal ideal.
    \end{itemize}
\end{lemma}

\begin{proposition}{Complementary of Multiplicative Contains Maximal Prime}{complementary_of_multiplicative_contains_maximal_prime}
    Input:
    \begin{itemize}
        \item $R$ is a commutative ring.
        \item $I\subset R$ is an ideal.
        \item $S$ is a multiplicative subset such that $S\cap I = \varnothing$.
    \end{itemize}
    Then
    \[ \mathcal{S} = \Set*{\text{ideals } J}{J\supset I \text{ and } J\cap S = \varnothing} \]
    has a maximal element $P$, and every such $P$ is prime.
\end{proposition}

\begin{lemma}{Prime Avoidance, Strikeout}{prime_avoidance}
    Input:
    \begin{itemize}
        \item $R$ is a commutative ring.
        \item $A$ is a subset of $R$ closed under addition and multiplication.
        \item $P_1,\cdots,P_n$ are ideals of $R$ where $P_3,\cdots,P_n$ are prime.
    \end{itemize}
    If $A\not\subset P_i$ for all $i$, then $A\not\subset \cup_i P_i$.
\end{lemma}

\begin{definition}{Radical}{radical}
    Let $R$ be a commutative ring and $A$ be a subset thereof.
    The radical of $A$ is defined by
    \[ \sqrt{A} = \Set*{x\in R}{x^n\in A \text{ for some } n\ge 1}. \]
\end{definition}

\begin{definition}{Nilradical, Reduced Ring}{nilradical}
    Let $R$ be a commutative ring.
    $\operatorname{nil}(R) = \sqrt{\langle 0 \rangle}$ is called the nilradical.
    An element $x\in R$ is called nilpotent if $x\in \operatorname{nil}(R)$.
    $R$ is called reduced if $\operatorname{nil}(R) = \langle 0 \rangle$.
\end{definition}

\begin{proposition}{Properties of Radical}{properties_of_radical}
    Let $R$ be a commutative ring and $A$ be a subset thereof.
    \begin{itemize}
        \item $\sqrt{\sqrt{A}} = \sqrt{A}$.
        \item $\sqrt{A} = A$ if $A$ in an intersection of prime ideals.
        \item $\operatorname{nil}(R) \in \operatorname{rad}(R)$.
    \end{itemize}
\end{proposition}

\begin{theorem}{Scheinnullstellensatz}{scheinnullstellensatz}
    Let $R$ be a commutative ring and $I\subset R$ be an ideal. Then
    \[ \sqrt{I} = \bigcap_{P\supset I} P, \]
    where $P$ runs through all the prime ideals containing $I$.
    The empty intersection is equal to $R$.
\end{theorem}

\section{Integral Domains}

\subsection{Unique Factorization Domains}

\begin{theorem}{Irreducible Equals Prime in UFD}{irreducible_equals_prime_in_ufd}
    The irreducible elements of a UFD are exactly the prime elements.
\end{theorem}

\subsection{Principal Ideal Domains}

\begin{definition}{Principal Ideal Domain}{principal_ideal_domain}
    A integral domain $R$ is called a principal ideal domain if every ideal is principal.
\end{definition}

\begin{theorem}{Every PID is a UFD}{every_pid_is_a_ufd}
    Every PID is a UFD.
\end{theorem}

\begin{proposition}{Irreducible Generates Maximal in PID}{irreducible_generates_maximal_in_pid}
    If $p\in R$ is an irreducible element of a PID $R$, then $\langle p \rangle$ is a maximal ideal.
\end{proposition}

\begin{theorem}{Ideals of Polynomial over PID}{ideals_of_polynomial_over_pid}
    Input:
    \begin{itemize}
        \item $R$ is a PID.
        \item $P = R[X]$.
        \item $I$ is a prime ideal of $P$.
    \end{itemize}
    The following statements hold.
    \begin{itemize}
        \item $I = \langle 0 \rangle$, or $I = \langle f \rangle$ with $f$ prime, or $I$ is maximal.
        \item If $I$ is maximal, then either
        \begin{itemize}
            \item $I = \langle f \rangle$ with $f$ prime, or
            \item $I = \langle p, g \rangle$ with $p\in R$ prime and $g\in P$ with image $g'\in (R/\langle p \rangle)[X]$ prime.
        \end{itemize}
    \end{itemize}
\end{theorem}

\section{Modules}

\begin{definition}{Annihilator}{annihilator}
    Let $M$ be a $R$-module where $R$ is a commutative ring.
    The annihilator of $m\in M$ is defined by
    \[ \operatorname{Ann}(m) = \Set*{x\in R}{xm = 0}. \]
    The annihilator of $M$ is defined by
    \[ \operatorname{Ann}(M) = \Set*{x\in R}{\forall m\in M, xm = 0}. \]
\end{definition}

$\operatorname{Ann}(m)$ and $\operatorname{Ann}(M)$ are ideals.

\begin{definition}{Faithful Module}{faithful_module}
    Let $M$ be a $R$-module where $R$ is a commutative ring.
    $M$ is faithful if $\operatorname{Ann}(M) = \langle 0 \rangle$.
\end{definition}

\begin{definition}{Cokernel \badge{UMP}}{cokernel}
    Let $R$ be a commutative ring.
    $M$ and $N$ are $R$-modules.
    $\alpha: M\rightarrow N$ is a homomorphism.
    The cokernel of $\alpha$ is defined by
    \[ \operatorname{Coker}(\alpha) = N/\operatorname{Im}(\alpha). \]
\end{definition}

\begin{proposition}{UMP of Cokernel}{ump_of_cokernel}
    Input:
    \begin{itemize}
        \item $R$ is a commutative ring.
        \item $M$ and $N$ are $R$-modules.
        \item $\alpha: M\rightarrow N$ is a homomorphism.
    \end{itemize}
    Then $\kappa: N \rightarrow \operatorname{Coker}(\alpha)$ satifies the following UMP:
    \begin{itemize}
        \item $\kappa \alpha = 0$, and
        \item for any $\beta: N\rightarrow P$ with $\beta \alpha = 0$, there is a unique homomorphism $\gamma: \operatorname{Coker}(\alpha) \rightarrow P$ such that $\gamma \kappa = \beta$, i.e. the following diagram commutes.
    \end{itemize}
    \begin{center}
        \begin{tikzcd}[execute at end picture={\draw[dashed,-latex,lightgray] (q1) -- (q2) -- (q3);}]
            M \arrow[r, "\alpha"] \arrow[rd] & N \arrow[d, "\forall \beta"'{name=q2}] \arrow[r, "\kappa"] & \operatorname{Coker}(\alpha) \arrow[dl, "\exists! \gamma"'{name=q3}, labels=below right] \\
            & |[alias=q1]| \forall P
        \end{tikzcd}
    \end{center}
\end{proposition}

\begin{theorem}{Free Submodule over PID}{free_submodule_over_pid}
    Input:
    \begin{itemize}
        \item $R$ is a PID.
        \item $E$ is a free $R$-module.
        \item $\Set*{e_\lambda}{\lambda\in\Lambda}$ is a free basis.
        \item $F$ is a submodule of $E$.
    \end{itemize}
    Then
    \begin{itemize}
        \item $F$ is free, and
        \item $F$ has a basis indexed by a subset of $\Lambda$.
    \end{itemize}
\end{theorem}

\begin{counterexample}{Module Product Scaled by Ideal}{module_product_scaled_by_ideal}
    Let $\Set*{M_\lambda}{\lambda\in\Lambda}$ be modules over $R$ and $I\subset R$ be an ideal thereof.
    Although
    \[ I\qty(\prod_\lambda M_\lambda) \subset \prod_\lambda \qty(I M_\lambda) \]
    always holds, the reverse
    \[ I\qty(\prod_\lambda M_\lambda) \supset \prod_\lambda \qty(I M_\lambda) \]
    may fail unless $I$ is finitely generated.
\end{counterexample}

\begin{counterexample}{Pairwise Disjoint $\nRightarrow$ Direct Sum}{pairwise_disjoint_nrightarrow_direct_sum}
    The vector spaces $\langle e_1 \rangle$, $\langle e_2 \rangle$, and $\langle e_1+e_2 \rangle$ are pairwise disjoint.
    However, $\langle e_1 \rangle + \langle e_2 \rangle + \langle e_1+e_2 \rangle$ is not a direct sum.
\end{counterexample}

\begin{counterexample}{$0\rightarrow A \rightarrow A\oplus B \rightarrow B \rightarrow 0$ $\nRightarrow$ Split}{direct_sum_not_split}
    It's true that
    \[ 0 \longrightarrow A \xlongrightarrow{\imath} A\oplus B \xlongrightarrow{\pi} B \longrightarrow 0 \]
    is split if $\imath$ and $\pi$ are the inclusion and projection, respectively.
    However, this is not true otherwise.
    The sequence
    \[ 0 \longrightarrow C_2 \xlongrightarrow{\imath \oplus 0} C_4 \oplus M \xlongrightarrow{\pi \oplus \operatorname{id}} C_2 \oplus M \longrightarrow 0 \]
    doesn't split for $\imath(a) = 2b$ and $\pi(b) = a$ where $a$ and $b$ are generators, and $M$ is any $\mathbb{Z}$-module.
    Now let
    \[ M = \bigoplus_{i=1}^\infty (C_2 \oplus C_4). \]
    Then $C_2 \oplus M \cong M \cong C_4 \oplus M$.
\end{counterexample}

\section*{Miscellany}

The following chain of inclusion holds.
\begin{center}
    rngs \\
    $\cup$ \\
    rings \\
    $\cup$  \\
    commutative rings  \\
    $\cup$  \\
    integral domains \\
    $\cup$  \\
    integrally closed domains \\
    $\cup$  \\
    GCD domains \\
    $\cup$  \\
    unique factorization domains \\
    $\cup$  \\
    principal ideal domains \\
    $\cup$  \\
    Euclidean domains \\
    $\cup$  \\
    fields \\
    $\cup$  \\
    algebraically closed fields \\
\end{center}

\subsection*{Tests}

\paragraph*{Idempotent}
\begin{itemize}
    \item $e_1$ and $e_2$ are complementary idempotent $\Leftrightarrow$ $e_1+e_2=1$ and $e_1 e_2 = 0$.
\end{itemize}

\paragraph*{Unit}
\begin{itemize}
    \item $u$ is a unit $\Leftrightarrow$ $u$ not in any maximal ideal.
\end{itemize}

\paragraph*{Maximal Ideal}
\begin{itemize}
    \item $M$ is maximal $\Leftrightarrow$ $R/M$ is a field.
    \item $M$ is maximal $\Leftrightarrow$ $\langle x \rangle + M = R$ for any $x\notin M$.
    \item In PID, $p$ is irreducible $\Leftrightarrow$ $p$ is prime $\Leftrightarrow$ $\langle p \rangle$ is maximal.
    \item In PID, $P$ is maximal $\Leftrightarrow$ $P$ is prime and nonzero.
\end{itemize}

\paragraph*{Prime Ideal}
\begin{itemize}
    \item $P$ is prime $\Leftrightarrow$ $R/P$ is a domain.
    \item $M$ is maximal $\Rightarrow$ $M$ is prime.
\end{itemize}

\paragraph*{Irreducible Element}
\begin{itemize}
    \item $p$ is prime $\Rightarrow$ $p$ is irreducible.
    \item In GCD domain, $p$ is irreducible $\Leftrightarrow$ $p$ is prime.
\end{itemize}

\paragraph*{Prime Element}
\begin{itemize}
    \item In GCD domain, $p$ is irreducible $\Leftrightarrow$ $p$ is prime.
    \item In PID, $p$ is prime $\Leftrightarrow$ $\langle p \rangle$ is prime.
\end{itemize}

\paragraph*{Local Ring}
\begin{itemize}
    \item $A$ is local $\Leftrightarrow$ nonunits of $A$ consists an ideal.
\end{itemize}

\subsection*{Application of UMP}

\begin{example}{How Do We Use UMP?}{how_do_we_use_ump}
    We may use UMP to prove, given $N\subset L\subset M$,
    \[ (M/N)/(L/N) = M/L. \]
    We use the following diagram.
    \begin{center}
        \begin{tikzcd}
            M \arrow[r] \arrow[rrd] & M/N \arrow[r] \arrow[rd] & (M/N)/(L/N) \arrow[d] \\
            & & X
        \end{tikzcd}
    \end{center}
    We prove that $(M/N)/(L/N)$ satisfy the following universal property (which is exactly that of $M/L$):
    \begin{itemize}
        \item there exists a map $M \rightarrow (M/N)/(L/N)$ such that $L \rightarrow 0$, and
        \item for any $M\rightarrow X$ such that $L\rightarrow 0$, there exists a unique map $(M/N)/(L/N) \rightarrow X$ such that the diagram commutes.
    \end{itemize}
    The first one is easy: $L \rightarrow L/N$ under $M\rightarrow M/N$, and $L/N \rightarrow 0$ under $M/N \rightarrow (M/N)/(L/N)$.
    \par
    The second one is also easy.
    If $L\rightarrow 0$ under $M\rightarrow X$, then $N\rightarrow 0$ under $M\rightarrow X$, and therefore there exists a unique map $M/N\rightarrow X$ such that left triangle commutes.
    $M/N\rightarrow X$ sends $L/N$ to $0$ since $L\rightarrow L/N$ under $M\rightarrow M/N$.
    Therefore, there exists a unique $(M/N)/(L/N)\rightarrow X$ such that the right triangle commutes.
    \par
    It remains to prove that there exists a unique $(M/N)/(L/N)\rightarrow X$ such that the whole triangle commutes.
    The existence is already shown.
    If the map is not unique, then two different such maps composes to two different $M/N \rightarrow X$ (since the same $M/N\rightarrow X$ corresponds to identical maps $(M/N)/(L/N) \rightarrow X$), and therefore composes to two different $M\rightarrow X$ (once again).
\end{example}

\begin{example}{$(X_0,X_1,\cdots)$ is not Finitely Generated}{x_0_x_1_is_not_finitely_generated}
    Let $R$ be a commutative ring.
    Then $I=(X_0,X_1,\cdots)\subset R[X_0,X_1,\cdots]$ is not finitely generated.
    Otherwise let $I=(f_0,\cdots,f_n)$.
    There is a maximal $\ell-1$ in the variables of $f_0,\cdots,f_n$.
    Let $Q=R[X_0,\cdots,X_{\ell-1},X_{\ell+1},\cdots]$.
    Then $Q[X_n] = R[X_0,X_1,\cdots]$.
    Then we prove
    \[ X_\ell \notin (f_0,\cdots,f_n) Q[X_n] \]
    where $(f_0,\cdots,f_n) \subset Q$.
    This is done by the following lemma.
\end{example}

\begin{lemma}{$I\neq(1)\Rightarrow X\notin IR[X]$}{i_neq_1_rightarrow_x_notin_irx}
    Let $R$ be a commutative ring, and $I$ be an ideal of $R$.
    If $I\neq (1)$ then $X\notin IR[X]$.
\end{lemma}
\begin{proof}
    To prove $X\notin IR[X]$, we prove that the image of $X$ under $R[X] \twoheadrightarrow R[X]/IR[X]$ is nonzero.
    This is true since there is an isomorphism
    \[ R[X]/IR[X] = (R/I)[X]. \]
    The image of $X$ in $(R/I)[X]$ is $X$ unless $I=(1)$.
\end{proof}

\subsection*{Application of Some Homological Algebra}

\begin{example}{$(M/N)/(L/N) = M/L$ with the Nine Lemma}{m_n_l_n_m_l_with_the_nine_lemma}
    We may apply the nine lemma to the following diagram.
    \begin{center}
        \begin{tikzcd}
              & 0 \arrow[d] & 0 \arrow[d] & 0 \arrow[d] & \\
            0 \arrow[r] & N \arrow[d] \arrow[r] & N \arrow[d] \arrow[r] & 0 \arrow[d] \arrow[r] & 0 \\
            0 \arrow[r] & L \arrow[d] \arrow[r] & M \arrow[d] \arrow[r] & M/L \arrow[d, "f"] \arrow[r] & 0 \\
            0 \arrow[r] & L/N \arrow[d] \arrow[r, "g"] & M/N \arrow[d] \arrow[r] & Q \arrow[d] \arrow[r] & 0 \\
              & 0 & 0 & 0
        \end{tikzcd}
    \end{center}
    With the nine lemma we can prove
    \[ M/L \cong (M/N)/(L/N). \]
    The arrows above are all defined in the obvious way except for $f$ and $g$.
    It's clear that the first two columns and all of the three rows are exact.
    ({\color{red}Not clear for the third row. See below.})
    To apply the nine lemma, we have to show that the diagram commutes.
    \begin{itemize}
        \item The upper-left square commutes: the arrows are trivial inclusions.
        \item The upper-right commutes: just to prove that if $x\in M$ is in $N$, then $x$ is zero in $M/L$, which is clear since $N\subset L$.
        \item The lower-left commutes: we exploit the freedom to define $g$ and defined it in such a way that the square commutes, i.e. for $[\ell] \in L/N$, take one preimage $\ell\in L$ and go along the $L\to M\to M/N$ path to get the image.
        This map is well-defined (we need this to see the square actually commutes: for now it only commutes for those elements $\ell\in L$ being representative elements we just chose), i.e. independent of the representative element $\ell$, since if we take $\ell'$ such that $\delta = \ell' - \ell\in N$, we can lift $\delta$ to the $N$ before $L$ and go along $N\to N\to M \to M/N$ and get $0$.
        {\color{red}We have to apply the five lemma to show $g$ is injective.}
        \item The lower-right commutes: we exploit the freedom to define $f$ in such a way that the square commutes.
        We lift $[m] \in M/L$ to a preimage $m\in M$.
        Then go along the $M\to M/N\to Q$ path to get the image.
        This map is well-defined (again we need this to show that the square actually commutes) by the same argument as above.
    \end{itemize}
\end{example}
In hindsight, we don't actually require the diagram to commute in every square.
We can leave some arrows undefined, and demand commutativity for only one square, and exactness for all but one line.
Then we can use the following propositions to complete the squares by defining the undefined arrows in such a way that the square commutes.

\begin{proposition}{Completing Square}{completing_square}
    \paragraph*{Completion to the right}%
    Given the following commutative diagram with exact rows,
    \begin{center}
        \begin{center}
            \begin{tikzcd}
                N' \arrow[d, "f"] \arrow[r] & N \arrow[d, "g"] \arrow[r] & N'' \arrow[r] & 0 \\
                M' \arrow[r] & M  \arrow[r] & M'' \arrow[r] & 0
            \end{tikzcd}
        \end{center}
    \end{center}
    there exists a unique $h\colon N''\to M''$ such that the following augmented diagram commutes.
    \begin{center}
        \begin{center}
            \begin{tikzcd}
                N' \arrow[d, "f"] \arrow[r] & N \arrow[d, "g"] \arrow[r] & N'' \arrow[d, "h"]\arrow[r] & 0 \\
                M' \arrow[r] & M  \arrow[r] & M'' \arrow[r] & 0
            \end{tikzcd}
        \end{center}
    \end{center}
    Moreover, $h$ is an isomorphism if $f$ and $g$ are isomorphisms.
    \paragraph*{Completion to the left}%
    Given the following commutative diagram with exact rows,
    \begin{center}
        \begin{center}
            \begin{tikzcd}
                0 \arrow[r] & N' \arrow[r] & N \arrow[d, "g"] \arrow[r] & N'' \arrow[d, "h"] \\
                0 \arrow[r] & M' \arrow[r] & M  \arrow[r] & M''
            \end{tikzcd}
        \end{center}
    \end{center}
    there exists a unique $f\colon N'\to M'$ such that the following augmented diagram commutes.
    \begin{center}
        \begin{center}
            \begin{tikzcd}
                0 \arrow[r] & N' \arrow[d, "f"] \arrow[r] & N \arrow[d, "g"] \arrow[r] & N'' \arrow[d, "h"] \\
                0 \arrow[r] & M' \arrow[r] & M  \arrow[r] & M'' 
            \end{tikzcd}
        \end{center}
    \end{center}
    Moreover, $f$ is an isomorphism if $f$ and $g$ are isomorphisms.
\end{proposition}
\begin{proof}
    This is just proposition 2.70 and 2.71 of \textit{An Introduction to Homological Algebra} by J. Rotman.
\end{proof}

% \bibliographystyle{plain}
% \bibliography{main}

\end{document}
