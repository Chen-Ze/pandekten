\documentclass{article}

\usepackage{pandekten}

\title{Dedekind Domain}
\author{Ch\=an Taku}

\begin{document}

\maketitle

This note accounts for
\begin{itemize}
    \item L. Pan's lecture on Week 4.
    \item L. Pan's lecture on Week 5.
\end{itemize}

\section{Dedekind Domain}

\begin{definition}{Dedekind Domain}{dedekind_domain}
    $A$ is a Dedekind domain if it satisfies all of the following conditions.
    \begin{itemize}
        \item $A$ is a Noetherian integral domain.
        \item $A$ is integrally closed.
        \item Nonzero prime ideals of $A$ are maximal, i.e. $A$ has dimension $0$ or $1$.
    \end{itemize}
\end{definition}

\begin{definition}{Krull Dimension}{krull_dimension}
    Let $A$ be a commutative ring.
    $\dim A$ is the maximal integer $l$ of ascending chains
    \[ p_0 \subsetneq p_1 \subsetneq \cdots \subsetneq p_l \]
    for $p_n \in \operatorname{Spec} A$.
\end{definition}

\begin{example}{Dimensions}{dimensions}
    $\dim A = 0$ is equivalent to that prime ideals are maximal.
    $\dim A = 1$ and $A$ is a domain implies that nonzero prime ideals are maximal.
    $\dim \mathbb{C}[X_1,\cdots,X_n] = n$.
    $\dim \mathbb{Z} = 1$.
\end{example}

\begin{lemma}{Dedekind has DVR Localization}{dedekind_has_dvr_localization}
    $A$ is a Noetherian integral domain.
    Then $A$ is Dedekind is equivalent to that $A_p$ is a DVR for all $p\in \operatorname{A}$.
\end{lemma}

\begin{example}{Dedekind Domains}{dedekind_domains}
    PID, $\mathbb{Z}$, $k[T]$ are all Dedekind domains.
    $\mathbb{Z}[\sqrt{-5}]$ is a Dedekind domain but not PID.
\end{example}

\subsection{Invertible Modules}

\begin{definition}{Fraction Ideal}{fraction_ideal}
    If $A$ is a integral domain, and $K = \operatorname{Frac}(A)$.
    $M\subset K$ is a fraction ideal if $M$ is a finitely generated $A$-submodule of $K$.
\end{definition}

\begin{example}{Fraction Ideal}{fraction_ideal}
    For any $x_1,\cdots,x_n\in K$, $Ax_1 + \cdots + Ax_n$ is a fraction ideal.
\end{example}

In the following we define $M'$ by
\[ M' = \Set*{y\in K}{xy\in A,\forall x\in M}. \]
$M'$ is a fraction ideal.
$M\cdot M' = \sum x_i y_i$ where $x_i\in M$ and $y_i\in M'$ is again a fraction ideal.

\begin{definition}{Invertible Ideal}{invertible_ideal}
    $M$ is invertible if $M\cdot M' = A$.
\end{definition}

\begin{example}{Invertible Ideals}{invertible_ideals}
    $xA$ is invertible.
\end{example}

In a DVR all the fraction ideal is given by the form $\pi^n A$ for $n\in \mathbb{Z}$.
Therefore, in a DVR, any nonzero fraction ideal is invertible.

\begin{proposition}{Fraction Ideal of Dedekind is Invertible}{fraction_ideal_of_dedekind_is_invertible}
    If $A$ is a Dedekind domain.
    $Q\neq 0$ is a fraction ideal.
    Then $Q$ is invertible.
\end{proposition}

\begin{lemma}{Localization Commutes with Dual}{localization_commutes_with_dual}
    If $p$ is a prime ideal of $A$, and if $A$ is Noetherian, then
    \[ (Q')_p = (Q_p)'. \]
\end{lemma}

\begin{lemma}{Localization Determines Fraction Ideal}{localization_determines_fraction_ideal}
    If $M$ is a fraction ideal, and $M_p = A_p$ for all prime ideal $p$, then $M=A$.
\end{lemma}

Non-zero fraction ideals of a Dedekind domain $A$ form an abelian group.

\begin{proposition}{Closed Sets are Finite in Dedekind Spectrum}{closed_sets_are_finite_in_dedekind_spectrum}
    If $x\neq 0\in A$ where $A$ is a Dedekind domain, then only finitely many prime ideals contain $x$.
\end{proposition}

\begin{corollary}{Local Evaluation of Dedekind is Almost Zero}{local_evaluation_of_dedekind_is_almost_zero}
    $K = \operatorname{Frac}(A)$, where $A$ is a Dedekind domain.
    Given $p\in \operatorname{Spec} A$ which is nonzero.
    $A_p$ is a DVR.
    Let $v_p: K^\times \rightarrow \mathbb{Z}$ be the evaluation map at $p$.
    For almost all $p$ (all but finitely many), $v_p(x) = 0$ for each $K^\times$.
\end{corollary}

\begin{corollary}{Local Evaluation of Dedekind is Almost Zero}{local_evaluation_of_dedekind_is_almost_zero}
    $v_p(Q) = 0$ for almost all prime ideals $p\neq 0$ of $A$.
\end{corollary}

\begin{proposition}{Unique Factorization}{unique_factorization}
    Any nonzero fraction ideal of $A$, a Dedekind domain, can be written uniquely as
    \[ \prod_{p\in \operatorname{Spec} A} p^{n_p}, \]
    where $n_p = 0$ for almost all $p$.
\end{proposition}

Let $J$ denote the group of all non-zero fraction ideals, and $P$ is the subgroup of $J$ of principal fraction ideals.
$A$ is a PID is equivalent to $P=J$.
$C = J/P$ is the ideal class group of $A$.
$C$ measures the failure of $A$ being a PID.
\par
The ideal $J$ has the form of a direct sum of abelian groups
\[ J = \bigoplus_{p\neq 0 \in \operatorname{Spec} A} \mathbb{Z}. \]
\par
Class groups are isomorphism classes of non-zero fractional ideals.

\begin{proposition}{Fractional Ideals are Isomorphic by Scaling}{fractional_ideals_are_isomorphic_by_scaling}
    Input:
    \begin{itemize}
        \item $R$ is a Dedekind domain.
        \item $A$ and $B$ are two fractional ideals.
    \end{itemize}
    Then the following two statements are equivalent.
    \begin{itemize}
        \item $A\cong B$ as an $R$-module.
        \item $A = xB$, for some $x\in \operatorname{Frac}(R)^\times$.
    \end{itemize}
\end{proposition}

\begin{example}{Class Group}{class_group}
    For $A = \mathbb{Z}[\sqrt{-5}]$, $C = \mathbb{Z}/2\mathbb{Z}$.
\end{example}

\begin{example}{Number Field}{number_field}
    Let $\mathbb{Q}\subset K$ be a finite algebraic extension.
    $\mathcal{O}_K$ is the integral closure of $\mathbb{Z}$ in $K$.
    We claim that $C_K$, the class group of $K$, has $\abs{C_K} < \infty$.
    $\abs{C_K}$ is called the class number of $K$.
\end{example}

\subsection{Classification of Modules}

\begin{theorem}{Finitely Generated Modules over a Dedekind Domain}{finitely_generated_modules_over_a_dedekind_domain}
    If $M$ is a finitely generated $A$-modules over a Dedekind domain $A$, then
    \[ M = Q_1 \oplus \cdots \oplus Q_m \oplus (A/P_1^{r_1}) \oplus \cdots \oplus (A/P_n^{r_n}), \]
    where $Q_i$ are fractional ideals of $A$ and $p_i$ are prime ideals of $A$.
\end{theorem}

\section{Extension}

In the following we frequently encounter the following objects.
\begin{center}
    \begin{tikzcd}
          & L \text{ (finite and separable)} \\
        B \text{ (integral closure)} \arrow[ur] & \\
          & K = \operatorname{Frac}(A) \arrow[uu] \\
        A \text{ (Dedekind)} \arrow[uu] \arrow[ur] &
    \end{tikzcd}
\end{center}

\begin{theorem}{Integral Closure in Finite Separable over\\ $\operatorname{Frac}(\text{Dedekind})$ \\is Finitely Generated over Dedekind}{integral_closure_in_separable_separable_over_frac_dedekind_is_finitely_generated_over_dedekind}
    Input:
    \begin{itemize}
        \item $A$ is a Dedekind domain.
        \item $K = \operatorname{Frac}(A)$.
        \item $L/K$ is a finite field extension and is separable.
        \item $B$ is the integral closure of $A$ in $L$.
    \end{itemize}
    Then $B$ is a finitely generated $A$-module.
    In particular, $B$ is Noetherian.
\end{theorem}

\begin{theorem}{Integral Closure in Finite Separable over\\ $\operatorname{Frac}(\text{Dedekind})$ \\is Dedekind}{integral_closure_in_separable_separable_over_frac_dedekind_is_dedekind}
    Input:
    \begin{itemize}
        \item $A$ is a Dedekind domain.
        \item $K = \operatorname{Frac}(A)$.
        \item $L/K$ is a finite field extension and is separable.
        \item $B$ is the integral closure of $A$ in $L$.
    \end{itemize}
    Then $B$ is a Dedekind domain.
\end{theorem}

\begin{definition}{Trace Map}{trace_map}
    If $L/K$ is a finite field extension.
    For all $x\in L$, define a linear transformation
    \[ \varphi_x: L\rightarrow L \]
    on vector spaces over $K$ by $\varphi_x(a) = ax$.
    Then
    \[ \tr_{L/K}(x) = \tr(\varphi_x) \in K. \]
\end{definition}

\begin{example}{Trace Map}{trace_map}
    \begin{itemize}
        \item $\tr_{L/K}({1}) = [L:K]$.
        \item $\tr_{\mathbb{C}/\mathbb{R}}(a+bi) = 2a$.
        \item If $L = K(\alpha)$ is a finite extension and $f(T)$ is a monic irreducible polynomial of $\alpha$ over $K$.
        Then $\tr_{L/K} = -a_{n-1}$, i.e. the sum of all roots of $f(T)$.
        \item If $L = \mathbb{F}_p(X^{1/p})$ then the irreducible polynomial of $X^{1/p}$ is $T^p - X^i$ for any $i$.
        $\tr (X^{i/p}) = 0$ and $\tr(1) = 0$.
        Therefore, $\tr = 0$.
    \end{itemize}
\end{example}

If $L/K$ is Galois then
\[ \tr_{L/K}(\alpha) = \sum_{\sigma\in\operatorname{Gal}(L/K)} \sigma(\alpha). \]

\begin{proposition}{Trace Criterion}{trace_criterion}
    Let $L/K$ be a finite field extension.
    The following statements are equivalent.
    \begin{itemize}
        \item A finite extension $L/K$ is separable.
        \item $\tr_{L/K}\neq 0$.
        \item $\tr: L\times L \rightarrow K$ as a bilinear form is non-degenerate.
    \end{itemize}
\end{proposition}

\begin{corollary}{Tower Law of Separability}{tower_law_of_separability}
    If $E/L$ is a separable field extension and $L/K$ is a separable field extension, then $E/K$ is separable.
\end{corollary}

\begin{corollary}{Extension Factors through Separable}{extension_factors_through_separable}
    If $L/K$ is a separable field extension, then $L_{\mathrm{S}}$ is a subfield of $L$ such that
    \begin{itemize}
        \item $L_{\mathrm{S}}/K$ is separable, and
        \item $L/L_{\mathrm{S}}$ is purely inseparable, i.e. all $x\in L \setminus L_{\mathrm{S}}$ are inseparable over $L_{\mathrm{S}}$,
    \end{itemize}
    where $L/L_{\mathrm{S}}$ is defined by
    \[ L_{\mathrm{S}} = \Set*{x\in L}{x\text{ is separable over } K}. \]
\end{corollary}
For each $\alpha\in L\setminus L_{\mathrm{S}}$, there is $k$ such that $\alpha^{p^k}\in L_{\mathrm{S}}$.
There is a Chain of extension
\[ L_{\mathrm{S}} \hookrightarrow L_1 \hookrightarrow \cdots \hookrightarrow L_{m-1} \hookrightarrow L, \]
where each extension has degree $p$ and has the form
\[ L_{i+1} = L_i(\beta^{1/p}) \]
for some $\beta \in L_i$.


{TODO: \color{red} Artin: Factorization, Gauss's lemma, Quadratic field, Galois Theory}
\par

\subsection{Finite Extension}

Let $K = \operatorname{Frac} A$ where $A\neq K$ is a Dedekind domain.
Let $L/K$ be a finite extension.
If $L/K$ is separable, and $B$ is the integral closure of $A$ in $L$, then
\begin{itemize}
    \item $B$ is a f.g. $A$-module, and
    \item $B$ is Dedekind.
\end{itemize}

\par

If $P\in \operatorname{Spec} A$ and $P\neq 0$, then
\[ PB = \prod_{\substack{p\cap A = P \\ p\in \operatorname{Spec}(B)}} p^{e_p}. \]
$e_p$ is the ramification index of $p$ in $L/K$.
$f_p$ is the residue index of $p$ in $L/K$, i.e. $f_p = [k(p):k(P)]$, where $k(p) = B/p$, and $k(P) = A/P$.
For $p\cap A = P$ we call $p\mid P$.

\begin{proposition}{Degree of Extension}{degree_of_extension}
    If $P\in \operatorname{Spec} A$ and $p\neq 0$,
    \[ [L:K] = \sum_{p\mid P} e_p f_p. \]
    If $[L:K]$ is separable, then $e_p=1$ for all but a finite number of $p\in\operatorname{Spec} B$.
\end{proposition}

\begin{proof}
    Note that
    \[ B/PB \cong \prod_{p\mid P} B/p^{e_p} \]
    and that
    \[ \dim_{k(P)} B/PB = [L:K] \]
    while
    \[ \dim_{k(P)} B/p^{e_p} = e_p f_p. \qedhere \]
\end{proof}

\begin{definition}{Total Ramification}{total_ramification}
    Let $P\in \operatorname{Spec} A$.
    $P$ is totally ramified in $L/K$ if there is only one $p\operatorname{Spec} B$ above $P$.
    In such case, $f_p = 1$, and
    \[ e_p = [L:K]. \]
\end{definition}

\begin{definition}{Unramification}{unramification}
    $P$ is unramified in $L/K$ if
    \begin{itemize}
        \item $e_p = 1$ for all $p\mid P$, and
        \item $k(p)/k(P)$ is separable for all $p\mid P$.
    \end{itemize}
\end{definition}

\begin{example}{Ramification of $\mathbb{Z}[i]$}{ramification_of_z_i}
    For $A = \mathbb{Z}$, $K = \mathbb{Q}$, $L = \mathbb{Q}[i]$, and $B = \mathbb{Z}[i]$, we find
    \[ (2) = (1+i)^2 \]
    and therefore $2$ is totally ramified.
    $p\neq 2$ is unramified.
    In fact,
    \[ \mathbb{Z}[i]/(p) = \mathbb{F}_p[x]/(x^2+1). \]
    If $p$ is odd, then $x^2+1$ is not a square in $\mathbb{F}_p[x]$.
    \begin{itemize}
        \item If $x^2+1$ is reducible, then there are two distinct prime ideals of $\mathbb{Z}_i$ containing $(p)$.
        Therefore, $e_p = f_p = 1$.
        Such case occurs if $p = 4n+1$.
        \item If $x^2+1$ is irreducible, then $\mathbb{F}_p[x]/(x^2+1)$ is a field.
        $(p)$ is a prime ideal of $\mathbb{Z}_i$.
        Therefore, $e_p = 1$.
        Such case occurs if $p = 4n+3$.
    \end{itemize}
\end{example}

\subsection{Simple Extension}

In the local case, let $(A,M)$ be a local ring, $K = \operatorname{Frac}(A)$, and $k = A/M$.
We start with a monic $f(x)\in A[x]$ and let
\[ B_f = A[X]/(f). \]
This is a free $A$-module of rank $n$.
Now
\[ B_f/MB_f = A[x]/(M,f) = k[x]/(\overline{f}) \]
where
\[ \overline{f} = \prod_{i=1}^g \varphi_i^{e_i} \]
and $\varphi_i$ are irreducible.
Therefore,
\[ B_f/MB_f = k[x]/(\overline{f}) = \prod_{i=1}^g k[x]/(\varphi_i)^{e_i} \]
For each $\varphi_i$, we have a polynomial $\tilde{\varphi}_i \in A[x]$ such that $\tilde{\varphi} - \varphi_i \in M$.

\begin{proposition}{Maximal Ideals of Quotient}{maximal_ideals_of_quotient}
    \[ (\tilde{\varphi}_i,m) \]
    for all $i$ are the maximal ideals of $B_f$.
\end{proposition}

\paragraph*{Unramified Case}
Assume that $\tilde{f}$ is irreducible, then $k[X]/(\overline{f})$ is a field.

\begin{proposition}{Unramified Case}{unramified_case}
    $B_f$ is a DVR with maximal ideal generated by $\pi$.
\end{proposition}

\begin{corollary}{Unramified Case over DVR}{unramified_case_over_dvr}
    If $A$ is DVR and $\overline{f}$ is irreducible, then $f$ is irreducible in $K[x]$, $B_f$ is the integral closure of $A$ in $L=K[X]/(f)$, and
    \[ e_{\pi B_f} = 1, f_{\pi B_f} = \deg f. \]
    In particular, $L/K$ is unramified at $\pi$.
\end{corollary}

\paragraph*{Totally Ramified Case}
Let $A$ be a DVR and $M = (\pi)$.
\begin{theorem}{Eisenstein}{eisenstein}
    Let
    \[ f(x) = x^n + a_{n-1}x^{n-1} + \cdots + a_0 \]
    where $a_i\in A$.
    $f$ is irreducible in $K[X]$ if $\pi\mid a_i$ for all $i$ but $\pi^2 \nmid a_0$.
\end{theorem}

\begin{proposition}{Totally Ramified Case}{totally_ramified_case}
    $B_f$ is a DVR with maximal ideal generated by the image $\alpha$ of $x$ in $B_f$.
\end{proposition}

The evaluations
\begin{align*}
    B_p \subset L && v_p:L^\times \rightarrow \mathbb{Z},\\
    A_P \subset K && v_P:K^\times \rightarrow \mathbb{Z}
\end{align*}
differs by
\begin{align*}
    \pi B_p &= PB_p = p^{e_p} B_p \Rightarrow v_p(\pi) = e_p,\\
    PA_P &= (\pi) \Rightarrow v_p(\pi) = 1.
\end{align*}

\begin{proposition}{Restriction of Valuation}{restriction_of_valuation}
    \[ \eval{v_p}_{K^\times} = e_p v_p. \]
\end{proposition}

\subsection{Galois Extension}

If $L/K$ is Galois, then
\[ \sigma: B \xrightarrow{\sim} B \]
is an isomorphism.
\[ p\in \operatorname{Spec} B \Rightarrow \sigma(P) \in \operatorname{Spec} B. \]
Therefore, the Galois group acts on $\operatorname{Spec} B$.
Since $P\in \operatorname{Spec} A$ is preserved, it acts on primes ideals above $P$, i.e.
\[ \sigma(p) \cap A = P \]
if $p\cap A = P$.

\begin{proposition}{Galois Group Acts Transitively on Prime Ideals above}{galois_group_acts_transitively_on_prime_ideals_above}
    $\operatorname{Gal}(L/K)$ acts transitively on $\Set*{p}{P}$.
\end{proposition}

\begin{corollary}{Ramification and Residue Index Depends Only on Intersection}{ramification_and_residue_index_depends_only_on_intersection}
    Let $L/K$ be a Galois extension.
    If there are $p,p'$ above $P$ then
    \[ e_p = e_{p'},\quad f_p = f_{p'}. \]
\end{corollary}

In such case, we denote
\[ e_P = e_p,\quad f_P = f_p, \]
for $p$ above $P$.
We also denote by $g_P$ the number of ideals above $P$.
Therefore,
\[ [L:K] = e_P f_P g_P. \]

\begin{definition}{Local Galois Group}{local_galois_group}
    The decomposition group of $p$ in $L/K$ is a subgroup of $\operatorname{Gal}(L/K)$ defined by
    \[ D_p(L/K) = \Set*{\sigma\in\operatorname{Gal}(L/K)}{\sigma(p) = p}. \]
\end{definition}

Therefore,
\[ \abs{G(L/K)} = g_p \abs{D_p(L/K)}, \]
and
\[ \abs{D_p(L/K)} = e_p f_p. \]
If $p$ and $p'$ are both above $P$ then $D_p$ and $D_{p'}$ are conjugate.
\par
If $\sigma\in\operatorname{Gal}(L/K)$ and
\[ \overline{L} = k(p), \quad \overline{K} = k(P), \]
then $\sigma$ induces
\[ \epsilon: D_p \rightarrow \operatorname{Aut}(\overline{L}/\overline{K}). \]

\begin{definition}{Inertia Subgroup}{inertia_subgroup}
    The inertia subgroup is defined by
    \[ I_p(L/K) = \ker \epsilon \]
    which is a normal subgroup of $D_p$.
\end{definition}

\begin{proposition}{Residue is Normal Extension}{residue_is_normal_extension}
    $\overline{L}/\overline{K}$ is normal, i.e. $\overline{L}/\overline{K}$ is a splitting field.
\end{proposition}

\begin{proposition}{Induces Map is Surjective}{induces_map_is_surjective}
    $\epsilon$ is surjective.
\end{proposition}

\paragraph*{Fact}
For
\[ \overline{L}/\overline{L}^{\mathrm{S}}/\overline{K} \]
we have
\[ \operatorname{Aut}(\overline{L}/\overline{K}) = \operatorname{Aut}(\overline{L}^{\mathrm{S}}/K) = \operatorname{Gal}(\overline{L}^{\mathrm{S}}/K). \]

\begin{corollary}{Residue Degree for Separable}{residue_degree_for_separable}
    If $\overline{L}/\overline{K}$ is separable, then
    \[ \operatorname{Gal}(\overline{L}/\overline{K}) = \operatorname{Aut}(\overline{L}/\overline{K}), \]
    and
    \[ f_P = \abs{\operatorname{Gal}(\overline{L}/\overline{K})}, \]
    and $I_P = e_P$.
\end{corollary}

If $\overline{K}$ is finite (thus $\overline{L}/\overline{K}$ is separable), then
\[ \overline{K} \cong \mathbb{F}_q,\quad q = p^n,\quad p = \operatorname{char} \overline{K}, \]
then any Galois extension $E/\mathbb{F}_q$ has cylic $\operatorname{Gal}(E/\mathbb{F}_q)$ generated by the Frobenius
\[ x \mapsto x^q. \]
If $e_P = 1$ i.e. $P$ is unramified in $L/K$, then
\[ \operatorname{Gal}(L/K) \supset D_P \cong \operatorname{Gal}(\overline{L}/\overline{K}) = \langle s_q \rangle, \]
where $s_q(a) - a^q \in \operatorname{p}$.

\par

Galois extension of $\mathbb{Q}(\xi_n)$.

% \bibliographystyle{plain}
% \bibliography{main}

\end{document}
