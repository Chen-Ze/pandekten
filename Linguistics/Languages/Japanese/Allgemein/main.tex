\documentclass{article}

\usepackage{xeCJK}
\setCJKmainfont{Hiragino Mincho ProN}
%\setCJKmainfont{Songti SC}

\usepackage{ruby}
\renewcommand\rubysep{0em}

\usepackage{pandekten}

\title{Allgemein}
\author{Ch\=an Taku}

\begin{document}

\maketitle

\section{Miscellany}

\subsection{Verb Pairs}

\begin{definition}{Verb Pair}{verb_pair}
    A verb pair is a pair of verbs where one is 自動詞 and another is 他動詞.
\end{definition}

\begin{definition}{Regular す-Pair}{regular_su_pair}
    A verb pair is called a regular す-pair if its 他動詞 ends with す.
\end{definition}

\begin{definition}{Regular ぁる-Pair}{regular_aru_pair}
    A verb pair is called a regular ぁる-pair if its 自動詞 ends with ぁる.
\end{definition}

\begin{definition}{Regular せる-Pair}{regular_seru_pair}
    A verb pair is called a regular せる-pair if its 他動詞 ends with せる.
\end{definition}

\begin{definition}{Regular む-Pair}{regular_mu_pair}
    A verb pair is called a regular む-pair if its 自動詞 ends with む while 他動詞 ends with める.
\end{definition}

\begin{definition}{Regular ぶ-Pair}{regular_bu_pair}
    A verb pair is called a regular ぶ-pair if its 自動詞 ends with ぶ while 他動詞 ends with べる.
\end{definition}

\begin{definition}{Regular つ-Pair}{regular_tsu_pair}
    A verb pair is called a regular つ-pair if its 自動詞 ends with つ while 他動詞 ends with てる.
\end{definition}

% \bibliographystyle{plain}
% \bibliography{main}

\end{document}
