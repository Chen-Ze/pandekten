\documentclass{article}

\usepackage{xeCJK}
\setCJKmainfont{Hiragino Mincho ProN}
%\setCJKmainfont{Songti SC}

\usepackage{ruby}
\renewcommand\rubysep{0em}

\usepackage{pandekten}

\title{Allgemein}
\author{Ch\=an Taku}

\begin{document}

\maketitle

\section{Miscellany}

\subsection{Verb Pairs}

\begin{definition}{Verb Pair}{verb_pair}
    A verb pair is a pair of verbs where one is 自動詞 and another is 他動詞.
\end{definition}

\begin{definition}{Regular す-Pair}{regular_su_pair}
    A verb pair is called a regular す-pair if its 他動詞 ends with す.
\end{definition}

\begin{definition}{Regular ぁる-Pair}{regular_aru_pair}
    A verb pair is called a regular ぁる-pair if its 自動詞 ends with ぁる.
\end{definition}

\begin{definition}{Regular せる-Pair}{regular_seru_pair}
    A verb pair is called a regular せる-pair if its 他動詞 ends with せる.
\end{definition}

\begin{definition}{Regular む-Pair}{regular_mu_pair}
    A verb pair is called a regular む-pair if its 自動詞 ends with む while 他動詞 ends with める.
\end{definition}

\begin{definition}{Regular ぶ-Pair}{regular_bu_pair}
    A verb pair is called a regular ぶ-pair if its 自動詞 ends with ぶ while 他動詞 ends with べる.
\end{definition}

\begin{definition}{Regular つ-Pair}{regular_tsu_pair}
    A verb pair is called a regular つ-pair if its 自動詞 ends with つ while 他動詞 ends with てる.
\end{definition}

\subsection{音読 and Cantonese}

\paragraph*{k vs. t vs. p}%
In many cases, Cantonese \textit{k} corresponds to Japanese く or き, Cantonese \textit{t} corresponds to Japanese つ or ち, and Cantonese \textit{p} corresponds to Japanese う.

\paragraph*{ai vs. aai}%
In many cases, Cantonese \textit{ai} corresponds to Japanese ぇい, while Cantonese \text{aai} corresponds to Japanese ぁい.

\paragraph*{au vs. aau}%
I didn't find a rule to tell them apart.

\paragraph*{ak vs. aak}%
In many cases, Cantonese \textit{ak} corresponds to Japanese ぉく, while Cantonese \text{aak} corresponds to Japanese ぁく.

\paragraph*{at vs. aat}%
In many cases, Cantonese \textit{at} corresponds to Japanese ぃつ or else, while Cantonese \text{aat} corresponds to Japanese ぁつ.

\paragraph*{ap vs. aap}%
I didn't find a rule to tell them apart.

\paragraph*{ok vs. uk}%
In many cases, Cantonese \textit{ok} corresponds to Japanese ぁく, while Cantonese \text{uk} corresponds to others.

\paragraph*{u, ng and p}%
In most cases, \textit{?u}, ng, and p endings corresponds to Japanese ぅ with two morae.

\subsection{Exceptions: 慣用音}

\begin{itemize}
    \item \ruby{告}{こく} (exists in middle Chinese but rare now).
    \item \ruby{法}{ほう} (matches middle Chinese but not Cantonese).
    \item \ruby{立}{りつ} (maybe back-formation of りっ in compounds).
\end{itemize}

% \bibliographystyle{plain}
% \bibliography{main}

\end{document}
