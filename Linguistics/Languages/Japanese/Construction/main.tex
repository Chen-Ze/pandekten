\documentclass{article}

\usepackage{xeCJK}
\setCJKmainfont{Hiragino Mincho ProN}
\setCJKmonofont{Hiragino Sans W3}


\usepackage{pandekten}

\title{Construction}
\author{Ch\=an Taku}

\begin{document}

\maketitle

\section{Partial Negation}

\begin{proposition}{必ずしも -ない}{kanarazushimo_nai}
    The construction
    \begin{center}
        \texttt{必ずしも + clause-ない}
    \end{center}
    where \texttt{clause-ない} ends with a plain or polite negative form of a verb, is equivalent to
    \begin{center}
        \textit{not always or necessarily } \texttt{clause}\textit{,}
    \end{center}
    where \texttt{clause} is \texttt{clause-ない} with the final verb replaced by its positive form.
\end{proposition}

\begin{proposition}{からといって -ない}{karatoitte_nai}
    The construction
    \begin{center}
        \texttt{clause1 + からといって + clause2-ない}
    \end{center}
    where \texttt{clause2-ない} ends with a plain or polite negative form of a verb, is equivalent to
    \begin{center}
        \textit{just because} \texttt{clause1}\textit{, it is not necessarily that} \texttt{clause2}\textit{,}
    \end{center}
    where \texttt{clause2} is \texttt{clause2-ない} with the final verb replaced by its positive form.
\end{proposition}

\subsection{Litotes}

\begin{example}{litotes}{litotes}
    The following example is from \href{https://www.wasabi-jpn.com/japanese-grammar/partial-negation-and-double-negative-in-japanese/}{Partial Negation and Double Negative in Japanese}.
    \begin{itemize}
        \item 日本人はどして英語が下手なの? 学校で勉強しないの? \\
        \textit{Why are japanese people so poor at English? Don't they study it at school?}
        \item[-] 学校で{\color{cyan}勉強しないんじゃない}. 勉強しても下手なんだよ.\\
        \textit{It's not that we didn't study at school. We studied but are still poor at it.}
        \item[-] でも, 全員が英語を{\color{cyan}話せない訳じゃない}よ.\\
        \textit{But, one should not think that no one in Japan can speak English.}
    \end{itemize}
\end{example}

% \bibliographystyle{plain}
% \bibliography{main}

\end{document}
