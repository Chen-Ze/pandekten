\documentclass{article}

\usepackage{xeCJK}
\setCJKmainfont{Hiragino Mincho ProN}

\usepackage{ruby}
\renewcommand\rubysep{0em}

\usepackage{pandekten}

\title{Syntax}
\author{Ch\=an Taku}

\begin{document}

\maketitle

\section{Clauses}

\subsection{Content Clause}

\begin{definition}{こと-Clause}{koto_clause}
    A こと-clause is a \href{https://en.wikipedia.org/wiki/Content_clause}{content clause}, possibly a \href{https://en.wikipedia.org/wiki/Non-finite_clause}{non-finite clause}, of the form
    \begin{center}
        \texttt{clause + こと,}
    \end{center}
    where \texttt{clause} ends with the 終止形 of plain form.
\end{definition}

\begin{definition}{の-Clause}{no_clause}
    A の-clause is a \href{https://en.wikipedia.org/wiki/Content_clause}{content clause}, possibly a \href{https://en.wikipedia.org/wiki/Non-finite_clause}{non-finite clause}, of the form
    \begin{center}
        \texttt{clause + の,}
    \end{center}
    where \texttt{clause} ends with the 終止形 of plain form.
\end{definition}

\begin{proposition}{こと-Only Cases}{koto_only_cases}
    の-clause should not be used in the following case.
    \begin{itemize}
        \item Used as argument of a predicate that is \href{https://www.wasabi-jpn.com/japanese-grammar/nominalizers-koto-and-no/}{related to communication or internal thoughts}.
    \end{itemize}
\end{proposition}

\begin{example}{こと-Only Cases}{koto_only_cases}
    \begin{itemize}
        \item 先生に学校を休む{\color{cyan}こと}を伝える.
        \item 世界が平和である{\color{cyan}こと}を願います.
    \end{itemize}
\end{example}

\begin{counterexample}{のだ-Clauses}{noda_clauses}
    の-clause before だ or です implies that it is an \href{https://japanese.stackexchange.com/questions/5398/what-is-the-meaning-of-%EF%BD%9E%E3%82%93%E3%81%A7%E3%81%99-%EF%BD%9E%E3%81%AE%E3%81%A0-etc}{explanation to some previous context}.
    \begin{itemize}
        \item 今日は出かけられない. 宿題がたくさんある{\color{cyan}の}です.
    \end{itemize}
\end{counterexample}

\begin{proposition}{の-Only Cases}{no_only_cases}
    こと-clause should not be used in the following case.
    \begin{itemize}
        \item Used as argument of a predicate that is \href{https://www.wasabi-jpn.com/japanese-grammar/nominalizers-koto-and-no/}{a perception verb}, e.g. 見る and 聞く.
        \item Used as argument of a predicate that is \href{https://www.wasabi-jpn.com/japanese-grammar/nominalizers-koto-and-no/}{\ruby{止}{と}める or \ruby{止}{や}める}.
        \item Used as argument of a predicate that is \href{https://www.wasabi-jpn.com/japanese-grammar/nominalizers-koto-and-no/}{an action which should be done while observing people's situations}, e.g. 手伝う and 待つ.
    \end{itemize}
\end{proposition}

\begin{example}{の-Only Cases}{no_only_cases}
    \begin{itemize}
        \item お父さんが話す{\color{cyan}の}を聞く.
        \item 子供が走る{\color{cyan}の}を見る.
        \item タバコを吸う{\color{cyan}の}を止める.
        \item 日本語を勉強する{\color{cyan}の}を手伝う.
    \end{itemize}
\end{example}

\subsection{Relative Clause}

\begin{counterexample}{Ambiguous Relative Clause}{ambiguous_relative_clause}
    See \href{https://en.wikipedia.org/wiki/Relative_clause#Japanese}{Relative Clause}.
    \begin{center}
        僕が記事を書いたレストラン
    \end{center}
    may refer to
    \begin{itemize}
        \item a restaurant about which I wrote an article; or
        \item a restaurant in which I wrote an article.
    \end{itemize}
\end{counterexample}

% \bibliographystyle{plain}
% \bibliography{main}

\end{document}
