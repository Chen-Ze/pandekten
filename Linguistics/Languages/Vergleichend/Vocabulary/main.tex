\documentclass{article}

\usepackage{xeCJK}
%\setCJKmainfont{Hiragino Mincho ProN}
\setCJKmainfont{Songti SC}

\usepackage{ruby}
\renewcommand\rubysep{0em}

\newcommand\coloredbox[3][black]{\fbox{\textcolor{#1}{\rule{#2}{#3}}}}

\usepackage{pandekten}

\title{Vocabulary}
\author{Ch\=an Taku}

\begin{document}

\maketitle

\section{Adjectives}

\subsection{Colors}

\subsubsection{Allgemein}

In the Japanese language, colors other than 赤い, 青い, 白い, and 黒い are named mainly after materials.
Some colors have only noun-form, i.e. are の-adjectives.

\begin{warning}
    The colors used in the following should be calibrated.
\end{warning}

\subsubsection{Vergleichend}

\begin{longtable}{m{2cm}m{2.5cm}m{5.5cm}}
    \toprule
    \multicolumn{3}{c}{\coloredbox[white]{1cm}{1cm}} \\
    \midrule
    English & white & \\
    German & weiß & cognate with English `white' \\
    Japanese & \ruby{白}{しろ}い \\
    \bottomrule
\end{longtable}

\begin{longtable}{m{2cm}m{2.5cm}m{5.5cm}}
    \toprule
    \multicolumn{3}{c}{\coloredbox[black]{1cm}{1cm}} \\
    \midrule
    English & black & \\
    German & schwarz & cognate with English `swarthy' \\
    Japanese & \ruby{黒}{くろ}い & cognate with \ruby{暗}{くら}い \\
    \bottomrule
\end{longtable}

\begin{longtable}{m{2cm}m{2.5cm}m{5.5cm}}
    \toprule
    \multicolumn{3}{c}{\coloredbox[red]{1cm}{1cm}} \\
    \midrule
    English & red & \\
    German & rot & cognate with English `red' \\
    Japanese & \ruby{赤}{あか}い & cognate with \ruby{明}{あか}るい \\
    \bottomrule
\end{longtable}

\begin{longtable}{m{2cm}m{2.5cm}m{5.5cm}}
    \toprule
    \multicolumn{3}{c}{\coloredbox[green]{1cm}{1cm}} \\
    \midrule
    English & green & \\
    German & grün & cognate with English `green' \\
    Japanese & \ruby{緑}{みどり}の & not sure if \href{https://gogen-yurai.jp/midori/}{related to みずみずしい} \\
    \bottomrule
\end{longtable}

\begin{longtable}{m{2cm}m{2.5cm}m{5.5cm}}
    \toprule
    \multicolumn{3}{c}{\coloredbox[blue]{1cm}{1cm}} \\
    \midrule
    English & blue & \\
    German & blau & cognate with English `blue' \\
    Japanese & \ruby{青}{あお}い &  \\
    \bottomrule
\end{longtable}

\begin{longtable}{m{2cm}m{2.5cm}m{5.5cm}}
    \toprule
    \multicolumn{3}{c}{\coloredbox[cyan]{1cm}{1cm}} \\
    \midrule
    English & cyan & \\
    German & türkis & turquoise, while cyan or zyan may also work \\
    Japanese & シアンの & from Dutch `cyaan' \\
    \bottomrule
\end{longtable}

\begin{longtable}{m{2cm}m{2.5cm}m{5.5cm}}
    \toprule
    \multicolumn{3}{c}{\coloredbox[magenta]{1cm}{1cm}} \\
    \midrule
    English & magenta & \\
    German & magentarot & \\
    Japanese & マゼンタの & from Italian `magenta' \\
    \bottomrule
\end{longtable}

\begin{longtable}{m{2cm}m{2.5cm}m{5.5cm}}
    \toprule
    \multicolumn{3}{c}{\coloredbox[yellow]{1cm}{1cm}} \\
    \midrule
    English & yellow & \\
    German & gelb & cognate with English `yellow' \\
    Japanese & \ruby{黄}{き}\ruby{色}{いろ}い & possibly \href{https://gogen-yurai.jp/ki-iro/}{related to \ruby{木}{き}} \\
    \bottomrule
\end{longtable}

\begin{longtable}{m{2cm}m{2.5cm}m{5.5cm}}
    \toprule
    \multicolumn{3}{c}{\coloredbox[orange]{1cm}{1cm}} \\
    \midrule
    English & orange & \\
    German & orange & from french `orange' \\
    Japanese & オレンジの & from English `orange' \\
     & オレンジ\ruby{色}{いろ}の & \\
     & \ruby{橙}{だいだい}\ruby{色}{いろ}の & possibly \href{https://en.wiktionary.org/wiki/%E6%A9%99#Japanese}{from \ruby{代}{だい}々} \\
    \bottomrule
\end{longtable}

\begin{longtable}{m{2cm}m{2.5cm}m{5.5cm}}
    \toprule
    \multicolumn{3}{c}{\coloredbox[brown]{1cm}{1cm}} \\
    \midrule
    English & brown & \\
    German & braun & cognate with English `brown' \\
    Japanese & \ruby{茶}{ちゃ}\ruby{色}{いろ}い & \\
     & \ruby{茶}{ちゃ}\ruby{色}{いろ}の & \\
    \bottomrule
\end{longtable}

\begin{longtable}{m{2cm}m{2.5cm}m{5.5cm}}
    \toprule
    \multicolumn{3}{c}{\coloredbox[violet]{1cm}{1cm}} \\
    \midrule
    English & violet & \\
    German & violett & \\
    Japanese & \ruby{菫}{すみれ}\ruby{色}{いろ}の & \href{https://gogen-yurai.jp/sumire}{from \ruby{墨}{すみ}\ruby{入}{い}れ}, where \ruby{墨}{すみ}\ is \href{https://gogen-yurai.jp/sumi/}{from \ruby{染}{そ}め} \\
    \bottomrule
\end{longtable}

\begin{longtable}{m{2cm}m{2.5cm}m{5.5cm}}
    \toprule
    \multicolumn{3}{c}{\coloredbox[purple]{1cm}{1cm}} \\
    \midrule
    English & purple & \\
    \sout{German} & & \\
    Japanese & \ruby{紫}{むらさき}\ruby{色}{いろ}の & possibly \href{https://en.wiktionary.org/wiki/%E7%B4%AB#Japanese}{from \ruby{群}{むら}, cognate with \ruby{村}{むら}, plus \ruby{咲}{さ}き} \\
    \bottomrule
\end{longtable}

\section{Idioms}

\begin{longtable}{m{2cm}m{8cm}}
    \toprule
    Latin & In v\=in\=o v\=erit\=as. \\
    German & Trunkner Mund verrät des Herzens Grund. \\
    Chinese & 酒后吐真言. \\
    Japanese & 酒は本心を表す. \\
    \bottomrule
\end{longtable}

\section{Miscellany}

For basic vocabularies, see the \href{https://en.wiktionary.org/wiki/Appendix:Swadesh_lists}{Swadesh list}.

% \bibliographystyle{plain}
% \bibliography{main}

\end{document}
