\documentclass{article}

\usepackage{xeCJK}
\setCJKmainfont{Hiragino Mincho ProN}
%\setCJKmainfont{Songti SC}

\usepackage{pandekten}

\title{Syntax}
\author{Ch\=an Taku}

\begin{document}

\maketitle

% \bibliographystyle{plain}
% \bibliography{main}

\section{Attribute}

\subsection{Genitive Attribute}

\begin{example}{Genitive Attribute}{genitive_attribute}
    In German, genitive attribute is formed by the genitive case, e.g.
    \begin{center}
        \textit{der Beruf des alten Mannes.}
    \end{center}
    In Japanese, the genitive case is formed by the particle の, e.g.
    \begin{center}
        田中さんの本.
    \end{center}
\end{example}

\subsection{Position Attribute}

\begin{example}{Position Attribute}{position_attribute}
    In German, position is indicated by prepositions, e.g.
    \begin{center}
        \textit{eine Wolke am Himmel.}
    \end{center}
    In Japanese, position is indicated by a noun plus の, e.g.
    \begin{center}
        山の中の屋.
    \end{center}
\end{example}

\subsection{Attribute Phrase}

\begin{example}{Attribute Phrase}{attribute_phrase}
    In written German, participles may be used to form \href{https://en.wikipedia.org/wiki/German_grammar#Extended_attribute_phrase}{lengthy nominal modifiers}, e.g.
    \begin{center}
        \textit{der während des Bürgerkrieges amtierende Premierminister.}
    \end{center}
    In Japanese, such usage is with 連体形 and is a special case of relative clause, e.g.
    \begin{center}
        カバンを持った人.
    \end{center}
\end{example}

\subsection{Relative Clause}

\begin{example}{Relative Clause}{relative_clause}
    In German, a relative clause may be finite (starting with a pronoun), e.g.
    \begin{center}
        \textit{Das Haus, in dem ich wohne, ist sehr alt,}
    \end{center}
    or non-finite (formed with participles), e.g.
    \begin{center}
        \textit{Die von ihm in jenem Stil gemalten Bilder sind sehr begehrt,}\\
    \end{center}
    and
    \begin{center}
        \textit{Die Regierung möchte diese im letzten Jahr eher langsam wachsende Industrie weiter fördern.}
    \end{center}
    In Japanese, a relative clause modifies a noun with 連体形, e.g.
    \begin{center}
        自分の住む家.
    \end{center}
\end{example}

\section{Clause}

\subsection{Content Clause}

\begin{example}{Content Clause}{content_clause}
    In German, content clauses may be finite (starting with e.g. \textit{dass}), e.g.
    \begin{center}
        \textit{Ich hoffe, dass ich bald perfekt Deutsch spreche,}
    \end{center}
    or non-finite (with \textit{zu}-infinitive), e.g.
    \begin{center}
        \textit{Ich hoffe, bald perfekt Deutsch zu sprechen.}
    \end{center}
    The finite clause should be used \href{https://en.easy-deutsch.de/sentence-structure/infinitive-with-zu/}{if the subject isn't in the main clause}.
    \par
    In Japanese, a content clause may be formed with こと or の, e.g.
    \begin{center}
        先生に学校を休むことを伝える.
    \end{center}
\end{example}

\end{document}
