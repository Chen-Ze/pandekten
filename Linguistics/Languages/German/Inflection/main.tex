\documentclass{article}

\usepackage{pandekten}

\title{Inflection}
\author{Ch\=an Taku}

\begin{document}

\maketitle

\section{Nouns}

\begin{theorem}{Declension of Nouns}{declension_of_nouns}
    Let \texttt{N} be a regular noun.
    \texttt{NS}, \texttt{GS} and \texttt{NP} are the nominative singular, genitive singular and nominative plural of \texttt{N}, respectively.
    Then the declension of \texttt{N} is given by the following table.
    \begin{longtable}{lll}
        \toprule
        Case & Singular & Plural \\
        \midrule
        Nominative & \texttt{NS} & \texttt{NP} \\
        Genitive & \texttt{GS} & \texttt{NP} \\
        Dative & Depends & \texttt{NP + 'n'} if \texttt{NP[-1]} not \texttt{/n|s/} \\
        & & \texttt{NP} otherwise \\
        Accusative & Depends & \texttt{NP} \\
        \bottomrule
    \end{longtable}
    The declension of singular cases are given by \cref{thm:strong_declension} and \cref{thm:weak_declension}, depending on \texttt{N}.
\end{theorem}

\begin{theorem}{Strong Declension}{strong_declension}
    Let \texttt{N} be a regular strong noun.
    \texttt{NS} and \texttt{GS} are the nominative singular and genitive singular of \texttt{N}, respectively.
    Then the singular declension of \texttt{N} is given by the following table.
    \begin{longtable}{ll}
        \toprule
        Case & Singular  \\
        \midrule
        Nominative & \texttt{NS} \\
        Genitive & \texttt{GS} \\
        Dative & \texttt{NS + 'e'} in fixed expressions if \texttt{N} is m. or n. \\
        & \texttt{NS} otherwise \\
        Accusative & \texttt{NS} \\
        \bottomrule
    \end{longtable}
\end{theorem}

\begin{example}{Dativ-e}{dativ_e}
    \begin{itemize}
        \item \textit{im Stande sein} (to be able);
        \item \textit{nach Hause} (adv., home).
    \end{itemize}
\end{example}

\begin{theorem}{Weak Declension}{weak_declension}
    Let \texttt{N} be a regular weak noun.
    \texttt{NS} and \texttt{GS} are the nominative singular and genitive singular of \texttt{N}, respectively.
    Then the singular declension of \texttt{N} is given by the following table.
    \begin{longtable}{ll}
        \toprule
        Case & Singular  \\
        \midrule
        Nominative & \texttt{NS} \\
        Genitive & \texttt{GS} \\
        Dative & \texttt{GS}, minus the final \texttt{'-s'} if there is \\
        Accusative & \texttt{GS}, minus the final \texttt{'-s'} if there is \\
        \bottomrule
    \end{longtable}
\end{theorem}

\begin{proposition}{Criteria of Weak Declension}{criteria_of_weak_declension}
    The following nouns are likely to be weak nouns.
    \begin{itemize}
        \item Masculine nouns that end with unstressed \texttt{'e'}.
        \item Many nouns that represent professions or nationalities of people.
        \item Latin or Greek words ending with \href{https://www.germanveryeasy.com/m/noun-declension}{\texttt{'-at'}, \texttt{'-ant'}, \texttt{'-ent'}, or \texttt{'-ist'}}.
    \end{itemize}
    Weak nouns that \href{https://www.vistawide.com/german/grammar/german_nouns03.htm}{do not refer to people or animals} have genitive singular ending in \texttt{'-ns'}.
\end{proposition}

\subsection{Genders}

Patterns for genders may be found in \href{https://germanwithlaura.com/noun-gender/}{German Noun Gender: Your Essential Guide}.

% \bibliographystyle{plain}
% \bibliography{main}

\end{document}
